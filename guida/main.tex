\documentclass[a4paper,12pt]{book}

\usepackage[utf8]{inputenc}
\usepackage[T1]{fontenc}
\usepackage{graphicx}
\usepackage{hyperref}
\usepackage{xcolor}

\title{Guida a Cnot\\{\small Un viaggio tra scienza e narrazione}}
\author{Generato da un'Intelligenza Artificiale}
\date{\today}

\begin{document}

\maketitle

\chapter*{Prefazione}
Benvenuto in questa guida a *Cnot*! 🚀 Questo librettino, generato da un'intelligenza artificiale, ha lo scopo di rendere accessibili i concetti scientifici e tecnologici che incontrerai nel libro. Non preoccuparti se qualcosa ti sembra complesso: cercheremo di spiegarlo con esempi semplici e curiosi!

\tableofcontents

\chapter{Introduzione a *Cnot*}
\section{Cos'è *Cnot*?}

\section{Come leggere questo libro}

\chapter{Concetti chiave}
\section{La computazione quantistica}


\section{Entanglement: quando due particelle si parlano}


\section{Il ruolo dell'informazione}


\chapter{Guida ai capitoli di *Cnot*}
\section{Capitolo 1: Il labirinto digitale}
\chapter{I Fondamenti di Informatica e Calcolo Quantistico}

In questo capitolo spieghiamo alcuni concetti scientifici e informatici fondamentali, presentati in modo semplice.

\section{Gli Algoritmi}
Un algoritmo è una serie di istruzioni che il computer segue per risolvere un problema. Per esempio, per cercare un numero in una lista possiamo usare un algoritmo chiamato \emph{ricerca lineare}. Il computer controlla ogni elemento della lista fino a trovare quello giusto.

\begin{lstlisting}[language=Python, caption={Esempio di Ricerca Lineare}]
def ricerca_lineare(lista, target):
    for i in range(len(lista)):
        if lista[i] == target:
            return i
    return -1
\end{lstlisting}

\section{Il Calcolo Modulare e la Fattorizzazione}
Un concetto importante è il \emph{calcolo modulo}, che ci dice il resto della divisione tra numeri. Ad esempio, nella funzione
\[
f(x)=2^x \mod 15
\]
calcoliamo il resto della divisione di \(2^x\) per 15. Notiamo che i risultati si ripetono: questo si chiama \emph{periodicità}. Trovare questo periodo è un passo chiave per fattorizzare un numero, cioè per trovare i suoi divisori.

\section{Il Calcolo Quantistico e l'Algoritmo di Shor}
I computer classici usano bit, che possono essere 0 o 1. I computer quantistici, invece, usano \emph{qubit} che possono essere 0 e 1 allo stesso tempo grazie alla \emph{sovrapposizione}. Questo permette loro di eseguire molti calcoli contemporaneamente.

L'algoritmo di Shor sfrutta il calcolo quantistico per fattorizzare numeri grandi in modo molto veloce. I punti chiave sono:
\begin{itemize}
  \item \textbf{Sovrapposizione}: Permette a un qubit di essere in più stati contemporaneamente.
  \item \textbf{Interferenza}: I qubit si combinano in modo da enfatizzare i risultati giusti.
  \item \textbf{Trasformata di Fourier Quantistica (QFT)}: Uno strumento che aiuta a trovare la periodicità nella funzione, un passaggio fondamentale per fattorizzare numeri.
\end{itemize}

Con il QFT, il computer quantistico riesce a trovare il periodo della funzione \( f(x)=2^x \mod N \) molto più velocemente rispetto a un algoritmo classico, riducendo il tempo necessario da esponenziale a polinomiale.

\section{Differenze tra Calcolo Classico e Quantistico}
Nei computer classici:
\begin{itemize}
  \item I dati sono rappresentati da bit (0 o 1).
  \item Gli algoritmi devono controllare ogni passo uno per uno.
\end{itemize}

Nei computer quantistici:
\begin{itemize}
  \item I dati sono rappresentati da qubit, che possono essere in uno stato di 0 e 1 allo stesso tempo.
  \item Grazie alla sovrapposizione e all'interferenza, si possono eseguire molti calcoli in parallelo.
\end{itemize}

Questo significa che alcuni problemi molto complessi, come la fattorizzazione di numeri grandi, possono essere risolti in tempi molto più brevi con il calcolo quantistico.

\section{Vecchie Tecnologie e Nuovi Concetti}
Anche se oggi usiamo computer avanzati, un po' come lo ZX Spectrum o il Commodore 64, i concetti di base della programmazione erano già presenti. Questi vecchi computer ci insegnano come le idee alla base degli algoritmi e del calcolo sono nate tanti anni fa, e come la tecnologia si è evoluta nel tempo.

\vspace{1em}
In sintesi, anche se i concetti di algoritmi, calcolo modulo e calcolo quantistico possono sembrare complicati, sono solo strumenti che ci aiutano a risolvere problemi e a far funzionare i computer in modo intelligente. Con pazienza e pratica, anche le idee più avanzate diventano più facili da capire!


\section{Capitolo 2: L'inganno quantistico}
g

\section{Capitolo 3: La prigione del supervisore}


\section{Capitolo 4: L'interrogatorio quantistico}


\section{Capitolo 5: La trappola per ioni}


\section{Capitolo 6: Il drone ribelle}


\section{Capitolo 7: Il salto di Hadamard}


\section{Capitolo 8: L'entanglement in azione}


\section{Capitolo 9: La sfida RSA}


\section{Capitolo 10: Il ritorno quantistico}


\chapter*{Conclusione}

\end{document}
