% chapters/chapter05.tex
\chapter{Che lavoro fa Mark?}

La domanda l'ha spiazzata e l'ipotesi di Laura ancora di più. Rimane attonita a guardare l'amica.

\par\medskip
«Vale vai a lavarti le mani, è pronto»\\
\par\medskip
«Deve venire anche Caterina però!»\\
\par\medskip
«Vai pure Cate, ne parliamo dopo cena, ma credo di aver capito il problema, e forse ho un’idea per sbloccare la situazione. Però ora mangiamo, le abitudini sono fondamentali per Valentina…»\\
«Vado a lavarmi le mani.»

\par\medskip
Hanno finito di mangiare: Rocky dorme ai piedi di Cate sazio di quanto gli è stato passato. Brutta abitudine dare da mangiare ai pulciosi da tavola, difficilmente vi rinunceranno!\\
Laura si alza per sparecchiare, Valentina scappa sul tappeto per finire i suoi disegni, mentre lo sguardo di Caterina torna serio e preoccupato:

\par\medskip
«Lo sapevo che il lavoro di Mark prima o poi mi avrebbe ostacolato!»

\par\medskip
«È probabile che la compagnia pretenda controlli stretti sui parenti. Non possono rischiare: tu hai già manifestato contro di loro.»

\par\medskip
«È il colmo. Non posso andare a New York perché il mio fidanzato lavora per una compagnia che detta le strategie energetiche. Quindi la mia libertà dipende dai miei affetti? Come posso accettarlo, Laura?»\\
«Cate…»\\
«no, scusami, tu non c’entri.»\\
«Cate, posso fare un paio di telefonate»\\
«Ma Laura, non volevo…»\\
«Non preoccuparti. Ora però lasciami da sola, metto a letto Vale e me ne occupo.»\\
«Sì, vado. Ormai deve tornare anche Alice.»\\
«Mi fai sapere?»\\
«Sì, tranquilla. Ti accompagno.»

\par\medskip
Laura e Caterina escono sulla pedana e raggiungono il prato.\\
«Fa fresco…»\\
«Mi sa che hai lasciato la giacca in casa. Te la porto.»

\par\medskip
Caterina rimane sola. Alza lo sguardo verso le stelle.\\
«Eccolo, Cate.»\\
«Grazie.»\\
«Allora, mi aggiorni?»\\
«Certo.»

\par\medskip
Caterina si dirige verso casa mentre Laura guarda l’amica diventare una luce nel firmamento.

\par\medskip
«Andiamo a letto!»\\
La voce di Valentina la raggiunge. La sente forte.

\par\medskip
Rientra. Valentina si è già lavata i denti e ha steso i due futon in camera.\\
«Hai proprio sonno?»\\
«No, è che voglio la storia. Dai, leggiamo!»\\
«Ma… ok, però solo un capitolo, perché poi devo fare una cosa per Cate.»\\
«Cosa devi fare?»\\
«Una cosa da grandi. Meglio se non te la dico, sennò poi dovrei tagliarti la lingua!»

\par\medskip
Valentina ride. Laura prende il manga dalla libreria in sala e si sdraia sul futon vicino a lei.\\
Raggiungono il segno dove erano arrivate a leggere la sera prima. Vale appoggia la testa sulle spalle di Laura, che comincia a leggere le nuvolette.

\par\medskip
Scorrono cinque pagine, poi Vale sbadiglia.\\
«Hai sonno? Non finiamo il capitolo?»\\
«È lo stesso, adesso dormo. Tu cosa fai?»

\par\medskip
«Devo fare quella cosa per Caterina»\\
«Posso venire con te?»\\
«Ma non hai detto che hai sonno»\\
«Si, ma voglio vedere che…»

\par\medskip
Valentina non finisce la frase, ha davvero sonno. Comunque Laura è abituata a vederla crollare di sera, d’altra parte anche se non è una psicologa conosce gli effetti della perdita anche su sé stessa.\\
Lascia la porta della loro stanza socchiusa. Prende la caraffa di tè freddo dal frigo e lo versa in un bicchiere di vetro grosso di color arancione. Dal credenzino sotto la libreria prende il DPL, un dispositivo di programmazione con kernel Linux. Non farà telefonate, forzerà il server della polizia municipale e cancellerà solo qualche dato, senza fare danni. Un’azione discutibile ma efficace.\\
«Vediamo la situazione»

\par\medskip
Laura schizza la rete: i dati di Cate devono svanire dal server giusto un attimo prima che il DHS, via TECS, chieda l’accesso. Poi tornano, invisibili come se nulla fosse.

\par\medskip
\begin{center}
  \includegraphics[width=.9\linewidth]{rete senza sfondo.png}
\end{center}

\par\medskip
«Abbiamo 128 minuti per mappare l'infrastruttura e segnare i punti deboli. Prima una perlustrazione a bassa intensità, senza far rumore. Poi agganciamo le difese, quasi sicuro sulle porte 80 e 22. Da lì pianifichiamo l’intrusione: passare il firewall, beccare uno user debole e la sua password. Se va bene troviamo pure una falla SQL, alteriamo quel che serve e spariamo prima che partano i controlli.\\
Ce la possiamo fare, Rocky.»

\par\medskip
Allunga la mano, d’istinto, verso la cuccia.

\par\medskip
«Rocky? Dove sei Rocky?»
