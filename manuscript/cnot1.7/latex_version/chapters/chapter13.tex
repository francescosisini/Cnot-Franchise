\chapter{Pane e giustizia}

«Finalmente…»

\par\medskip
C’è una candela sul tavolo e alla sua luce Luca, Giovanni e il vagabondo stanno per condividere il cibo del pacco sulla mensa rischiarata dalla sua luce.  
Nella sala ora ci sono solo loro quattro, tre più Ippa.  
Chi stava cenando quando Luca è caduto nel pozzo ha preferito mantenersi alla larga.

\par\medskip
Dividono il contenuto del pacco e tengono da parte quello che si può per i prossimi giorni.

\par\medskip
Luca è vestito solo con un asciugamano. I suoi vestiti sono appesi fuori.  
Il sole del mattino li asciugherà.

\par\medskip
«Guarda, Ippa, c’è anche la scatoletta per te, che gentili.»

\par\medskip
«Grazie, Luca. Io, l’avevo scambiata per pomodoro.»

\par\medskip
«Peccato non possa mangiarlo con Rocky…»

\par\medskip
«Una bocca in meno da sfamare.»

\par\medskip
«Se non ci aiutavano Laura e Caterina ci sarebbero state due bocche in meno da sfamare…»

\par\medskip
«Sì, ma se ti ascoltavano e restavano a cena ora c’erano due bocche in più da sfamare.»  
Luca ride.

\par\medskip
«Ridi perché non comprendi, caro Luca, ma… “Dare da mangiare è il primo gesto di giustizia.”»

\par\medskip
Luca apre i bocconcini e ne dà un po’ a Ippa.  
Lei si avvicina con il muso alla ciotola ed ogni boccone ritrae la testa di qualche centimetro.  
Si guarda rapidamente intorno e torna a mangiare.  
Mangia diversamente da Rocky.

\par\medskip
«Così mangiamo tutti insieme da veri amici.»

\par\medskip
«Ma l’amicizia non è nutrire. È digiunare insieme.»

\par\medskip
«Non vi sopporto voi filosofi. La prossima volta mangio con le casalinghe.»

\par\medskip
«Ti mangiano loro.»

\par\medskip
Giovanni tasta il braccio di Luca.

\par\medskip
«Sei bello pienotto.»

\par\medskip
«Mi sa che se non la liberavi tu, Caterina, finiva mangiata.»

\par\medskip
Ridono. Si stanno rilassando.  
Il momento non deve essere stato facile.  
Questo convitto è una piccola comunità, ma sembrano tagliati fuori dai servizi base.  
Non sono coperti neanche dalla connettività globale.  
Però su al secondo piano altri ragazzini come Luca giocavano con vecchie console,  
per cui o si sono procurati qualche accumulatore o hanno un generatore elettrico.

\par\medskip
\begin{center}
  \includegraphics[width=0.75\linewidth]{nintendo box.png}
\end{center}

\par\medskip
Comunque non ho visto luce eccetto in casa, se non qualche candela.  
È un ecosistema a parte.  
A quanto pare Caterina aveva ragione, anche se ha avuto qualche difficoltà  
nel passare dalla teoria alla pratica.

\par\medskip
Comunque, ovunque siate, se un carrello della spesa vi sbarra la strada,  
aspettate a spingerlo via. Potrebbe non essere la scelta migliore.
