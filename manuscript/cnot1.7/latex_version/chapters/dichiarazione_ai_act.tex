\clearpage
\thispagestyle{empty}

\begin{center}
\vspace*{2cm}
{\Large \textbf{Dichiarazione sull’uso dell’Intelligenza Artificiale}}\\[1.5cm]
\end{center}

Ai sensi del nuovo regolamento europeo sull’Intelligenza Artificiale (AI Act),
gli autori dichiarano che, nella realizzazione di questa opera, sono stati
utilizzati sistemi di IA (tra cui ChatGPT, modello 5) in modo limitato e
controllato, con finalità di supporto creativo, analitico e redazionale.

L’IA non ha elaborato in autonomia la costruzione narrativa,
ma è stata impiegata come strumento di assistenza nelle seguenti attività:

\begin{enumerate}
  \item \textbf{Analisi della prima stesura e individuazione di incongruenze}:  
  suggerimenti di coerenza interna, segnalazione di passaggi poco chiari e
  rilevazione di discontinuità, sempre sotto revisione dell’autore.

  \item \textbf{Analisi dei dialoghi e verifica della coerenza psicologica}  
  con i profili NEO PI-R dei personaggi:  
  valutazione della compatibilità tra linguaggio, temperamento e azioni dei
  personaggi, senza generazione autonoma di nuove scene.

  \item \textbf{Sviluppo e revisione delle schede tecniche}:  
  supporto nella formulazione delle componenti tecniche, scientifiche e
  infrastrutturali presenti nell’opera.
\end{enumerate}

La Prefazione è l’unica sezione interamente generata dall’IA, come chiaramente
indicato nel testo, mentre ogni altro contenuto narrativo è stato concepito,
sviluppato e finalizzato dagli autori, con l’IA utilizzata esclusivamente come
strumento di supporto editoriale.

\medskip
\begin{flushright}
\textit{Ferrara, 2025}
\end{flushright}

\clearpage
