\chapter{Il tranello della mensa}

«Piano Luca, non sono pratico di questo piano, scusami.»

\par\medskip
La scena è da manuale di arte drammatica: Luca procede al buio del secondo piano del convitto del Cardinal Mora, reggendo il pacco con la spesa. Giovanni lo segue a ruota, seguito a sua volta da Ipparchia che seguita da Rocky.

\par\medskip
Il trenino poli-specie vuole raggiungere la cucina al piano terra senza dare nell’occhio, perché le casalinghe non gradirebbero sicuramente un secondo cane. Giovanni non ha mai subito le loro punizioni: è grande, e rispettando sa farsi rispettare, ma Luca, con i suoi problemi di iperattivismo, è finito diverse volte in gattabuia e preferisce non finirci ancora.

\par\medskip
«Passiamo per la palestra e raggiungiamo le scale.»\\
«Va bene, Luca, fai tu strada.»

\par\medskip
Altri ragazzini sono presenti nella palestra e stanno giocando con videogiochi retrò. Luca li saluta a cenni, ma non si ferma con loro.

\par\medskip
L’attenzione di tutti è verso Rocky: la curiosità è alta, ma nessuno lascia la console, anche se gli occhi vanno e vengono dal cane.

\par\medskip
Il trenino esce dalla seconda porta e prende le scale. Come vedi qui sotto le due porte sono allineate alla tromba.

\par\medskip
\begin{center}
  \includegraphics[width=0.55\linewidth]{piano 1.png}
\end{center}

\par\medskip
\begin{center}
  \includegraphics[width=0.55\linewidth]{piano 2.png}
\end{center}

\par\medskip
Giovanni si tiene radente al muro. Raggiungono il primo piano.  
Le casalinghe sono al lavoro e si sente benissimo.

\par\medskip
Ma basta passare un attimo, basta continuare a scendere le scale.

\par\medskip
In mezzo alla rampa c’è un buco. Giovanni lo sa.  
È radente al muro, ma conosce quel buco.

\par\medskip
Per un attimo pensa a Rocky, ma anche lui è e rimane radente al muro.

\par\medskip
Per ora nessun problema.\\
Un’altra rampa e raggiungono la Hall al piano terra.

\par\medskip
\begin{center}
  \includegraphics[width=0.55\linewidth]{piano terra.png}
\end{center}

\par\medskip
Qui Giovanni si muove sicuro, punta dritto alla porta della ex Mensa, mentre Luca lo segue con affanno.  
Il pacco comincia a farsi sentire.

\par\medskip
Rocky si guarda intorno. I cavalletti devono sembrargli ottimi per una pausa pipì, ma per fortuna non ne approfitta.

\par\medskip
Una candela è accesa. Qualcuno sta già mangiando.  
Ippa non cede al richiamo dell’odore, ma Rocky non è ancora altrettanto disciplinato.  
Si distacca dalla comitiva e va verso il tavolino.

\par\medskip
Luca lo guarda.\\
«No, Rocky, fermo!»

\par\medskip
È rapido. Non è ancora un ninja, ma ha la prontezza di lasciare andare i beni sostituibili per provare a salvare una vita non sostituibile.  
Il pacco cade mentre lui si lancia per afferrare il nuovo amico.

\par\medskip
Dopo sessanta secondi la situazione è questa: in mensa è rimasto solo Giovanni.  
Chi altro c’era si è dileguato.

\par\medskip
Luca è vigile e cosciente, ma malamente incastrato non riesce a far forza.  
Giovanni parla con lui mentre Rocky abbaia.

\par\medskip
«Mi arriva alla bocca. Non riesco a tenere su la testa.»\\
«Stai tranquillo, non salirà di livello.»\\
«Sì, sì.»\\
«Ci sono io, ti faccio uscire.»

\par\medskip
Giovanni chiama ancora aiuto, ma anche qui nel convitto le cose sembrano andare come fuori.

\par\medskip
«Vado a chiamare qualcuno. Torno subito.»\\
«Non lasciarmi da solo!»\\
«C’è Rocky, arrivo subito.»

\par\medskip
Ippa lo segue. Corre nella hall e poi verso l’entrata principale.  
Solleva la barra e si fa giusto lo spazio per passare.

\par\medskip
«Ehi amico, che passa?»\\
Una voce lo coglie appena si fa fuori.\\
«Ho bisogno. Uno dei ragazzini è finito nel pozzo.»
