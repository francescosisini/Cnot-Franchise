\chapter{Missione per Rocky}

Ha lasciato Valentina in casa da sola per cercare Rocky.\\
Ma nei pressi di casa non lo trova.

\par\medskip
Adesso si trova di fronte a un dilemma: proseguire la sua intrusione etica ed aiutare l’amica nel poco tempo rimasto, o cercare il suo cucciolo prima che si ritrovi in qualche guaio?

\par\medskip
«Cate, ho bisogno di un favore. Rocky è scappato.»\\
«Oddio, quando?»\\
«Non lo so. Forse quando sono entrata per prenderti la giacca. Comunque devo andare a cercarlo. Però ho bisogno per Vale.»\\
«Certo. Dammi dieci minuti.»

\par\medskip
Dopo tredici minuti Caterina e Alice atterrano nel giardino di Laura. Quando sente il rumore lei esce sulla pedana, già pronta per la sua \emph{rescue action}. Non ha lasciato nulla al caso.

\par\medskip
Un cane in un territorio misto può perdersi ovunque: da una linea abbandonata della metro a un centro di assistenza per l’intelligenza artificiale. Comunque, a quanto risulta, dal suo segnalatore non è andato troppo lontano. Sembra essere in un boschetto a poche centinaia di metri.

\par\medskip
I problemi si risolvono creando ordine: Laura sistema il suo zaino tattico perché il suo contenuto determinerà il successo della missione. Come a scuola: infilare nello zaino il quaderno sbagliato potrebbe compromettere la giornata!\\
– una corda in microfibra mano tessuta,\\
– una carrucola smart a riduzione di carico,\\
– guanti grip GECO,\\
– un micro drone da zaino con telecomando e telecamera termica,\\
– un visore Augmented Reality multifunzionale,\\
– sfere da esplorazione,\\
– spray a schiuma rapida,\\
– una batteria al grafene ultracompatta,\\
– una radio Mesh Network autonoma e un beacon personale,\\
– infine un esoscheletro pieghevole e una barella smart ultracompatta.

\par\medskip
Alice la osserva ammirata: Laura nella sua tuta nera attillata.

\par\medskip
«Sembri Tomb Rider! Ma cosa c’è nello zaino?»\\
«Niente, amore, ho preso i suoi croccantini per chiamare Rocky.»

\par\medskip
Non mente, ci sono anche croccantini.

\par\medskip
«Dovrebbe essere qui vicino. Arrivo subito.»\\
«Laura, posso venire con te? Può restare Alice con Vale?»\\
«In effetti…»\\
«Sì, ci penso io.»

\par\medskip
Caterina indossa una minigonna, una camicetta e un paio di scarpette con un po’ di tacco.  
Ha un abbigliamento perfetto per quello che le attende.

\par\medskip
«Ok, mi raccomando, Alice.»

\par\medskip
Alice non risponde. Sorride, entra in casa e le guarda dalla vetrata.

\par\medskip
Prenderanno il drone? No, sembra si avviino a piedi lungo la capezzagna.  
Qui ci sono già passate insieme, quella sera, prima della loro avventura nel computer quantistico.  
Sì, quella è stata davvero una bella sfida: la posta in gioco era alta, la libertà di essere decoerenti.  
Ma ora la questione è altrettanto seria.

\par\medskip
La strada termina in un incrocio a T.  
Il sensore indica che Rocky si trova davanti a loro a circa ottanta metri, non a destra e non a sinistra.

\par\medskip
\begin{center}
  \includegraphics[width=0.78\linewidth]{piantina rocky senza sfondo.png}
\end{center}

\par\medskip
«Dovremo entrare nel campo?»\\
«Se procediamo e prendiamo qui a sinistra, seguendo la strada, ci possiamo avvicinare un po’.  
Ma alla fine credo che dovremmo comunque entrare. Rocky mi sembra in mezzo al campo.»

\par\medskip
«Di quanto la allunghiamo?»\\
«Circa seicento metri, direi. Solo cinque minuti. Cosa ne dici?»\\
«Forse era meglio venire con il drone?»\\
«Rocky ha paura dei droni, Cate, sarebbe stato peggio.»\\
«Dai, decidi tu allora, Laura.»\\
«Beh, allora avviciniamoci seguendo la strada, poi valutiamo.»\\
«Cosa dice la posizione?»\\
«Sembra fermo, speriamo stia bene, non rilevo i parametri biometrici.»\\
«Cosa significa, Laura?»\\
«Ma niente, mi sa che semplicemente non ho rinnovato l’abbonamento.»\\
«Qui svoltiamo a destra.»\\
«Ehi! È messa male questa strada.»\\
«Tutta l’area ormai è un po’ andata.»\\
«Ecco, di nuovo a destra e tra un centinaio di metri dovremmo esserci.»\\
«Non siamo mai venute qui insieme, vero?»\\
«No, ci siamo fermate alla panchina. Comunque anch’io erano anni che non facevo questa strada.  
Forse l’ultima volta sono venuta con mio padre, poco prima che nascesse Valentina.»\\
«A proposito, chissà come se la cavano quelle due.»\\
«Dietro il cespuglio c’è un cancello, ma non mi risulta nella mappa.»\\
«Comunque Rocky dovrebbe essere qui a trenta metri, solo che non vedo come entrare.»\\
«Vediamo se c’è un varco. Qui, Laura.»\\
«Grazie, Cate, sono contenta che tu sia venuta con me.»\\
«Dai, proviamo a entrare.»\\
«Aspetta, ma c’è una scritta: Convitto Cardinal Mora.  
Lavori di ricostruzione. Ma la data è del 1970! Ecco perché non è più nella mappa.»\\
«Quindi Rocky sarà qui dentro.»\\
«Ora lo scopriamo.»

\par\medskip
Laura e Caterina superano il confine tra il presente e questo mondo che appartiene a un passato dimenticato.

\par\medskip
Riusciranno a uscire indenni anche questa volta?
