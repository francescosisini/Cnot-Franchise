\chapter{La torta di mezzanotte}

Con un colpo secco e forza ben calibrata, ne sbatte uno contro lo spigolo del tavolo.  
Lo guarda colare dalla frattura.  
Nella mano sinistra regge un pacchetto di farina, che aggiunge al burro e all’uovo.

\par\medskip
«Ora mescoliamo per bene.»
\par
«Sì!»
\par\medskip
Valentina è entusiasta. Laura cucina raramente con lei dei dolci, mentre Alice sembra esperta.
\par\medskip
«Vuoi provare? Spingi bene in modo che si sciolga il burro.»\\
«Così?»\\
«Sì, così va bene. Vale, continua a mescolare.»
\par\medskip
Mentre Valentina mescola, Alice, a suo agio, sceglie la musica per accompagnare la preparazione.
\par\medskip
\begin{center}
  \includegraphics[width=0.8\linewidth]{crostata senza sfondo.png}
\end{center}

\par\medskip
«Ci siamo, Vale? Adesso in frigo per dieci minuti.»
\par\medskip
Si lavano le mani e insieme riprendono a guardare la foto dove Alice aveva interrotto.
\par\medskip
«Questa sei tu, piccolina?»\\
«Sì, ero appena nata. E quella è Laura, con la mamma. Il papà sta facendo la foto.»\\
«Ti mancano tanto?»\\
«Non le ricordo più tanto.»
\par\medskip
Il timer suona. Il burro si è raffreddato.  
E ora hanno un bel panetto di farina, uova, zucchero e burro.  
Da trasformare in pasta frolla.
\par\medskip
«E quello?»\\
«Con questo ci facciamo i bigoli. Non hai mai mangiato la crostata?»\\
«Ah, sì, adesso ho capito. Quelli sopra la marmellata?»\\
«Esatto.»\\
«Questa l’abbiamo preparata io e Laura.»\\
«Fantastico. Adesso la spalmiamo…»\\
«E intanto facciamo i bigoli.»\\
«Bravissima. Ora in forno trenta minuti.»\\
«Chissà che sorpresa quando tornano.»
\par\medskip
Vale va a prendere il suo futon in camera e lo porta in soggiorno.  
Si sdraiano insieme, lei e Alice.
\par\medskip
«Laura ti assomiglia molto da piccola.»\\
«E tu e Caterina vi assomigliate?»\\
«Non tanto. Lei dice, io ho i capelli lisci.»\\
«Allora vi assomigliate solo quando avete i capelli bagnati.»\\
«È vero, almeno così ci assomigliamo.»\\
«Sei simpatica, Alice.»\\
«Anche tu sei simpatica.»
\par\medskip
Valentina si addormenta, mentre Alice aspetta perché vuole controllare la cottura.  
Sono ormai le 23. Alle 24 sarà troppo tardi per la revisione del visto,  
ma di questo Alice non sa nulla.  
È solo ansiosa per il fine serata, che sembra aver preso una piega diversa da come era cominciato.  
Sta ripensando alle parole della sua canzone,  
a come SUNO ha trasformato il suono della sua chitarra in una polifonia quasi perfetta,  
agli archi che hanno reso i suoi sentimenti reali, o meglio tangibili quanto lo può essere il timbro di un'onda sonora.
\par\medskip
Le sue parole hanno cristallizzato il suo dolore,  
ma il profumo della crostata la riporta alla vita che non si cristallizza,  
ad una serata che potrebbe finire con un dolce da mangiare insieme.  
Anche insieme a sua sorella.
