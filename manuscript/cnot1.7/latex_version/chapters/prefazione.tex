\clearpage
\thispagestyle{empty}

\begin{center}
\vspace*{2cm}
{\Large \textbf{Prefazione}}\\[1.5cm]
\end{center}

Questa prefazione è stata scritta da ChatGPT, modello 5,
sulla base di una serie articolata di indicazioni fornite dall’autore.  
Il testo è nato da un processo dialogico: l’autore ha definito l’orientamento
tematico, il ruolo del volume all’interno della trilogia e la sensibilità
narrativa complessiva; il modello ha tradotto tali elementi in una forma unitaria,
con uno stile coerente e rispettoso dell’opera.  
La prefazione, e solo la prefazione, è quindi di origine interamente generativa,
mentre il romanzo rimane integralmente opera dell’autore.

\medskip

\noindent
Questo libro segue il primo \textit{Cnot} a distanza di un anno, ma la sua
prospettiva è completamente diversa.  
Se nel primo volume il mondo era osservato dal punto di vista della persona,
qui è l’Europa stessa a entrare in scena: non come entità politica,
non come cornice istituzionale, ma come spazio umano che comincia a
riformarsi, lentamente, nelle pieghe della vita quotidiana.

La grande infrastruttura di calcolo — reti neurali, sistemi quantistici,
sensori distribuiti, ambienti predittivi — non è più un’astrazione tecnica:
è diventata l’aria sociale in cui i personaggi si muovono.  
È un campo che reagisce alle loro scelte, che amplifica tensioni, che offre
protezione e allo stesso tempo impone nuove vulnerabilità.

In questo contesto Caterina, Laura e Giovanni non sono eroi
né rappresentazioni simboliche: sono persone  
che cercano di capire cosa significhi vivere dentro un continente che
sta cambiando la propria forma, e che per la prima volta in molti decenni
prova a immaginarsi come un’unica storia.

\textit{Cnot 1.7} esplora questo passaggio:  
il momento fragile in cui un sistema antico non è più sufficiente,
e uno nuovo non è ancora nato del tutto.  
È un tempo intermedio, in cui l’Europa appare come una promessa non detta,
un progetto che non si afferma con dichiarazioni, ma attraverso piccole
riconfigurazioni della vita ordinaria: la gestione delle risorse,
le relazioni affettive, il movimento delle persone, la sicurezza,
la fatica di restare se stessi quando il mondo intorno accelera.

Nessuno dei personaggi comprende appieno ciò che accade, e proprio per questo
sono perfetti testimoni del mutamento:  
perché lo percepiscono nei dettagli — nei ritardi burocratici,
nelle tecnologie che cominciano a parlare una lingua nuova,
nella trasformazione silenziosa dei luoghi che abitano.

Se il primo \textit{Cnot} affrontava il problema della decoerenza personale,
qui la domanda diventa collettiva:  
\textit{come si vive dentro un continente che tenta, per la prima volta,
di costruire un campo cognitivo comune?}

Non è un romanzo di istituzioni, né un trattato politico.  
È la storia di come una società si accende dal basso:  
dal dolore, dal vincolo, dall’amicizia, dai conflitti,
dalla capacità di cambiare senza spezzarsi.

È da qui che prende forma la trilogia:  
dal punto in cui il destino individuale comincia a disegnare,
senza volerlo, il destino di un intero continente.

\medskip
\begin{flushright}
\textit{Ferrara, 2025}
\end{flushright}

\clearpage
