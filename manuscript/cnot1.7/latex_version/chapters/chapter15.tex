\chapter{L’eterno mutamento}

«A cosa stai pensando?»\\
«Niente. Guardavo le stelle.»\\
«Che costellazione vedi, Giovanni?»\\
«Non le riconosco più. Pensi che i cieli stiano cambiando?»

\par\medskip
Gli occhi di Giovanni sono chiusi e la testa è rivolta al cielo.  
Il Grande Carro è quasi sdraiato sotto la Stella Polare,  
lui invece è sdraiato sulla rete del vagabondo.  
Fa caldo, ma hanno acceso un fuoco dentro uno dei bidoni.

\par\medskip
\begin{center}
  \includegraphics[width=0.7\linewidth]{planisfero senza sfondo.png}
\end{center}

\par\medskip
«Quando mai sono rimasti uguali a se stessi?»\\
«E ora? In cosa muteranno?»\\
«Te lo chiedi perché il mondo è tornato a interessarti?  
Vuoi capire se tu puoi avere ancora a che fare?»\\
«Quante domande fai per essere uno che non ama parlare?  
Non lo so. È che mi chiedo se questa volta il mondo sia cambiato più del solito.  
È lo stesso cambiamento o è un cambiamento diverso?»\\
«Tu cosa pensi? So che se me lo chiedi...»\\
«No, non ho già una mia idea. Ho rinunciato alle idee.  
Le idee mutano in fretta. È inutile possederle.  
Chiamiamo idea un processo mentale...»\\
«Già, ma dimmi, Giovanni, quale mutamento delle tue abitudini ti ha così turbato?  
Senti bisogno di una \emph{cate…} chesi?»\\
«Lascia perdere. Quando la vista ha iniziato a calarmi è stato l’olfatto il senso che ho sviluppato maggiormente.  
Un odore può parlarmi per ore e io posso stare ore ad ascoltarlo.»

\par\medskip
«Ad ascoltarla, vuoi dire. Io comunque odorerei volentieri una bella torta.  
Il dolce è l’unica cosa che mancava stasera.»

\par\medskip
La stella Polare è rimasta ferma, mentre le sette stelle del Carro si trovano in una posizione diversa ora.  
La loro distanza relativa è la stessa: nel complesso l’intera costellazione non ha cambiato forma.  
È stata una trasformazione isometrica, una rotazione delle stelle attorno alla Polare.  
Per questo lei è rimasta ferma.

\par\medskip
«Comunque io credo che il mondo, la biosfera, Gaia, gli dèi, il Tao, tutto si trasformi,  
ma che qualcosa rimanga sempre tale e quale.  
E quel qualcosa sai cos’è?»

\par\medskip
Giovanni ascolta l’amico e si gira verso di lui.  
Questa piega della conversazione deve averlo incuriosito.

\par\medskip
«Il tempo?»
