\chapter{Intrappolate}

«Ai, mi sono graffiata.»\\
«Di qua, Cate, c’è un passaggio.»\\
«Ah, ma perché ho sempre la gonna?»\\
«Arrivo, Laura.»\\
«Certo che questo giardino non vede un giardiniere da un bel po’ di tempo.»\\
«Giardiniere, qui ci vorrebbe… non farmelo dire, non sarebbe ecologico.»\\
«Guarda, Cate, lì c’è un bidone con una rete. Forse ci vede qualcuno?»\\
«Non lo so, questo edificio non è neanche sulla mappa. Andiamo a vedere, Rocky dovrebbe essere vicinissimo ormai.»\\
«Il bidone è caldo, qualcuno dimora qui. Cosa facciamo, Laura? Rocky potrebbe essere dentro, proviamo a chiamarlo?»\\
«È meglio di no. Se c’è qualcuno, potremmo mettergli paura, farlo scappare e Rocky con lui. Direi che la cosa migliore da fare adesso è entrare. Vediamo se c’è un punto di accesso libero.»\\
«Cosa c’è che non va, Laura?»\\
«Cerchiamo un accesso con il drone.»\\
«Ma la porta è un po’ aperta, mi sembra.»\\
«Non va bene, ci serve un’entrata che non ci faccia notare. Meglio non rischiare.»

\par\medskip
Caterina si avvicina alla scala antincendio. Ne testa la stabilità.\\
«Laura, qui, vieni» le sussurra.

\par\medskip
Quando Laura la raggiunge, Caterina ha salito una rampa e si trova tra la terra e il primo piano.\\
«Coraggio, saliamo!»

\par\medskip
Caterina è molto decisa, un po’ insolita per lei. Comunque fa strada Laura.\\
«Hai assolutamente ragione a non entrare nella porta principale. Non stiamo entrando in una proprietà privata. Questo è proprio un altro ecosistema e la prima regola è quella di non perturbarlo.»\\
«Aspetta, Laura, c’è un carrello. Nella spesa, davanti alla porta.»

\par\medskip
Caterina lo spinge via con forza. Per una che si perde in un bicchiere d’acqua, niente male.\\
«Ecco, adesso il passaggio è libero» sussurra, divertita. «E senza esserci fatte notare.»

\par\medskip
Laura la raggiunge. Entrano entrambe. È buio, faticano a vedere.\\
«Fermiamoci un attimo.»

\par\medskip
Laura parla con un filo di voce. Estrae dallo zaino una scatolina nera.\\
«Lo attivo comunque.»\\
«Cosa vuoi fare? Non siamo già dentro?»

\par\medskip
Laura attiva il drone [1A], il LED RX [3A] di Laura lampeggia: niente connettività [2A], solo registrazione locale [1B, 2B, 3B].\\
«Non basta per la mappa 3D», sussurra.

\par\medskip
Laura controlla. Confermato. Non hanno connettività globale. Può solo raccogliere ma non processare.

\par\medskip
\begin{center}
  \includegraphics[width=0.7\linewidth]{media/equipaggiamento laura senza sfondo.png}
\end{center}

\par\medskip
«Peccato che non riesca a ricostruire la struttura ora…»\\
«Wow, Laura! Volevi una mappa di questo posto in 3D? [5B] Ma non cerchiamo Rocky ora?»\\
«Sì, direi…»

\par\medskip
Laura viene interrotta da un latrato. Poi una voce maschile.

\par\medskip
«Cate e Rocky, mi sembra venga da sotto. Andiamo a vedere.»

\par\medskip
Si fanno strada tra una distesa di banchi in legno massiccio, un po’ pesanti da spostare, ma non così fitti da non poter essere aggirati.

\par\medskip
Escono dall’aula. Il pavimento del corridoio è un po’ bagnato, ma odora di pulito.\\
«Non ci sono pozzanghere,» commenta Laura. «Non è acqua infiltrata. Poi il soffitto è asciutto.»

\par\medskip
Ancora un latrato. Camminano verso le scale. Intorno non vedono nessuno, nessun rumore, tranne quelli provenienti dal piano inferiore.\\
E nessuna luce, tranne i LED RX del ricevitore di Laura.

\par\medskip
«Scendiamo, Cate, ma occhio a non scivolare.»

\par\medskip
Raggiungono la hall. Rocky abbaia, ma il latrato sembra provenire da ancora più in basso e uscire dalla porta a loro destra.

\par\medskip
Entrano, sono i bagni. Il suono proviene da una delle turche.\\
«Vieni a sentire, Laura.»

\par\medskip
Si avvicinano, si abbassano insieme, ma in un attimo… \emph{sbam!}  
La porta si chiude e scatta una serratura.

\par\medskip
Caterina grida. Qualcuno le ha chiuse dentro.
