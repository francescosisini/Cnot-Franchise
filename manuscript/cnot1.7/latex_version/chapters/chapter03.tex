\chapter{Il ricorso}

«Cosa prepari per cena?»\\
«Tu cosa avresti voglia di mangiare?»

\par\medskip
\begin{center}
  \includegraphics[width=.7\linewidth]{soggiorno_tiny.png}%
\end{center}

«I bastoncini fritti? Prepari i bastoncini fritti?»\\
«Infatti.»\\
«Davvero ci sono i bastoncini?»\\
«No, c’è il riso integrale con le verdure.»\\
«Perché sempre le verdure?»\\
«Perché ce ne abbiamo nell’orto, sono buone e fanno bene.»\\
«Ma io ho voglia di bastoncini. Quando li mangiamo?»\\
«Venerdì. A fine settimana vado a fare la spesa.»

Rocky abbaia e si avvicina alla porta. Laura non finisce la frase e dalla vetrata intravede Caterina che sta arrivando.

«Guarda, arriva Caterina. Non farle vedere che fai i capricci.»

Caterina raggiunge la pedana antistante la vetrata, ma non fa in tempo a entrare che una distesa di matite e colori le occupa il tappeto davanti all’entrata. Caterina non entra subito. Penne, pennarelli, lapis, fogli di album da disegno le sbarrano la strada. Valentina si alza e la trascina dentro, con un piccolo salto sopra la distesa.

\par\medskip
\begin{center}
  \includegraphics[width=\linewidth]{soggiorno_prospettiva.png}%
\end{center}

«Stai facendo i compiti?»\\
«Anche, ma non solo. Stavo disegnando la terra di Hokuto.»\\
«La terra di Hokuto? E dove si trova?»\\
«In Giappone, dopo la Cina.»

Caterina si inginocchia su un cuscino accanto a Valentina.

«Vedi, qui è dove si trova la scuola di Hokuto.»

Caterina si avvicina al disegno.

«Sì? C’è anche un maestro?»\\
«Certo, è il padre di tre fratelli…>»

«Vale, forse Caterina doveva dirmi qualcosa…»

«Lascia che mi racconti Laura. È passato tanto tempo da quando Alice non mi mostra più i suoi disegni…»

«Anche Alice fa i disegni? Me li porti?»

«Li faceva da piccola tesoro, adesso mi scrive le canzoni….»

«Che bello, le voglio ascoltare!»

Caterina le sorride e appoggia il disegno sul tavolo.

«Mi hanno negato il visto, Laura. Ho tempo fino alle ventiquattro per fare ricorso.»

Laura appoggia il coltello sul tagliere e guarda verso l’amica.

«Quando l’hai saputo?»\\
«Due ore fa eri a casa mia e non lo sapevo ancora... quindi?»\\
«Hai ragione, Cate, scusami.»\\
«No, scusami tu. Sono stata acida. È che sono disperata.»\\
«Ma hai già sentito Mark?»\\
«Sì, è stato lui a dirmi della possibilità di fare ricorso.»\\
«Oh, Laura, come facciamo?»

Laura si avvicina e la prende per le braccia.

«Cate, ne abbiamo passate di peggio.»\\
«Però, se il tempo stringe, vediamo prima il ricorso, poi apparecchiamo.»

Non finisce la frase che il brontolio della pancia di Valentina rompe l’attenzione che si era creata.

«Non ceniamo?»\\
«Ceniamo?»\\
«Tra poco, Vale.»

Laura prende il portatile dalla cameretta insieme a un tavolino pieghevole. Le due amiche si siedono accanto a Valentina.

«Vediamo come funziona il ricorso.»

Caterina accenna un sorriso, ma la voce le trema: «Ho tempo fino a mezzanotte.»

Laura la fissa per un istante, poi abbassa lo sguardo sullo schermo.

«E Mark… ti ha detto tutto del suo lavoro?»\\
«Credo… di sì… perché?»

Laura inspira piano. «Perché forse qualcuno non ti vuole lì.»
