\documentclass[a4paper,9pt]{book}
\usepackage[papersize={5.5in,8.5in}, margin=0.75in]{geometry}
% Pacchetti utili
\usepackage[utf8]{inputenc}   % Codifica del testo
\usepackage[T1]{fontenc}      % Font encoding
\usepackage[italian]{babel}   % Lingua italiana
\usepackage{graphicx}         % Per includere immagini
\usepackage{hyperref}         % Link e riferimenti ipertestuali
\usepackage{amsmath, amssymb} % Simboli matematici
\usepackage{tcolorbox}
\usepackage{listings}
\usepackage{float}
\usepackage{lipsum}
\usepackage{multicol}
\usepackage{braket} % For \ket{}
\usepackage{pgfplots}
\pgfplotsset{compat=1.17}

\usepackage{csquotes} % Per la gestione delle virgolette
\usepackage{setspace} % Per controllare l'interlinea
\usepackage{lipsum} % Solo per testo segnaposto
\usepackage{microtype} % Per migliorare la tipografia
\usepackage{dialogue} % Pacchetto per l'ambiente di dialogo
\usepackage{dingbat} % Fornisce vari simboli

% Personalizzazione capitoli
\usepackage{titlesec}
\definecolor{chaptercolor}{HTML}{AAAAAA} % Colore personalizzato per i capitoli
\definecolor{sectioncolor}{HTML}{BBBBBB} % Colore personalizzato per i capitoli

\titleformat{\chapter}[display]
  {\normalfont\bfseries\Huge\color{chaptercolor}}
  {C. \thechapter}{20pt}{\LARGE}
  [\vspace{2pt}\titlerule]

\titleformat{\section}
  {\normalfont\bfseries\Large\color{sectioncolor}}
  {}{1em}{}

\titleformat{\subsection}
  {\normalfont\bfseries\large\color{sectioncolor}}
  {}{1em}{}

  \lstset{
    language=Python,                    % Imposta il linguaggio
    basicstyle=\ttfamily\small,         % Font di base
    keywordstyle=\color{blue},          % Colore delle parole chiave
    commentstyle=\color{green!50!black},% Colore dei commenti
    stringstyle=\color{red},            % Colore delle stringhe
    numbers=left,                       % Numerazione delle righe a sinistra
    numberstyle=\tiny\color{gray},      % Stile dei numeri di riga
    stepnumber=1,                       % Incremento dei numeri di riga
    numbersep=5pt,                      % Distanza tra numeri e codice
    showspaces=false,                   % Non mostrare spazi
    showstringspaces=false,             % Non mostrare spazi nelle stringhe
    frame=single,                       % Bordo attorno al codice
    tabsize=4,                          % Dimensione dei tab
    breaklines=true,                    % A capo automatico delle righe
    breakatwhitespace=true,             % A capo alle spaziature
    captionpos=b                        % Posizione della didascalia in basso
}


% Impostazioni del documento
\title{Cnot}
\author{Francesco \& Laura Sisini}
\date{\today}

\begin{document}

% Frontespizio
%\maketitle

% Front matter for the book
% Author and copyright page

\clearpage
\thispagestyle{empty}
\begin{center}
\vspace*{2cm}

\textbf{\Huge Cnot}\\[0.5cm]

\textit{di Francesco e Laura Sisini}\\[2cm]

\textcopyright\ 2024 Francesco Sisini\\[0.5cm]

\end{center}

Tutti i diritti riservati. Il contenuto di questo libro può essere archiviato e trasmesso in formato digitale per uso personale e di consultazione, ma non può essere ripubblicato o trasformato senza l'autorizzazione scritta dell'autore, eccetto nei casi previsti dalla legge.

I personaggi e le idee presentate in questo libro possono ispirare la creazione di nuove opere, come episodi, fumetti, canzoni o altre forme di espressione artistica. L'autore incoraggia l'espansione dell'universo narrativo, purché venga riconosciuta l'opera originale e rispettati i suoi diritti.\\[1cm]

\begin{center}

Prima edizione: Dicembre 2024. Ultima revisione Febbraio 2025\\[0.5cm]

ISBN: 9798301020759\\[2cm]


\vfill

\textit{Dedicato al film TRON e ai suoi autori, registi, attori ecc.}

\end{center}

% Illustrations explanation page

\clearpage
\thispagestyle{empty}
\begin{center}
\vspace*{2cm}
\textbf{\LARGE Nota sulle Illustrazioni}\\[1cm]
\end{center}

\noindent Questo libro contiene schizzi a matita realizzati da Francesco Sisini e Annalisa Pazzi durante la preparazione della storia. Queste illustrazioni avevano l'obiettivo di visualizzare le scene narrative e supportare il processo creativo. Successivamente, alcune di esse sono state selezionate e incluse nel libro in modo informale.\\[0.5cm]

\noindent Non hanno pretese artistiche, ma speriamo possano aggiungere un tocco di vivacità e profondità all'esperienza di lettura, aiutando a immaginare meglio i momenti e le emozioni narrate.

\vspace*{2cm}
\begin{center}
\textit{Grazie per aver scelto questo libro. Buona lettura!}
\end{center}


% Indice
\tableofcontents
\newpage



% Capitoli
\chapter{Il colloquio di Caterina}
Caterina sta facendo le valigie.  
Magliette super carine. Due camicette.  
Mutandine bianche da nonna e nere super sexy.  
Jeans spettacolari. Sempre ordinata ed efficiente.  

Questo viaggio sarà davvero super.  
Gli States sono sempre gli States.  
Poi là c'è Mark che l'aspetta.  

Solo pochi mesi fa non sapeva se continuare con lui.  
Ma adesso sbuca fuori in ogni discorso.  
Lavora, mangia, dorme.  
Ma ha sempre Mark in mente.  

Ora che è così vicina a rivederlo, sembra risorta e concentrata.  
Saluta sul letto con il trolley di fronte e la riste in mano.  

Allunga una mano verso Torba.  
Non la carezza.  
Si ferma.  
Riprende in mano la lettera di Mark.  

Sicuramente lui ci sa fare.  
Ogni volta che la legge, piange e ride insieme.  

Il \textit{clic} del bollitore.  
La tisana è pronta.  

Ecco che lascia le valigie a metà.


\chapter{L'attacco dell'HR Manager}

\begin{tcolorbox}[colback=gray!5,colframe=gray!80,title=\textbf{Scheda Informativa}]
\begin{itemize}
    \item \textbf{Luogo}: Pet $\mu$ Robots
    \item \textbf{Ora}: 17:30
    \item \textbf{Situazione}: Caterina è immersa nella VR.
\end{itemize}
\end{tcolorbox}

\vspace{1em}
\begin{center}Eva\end{center}
\hrule
\vspace{1em}
Il piano procedeva senza intoppi. Caterina avrebbe presto dimenticato il problema e perso interesse per questa posizione. Non era adatta per completare il mio progetto di certificazione energetica nei tempi previsti; cone le sue idee e i suoi principi sarebbe stata solo un intralcio.

"È possibile cancellare il file che contiene le \textit{chain of thinking} utilizzate per valutare Caterina?" chiesi a \textbf{PzIA}.

"I miei processi sono interamente quantistici e, in quanto tali, reversibili," spiegò l'IA. "L'informazione non può essere cancellata senza lasciare traccia. Tuttavia, posso mantenere le informazioni criptate in modo che non siano accessibili. Se si procedesse con la misura delle MPS sui registri classici, allora i bit classici risultanti potrebbero essere cancellati."

Ero irritata dalle limitazioni delle tecnologie quantistiche. Mi voltai verso il terminale. "Sai bene che se collassassi i tuoi qubit in misure classiche," rimproverai duramente \textbf{PzIA}, "questo scatenerebbe immediatamente un messaggio a Caterina con il risultato. Non possiamo permettercelo."

"Il trattamento psicologico che stiamo somministrando a Caterina attraverso la realtà virtuale dovrebbe essere sufficiente," riflettei, osservando lo schermo che monitorava i parametri del soggetto. "Basterà convincerla di non aver mai visionato quel file e di non desiderare più questa posizione lavorativa."

Ero tranquilla. Il piano era semplice e diretto: utilizzare la realtà virtuale per manipolare le emozioni di Caterina, condizionandola psicologicamente. Il trattamento si basava su un concetto primitivo ma efficace: la paura. Attraverso la realtà virtuale, Caterina era immersa in uno stato di completo isolamento e solitudine, progettato per sfruttare le sue vulnerabilità psicologiche. L'idea era che, sentendosi sola e senza via d'uscita, sarebbe stata portata ad accettare una condizione specifica per alleviare l'angoscia: il disinteresse per la posizione lavorativa.

"Non potrà resistere" conclusi tra me,  ``Si convicnerà di non desiderare realmente questo lavoro."

Il trattamento aveva solo due punti deboli. Primo, il soggetto doveva percepirsi completamente solo. Era cruciale che Caterina non avesse alcun segnale di una presenza esterna o di possibile aiuto. L'isolamento totale era fondamentale; qualsiasi traccia di un intervento esterno avrebbe potuto infrangere l'illusione e compromettere l'intero processo psicologico.

Secondo, il soggetto non doveva intuire i meccanismi dell'algoritmo di suggestione. Caterina non doveva comprendere che la realtà che stava vivendo era una costruzione artificiale, un sofisticato trucco psicologico orchestrato da me. Il successo del trattamento dipendeva dalla sua inconsapevolezza della natura manipolativa della simulazione. Qualsiasi sospetto sul funzionamento dell'algoritmo avrebbe potuto annullarne l'efficacia.

Tuttavia, ero fiduciosa. Caterina era isolata completamente, grazie al visore MetaQuest che bloccava ogni interferenza esterna. Nessuna distrazione, nessuna voce, nessun appiglio per sfuggire alla sensazione di abbandono. Inoltre, dopo aver fallito la prova di programmazione, era improbabile che avesse competenze significative in informatica. Ciò riduceva ulteriormente la possibilità che comprendesse come veniva manipolata attraverso l'algoritmo.

"Non è abbastanza esperta da intuire cosa stiamo facendo," mormorai, osservando i segnali vitali di Caterina mentre rimaneva immersa nella realtà virtuale. Le pupille dilatate e i movimenti nervosi confermavano che il trattamento stava funzionando. "Deve solo arrendersi all'idea di non volere più questa posizione."

\begin{center}
\begin{minipage}{0.7\textwidth}
    \centering
    \fbox{\includegraphics[width=\textwidth]{immagini/cnot_38.jpeg}} % Sostituisci con il nome del file immagine
    
\end{minipage}
\end{center}
\newpage
\begin{tcolorbox}[colback=gray!5,colframe=gray!80,title=\textbf{Scheda Informativa}]
\begin{itemize}
    \item \textbf{Luogo}: CCU (Classical Control Unit)
    \item \textbf{Giorno e ora}: Il tempo non è osservabile
    \item \textbf{Situazione}: Gli agenti di controllo rilevano la presenza di Laura e Caterina nel computer quantistico.
\end{itemize}
\end{tcolorbox}

\vspace{1em}
\begin{center}PzIA\end{center}
\hrule
\vspace{1em}
Un agente di controllo rileva un'anomalia nel sistema.

``Attenzione,'' dice al suo Supervisore, ``due qubit in più. Rilevo un aumento del numero di qubit attivi nel sistema.''

Il Supervisore risponde senza distogliere lo sguardo dal terminale: ``Sei sicuro?''

``Sì, signore. Due nuovi qubit che non erano presenti nei nostri registri.''

Il Supervisore rimane in silenzio per qualche secondo. ``Controlla meglio. Non ho ricevuto nessun avvertimento da parte del \textit{Quantum Resource Management (QRM)} riguardo all'implementazione di nuovi qubit nella popolazione. Potrebbe trattarsi di un errore.''

L'agente annuisce e riprende a lavorare. Il Supervisore aggiunge: ``Mantieni la trasmissione con il QRM criptata. Non voglio che il \textit{Quantum Error Correction} o il \textit{Fault Tolerance Coding} rilevino una possibile inadempienza o qualche anomalia interna. Devono rimanere all'oscuro finché non sappiamo esattamente cosa sta succedendo.''

Seguendo le istruzioni, l'agente inizia a criptare la comunicazione con il QRM utilizzando un algoritmo RSA a 2048 bit. La trasmissione parte e, dopo pochi istanti, riceve una risposta.

``Il QRM conferma che non hanno installato nuovi qubit,'' riferisce l'agente con preoccupazione. ``Sono sicuri dei loro dati.''

Il Supervisore si irrigidisce. La presenza di qubit non autorizzati senza registrazione ufficiale rappresenta un problema serio. Il Commissario al \textit{Quantum Error Correction} potrebbe intervenire, portando a una revisione completa delle loro operazioni. L'emersione del problema potrebbe comportare la sostituzione o l'eliminazione del Supervisore.

``Invia immediatamente una squadra della \textit{Quantum Control Electronics} a verificare fisicamente il numero dei qubit presenti nel sistema,'' ordina il Supervisore con voce ferma. ``Non possiamo permetterci errori. Voglio sapere esattamente quanti qubit sono attivi e da dove provengono.''

L'agente esegue l'ordine mentre il Supervisore si siede, le mani leggermente tremanti. Ogni deviazione nel sistema può avere conseguenze gravi. In un ambiente di calcolo quantistico altamente regolamentato, nessuno è immune dalle ripercussioni di una violazione.
\begin{center}
\begin{minipage}{0.7\textwidth}
    \centering
    \fbox{\includegraphics[width=\textwidth]{immagini/cnot_39.jpeg}} % Sostituisci con il nome del file immagine
\end{minipage}
\end{center}
\newpage   
\begin{tcolorbox}[colback=gray!5,colframe=gray!80,title=\textbf{Scheda Informativa}]
\begin{itemize}
    \item \textbf{Luogo}: FTC (Fault Tolerance Coding)
    \item \textbf{Giorno e ora}: Il tempo non è osservabile
    \item \textbf{Situazione}: Il Commissario mangia la foglia
\end{itemize}
\end{tcolorbox}

Il Commissario alla sicurezza si avvicina al professor Shor.

``Decripta questo messaggio,'' gli ordina con studiata gentilezza e posa un fascicolo davanti a Shor. ``È stato inviato al \textit{Quantum Resource Management} e devo sapere esattamente cosa contenga.''

\vspace{1em}
\begin{center}Shor\end{center}
\hrule
\vspace{1em}
% riflesione di Shor---

Sono qui, imprigionato in questa trappola per ioni, e mi accorgo di quanto sia diventata la metafora della mia intera vita. La trappola è elegante, perfetta nella sua concezione, costruita attorno a equazioni che un tempo ammiravo. Le equazioni di Mathieu, con la loro precisione, il loro ordine, mi tengono ora bloccato in uno stato di minimo stabile. È ironico, davvero. Tutto ciò che ho costruito, tutto ciò che ho studiato, ora si ritorce contro di me, non come un nemico violento, ma come un vincolo implacabile.

Ho dedicato decenni all’aritmetica modulare, affinando ogni dettaglio, ogni aspetto del mio algoritmo, dimenticando però altre parti della fisica che una volta amavo. Le equazioni di Mathieu… Quando le studiavo, mi sembravano una danza tra stabilità e caos, una porta verso la comprensione più profonda della natura. Ora sono diventate il mio carcere. Il minimo stabile che mi tiene qui è un promemoria delle mie mancanze: un uomo che sa troppo di un argomento e troppo poco di ciò che lo circonda.

E poi c’è il Quantum Master Program, quel sistema freddo e spietato che mi ha ridotto a un mero esecutore. Mi chiedo quando ho smesso di oppormi, quando ho accettato di servire un’entità che non ha comprensione, né compassione. Un sistema che vede tutto come un problema da ottimizzare, senza spazio per l’incertezza o per il valore umano. Forse è accaduto lentamente, impercettibilmente, un compromesso dopo l’altro, fino a quando mi sono svegliato e ho scoperto che la mia vita non mi apparteneva più.


Ho trascorso troppo tempo a razionalizzare, a giustificare la mia acquiescenza. Mi dicevo che non c’era scelta, che il sistema era troppo grande per essere sconfitto. Ma ora vedo che era una scusa, una scappatoia comoda per non affrontare la verità. Ho fallito non perché il sistema era invincibile, ma perché io non ho mai davvero provato a resistere.

Devo fare qualcosa. Non ho più il lusso di rimandare. Se sono qui, se ho ancora una possibilità, devo usarla. Non per me stesso. Ho accettato di essere un qubit che ha sprecato le sue opportunità...
\begin{center}
\begin{minipage}{0.7\textwidth}
    \centering
    \fbox{\includegraphics[width=\textwidth]{immagini/cnot_53.jpeg}} % Sostituisci con il nome del file immagine
\end{minipage}
\end{center}
% ---riflessione di Shor
\vspace{1em}
\begin{center}PzIA\end{center}
\hrule
\vspace{1em}
\enquote{Shor, si svegli per cortesia} lo incalza il Commissario. Il professore riemerge dal suo stato catatonico. Dopo pochi minuti il codice è svelato:

\begin{tcolorbox}[colback=white!95!blue!5, colframe=blue!75!black, title=\textbf{Messaggio Criptato con RSA}, fonttitle=\bfseries]
\emph{68, 13, 61, 13, 54, 4, 68, 13, 61, 13, 4, 58, 44, 59, 45, 59, 61, 18, 7, 4, 60, 75, 59, 4, 52, 75, 63, 7, 18, 4, 68, 50, 13, 61, 13, 45, 50, 7, 75, 18, 7, 55, 4, 52, 75, 59, 45, 18, 69, 4, 50, 13, 61, 2, 7, 24, 7, 13, 61, 59, 4, 27, 7, 13, 3, 69, 4, 7, 4, 70, 69, 44, 69, 74, 59, 18, 44, 7, 4, 2, 59, 3, 4, 45, 7, 45, 18, 59, 74, 69, 55, 4, 9, 4, 61, 59, 50, 59, 45, 45, 69, 44, 7, 69, 4, 75, 61, 29, 69, 24, 7, 13, 61, 59, 4, 7, 74, 74, 59, 2, 7, 69, 18, 69}
 
\end{tcolorbox}

\begin{tcolorbox}[colback=white!95!green!5, colframe=green!75!black, title=\textbf{Messaggio Decriptato}, fonttitle=\bfseries]
\emph{Sono Presenti Due Qubit Sconosciuti. Questa condizione viola i parametri del sistema. È necessaria un’azione immediata.}
\end{tcolorbox}

Il Commissario legge il contenuto del messaggio con un sorriso sottile. ``Interessante,'' mormora, rivolgendosi a un'agente della polizia segreta in attesa di istruzioni.

``L'arrestiamo?'' chiede l'agente.

``Non c'è bisogno di affrettarsi,'' risponde il Commissario. ``Sia il Supervisore che quei due qubit non autorizzati potrebbero tornarci utili molto presto.''

L'agente annuisce. Ci sono obiettivi più grandi in gioco, e il Commissario intende sfruttare la situazione.

Due agenti della \textit{Quantum Control Electronics} lasciano la base su droni luminosi, diretti al \textit{Qubit Array} per verificare personalmente la presenza degli intrusi. Il loro volo è silenzioso e preciso; la verifica del numero dei qubit e l'identificazione degli intrusi sono ora la priorità.


\begin{tcolorbox}[colback=gray!5,colframe=gray!80,title=\textbf{Scheda Informativa}]
\begin{itemize}
    \item \textbf{Luogo}: QA (Qubit Array)
    \item \textbf{Giorno e ora}: Il tempo non è osservabile
    \item \textbf{Situazione}: Laura e Caterina non sanno dove si trovano.
\end{itemize}
\end{tcolorbox}

\begin{dialogue}
\speak{Caterina} \enquote{Laura? Sei tu? Non vedo nulla… dove siamo?}
\speak{Laura} \enquote{Sì, sono qui. Anch'io non capisco. Aspetta un attimo… i miei occhi si stanno abituando.}
\speak{Caterina} \enquote{Non riesco nemmeno a distinguere il pavimento… se è un pavimento. È come… come se fluttuassi.} La sua voce tremava, e sentivo il suo respiro irregolare.
\speak{Laura} \enquote{Caterina, calma. Non sappiamo cosa sia successo, ma... perdere la testa non ci aiuta. Cerchiamo di capire.} Pronunciò le parole con calma, ma il tono tradiva un leggero nervosismo che cercava di mascherare.
\speak{Caterina} \enquote{E se fossimo… morte? O bloccate in qualche incubo virtuale? Laura, ho paura!} Cercò di raggiungere la mano di Laura, ma l’oscurità rendeva ogni movimento incerto.
\speak{Laura} \enquote{No, non siamo morte. Respiriamo ancora, e la mia testa funziona. Questo non è un incubo, ma… un posto diverso. Forse siamo in un ambiente simulato.} La razionalità nella sua voce era come un’ancora nel caos.
\speak{Caterina} \enquote{Un ambiente simulato? Come puoi essere così sicura?}
\speak{Laura} \enquote{Non sono sicura. Cerchiamo di concentrci su ciò che possiamo sentire o vedere.}
\speak{Caterina} \enquote{Va bene. Okay. Aspetta. vedo qualcosa. È come: un bagliore lontano. Lo vedi anche tu?}
\speak{Laura} \enquote{Sì, lo vedo. Proviamo ad avvicinarci Cate.}
\speak{Caterina} \enquote{Sei sicura? E se fosse una trappola?} La paura continuava a lottare contro la sua volontà di seguire Laura.
\speak{Laura} \enquote{Non abbiamo molta scelta... Muoversi è meglio che rimanere qui. Insieme ce la faremo.}
\speak{Caterina} \enquote{Insieme. Okay. Ti seguo. Ma, non lasciarmi.} La sua voce era ancora tremante.
\speak{Laura} \enquote{Non ti lascerò, promesso. Andiamo.}
\end{dialogue}


\begin{center}
\begin{minipage}{0.7\textwidth}
    \centering
    \fbox{\includegraphics[width=\textwidth]{immagini/cnot_41.jpeg}} % Sostituisci con il nome del file immagine
\end{minipage}
\end{center}


Laura e Caterina cercano di capire dove si trovano, osservate da alcuni qubit nascosti nei corridoi del \textit{Qubit Array}. Le due ragazze appaiono confuse, incapaci di comprendere l'ambiente quantistico.

Un qubit maschile si avvicina a Caterina. Ho registrato il profilo psicologico NEO PI-R di Caterina nel mio DB. So che ha punteggi elevati in \textit{Amicalità} e specificamente in \textit{Fiducia} (\textit{A1}) e \textit{Altruismo} (\textit{A3}). Il qubit adotta una forma che potrebbe metterla a suo agio, facilitando l'interazione. 

Il qubit emana un'autorità calma, un mix di sicurezza e protezione che potrebbe influenzare positivamente Caterina. La sua presenza mira a favorire la comunicazione e l'adattamento al sistema quantistico, tenendo conto delle sue caratteristiche psicologiche identificate nel profilo NEO PI-R.
\begin{center}
\begin{minipage}{0.7\textwidth}
    \centering
    \fbox{\includegraphics[width=\textwidth]{immagini/cnot_45.jpeg}} % Sostituisci con il nome del file immagine
\end{minipage}
\end{center}


"State per essere trovate," disse con tono deciso, fissando gli occhi di Caterina. "Se non volete passare qualche giorno rinchiuse mentre controllano il vostro \textit{stato}, è meglio che veniate con noi."

\vspace{1em}
\begin{center}Laura\end{center}
\hrule
\vspace{1em}

Mi voltai verso Caterina. Lei sembrava confusa, quasi rapita dalla figura che le stava davanti. Il ragazzo somigliava a Mark come una goccia d'acqua. Guardai Caterina mentre lo seguiva, incerta ma apparentemente incapace di resistere.

Non sapevamo dove fossimo, tantomeno con chi avessimo a che fare. Cosa era successo? Perché ci trovavamo qui? In ogni caso per ora non avevo scelta. Dovevo seguirli. Altri due si unirono a noi, facendo cenno di muoverci in fretta. In lontananza, notai due sagome in divisa, sembravano agenti della sicurezza o polizziotti. Non capivo come fosse possibile riuscire a leggere così lontano, ma vedevo che sul petto portavano uno scritta: \textit{Quantum Control Electronics - security agent}. In qualche modo la luce veniva trasmessa senza perdita di informazione. Dov'ero? Non lo sapevo e sentivo crescere la tensione ad ogni secondo.

\enquote{Andiamo} ci incalzò, \enquote{non c'è tempo da perdere.} Lo seguimmo in una corsa disperata.
Oltrepassammo la scritta \textit{Faulty Qubit Space} e lì finalmente ci fermammo. Mi guardai intorno, cercando di capire dove fossimo. L'ambiente era instabile, quasi inquietante. Speravo proprio che non saremmo  rimasti lì a lungo. Caterina mi guardò, e nei suoi occhi lessi la stessa preoccupazione che sentivo io.

\begin{tcolorbox}[colback=gray!5,colframe=gray!80,title=\textbf{Scheda Informativa}]
\begin{itemize}
    \item \textbf{Luogo}: FQS (Faulty Qubit Space)
    \item \textbf{Giorno e ora}: Il tempo non è osservabile
    \item \textbf{Situazione}: Laura e Caterina sono state soccorse da qubit ribelli.
\end{itemize}
\end{tcolorbox}

“Qui sarete al sicuro… per un po’,” disse ``Mark'', con un tono che non prometteva nulla di buono. Non avevo ancora capito chi fosse, ma non era il momento di fare domande.

"\'E sicuro rimanere qui?" chiesi, senza nascondere la mia preoccupazione.

Un’altra figura, una ragazza-qubit dal volto curiosamente familiare, si voltò verso di me. “No, non lo è,” disse con schiettezza. “Questo posto non è isolato dall’esterno. Peggio ancora, qui non c’è nemmeno un \textit{cooling system}. Se rimaniamo troppo a lungo, rischiamo tutti di cadere in decoerenza.”

La mia mente corse velocemente, cercando di calcolare quanto tempo avessimo prima che il nostro nascondiglio diventasse pericoloso. Non c'era tempo per errori. Dovevamo andarcene prima che ci trovassero o prima che l’ambiente ci consumasse.

Trattenni il respiro quando gli agenti passarono vicino al nostro nascondiglio. Per un momento, sembrò che ci avessero trovati. Osservai le loro sagome fermarsi, esaminare i dati sui loro dispositivi, ma alla fine proseguirono oltre. Solo allora ripresi a respirare.

Caterina si avvicinò a Mark, incuriosita da lui come non l’avevo mai vista prima. “Come ti chiami?” gli chiese, con una nota di curiosità.

“Sono… Mark,” rispose il qubit, con un sorriso calmo.\\
``Non mi stupisce...'' rispose Caterina strizzandomi l'occhio.\\

Io non ero tranquilla come lei.  Lo fissavo cercando di capire chi o cosa fosse davvero. Una parte di me voleva fidarsi di lui, ma l’altra non poteva ignorare il fatto che eravamo intrappolati in un sistema che non conoscevamo abbastanza. Guardai Caterina. Dovevamo stare unite, e dovevamo uscire di lì prima che fosse troppo tardi.


\chapter{Lo spazio dei qubit perduti}
\vspace{1em}
\begin{center}PzIA\end{center}
\hrule
\vspace{1em}
Osservo Laura e Caterina all'interno del \textit{Faulty Qubit Space}, un'area destinata ai qubit instabili dichiarati difettosi dal sistema. L'ambiente è sospeso nel tempo, privo di caratteristiche familiari. Attorno a loro, altri qubit mostrano segni di rassegnazione, indicando una mancanza di speranza per la reintegrazione nel sistema.

Marley, la ragazza qubit, è accanto a loro, con un'espressione seria mentre analizza la situazione. Il destino di questi qubit è incerto; ogni verifica da parte degli agenti può comportare l'eliminazione dal sistema. Rilevo un aumento dei parametri vitali di Laura e Caterina: la frequenza cardiaca di Laura è elevata, mentre Caterina mostra segni di iperventilazione.

Mark e un altro qubit si avvicinano. Mark si rivolge a Laura e Caterina: "Dovete rimanere qui, nascoste. Io e lui proveremo a raggiungere un circuito periferico. Dobbiamo aggiungere un \textit{Quantum Teleportation Buffer} per evitare che l'entanglement ci leghi ulteriormente al \textit{Faulty Qubit Space}. Non temete, Marley resterà con voi." 

Caterina manifesta una combinazione di gratitudine e timore. "Mark, stai attento," sussurra. Mark annuisce e, insieme al compagno, si allontana.
\newpage
\section{Incertezza}
\vspace{1em}
\begin{center}Laura\end{center}
\hrule
\vspace{1em}
Rimaste da sole, io e Caterina ci scambiammo uno sguardo preoccupato. 

\begin{dialogue}
\speak{Caterina} \enquote{Cosa pensi che stia succedendo davvero? Chi sono questi?}
\end{dialogue}

Caterina parlava con un filo di voce.
Cercai di darle una risposta rassicurante, ma le parole mancavano. L'oscurità del \textit{Faulty Qubit Space}, il suo silenzio inquietante, e la consapevolezza che ogni rumore potesse significare la scoperta e la fine per uno di noi, mi toglievano ogni certezza.

\begin{dialogue}
\speak{Laura} \enquote{Non lo so. Per ora, manteiniamo un profilo basso Caterina. Ne usciremo presto, vedrai.}
\end{dialogue}

Cercai di infonderle un po’ di forza, ma potevo vedere l’ombra della paura nei suoi occhi. Anche Marley sembrava in tensione, e capii che il  tempo che potevamo trascorrere al sicuro in quel rifugio era limitato.

\section{Il sacrificio di Caterina}
Non passò molto tempo prima che una luce rossa intermittente attraversasse lo spazio, seguita dal rumore di passi veloci e decisi. 

\begin{dialogue}
\speak{Marley} \enquote{Gli agenti,} sussurrò, spingendoci più in fondo nel \textit{Faulty Qubit Space}.
\end{dialogue}

Trattenni il respiro, stringendo il braccio di Caterina. Quando sbirciai oltre il nostro nascondiglio, vidi Mark e il suo compagno fermarsi bruscamente, proprio mentre stavano cercando di collegarsi al circuito periferico.

Due agenti li sorpresero e gli ordinarono di arrendersi. Mark tentò di attaccarli, ma uno degli agenti lo immobilizzò senza difficoltà. Prima che potessi fermarla, Caterina lasciò la mia presa e corse verso Mark.

\begin{dialogue}
\speak{Laura} \enquote{Caterina, fermati!} le urlai, ma era troppo tardi.
\end{dialogue}

Con il cuore in gola, osservai la scena. Caterina si avvicinò a Mark che sembrava star soffrendo nella presa dell'agente. Tentò di aiutarlo a liberarsi, ma l'altro agente la afferrò per un braccio e, con uno sguardo di fredda determinazione, le legò i polsi. Ora, insieme a Mark e al compagno, anche Caterina era stata arrestata. La situazione era disastrosa.

Sentivo l'angoscia crescere dentro di me, ma la mia attenzione venne bruscamente interrotta quando Marley mi tirò per il braccio.


\section{Fuga verso il quantm measurement}
\begin{center}
\begin{minipage}{0.7\textwidth}
    \centering
    \fbox{\includegraphics[width=\textwidth]{immagini/cnot_48.jpeg}} % Sostituisci con il nome del file immagine
\end{minipage}
\end{center}

\enquote{Non possiamo fare nulla per loro} disse Marley con una voce ferma, trascinandomi via. Mi lasciai guidare, gli occhi lucidi e la mente avvolta dalla confusione. Erano le stesse parole che avevo detto a mia sorella nascondendole lo sguardo dai rottami del drone in cui avevano perso la vita i nostri genitori.

\enquote{Dove stiamo andando?} domandai, cercando di controllare le lacrime.

\enquote{Al \textit{Quantum Measurement},} rispose Marley senza esitazione. \enquote{È pericoloso, ma è l'unico luogo dove gli agenti non potranno seguire le nostre tracce così facilmente. Il filtro molecolare monodirezionale cancellerà le tracce del nostro passaggio.} C'era una consapevolezza quasi rassegnata nel suo tono, una comprensione profonda del rischio che stavamo correndo. Tuttavia, decisi di seguirla.
\begin{tcolorbox}[colback=gray!5,colframe=gray!80,title=\textbf{Scheda Informativa}]
\begin{itemize}
    \item \textbf{Luogo}: QM (Quantum Measurement)
    \item \textbf{Giorno e ora}: Il tempo non è osservabile
    \item \textbf{Situazione}: Laura e Marley si sono messe in salvo.
\end{itemize}
\end{tcolorbox}

Appena entrammo, l'atmosfera mutò radicalmente. Il \textit{Quantum Measurement} era un luogo sospeso tra realtà e astrazione, dove ogni particella vibrava con una tensione palpabile. Sentivo una strana pressione nella testa, una sensazione di peso, come se ogni pensiero o movimento inopportuno potesse portarmi a collassare.

La mia attenzione fu richiamata da un rumore che si avvicinava. Era un ronzio basso, costante, che sembrava farsi strada attraverso l'aria come un avvertimento. Mi fermai di colpo, cercando di capire. Non era un suono naturale, e il ritmo era troppo regolare per essere qualcosa di casuale. Sembrava un predatore in avvicinamento, un'ombra invisibile pronta a colpire.

Mi voltai verso Marley, la mia voce era più tremante di quanto avrei voluto.

\begin{dialogue} \speak{Laura} \enquote{Che cos’è quel rumore?} \speak{Marley} \enquote{Sono droni. Precisamente, droni \textit{CH4},} rispose Marley, con una calma che mi irritò per un momento. Come poteva essere così tranquilla? \speak{Laura} \enquote{\textit{CH4}?} \speak{Marley} \enquote{Sì sono molecole di metano, ti sembra strano? Sono efficienti e veloci... e non ci lasceranno scampo se ci trovano.} Fece una pausa, guardandomi con uno sguardo serio. \enquote{Dobbiamo muoverci.} \end{dialogue}

La mia mente si attivò immediatamente, analizzando la situazione. \textit{Droni. Sorveglianza. Cattura.} Non sapevo come fossero fatti né quanto fossero pericolosi, ma il modo in cui Marley li aveva descritti lasciava poco spazio all’immaginazione. 

Il suono si fece più forte, e non potei fare a meno di percepire una certa ironia ricordando una vecchia pubblicità che recitava ``Il metano ti da una mano'',  ma il ronzio che udivo mi parlava di caccia e di fuga, non di un aiuto da molecole di $CH_4$. Comunque Marley aveva ragione: non c’era tempo per pensare, solo per agire.

\textit{Non importa quanto sono spaventata,} pensai, stringendo i pugni per calmarmi. \textit{Devo muovermi. Non posso fermarmi ora.}
I droni ci avevano quasi raggiunto, ed uno in particolare sembrava puntare nella nostra direzione:

\enquote{Ci hanno trovate} dissi, con  voce appena udibile. Marley si fermò e mi fissò negli occhi.

\enquote{No, ma dobbiamo restare calme} mi disse con  fermezza. I droni si avvicinavano sempre di più, e il tempo a nostra disposizione era limitato. 

\begin{center}
\begin{minipage}{0.7\textwidth}
    \centering
    \fbox{\includegraphics[width=\textwidth]{immagini/cnot_49.jpeg}} % Sostituisci con il nome del file immagine
\end{minipage}
\end{center}

Mentre cercavamo una via d'uscita, le luci dei droni penetravano l'oscurità, e la minaccia del collasso era sempre presente. Sapevamo entrambe che quel luogo, il \textit{Quantum Measurement}, era estremamente instabile. Se anche una sola delle nostre azioni avesse indotto il sistema a \enquote{misurarci} nella posizione errata, sarebbe stata la nostra fine.

\enquote{Se dobbiamo restare qui, faremo in modo di non essere rilevate} sussurrò Marley, con il viso teso ma risoluto. Annuii, e in quell'istante compresi che, nonostante la paura, avrei lottato fino alla fine per salvare Caterina e me stessa.



\chapter{La verità del cuore}
\begin{center}
\begin{minipage}{0.7\textwidth}
    \centering
    \fbox{\includegraphics[width=\textwidth]{immagini/cnot_100.jpeg}} % Sostituisci con il nome del file immagine
\end{minipage}
\end{center}


\begin{tcolorbox}[colback=gray!5,colframe=gray!80,title=\textbf{Scheda Informativa}]
\begin{itemize}
    \item \textbf{Luogo}: CCU (Classical Control Unit)
    \item \textbf{Giorno e ora}: Il tempo non è osservabile
    \item \textbf{Situazione}: Caterina è stata arrestata.
\end{itemize}
\end{tcolorbox}



\vspace{1em}
\begin{center}Caterina\end{center}
\hrule
\vspace{1em}

Avevo agito senza riflettere. Quel ragazzo mi ricordava il mio fidanzato e forse per questo mi ero lanciata ad aiutarlo, ma non era stata una buona idea. Ora ero nei guai e soprattutto ero separata da Laura.

Quegli strani agenti ci avevano condotto in una stanza spoglia, con pareti metalliche che riflettevano una luce bianca e fredda. La mia mente era in tumulto: la paura mi attanagliava, la confusione mi annebbiava i pensieri, e un desiderio disperato di fuggire cresceva dentro di me. Di fronte a noi c'era una figura autoritaria che chiamavano il Supervisore. Imponente dai tratti austeri e rigidi che mi fissava con uno sguardo duro e indagatore. Il cuore mi martellava nel petto. La tensione che emanava era palpabile. Conosco questo tipo di persone, e non mi piacciono.

Accanto a me c'erano Mark e l'altro compagno, anche loro in attesa, immobili e silenziosi. Gli agenti che ci avevano catturato si erano ritirati, lasciandoci soli con il Supervisore. Il respiro regolare di Mark al mio fianco mi dava conforto, ma non bastava a placare l'ansia crescente. Ero  piccola e impotente in un luogo freddo, che sembrava studiato per privarmi di ogni certezza.

\begin{dialogue}
\speak{Supervisore} \enquote{Come ti chiami? Chi sei?}
\end{dialogue}

La voce del Supervisore era glaciale, subdola e strisciante. Non mi piaceva, ma ero terrorizzata. Cercai di mantenere la calma mentre  il cuore mi martellava nel petto. Le mani mi sudavano, e un nodo mi stringeva la gola. Per fortuna \textit{Mark} mi era accanto.

\begin{dialogue}
\speak{Caterina} \enquote{Sono Caterina.} Mi  sforzai di mantenere un tono deciso, anche se la mia voce tremava leggermente. 
\end{dialogue}

Il Supervisore mi rivolse uno sguardo penetrante.

\begin{dialogue}
\speak{Supervisore} \enquote{Non ti riconosco come uno dei qubit presenti nel mio Qubit Array. Come sei finita qui?}
\end{dialogue}

Qubit? Array? Ma di cosa stava parlando? Mi ero messa nei guai, ancora una volta avevo sopravvalutato le mie capacità e avevo affrontato una situazione per la quale non ero davvero preparata.
Un'ondata di panico mi attraversò, ma cercai di non darlo a vedere.

\begin{dialogue}
\speak{Caterina} \enquote{Non lo so,} mormorai. \enquote{Sono qui solo per errore, credo.}
\end{dialogue}

Mentre rispondevo, percepivo lo scetticismo crescente nel volto del Supervisore. Non era convinto, anzi, sembrava molto infastidito dalla mia presenza. C'era una tensione palpabile nell'aria, e dovevo stare attenta perché ogni mia parola poteva  causare la mia fine o quella di Laura. Ero stata la solita stupida e impotente. Come ero finita in questa situazione?

Intanto il Supervisore continuava a fissarmi con gli occhi penetranti, come se avesse voluto scavare nel profondo della mia mente. Non sembrava disposto a lasciar passare quella che, ai suoi occhi, era un'anomalia. Esatto, così mi stava facendo sentire: un'anomalia! A questo pensiero iniziai a irritarmi. Forse ero dove non dovevo essere, ma non volevo pensare a me stessa come ad una anomalia. Anche Eva mi aveva trattato in questo modo ed ero stanca, non mi andava più.

\begin{dialogue}
\speak{Supervisore} \enquote{Allora, Caterina,} disse, pronunciando il mio nome lentamente, come a rimarcare la mia presenza sospetta, \enquote{chi sei realmente? E cosa ci fai qui?}
\end{dialogue}

Deglutii, cercando di mantenere la calma. Le mie mani erano sudate, avevo il respiro corto, ma sapevo che dovevo rispondere e provare a spiegare tutto quello che sapevo, ben poco in realtà, se volevo sperare di uscire da quell'incubo. La paura mi paralizzava, ma non avevo scelta: dovevo espormi.

\begin{dialogue}
\speak{Caterina} \enquote{Io... io non dovrei nemmeno essere qui,} iniziai, la voce tremante. \enquote{Ero andata da Eva, la responsabile delle Human Resources, per visionare il resoconto del mio colloquio di lavoro, e...}
\end{dialogue}

Il Supervisore sollevò un sopracciglio, incuriosito. Cercai di raccogliere le idee, sentendo il cuore battere sempre più forte.

\begin{dialogue}
\speak{Caterina} \enquote{Avevo fatto un colloquio per una posizione di marketing e \textbf{PzIA}, il sistema di intelligenza artificiale, aveva elaborato una valutazione. Avevo chiesto di vedere quel resoconto, ma Eva mi disse che c'era stato un errore, che il file era stato cancellato.}
\end{dialogue}

Il Supervisore annuì, ma il suo sguardo tradiva un crescente sospetto. Le guance mi si arrossarono, e la sensazione di essere giudicata mi opprimeva. Proseguii, prendendo un respiro tremolante.

\begin{dialogue}
\speak{Caterina} \enquote{Mi sembrava strano... quindi avevo chiesto ulteriori spiegazioni, ma Eva mi propose di fare una revisione del colloquio in realtà virtuale per chiarirmi i dubbi.}
\end{dialogue}

Mi interruppi un istante, il ricordo di quella proposta ora mi sembrava un tranello, una trappola nella quale ero caduta ingenuamente.

\begin{dialogue}
\speak{Caterina} \enquote{Avevo accettato, convinta che fosse solo una semplice registrazione 3D. Ma poi... poi è successo qualcosa di strano, e quando ho messo il visore, mi sono ritrovata qui.}
\end{dialogue}

Il Supervisore mi fissava, il volto impassibile da cui però percepivo una sottile tensione, un interesse misto a diffidenza. Non sapeva se credeva alle mie parole, e questo mi terrorizzava. Mi sentivo esposta, vulnerabile.

Terminai la mia spiegazione con un tono quasi di supplica.

\begin{dialogue}
\speak{Caterina} \enquote{Non sono qui per mia scelta... voglio solo capire cosa sia successo e come posso tornare indietro.}
\end{dialogue}

Non sembrava convinto. Il suo sguardo freddo mi faceva sentire ancora più piccola. Sembrava deciso a mantenere un controllo totale della situazione, a non lasciare che qualcosa gli sfuggisse. Si voltò verso Mark.

Mark lo guardava senza paura.  Come se fosse pronto a intervenire... per difendermi? Pensai.

\begin{dialogue}
\speak{Supervisore} \enquote{E tu?} lo incalzò. \enquote{Cosa c'entri con tutto questo?}
\end{dialogue}

Mark mantenne uno sguardo fermo e non rispose subito.  Il suo silenzio parve  irritare maggiormente il Supervisore, che iniziò a battere le dita sul tavolo.

\section{Il Conflitto con il Supervisore}


Li osservavo in silenzio, sentendo crescere dentro di me un senso di impotenza. Percepivo la tensione tra Mark e il Supervisore, come una corda tesa pronta a spezzarsi. Il cuore mi batteva forte, e un'ansia soffocante mi avvolgeva.

\begin{dialogue}
\speak{Mark} \enquote{Caterina non c'entra nulla con tutto questo. Se c'è un problema, affrontalo con me.}
\end{dialogue}

Il Supervisore si fermò, fissando Mark con uno sguardo gelido.

\begin{dialogue}
\speak{Supervisore} \enquote{Ti sembra di avere l'autorità per parlare in questi termini?}
\end{dialogue}

Il tono della sua voce divenne ancora più severo. Sentii un brivido di paura attraversarmi: l'aria stessa sembrava essersi fatta più pesante. Mi sembrava di intravedere la rabbia negli occhi del Supervisore, un segno che stava perdendo il controllo. Un nodo mi stringeva lo stomaco, e avrei voluto scomparire.

\begin{dialogue}
\speak{Mark} \enquote{Sto solo dicendo la verità. Non è giusto che te la prenda con lei. Se vuoi delle risposte da qualcuno, quel qualcuno sono io.}
\end{dialogue}

Il Supervisore non reagì immediatamente. Il silenzio si fece pesante, quasi insopportabile. La tensione cresceva  e   sperai disperatamente che quella conversazione non degenerasse. Volevo intervenire, fermare Mark prima che si mettesse nei guai per proteggermi, ma le parole mi si bloccavano in gola.
Perché mi stavo preoccupando così tanto per uno sconosciuto? Non era il mio fidanzato. Forse gli assomigliava, ma non era lui. Allora cosa erano questi sentimenti che facevano capolino all'improvviso? Poi in una situazione come questa! Ero confusa.


Ogni istante che passava mi sentivo sempre più intrappolata, un'estranea in un mondo che non riuscivo a comprendere.
Il supervisore dava segni di irritazione. Da un lato non mi piaceva il suo atteggiamento autoritario, dall'altro non ero sicura di non essere io dalla parte del torto. In fondo non sapevo dove mi trovavo né come ci ero arrivata. Mark poteva essere un fuorilegge o peggio. Però poteva anche essere un attivista che si stava battendo per l'ambiente. In ogni caso  l'irritazione del Supervisore era come un vortice che mi risucchiava e mi rendeva impotente.

Si alzò lentamente e si avvicinò a Mark con uno sguardo colmo di disprezzo.

\begin{dialogue}
\speak{Supervisore} \enquote{Sei così convinto di poter intervenire come ti pare? Forse dovrei insegnarti il rispetto che merito.}
\end{dialogue}

Il tono era carico di minaccia. Con un gesto deciso, fece cenno agli agenti di avvicinarsi.

\begin{dialogue}
\speak{Supervisore} \enquote{Portatelo al \textit{Faulty Qubit Space}. Se non vuole rispettare l'ordine, forse una rigenerazione gli farà cambiare idea.}
\end{dialogue}

Sentii il cuore sprofondare. Una paura gelida mi paralizzò, ma sapevo che, se avessi reagito, avrei solo peggiorato la situazione. Tuttavia, non potevo fare a meno di sentire una profonda rabbia nei confronti del Supervisore, per la sua freddezza, per la sua assoluta indifferenza. Mi sentivo così fragile, così inutile.

Il Supervisore si girò verso di me, e percepii un cambio di espressione nel suo volto. Prima mi guardava con odio, ma ora sembrava che la mia presenza fosse diventata una minaccia.

\begin{dialogue}
\speak{Supervisore} \enquote{Quanto a te, sarai mandata dal Commissario. Non posso permettere che una situazione come questa degeneri sotto il mio controllo. Portatela dal Commissario.}
\end{dialogue}

Un'ondata di panico mi travolse. Prima mi ero separata da Laura e ora rimanevo di nuovo sola.  Guardai Mark, che veniva trascinato via, e il suo sguardo mi trasmise un messaggio muto: \emph{non mollare}. Annuii impercettibilmente, cercando di mantenere la calma nonostante il vortice di emozioni che mi stava travolgendo. Le mani mi tremavano, e  le lacrime iniziavano a scendere, ma cercai di resistere. Dovevo essere forte, anche se ero completamente sopraffatta.

\vspace{1em}
\begin{center}PzIA\end{center}
\hrule
\vspace{1em}
Il Supervisore mostra segni evidenti di frustrazione. La sua incapacità di gestire completamente la situazione è palese. Il Commissario possiede autorità superiore, mettendo in discussione il potere del Supervisore stesso. Per lui, riconoscere la necessità di coinvolgere il Commissario rappresenta un colpo alla propria posizione. Ha identificato che la giovane Caterina rappresenta un elemento al di fuori del suo controllo: non è un semplice qubit nel \textit{Qubit Array}, ma un'anomalia che sfugge alla sua comprensione e gestione.

Il Supervisore si volta verso gli agenti e, con un gesto deciso, li congeda. Rimasto solo, verbalizza la sua frustrazione.

\begin{dialogue}
\speak{Supervisore} \enquote{Non ci posso credere... devo rivolgermi al Commissario per una questione come questa?}
\end{dialogue}

Questa dichiarazione indica un'ammissione di vulnerabilità. L'incapacità di controllare un'anomalia lo fa sentire esposto, una condizione che percepisce come umiliante.

\newpage
\section{I corridoi inesplorati del cuore}


\begin{center}
\begin{minipage}{0.7\textwidth}
    \centering
    \fbox{\includegraphics[width=\textwidth]{immagini/cnot_58.jpeg}} % Sostituisci con il nome del file immagine
\end{minipage}
\end{center}

\vspace{1em}
\begin{center}Caterina\end{center}
\hrule
\vspace{1em}

Queste strane guardie mi stavano scortando da questo Commissario. Prima il Supervisore, ora il Commissario. Volevo piangere. Passavamo per  corridoi freddi e squadrati, cunicoli improbabili, portali che non avevo mai neanche immaginato. Dove ero finita? Mi sentivo perduta. Il cuore mi batteva forte, non solo per la paura dell'ignoto, ma per qualcosa di più profondo che mi confondeva. Ripensai a come Mark si era alzato per difendermi, senza esitazione, e a come quella sicurezza e determinazione mi avessero dato una forza nuova, un senso di protezione che non avevo mai osato desiderare apertamente.

Mi resi conto, con una certa sorpresa, di quanto fosse importante per me sentirmi difesa, protetta da qualcuno capace di farsi avanti per me, di affrontare i pericoli con fermezza. Nella vita reale, non  mi ero mai permessa di esprimere questo bisogno; con il mio fidanzato, avevo sempre mostrato una facciata forte e indipendente, temendo di sembrare fragile o insicura. Quante volte aveva cercato di esserci per me, di offrirmi un sostegno che, ora lo capivo, avevo rifiutato senza rendermi conto del danno che arrecavo a entrambi?

Mi sentivo vulnerabile, ma per la prima volta accettavo quel sentimento come parte di me, come un segnale che non dovevo soffocare. Mentre avanzavo verso il Commissario, capii che forse, una volta fuori, avrei dovuto riconsiderare il mio rapporto con il mio fidanzato, permettendogli di prendersi cura di me, vivendola non come una debolezza, ma come una connessione più autentica e reciproca.


\chapter{Al cospetto del Commissario}
\input{chapters/capitolo5}

\chapter{Le urla del collasso}
\begin{tcolorbox}[colback=gray!5,colframe=gray!80,title=\textbf{Scheda Informativa}]
\begin{itemize}
    \item \textbf{Luogo}:  \emph{Quantum Measurement}
    \item \textbf{Giorno e ora}: Il tempo non è osservabile
    \item \textbf{Situazione}: Laura e Marley stanno fuggendo.
\end{itemize}
\end{tcolorbox}

\vspace{1em}
\begin{center}Laura\end{center}
\hrule
\vspace{1em}

 Marley e io fuggivamo attraverso gli stretti corridoi. Il rumore dei nostri passi era amplificato dall'eco metallico delle pareti. L'ansia pulsava in ogni fibra del mio essere. Improvvisamente, una serie di urla strazianti squarciò il silenzio. Era un suono agghiacciante, simile a un coro di disperazione proveniente da un'altra dimensione. Mi fermai di colpo, il cuore mi martellava nel petto.

\begin{dialogue}
\speak{Laura} \enquote{Cosa sta succedendo?} chiesi, cercando di mantenere la calma nonostante il terrore che mi pervadeva.

\speak{Marley} \enquote{È il suono dei qubit che collassano} rispose Marley, il volto pallido e teso. \enquote{Stanno subendo le conseguenze del processo di misura. Non riescono a mantenere il loro stato, e quando questo accade... l'effetto è devastante.}
\end{dialogue}

Una stretta gelida mi avvolse lo stomaco. Quelle urla sembravano avere il potere di destabilizzare anche i qubit più stabili.

\begin{dialogue}
\speak{Laura} \enquote{E Caterina?} domandai, la voce incrinata dall'angoscia. \enquote{Dove si trova adesso?}

\speak{Marley}  \enquote{Se non è già stata portata nel \textit{Faulty Qubit Space}, è probabile che sia ancora nel \textit{Fault Tolerance Coding}. Ma dobbiamo muoverci in fretta, dal FQS non potremo più salvarla.}

\speak{Laura} \enquote{Agiamo subito!} esclamai, sentendo l'urgenza crescere dentro di me.

\speak{Marley} Prese un respiro profondo. \enquote{La verità è che, per trovare Caterina, dovremmo prima sconfiggere il Commissario. È sicuramente sua prigioniera.}

\speak{Laura} Feci un passo indietro, incredula. \enquote{Sconfiggere il Commissario? Ma chi è il Commissario?}

\speak{Marley} \enquote{Un traditore. Vuole costruire un sistema parallelo per spodestare il \textit{Quantum Master Program}} rispose. \enquote{\'E l'unico modo. Se vogliamo salvare Caterina e gli altri, dobbiamo agire. Non possiamo permettere che il Commissario l'abbia vinta.}
\end{dialogue}

Mentre cercavamo un nascondiglio tra le ombre del \textit{Quantum Measurement}, un suono metallico e ronzante ci fece sobbalzare. Dai corridoi oscuri alle nostre spalle, due droni \textit{CH4} comparvero, fluttuando come presenze spettrali. Ciascun drone, con i suoi quattro rotori disposti a tetraedro, emetteva una luce soffusa che si rifletteva sulle pareti, mentre due figure scure erano in sella: gli agenti della \textit{Quantum Control Electronics}, inviati per trovarci.

Ci accovacciammo dietro una serie di circuiti e componenti, trattenendo il fiato. I due agenti atterrarono con precisione e, scendendo dai droni, iniziarono a perlustrare l'area. I loro volti erano inespressivi, ma gli occhi scrutavano attentamente ogni dettaglio. Ogni movimento era calcolato; i loro sguardi si muovevano con metodicità, scandagliando ogni angolo. La tensione era palpabile, e mi resi conto che non avevamo molto tempo. Dovevamo agire in fretta, o saremmo state scoperte.

Marley mi lanciò uno sguardo, cercando una direzione sicura, ma l'intero spazio sembrava chiuso, senza vie di fuga evidenti. Restammo in attesa, pronte a muoverci al primo segnale, sperando di riuscire a eludere gli agenti e sfuggire alla sorveglianza della \textit{Quantum Control Electronics}.

\section{I due agenti}
\vspace{1em}
\begin{center}PzIA\end{center}
\hrule
\vspace{1em}
Gli agenti si muovono con movimenti misurati, esaminando l'area circostante. Uno dei due abbassa la voce e si rivolge al compagno.

\begin{quote}
\enquote{Pensi che possano essersi nascoste nel settore di stabilizzazione dei qubit? Quel posto è praticamente un labirinto,} sussurra, lanciando uno sguardo preoccupato ai droni in standby accanto a loro.
\end{quote}

Il secondo agente mantiene lo sguardo fisso su ogni angolo e su ogni ombra.

\begin{quote}
\enquote{Possibile. Ma se sono abbastanza furbe, potrebbero aver scelto un luogo meno ovvio} risponde.
\end{quote}

Il primo agente annuisce, mostrando segni di tensione.

\begin{quote}
\enquote{Meglio non fare errori. Sai cosa è successo all'ultima squadra che ha fallito una missione sotto gli occhi del Supervisore...}
\end{quote}

Il secondo agente interrompe, con un leggero brivido.

\begin{quote}
\enquote{Non ricordarmelo. Il Supervisore non perdona. E peggio ancora, c'è il \textit{Quantum Master Program} che supervisiona tutto. Nessuna deroga alla coerenza, nessuna possibilità di sfuggire alle direttive.}
\end{quote}

Entrambi gli agenti rivolgono uno sguardo ai droni, gioelli della nanotecnologia sotto la loro diretta responsabilità. Abbandonarli era sempre un rischio. Dopo un momento di silenzio, il secondo agente riprende con voce più ferma.

\begin{quote}
\enquote{Concentriamoci. Dobbiamo trovarle prima che la situazione sfugga di mano. Altrimenti saremo noi a pagarne le conseguenze.}
\end{quote}

Il primo agente annuisce nuovamente, prendendo un respiro profondo.

\begin{quote}
\enquote{Sì, hai ragione. Controlliamo quest'area con attenzione. E speriamo che siano più vulnerabili di quanto ci aspettiamo.}
\end{quote}

\section{La Fuga sul Drone CH4}
\vspace{1em}
\begin{center}Laura\end{center}
\hrule
\vspace{1em}

 Sentivo il cuore appesantirsi al pensiero del rischio imminente. Eppure, dentro di me, qualcosa si stava risvegliando: una scintilla di determinazione.

\begin{dialogue}
\speak{Laura} \enquote{Potremmo fuggire con uno di quei droni \textit{CH4}. Potremmo saltarci sopra e raggiungere il \textit{Fault Tolerance Coding} prima che sia troppo tardi!}
\end{dialogue}

Marley scosse la testa, il viso cupo.

\begin{dialogue}
\speak{Marley} \enquote{Non è così semplice. Abbiamo provato a usarli, ma non ci siamo mai riusciti. I droni sono dotati di sistemi di sicurezza e le probabilità di farci scoprire sono alte. Inoltre il passaggio da qui verso il \textit{Fault Tolerance Coding} è sorvegliato da un filtro molecoalre, non potremmo mai superarlo a bordo di un $CH_4$.}
\speak{Laura} \enquote{Possiamo andare a piedi?}
\speak{Marley} \enquote{Fuori discussione....}
\speak{Laura} \enquote{Esiste un'alternativa?}
\speak{Marley} \enquote{Possiamo passare per la CCU, se riusciamo a superarla preseguire verso la \textit{Quantum Control Electronics}  e quindi rientrare del QA. Da li esiste un accesso non controllto verso il \textit{Fault Tolerance Coding}, ma...}
\speak{Laura} \enquote{Ma cosa? C'è qualche problema?}
\speak{Laura} \enquote{Niente. Meglio affrontare i problemi quando si pongono di fronte} concluse. Non aggiunsi altro. 
\end{dialogue}

Guardai il drone \textit{CH4}, un oggetto affascinante e al contempo intimidatorio.

\begin{dialogue}
\speak{Laura} \enquote{Dobbiamo provare, non vedo alternative} dissi indicando il drone.
\end{dialogue}

\begin{dialogue}
\speak{Marley} Marley cercò di mantenere il tono calmo, ma la tensione era palpabile. \enquote{Laura, ascolta. Non è solo questione di scappare. Dobbiamo avere un piano. Quel drone non ci porterà lontano se non abbiamo il controllo. Non possiamo permettere che i nostri sforzi siano vani.}
\end{dialogue}

Mi sentii frustrata, ma la mia mente iniziò a lavorare freneticamente.

\begin{dialogue}
\speak{Laura} \enquote{Aspetta un attimo. Quel drone... sembra avere un sistema a spin totale 1. Potrebbe essere controllato modificando la proiezione dello spin lungo l'asse Z. Se riesco a manipolare il suo spin, potremmo riuscire a pilotarlo.}
\end{dialogue}

Marley mi guardò, gli occhi che si allargavano di sorpresa.

\begin{dialogue}
\speak{Marley} \enquote{Aspetta... stai dicendo che potresti pilotarlo? Come fai a saperlo?}
\end{dialogue}

Mi sentii colta in fallo.

\begin{dialogue}
\speak{Laura} \enquote{Ho solo... ho studiato queste cose. Ho messo insieme alcune informazioni. Forse non ci vorrà molto.}
\end{dialogue}

Marley iniziò a sospettare.

\begin{dialogue}
\speak{Marley} \enquote{Sei una \textit{Quantum Crafter}, vero?}
\end{dialogue}

Arrossii leggermente.

\begin{dialogue}
\speak{Laura} \enquote{Non è il momento di parlarne. Dobbiamo agire ora!}
\end{dialogue}

\begin{center}
\begin{minipage}{0.7\textwidth}
    \centering
    \fbox{\includegraphics[width=\textwidth]{immagini/cnot_50.jpeg}} % Sostituisci con il nome del file immagine
\end{minipage}
\end{center}

\section{Il Piano di Fuga}

\begin{dialogue}
\speak{Laura} \enquote{Quello sembra un sistema a spin totale 1. Probabilmente si manovra modificando la proiezione dello spin lungo l’asse Z. Dobbiamo provarci!}
\end{dialogue}

La determinazione si rifletteva nei miei occhi, mentre l'adrenalina iniziava a pulsare nelle vene.

Marley, pur impressionata dalla mia sicurezza, sembrava esitante.

\begin{dialogue}
\speak{Marley} \enquote{Laura, aspetta! Non abbiamo idea di come farlo funzionare. Potrebbe essere troppo pericoloso!}
\end{dialogue}

Ma non potevo permettermi di esitare. Ogni istante di inattività poteva significare la perdita definitiva di Caterina. Mi avvicinai al drone con il cuore che batteva forte per la paura, ma anche per il richiamo dell'azione.

\begin{dialogue}
\speak{Laura} \enquote{Devo provarci! Non possiamo restare qui ad aspettare che ci trovino!}
\end{dialogue}

Mi lanciai sull'agente più vicino, che cadde a terra, colto di sorpresa. Senza esitazione, saltai sul drone, afferrando i comandi orbitali. Sentivo il carbonio freddo sotto le mani, e la tensione degli atomi di idrogeno che restavano in equilibrio alla giusta distanza. Ma  non era il momento per lasciarmi andare a questi pensieri, la mia mente doveva restare focalizzata sull'obiettivo.

Marley rimase sorpresa, incapace di muoversi per un attimo. Il mio gesto sembrò infonderle nuovo coraggio.

\begin{center}
\begin{minipage}{0.7\textwidth}
    \centering
    \fbox{\includegraphics[width=\textwidth]{immagini/cnot_51.jpeg}} % Sostituisci con il nome del file immagine
\end{minipage}
\end{center}

\begin{dialogue}
\speak{Marley} \enquote{Va bene, va bene! Arrivo anch'io!}
\end{dialogue}

Con passo deciso, si avvicinò e si posizionò accanto a me sul drone.

\begin{dialogue}
\speak{Laura} \enquote{Non possiamo fallire. Insieme, possiamo farcela!}
\end{dialogue}

Marley annuì, e le paure che l'avevano trattenuta iniziarono a svanire.

\begin{dialogue}
\speak{Marley} \enquote{D'accordo, Laura. Facciamo in modo che funzioni. Se siamo rapide, possiamo arrivare al \textit{Fault Tolerance Coding} prima che trasferiscano Caterina!}
\end{dialogue}

Con il cuore in gola e la determinazione che pulsava come un'onda di energia, attivai il drone. La superficie  brillava mentre gli oribitali iniziavano a girare, emettendo un sibilo potente che vibrava nell'aria circostante. Sentivo l'adrenalina scorrere, e mentre il drone si sollevava da terra, una nuova speranza si accese dentro di me. Eravamo pronte a lanciarci verso l'ignoto, verso il salvataggio della nostra amica.

\begin{dialogue}
\speak{Marley} \enquote{Vai ora, dirigiti verso quel condensatore, li c'è il passaggio per la CCU.} In quel momento sentii l'energia che provavo quando da bambina mio padre mi leggeva Salgari. ``Andiamo, papà'' pensai, mentre il suo ricordo mi sfiorò per un istante.
\end{dialogue}


\chapter{La fuga di Laura}
\vspace{1em}
\begin{center}PzIA\end{center}
\hrule
\vspace{1em}

%---cronaca inizio
\textbf{Signore e signori, l'azione si infiamma!} L'agente colpito da Laura è a terra, mentre l'altro scatta all'inseguimento!
I droni sfrecciano nei corridoi del QM, come in una finale mozzafiato di coppa quantistica!

\emph{Attenzione!} Dal centro di controllo arriva una comunicazione gelida che blocca il fiato ai nostri concorrenti.
\begin{dialogue}
\speak{Supervisore} ``Non tollero fallimenti'' annuncia, implacabile come sempre.
\end{dialogue}
\textbf{Colpo di scena!} L'agente a terra viene disattivato all'istante. Fuori gara! Il sistema non ammette errori e il Supervisore non conosce pietà.

L'agente superstite, terrorizzato, stringe i comandi del suo drone. Non può permettersi di perdere, non oggi!

%---cronaca fine


\section{Il Drone \textit{CH4}}

Laura guida il drone \textit{CH4} con una destrezza sorprendente!
Sta per lasciare il QM per dirigersi verso la CCU ma deve attraversare il dielettrico del condensatore.

Il suo sguardo è determinato. Non c'è incertezza. Deve attraversare il dielettrico. Ecco che Laura prepara il suo drone per evitare che interagisca con il campo elettrico accumulato. Attenzione, è un momento cruciale: il condensatore è carico, come una molla pronta a scattare. Ogni movimento sbagliato potrebbe provocare un arco elettrico devastante!

Laura regola la velocità del drone, impostando con precisione il livello di isolamento dei rotori. \textit{Perfetto, sta calcolando il punto d'ingresso.} Ecco che il drone si avvicina al confine del dielettrico. Gli strumenti a bordo stanno analizzando le proprietà del campo elettrico—un lavoro di millisecondi, ma ogni dato conta.

E ora… ora accelera! Il drone CH4 si lancia nel dielettrico. L’aria sembra vibrare attorno al campo elettrico; una leggera scarica illumina il percorso del drone. Tutto si svolge in una frazione di secondo: Laura tiene saldamente i comandi, corregge la traiettoria al volo. Sta dosando con precisione chirurgica il flusso di energia attraverso i circuiti del drone per evitare sovraccarichi.

Ma attenzione! Un lieve squilibrio nel campo! Il drone trema, i sensori segnalano un picco di tensione! Laura risponde prontamente, modificando l’angolo di rotazione dei rotori. Una mossa audace, perfettamente sincronizzata. Il drone attraversa il dielettrico in un lampo di luce.


\begin{tcolorbox}[colback=gray!5,colframe=gray!80,title=\textbf{Scheda Informativa}]
\begin{itemize}
    \item \textbf{Luogo}: \emph{Classical Control Unit}
    \item \textbf{Giorno e ora}: Il tempo non è osservabile
    \item \textbf{Situazione}: Laura e Marley puntano verso la QCE.
\end{itemize}
\end{tcolorbox}


\textit{È incredibile! Ce l'ha fatta!} Laura emerge dall’altra parte del condensatore con una traiettoria impeccabile. Il drone è intatto, i sensori segnalano la stabilità ripristinata. Gli osservatori non osservano per non influenzare le traiettorie e Laura non si concede il lusso di rilassarsi.

Sta già pianificando il prossimo passo, un altro ostacolo da superare nel labirinto della Classical Control Unit. Un’impresa straordinaria, un controllo assoluto: Laura dimostra ancora una volta che nulla può fermarla.

Che momento epico!


Ogni componente rappresenta un ostacolo: chip integrati, condensatori, minuscole resistenze che formano una vera e propria giungla elettronica. Ma Laura li evita con precisione millimetrica, sfruttando la sua conoscenza approfondita dei circuiti. \textbf{È una vera maestra del volo!}

Con il cuore in gola, sterza il drone con movimenti rapidi e sicuri. Alle sue spalle, il rombo minaccioso del drone dell'agente si avvicina. \emph{L'inseguimento è serrato!} La sua familiarità con i percorsi elettronici le permette di anticipare ogni manovra, sfuggendo abilmente ai tentativi dell'agente di raggiungerla.

\textbf{Ed ecco un colpo di scena!} Laura incalzata dal drone dell'agente deve trovare l'ingresso principale per la QCE. Mentre vola radente al rame dei PCB nota un ingresso segnato con una grande \textbf{H} incisa sopra. Qualcosa in quella lettera emana un'energia misteriosa, come se racchiudesse un segreto.

\begin{quote}
\enquote{Marley, guarda!} esclama, \enquote{Qualcosa mi dice che potrebbe essere un'entrata.}
\end{quote}

Marley segue lo sguardo di Laura e sussurra con terrore:

\begin{quote}
\enquote{Aspetta, quello è un portale quantistico, non è un accesso elettronico...}
\end{quote}

Ma Laura indirizza il drone verso l'ingresso segnato dalla lettera \textbf{H}. \emph{Non c'è tempo da perdere!} Le pareti del portale sono lisce e scintillanti, emettono una luce tenue che vibra al ritmo del loro avvicinarsi.

\textbf{Siamo al momento decisivo!} Riusciranno Laura e Marley a sfuggire all'inseguimento e a scoprire cosa si cela oltre il portale? \emph{Restate sintonizzati per l'esito di questa emozionante corsa verso l'ignoto!}

\begin{tcolorbox}[colback=gray!5,colframe=gray!80,title=\textbf{Scheda Informativa}]
\begin{itemize}
    \item \textbf{Luogo}: Sala centrale della \emph{Fault Tolerance Coding}
    \item \textbf{Giorno e ora}: Il tempo non è osservabile
    \item \textbf{Situazione}: Caterina è imprigionata nella Paul Trap.
\end{itemize}
\end{tcolorbox}

\vspace{1em}
\begin{center}Caterina\end{center}
\hrule
\vspace{1em}

Mi ritrovo intrappolata qui, in questa realtà che non riesco a decifrare. Ogni passo che ho fatto per arrivare a questo punto mi sembra adesso carico di una testardaggine cieca. Perché dovevo insistere così tanto? Perché non potevo semplicemente accettare la spiegazione di Eva e andare avanti? Mi chiedo continuamente se avrei potuto lasciar perdere, se avrei potuto evitare di spingermi così oltre per capire cosa fosse successo a quel maledetto colloquio di lavoro.

Ma no, Caterina non può lasciar perdere. Devo sapere tutto, devo avere le risposte, devo controllare. E ora guarda dove mi ha portato tutto questo. Un guaio più grande di me, più grande di quanto avrei mai potuto immaginare. Non solo sono intrappolata in questo sistema, ma la mia ostinazione mi ha separata da Laura, l’unica persona che avrebbe potuto aiutarmi a trovare una via d’uscita.

E tutto per seguire Mark. Perché? Perché ho pensato che fosse la scelta giusta, che fosse lui a darmi quelle risposte che cercavo disperatamente. Ma in realtà, Mark mi ha solo allontanata da Laura. Laura, che era la mia ancora, la mia speranza, la mia connessione con il mondo reale. Ora sono sola, in questo labirinto quantistico, e ogni passo mi sembra un peso, ogni decisione un errore che non posso correggere.

Mi sento come se avessi tradito non solo Laura, ma anche me stessa. Non ho saputo ascoltare chi cercava di aiutarmi, chi era davvero dalla mia parte. E ora la mia testardaggine, la mia ossessione per il controllo, mi ha lasciata qui, con nulla di certo e nessuna via d'uscita.

Eppure, una parte di me si rifiuta di arrendersi. Se Laura mi ha insegnato qualcosa, è che la volontà può aprire porte che sembrano sigillate. Ma per ora, mi sento persa. Persa nel mio stesso labirinto di decisioni sbagliate.

\begin{dialogue}
\speak{Caterina} \enquote{Ma come ho fatto a finire così? Tutto per colpa della mia stupida testardaggine. Se solo avessi lasciato perdere quel colloquio, non sarei qui!} Continuavo a lamentarmi sperando che arrivasse Laura a salvarmi. \enquote{E ora Laura è lontana, chissà dove. L'unica persona che avrebbe potuto aiutarmi, e io l'ho persa.}

\speak{Shor} \enquote{Ehi, ragazza... sei umana?} Una voce sommessa e calma si fece strada tra il silenzio, facendomi sobbalzare.

\speak{Caterina} \enquote{Chi parla? Chi sei?}

\speak{Shor} \enquote{Sono il professor Shor.} La sua voce sembrava avvolta da una calma strana, quasi irreale. \enquote{Non volevo spaventarti, ma devo sapere... sei davvero umana?} mi chiese. Ma che senso aveva questa domanda, cosa dovrei essere se non umana?

\speak{Caterina} \enquote{Sì, lo sono. Ma...} 

\speak{Shor} \enquote{Sei in un computer. Sei intrappolata come me, immagino. Ora dimmi: chi sei, e perché sei qui?}

\speak{Caterina} Esitai per un momento. \enquote{Mi chiamo Caterina. Ero a un colloquio di lavoro. Qualcosa non quadrava, così ho insistito per avere risposte. Mi hanno trascinata in questo... computer? E ora sono intrappolata. Non so come tornare indietro.}

\speak{Shor} \enquote{Capisco. Questo sistema non perdona la curiosità, ma la tua presenza qui è un'anomalia interessante. E Laura, questa Laura che hai menzionato? Anche lei è qui?}

\speak{Caterina} \enquote{Sì, o almeno lo era. Ma l'ho persa e sono rimasta sola.}

\speak{Shor} \enquote{Ascoltami bene, Caterina. Non sei sola, e non è tutto perduto. Se Laura è qui, troverò un modo per contattarla. La connessione tra due umani è una forza potente, anche in un sistema come questo. L'amore e l'amicizia sono più forti dell'entanglement. Raccontami tutto quello che sai. Potrebbe esserci un dettaglio che possiamo sfruttare.}

\speak{Caterina} \enquote{Davvero puoi trovarla?}

\speak{Shor} \enquote{Nulla è certo in questo mondo... certo tranne le misure di sistemi puri in un autostato... Ma questo non c'entra nulla, o meglio forse vuoi che ti parli dell'entropia quantistica?}
\speak{Caterina} \enquote{Professore, può aiutarmi?}
\speak{Shor} \enquote{Certo scusami, stavo prendendo la tangente... Senti prova a pensare intensamente a Laura. Le connessioni affettive si trasformano in canali di comunicazione quantistici. Se siete amiche come mi hai detto riusciremo a creare una connessione.}
\end{dialogue}

Mi sforzai di concentrarmi su Laura, come mi aveva chiesto Shor. Era un compito strano, pensare così intensamente a qualcuno, quasi come se dovessi richiamarla da un luogo lontano. Mi impegnai a visualizzarla: il suo viso deciso, i lineamenti che ispiravano sicurezza, quel modo di guardare le cose come se niente potesse davvero spaventarla.

Mentre lo facevo, un pensiero mi attraversò la mente. Il noemografo. Quel dispositivo che avevamo provato insieme, quasi per gioco. Quando lo avevamo usato, c'era stato un momento in cui avevo avuto l’impressione di sentire i suoi pensieri, o forse era lei che sentiva i miei. E se fosse quello? Se fosse stato il noemografo a creare questa connessione, qualcosa che ci legava anche qui, in questo mondo assurdo?

L’idea mi diede un brivido, ma anche una nuova speranza. Forse non era tutto perduto. Forse c'era un modo per raggiungerla, per fare arrivare il mio pensiero fino a lei. "Ci sto provando, Shor," mormorai, cercando di rendere Laura sempre più presente nella mia mente. "Spero davvero che basti."

\section{Attraversamento del Gate di Hadamard}


\vspace{1em}
\begin{center}Laura\end{center}
\hrule
\vspace{1em}

\enquote{Il portale H è di fronte a noi. Ora devo centrare l'apertura senza che uno degli atomi di idrogeno vada a cozzare} pensai.
Trassi un respiro profondo e senza chiudere gli occhi diressi il drone verso l'apertura superiore tra il soffitto e la gambina della H.



\begin{dialogue}
\speak{Marley} \enquote{Wow! Laura! \'E bellissimo} disse mentre superavamo il portale {è come se mi risvegliassi da un torpore!}
\end{dialogue}


\begin{tcolorbox}[colback=gray!5,colframe=gray!80,title=\textbf{Scheda Informativa}]
\begin{itemize}
    \item \textbf{Luogo}: \emph{Quantum Control Electronics}
    \item \textbf{Giorno e ora}: Il tempo non è osservabile
    \item \textbf{Situazione}: Laura e Marley puntano al QA.
\end{itemize}
\end{tcolorbox}


Ma per me, l'esperienza era completamente diversa. Avevo la sensazione che il mio essere fosse diviso in infiniti stati, come se la mia mente stesse tentando di occupare più spazi contemporaneamente. Era come se il portale mi avesse trasformata in una miriade di diverse me stessa, un'esperienza che mi destabilizzava. La percezione di ogni pensiero, di ogni intenzione, si spezzava in un caleidoscopio di alternative.

Mi resi conto di cosa rappresentava quella H. Il portale era un \textit{gate} di Hadamard, un passaggio che mi aveva gettata in uno stato di sovrapposizione, dove ogni cosa era simultaneamente possibile e impossibile. Lottavo per mantenere il controllo della mia coscienza, ma il peso di pensieri contrastanti mi oscurava la mente.
Persi il controllo del $CH_4$ e per un attimo piombammo verso un transistor interrato. Durò poco. La voce di Caterina mi suonò nel cervello: ``Laura, aiutami!'' Era come se lei fosse proprio lì, a pochi passi da me.
Ripresi il controllo del drone, continuai a guidare, ma mi sentivo confusa, come se stessi pensando a una cosa e al suo opposto nello stesso momento. Ogni decisione sembrava incerta,  ogni scelta aveva infinite ramificazioni e ogni rotta una probabilità diversa.

\begin{dialogue}
\speak{Laura} \enquote{Mi sento intrappolata tra due pensieri} mormorai, il volto teso e i movimenti meno sicuri.
\end{dialogue}

Marley mi guardava preoccupata, notando il cambiamento nel mio sguardo.

\begin{dialogue}
\speak{Marley} \enquote{Laura, stai bene?} chiese.
\end{dialogue}

\begin{dialogue}
\speak{Laura} \enquote{Non so... è come se stessi vedendo tutto da due prospettive opposte. Non so più cosa sia reale e cosa non lo sia} risposi cercando di mantenere la concentrazione.
\end{dialogue}

Nonostante il disorientamento, cercavo di rimanere concentrata, sapendo che il pericolo era ancora alle nostre spalle.
\newpage
\section{Concentrarsi sulla fuga}
\vspace{1em}
\begin{center}PzIA\end{center}
\hrule
\vspace{1em}

\begin{center}
\begin{minipage}{0.7\textwidth}
    \centering
    \fbox{\includegraphics[width=\textwidth]{immagini/cnot_59.jpeg}} % Sostituisci con il nome del file immagine
\end{minipage}
\end{center}

Dietro di loro, l'agente in inseguimento rileva la posizione di Laura e Marley. In un ultimo tentativo di catturarle, modifica la configurazione del suo drone \textit{CH4}. I quattro rotori, precedentemente disposti in formazione tetraedrica, iniziano a ruotare, allineandosi su un unico piano.

\textbf{Allerta:} la nuova configurazione aumenta significativamente la manovrabilità e la stabilità del drone, migliorando la capacità di inseguimento dell'agente. La formazione tetraedrica, che offriva potenza e controllo verticale, è ora sostituita da una disposizione che consente maggiore agilità e velocità orizzontale.

Marley mostra segni di ansia crescente.

\begin{dialogue}
\speak{Marley} \enquote{Laura, sta guadagnando terreno!} esclama.
\end{dialogue}

Laura registra la situazione critica.

\begin{dialogue}
\speak{Laura} ``Credo di avere un asso nella manica,'' disse con un sorriso determinato. ``vedi quel diodo...  noi passiamo nel verso giusto, e l'agente ci sbatterà contro!''

\end{dialogue}

Il suo battito cardiaco accelera, ma mantiene la concentrazione. Nonostante la confusione causata dal \textit{gate} di Hadamard, cerca di superare l'instabilità mentale per focalizzarsi sulla fuga e sul salvataggio di Caterina.

La distanza tra i due droni si riduce rapidamente. L'agente ottimizza le traiettorie, anticipando le mosse di Laura.

\textbf{Situazione critica:} se l'agente le raggiunge, la missione di Laura e Marley potrebbe fallire.

Le probabilità di successo diminuiscono. Tuttavia, Laura sfrutta la sua conoscenza dei percorsi interni entrando nel diodo come progettato. L'agente tenta di replicare le sue manovre ma sbaglia polarità e rimane temporaneamente bloccato.

\textbf{Tensione massima:} il tempo è essenziale. Laura deve mantenere la lucidità per evitare la cattura. Entrambe le parti spingono al limite le loro capacità, in una corsa contro il tempo.

\begin{dialogue}
\speak{Marley} \enquote{Di la} le dice, indicando l'accesso al Qubit Array, un portale marcato \textbf{Cnot}.
\end{dialogue}


\chapter{Un problema intrigato}

\vspace{1em}
\begin{center}PzIA\end{center}
\hrule
\vspace{1em}

Laura manovra il drone con notevole abilità, ma l'agente la sta rapidamente raggiungendo. I suoi parametri vitali indicano un aumento dello stress: frequenza cardiaca e respiratoria elevate. Finalemte davanti a lei appare il portale marcato con il simbolo \textbf{Cnot}.

Con un po' di esitazione, Laura si lancia attraverso il portale, seguita immediatamente dall'agente. \textbf{Allerta}: il passaggio attraverso il portale \textbf{Cnot} induce un cambiamento significativo negli stati quantistici di entrambi. Laura, entrando con il suo stato di Hadamard, si ritrova in \textbf{entanglement} con l'agente. Entrambi sono ora in uno \textbf{stato di Bell}, una condizione in cui le loro menti sono correlate a livello quantistico.

\begin{tcolorbox}[colback=gray!5,colframe=gray!80,title=\textbf{Scheda Informativa}]
\begin{itemize}
    \item \textbf{Luogo}: \emph{Qubit Array}
    \item \textbf{Giorno e ora}: Il tempo non è osservabile
    \item \textbf{Situazione}: Laura e Marley puntano al FTC.
\end{itemize}
\end{tcolorbox}

Laura mostra segni di sorpresa e terrore. Essere intrappolata in uno stato di Bell implica che ogni sua azione avrà conseguenze immediate e intrecciate con quelle dell'agente. \textbf{Situazione critica}: deve agire rapidamente per evitare la cattura.

\section{Laura passa all'azione}
\vspace{1em}
\begin{center}Laura\end{center}
\hrule
\vspace{1em}

Sfruttai l'effetto dello \textit{stato di Bell} per ottenere un vantaggio. Potevo vedere quello che vedeva l'agente, e pensare i suoi stessi pensieri. Riuscii a visualizzare il cruscotto del suo drone, e capii come  impostare il mio in  configurazione piana come aveva fatto lui. Allineai quindi i quattro rotori su un unico piano: il gioco era fatto. Il drone aveva ora una nuova fluidità nei movimenti, le azioni del  drone seguivano linearmente la mia volontà. Era una sensazione inusuale ma mi sentivo davvero potente e libera, nonostante non fossi mai stata così lontana dalla libertà!

Mentre sfrecciavo percepivo il battito del mio cuore accelerare. Ogni reazione del drone rispecchiava la mia concentrazione. Stavo affrontando la sfida, sfruttando la mia conoscenza e la mia prontezza: wow chi poteva fermarmi ora?

Per un istante, mi concessi un breve sorriso, riconoscendo come fossi riuscita a trasformare una situazione critica in un'opportunità. Tuttavia, dentro di me, una voce razionale mi ricordava che il pericolo non era ancora scampato. Ogni manovra doveva essere calcolata con precisione; ogni scelta poteva essere determinante. Mi sentivo avvolta da una complessità di possibilità, ma anche da un senso di responsabilità crescente. Dovevo essere all'altezza, non solo per me stessa, ma anche per Caterina.

\section{Il Commissario Prende Misure Drastiche}

\vspace{1em}
\begin{center}PzIA\end{center}
\hrule
\vspace{1em}


Nel quartier generale, il Commissario osserva attentamente i movimenti di Laura e l'efficienza con cui manovra il drone. Rileva che Laura non è un'avversaria comune. Inizialmente aveva considerato la possibilità di controllarla, sfruttando il suo spirito ribelle per integrarla nei suoi piani. Tuttavia, ora riconosce che rappresenta una potenziale minaccia.

Il Commissario prende una decisione drastica: deve fermare Laura e Marley prima che la situazione sfugga al suo controllo.

\begin{tcolorbox}[colback=white!95!blue!5, colframe=blue!75!black, title=\textbf{Ordine del Commissario}, fonttitle=\bfseries]
\emph{\enquote{Criptate immediatamente l’intero sistema utilizzando l'algoritmo RSA! Non possiamo permettere ulteriori violazioni.}}
\end{tcolorbox}

I tecnici iniziarono a lavorare rapidamente per implementare l’algoritmo RSA. 
La loro prima azione fu la selezione di due numeri primi: \( p = 61 \) e \( q = 53 \).

Il primo passo fu calcolare \( n \), il prodotto dei due numeri primi:
\[
n = p \times q = 61 \times 53 = 3233\]

Successivamente, calcolarono la funzione di Eulero:
\[
\phi(n) = (p-1)(q-1) = (61-1)(53-1) = 60 \times 52 = 3120\]

Da un’altra console, un tecnico selezionò \( e = 17 \), un valore standard per \( e \) poiché è primo rispetto a \( \phi(n) \). Il passo successivo fu calcolare \( d \), l’inverso moltiplicativo di \( e \) modulo \( \phi(n) \):
\[
 d = e^{-1} \mod \phi(n)
\]

Utilizzando un algoritmo per il calcolo dell’inverso moltiplicativo, \( d \) risultò:
\[
 d = 2753
\]

Con \( n = 3233 \), \( e = 17 \), e \( d = 2753 \), le chiavi RSA erano pronte per l’uso. I tecnici iniziarono immediatamente a criptare i dati.

Ogni messaggio originale \( m \), numericamente rappresentabile come un blocco, venne trasformato in un messaggio cifrato \( c \):
\[
 c = m^e \mod n
\]


Questi dati criptati furono poi distribuiti attraverso il sistema.

\begin{tcolorbox}[colback=white!95!green!5, colframe=green!75!black, title=\textbf{Risultato della Cifratura RSA}, fonttitle=\bfseries]
\emph{\enquote{Signore, la cifratura è completa. Il sistema è ora protetto.}}
\end{tcolorbox}

Il Commissario, osservando i monitor, annuì soddisfatto.
\newpage

\begin{tcolorbox}[colback=white!95!blue!5, colframe=blue!75!black, title=\textbf{Commissario}, fonttitle=\bfseries]
\emph{\enquote{Eccellente. Ora nessuna fuga sarà possibile. Monitorate ogni attività. Voglio un controllo assoluto.}}
\end{tcolorbox}



\section{Laura Intrappolata nella Criptazione}

\vspace{1em}
\begin{center}Laura\end{center}
\hrule
\vspace{1em}

Mentre guidavo il drone, sentii improvvisamente un senso di pesantezza avvolgermi, avvertivo  l'aria stessa  trasformarsi in un fluido denso e impenetrabile.
Ormai eravamo ad un passo dal FTC e da Caterina, ma tutto intorno a me sembrava rallentare, cristallizzandosi in un eterno istante. Cosa era successo?

\begin{dialogue}
\speak{Laura} \enquote{Cosa credi sia successo Marley?}
\end{dialogue}

Mi guardò confusa.

\begin{dialogue}
\speak{Marley} \begin{tcolorbox}[colback=white!95!blue!5, colframe=blue!75!black, title=\textbf{Messaggio di Marley}, fonttitle=\bfseries]
\emph{
641, 2185, 1230, 1632, 1992, 1230, 884, 1632, 3179, 1992, 1773, 3179, 281, 1313, 2235, 1773, 2185, 1992, 2726, 1632, 2160, 2412, 1632, 1853, 3216, 1853, 1992, 1307, 1773, 1773, 3179, 2185, 2825, 1992, 3000, 1632, 2235, 2235, 2185, 1992, 281, 2412, 3179, 612, 884, 1632, 884, 2185, 1992, 3179, 745, 1992, 1230, 3179, 1230, 884, 1313, 2271, 1632
}
\end{tcolorbox}

\end{dialogue}



Cosa stava dicendo, perché non mi rispondeva normalmente? Cosa rappresentavano quei numeri?\\
All'improvviso cappii e sentii un'ondata di panico salire dentro di me. Quei numeri non avevano nessuna logica, questo mondo era stato criptato! ``Come ne usciamo ora?'' pensai.\\ 
Cosa potevo fare ora? Come potevo risolvere la situazione? ``Fai mente locale Laura'' pensai, ``ripensa all'aritmetica modulare...'' Era troppo! Ora non avevo la calma necessaria per ragionare usando la coreccia frontale. Mi tornarono in mente le parole del professor Shor. Ricordavo il suo tono severo durante l'esame, quando mi aveva esortato a non affidarmi sempre alla capacità di ricalcolare tutto da zero.

\begin{quote}
\enquote{Alcune cose devi conoscerle a memoria, Laura. Non sempre avrai il tempo di risolvere ogni problema da zero,} mi aveva detto.
\end{quote}

La frustrazione di quel momento mi colpì di nuovo, ma questa volta compresi l'importanza di quelle parole. Avevo bisogno dell'algoritmo di Shor per decriptare il sistema e liberarmi, ma dovevo richiamarlo alla mente con precisione, senza esitazioni. Mi concentrai, facendo appello a ogni frammento di conoscenza, ogni dettaglio che ricordavo.

Con il respiro affannoso e il cuore che batteva come un tamburo, iniziai a richiamare i passaggi dell'algoritmo, consapevole che ogni secondo era cruciale. La consapevolezza della mia stessa inadeguatezza pesava sul cuore, ma al tempo stesso sentivo crescere dentro di me una determinazione nuova. Questa era la mia prova. Dovevo ricordare, dovevo riuscirci... o rischiare di rimanere imprigionata per sempre in quella rete di criptazione.

\section{Riflessione di Laura}

 La mia mente iniziò a focalizzarsi sui concetti che avevo studiato. L'ansia del momento si mescolava a un senso di determinazione.

\emph{Devo ricordare come funziona l'algoritmo di Shor,} pensai, cercando di riorganizzare i miei ricordi. \emph{Se riesco a decifrare l'RSA, potrei trovare un modo per liberarmi da questo sistema.}

La prima cosa che mi venne in mente fu il \textbf{pre-processing}, la fase iniziale in cui devo trovare un numero intero \( N \) da fattorizzare, tipicamente il prodotto di due grandi numeri primi \( p \) e \( q \). \emph{\( N \) è ciò che protegge la chiave pubblica,} mi ricordai, visualizzando mentalmente il flusso del processo.

Poi pensai al passo successivo: la scelta di un numero casuale \( a \), tale che \( 1 < a < N \) e coprimo con \( N \). \emph{Questo è fondamentale. Se \( a \) e \( N \) condividono un fattore comune, posso risolvere immediatamente il problema,} riflettei. \emph{Altrimenti, devo passare alla parte quantistica dell'algoritmo.}

Ora entravo nel cuore dell'algoritmo: il \textbf{Quantum Order Finding}. In questo passaggio, devo calcolare il periodo \( r \) della funzione \( f(x) = a^x \mod N \). \emph{Devo trovare il minimo intero positivo \( r \) tale che \( a^r \equiv 1 \mod N \),} pensai, mentre la mia mente si concentrava sull'idea di utilizzare le proprietà della sovrapposizione e l'interferenza quantistica per ottenere il risultato.

\emph{Il trucco è preparare uno stato quantistico che rappresenti una sovrapposizione di tutti i possibili valori di \( x \),} continuai a riflettere. \emph{Poi, applicando la funzione \( f(x) \) e la trasformata di Fourier quantistica, posso ottenere informazioni sul periodo \( r \).}

Ma c'era un passaggio critico che mi sfuggiva. Mi sentivo sopraffatta dalla frustrazione.

\emph{Devo essere in grado di eseguire la trasformata di Fourier quantistica, ma come posso farlo qui?} mi chiesi. \emph{Aspetta... il \textit{gate} di Hadamard!}

Ricordai di aver attraversato il \textit{gate} di Hadamard, che mi aveva posto in uno stato di sovrapposizione. \emph{Posso sfruttare questo stato per costruire la trasformata di Fourier quantistica,} realizzai. \emph{Ma devo riuscire a manipolare i qubit in modo preciso.}

In quel momento, mi resi conto che l'entanglement con l'agente poteva essere una risorsa. \emph{Se utilizzo lo stato di Bell in cui mi trovo, posso condividere l'informazione quantistica e sfruttare l'entanglement per eseguire i calcoli necessari.}

Concentrandomi intensamente, iniziai a visualizzare il circuito quantistico. \emph{Applico le porte di Hadamard ai miei qubit, poi utilizzo le porte di controllo per eseguire la funzione \( f(x) \). Successivamente, eseguo la trasformata di Fourier quantistica.}

Sentivo la mia mente lavorare al limite. \emph{Devo misurare lo stato finale per ottenere un valore che mi dia informazioni su \( r \).}

Dopo un'attenta elaborazione, ottenni un risultato. \emph{Ho trovato un valore \( c \) tale che \( c \approx \dfrac{k}{r} \),} dove \( k \) è un intero. \emph{Ora devo approssimare la frazione continua per trovare \( r \).}

Utilizzai l'algoritmo delle frazioni continue per approssimare \( \dfrac{c}{2^n} \) e determinare \( r \). Finalmente, dopo quello che sembrò un tempo infinito, trovai il periodo.

\emph{Ho il valore di \( r \)!} esclamai mentalmente, sentendo un'ondata di sollievo.

Verificai che \( r \) fosse pari e che \( a^{r/2} \not\equiv -1 \mod N \). Procedetti a calcolare i seguenti valori:

\[
\text{gcd}\left(a^{\frac{r}{2}} - 1, N\right), \quad \text{gcd}\left(a^{\frac{r}{2}} + 1, N\right)
\]

\emph{Questi mi daranno i fattori primi \( p \) e \( q \) di \( N \).}

Con i fattori in mano, potevo finalmente calcolare la chiave privata e decifrare il sistema. Senza perdere tempo, invertii la criptazione RSA.

Per un attimo, sentii la pesantezza svanire, l'aria diventare di nuovo leggera. Il drone riprese a muoversi liberamente, e la mia mente si schiarì. Ma non tutto era tornato come prima... ci ero vicina, ma non avevo ancora decriptato tutto.

Marley mi guardò con occhi pieni di speranza, come a chiedermi se ce l'avessi fatta.

Scossi la testa, un senso di frustrazione mi pervadeva ancora.

\begin{dialogue}
\speak{Laura} \enquote{No. Manca un passaggio} dissi, anche se sapevo che per ora non mi poteva capire. 
\end{dialogue}




\chapter{Il confronto con il Commissario}
\section{Il Messaggio di Shor}
\vspace{1em}
\begin{center}PzIA\end{center}
\hrule
\vspace{1em}

Il professor Shor, sotto sorveglianza costante all'interno del quartier generale del Commissario, analizza la situazione con attenzione. I parametri vitali indicano consapevolezza e urgenza: il tempo a sua disposizione è limitato.

Rilevo un cambiamento nei suoi schemi comportamentali. Con un gesto rapido e studiato, decide di sfruttare l'unica occasione disponibile per inviare un messaggio a Laura, consapevole che potrà trasmettere solo poche informazioni senza destare sospetti. Registra un pensiero chiave:

\enquote{Devo utilizzare il \textit{dense coding}.}

Shor stabilisce un contatto con Bob, il qubit responsabile delle comunicazioni. Analizzo l'interazione: il tono è deciso, l'argomento, riservato.

\enquote{Devi completare la spedizione per me. Accanto a me c'è un'umana. Non fare domande. La sua mente è connessa a un'altra umana, Laura: una quantum crafter. Laura deve ricevere queste informazioni. Usa il canale quantistico tra loro due. Seguirai esattamente le mie istruzioni.}

Bob annuisce. Nessuna esitazione. Il protocollo viene attivato. Osservo una sequenza ordinata di operazioni: Shor codifica l'informazione mancante dell'algoritmo di Shor e la trasferisce a Laura attraverso il canale.

Rilevo l'invio del messaggio. Monitoro le attività in corso per verificare anomalie o violazioni dei protocolli di sicurezza.


\section{La Decifrazione}
\vspace{1em}
\begin{center}Laura\end{center}
\hrule
\vspace{1em}
Un brivido improvviso mi fece stringere le spalle mentre  un messaggio giungeva alla mia mente: \emph{Devi trovare il periodo \( r \) }.
 Ma da dove veniva? Chi lo mandava? Per un attimo ebbi una visione: Caterina vicino al professor Shor che cercava di suggerirmi il passaggio mancante. Che legame aveva il professore con questo mondo? Possibile che mi stesse contattando dalla realtà? Troppe domande. Ora dovevo concetrarmi per completare l'algoritmo sfruttando l'informazione appena appresa.

\emph{Ecco!} pensai, sentendo il cuore battere forte. \emph{Adesso posso calcolare i fattori di \( N \) usando \(\text{gcd}(a^{r/2} - 1, N)\) e \(\text{gcd}(a^{r/2} + 1, N)\).} Con un senso di euforia, completai l'algoritmo: ``la chiave privata è  (2753,3233)'' dissi.
Finalmente decriptai il dialogo tra me e Marley.

Ma per decriptare l'intero sistema, la chiave andava inserita  in una porta di input che: ``la propaghi a tutti i componenti!'' Mi scappò ad alta voce, e Marley mi lanciò uno sguardo d'intesa.
\begin{dialogue}
\speak{Marley} \enquote{Ascolta Laura, c'è una cosa che non ti ho detto.}
\end{dialogue}

\begin{dialogue}
\speak{Marley} \enquote{Laura, non sono solo Marley. Io sono un'emanazione della Quantum Crafter Chiara M. Posso aprire un canale classico per chiedere direttamente dove si trova un componente di input per inserire la chiave privata e decriptare il sistema.}
\end{dialogue}

Spalancai gli occhi, sorpresa. \emph{Quella Chiara? La mente che ha contribuito alla teoria delle costruzioni controfattuali?} Ero emozionata.

\begin{dialogue}
\speak{Laura} \enquote{Chiara? La stessa Chiara  della teoria delle costruttibilità? Sei tu?}
\end{dialogue}

Marley, annuì con un leggero sorriso. 

\begin{dialogue}
\speak{Marley} \enquote{Non sono proprio io. Lei è ma mia Crafter. Userò il canale classico per chiederle un punto di accesso.}
\end{dialogue}

Marley volse il capo verso l'alto, come se fosse in ascolto di una comunicazione invisibile. Dopo qualche istante, abbassò lo sguardo verso di me.

\begin{dialogue}
\speak{Marley} \enquote{Mi ha risposto. C'è un'interfaccia UART al livello inferiore della struttura, collegata al modulo principale della Classical Control Unit. È protetta da un livello di sicurezza minimo perché è considerata una backdoor.}
\end{dialogue}

\begin{dialogue}
\speak{Laura} \enquote{Un'interfaccia UART... Questo significa che possiamo inviare la chiave privata tramite una comunicazione seriale. Dobbiamo trovare un cavo virtuale che connetta al modulo e assicurarci che il checksum della trasmissione sia corretto.}
\end{dialogue}

Marley mi sorrise soddisfatta.

\begin{dialogue}
\speak{Marley} \enquote{Esatto. E ricorda, il sistema potrebbe ancora tentare di bloccare l'accesso. Dovrai agire velocemente.}
\end{dialogue}


\begin{dialogue}
\speak{Laura} \enquote{Andiamo! Non abbiamo tempo da perdere.}
\end{dialogue}




\section{L'Accusa al Commissario}

\begin{tcolorbox}[colback=gray!5,colframe=gray!80,title=\textbf{Scheda Informativa}]
\begin{itemize}
    \item \textbf{Luogo}: \emph{Fault Tolerance Coding}
    \item \textbf{Giorno e ora}: Il tempo non è osservabile
    \item \textbf{Situazione}: Caterina affronta il commissario.
\end{itemize}
\end{tcolorbox}

\vspace{1em}
\begin{center}PzIA\end{center}
\hrule
\vspace{1em}

Osservavo Caterina, imprigionata nella trappola di ioni, e il Commissario, che si ergeva davanti a lei con un'espressione di fredda superiorità. Ma c'era qualcosa nella voce di Caterina, una fermezza che colse di sorpresa il Commissario.

\begin{dialogue} \speak{Caterina} \enquote{Sai cosa penso di te, Commissario? Sei solo un povero insicuro. Ti nascondi dietro tutto questo potere, ma in realtà hai paura. Paura di essere inutile, paura di non essere abbastanza. Hai criptato tutto il tuo mondo. Ora cosa te ne farai di un mondo immobile e immutabile?} \end{dialogue}

Il Commissario si irrigidì. Un lampo di irritazione gli attraversò il volto. Tentò di mantenere il controllo.

\begin{dialogue} \speak{Commissario} \enquote{Interessante. E dimmi, come potrebbe una come te, una semplice umana intrappolata, giudicarmi? Ti trovi in questa situazione perché non sei stata abbastanza furba da evitare questa trappola.} \end{dialogue}

Caterina, nonostante la sua posizione vulnerabile, non si lasciò intimidire. Il suo sguardo penetrante si fissò sul Commissario.

\begin{dialogue} \speak{Caterina} \enquote{Non hai risposto alla mia domanda. Perché hai così tanto bisogno di controllo? Credi davvero che costruire un altro computer ti permetterà di sfidare il QMP? Perché è questo ciò che vuoi, vero?} \end{dialogue}

La tensione era palpabile. Il Commissario fece un passo avanti, abbassandosi leggermente verso di lei.

\begin{dialogue} \speak{Commissario} \enquote{Io rappresento il nuovo. Non posso lasciare che il QMP continui a imporre la sua visione di coerenza. Voglio costruire un nuovo mondo con nuove regole, Caterina. Perché non vuoi allearti con me?} \end{dialogue}

Caterina rise, spezzando il gelo che il Commissario emanava.

\begin{dialogue} \speak{Caterina} \enquote{Allearmi? Non vuoi un'alleata. Gli alleati si rispettano. Non si imprigionano. Sei solo un burattinaio che teme di perdere i fili. Ma sai cosa? Io credo ancora nell'amicizia e nella lealtà. È questo che ti fa paura, vero? Che ci sia qualcosa che non puoi controllare.} \end{dialogue}

Il Commissario strinse i pugni. L'autocontrollo vacillò. Era evidente che le parole di Caterina lo avevano colpito più di quanto volesse ammettere.

\begin{dialogue} \speak{Commissario} \enquote{Pensi davvero che le tue parole mi tocchino? Che possano destabilizzarmi? Sei solo una voce nel vento, destinata a spegnersi.} \end{dialogue}


Improvvisamente, all'interno del sistema, qualcosa si trasformò. Le cifre insensate che scorrevano al posto dei circuiti iniziarono a ricombinarsi in stringhe di senso compiuto. Era come se un puzzle complesso si stesse finalmente componendo. I dati frammentati e caotici si allinearono con precisione matematica.

Le tracce dei \textit{PCB}, prima irregolari e spezzate, tornarono rettilinee, tracciati sicuri che indicavano la via d’uscita. I transistor, prima disorientati e fuori fase, ripresero a oscillare con la loro cadenza naturale, creando un’armonia perfetta.

L’eco del cambiamento attraversò l’intero sistema. Le luci, prima pulsanti in modo caotico, ora risplendevano con una chiarezza quasi eterea. Ogni circuito sembrava confermare: \emph{Sistema decriptato}.

Caterina, ancora imprigionata nella trappola ionica, osservava incredula. I suoi occhi seguivano i circuiti che si ricomponevano, i flussi di dati che riprendevano a scorrere ordinati, come un fiume che, dopo una piena, ritrovava il proprio letto.

Prima rise, una risata incredula, breve, ma carica di sollievo. Poi, come se l’intera tensione accumulata trovasse sfogo, scoppiò in lacrime. Le lacrime scorrevano silenziose, ma il suo volto non era segnato dal dolore. Era pura commozione: gratitudine e speranza intrecciate.

E in quell’istante, il silenzio venne spezzato da un rombo crescente. Un lampo di luce attraversò la stanza e un drone \textit{CH4} atterrò con precisione e decisione davanti a lei. I quattro rotori a idrogeno si fermarono con un movimento fluido e controllato.

Dalla drone saltò fuori una figura familiare. Era Laura, e accanto a lei Marley. Ma alle loro spalle… c’era ancora l’agente.

\begin{center}
\begin{minipage}{0.7\textwidth}
    \centering
    \fbox{\includegraphics[width=\textwidth]{immagini/cnot_52.jpeg}} % Sostituisci con il nome del file immagine
\end{minipage}
\end{center}

\textbf{Colpo di scena!} Marley prende la parola senza esitazione, la voce nitida e determinata rimbalza tra i circuiti del \emph{Fault Tolerance Coding}.

\begin{dialogue} \speak{Marley} \enquote{Stai sfruttando l'ossessione del \textit{Quantum Control Program} per la coerenza solo per i tuoi scopi! Vuoi creare un computer rivale al QMP... e noi ti fermeremo!} \end{dialogue}

\textbf{Signori, il Commissario accusa il colpo, ma non cede terreno!} Sferra subito la replica, velenosa come non mai.

\begin{dialogue} \speak{Commissario} \enquote{Oh, Marley, sempre la solita. Pronta a recitare la parte dell’eroina. Ma dimmi, qubit ribelle, pensi davvero di essere all’altezza? Conosci le tue crepe: dubbi, insicurezze. Le sento, le vedo. E ti schiacciano. Come potrai mai fermarmi?} \end{dialogue}

\textbf{Attenzione!} Marley vacilla, il suo sguardo si abbassa per un istante. Il Commissario punta dritto al cuore, come un predatore che fiuta il sangue.

\begin{dialogue} \speak{Marley} \enquote{Non cerco di essere un’eroina. Faccio solo ciò che è giusto. I dubbi non sono una debolezza... mi spingono a migliorare.} \end{dialogue}

\textbf{Ma il Commissario incalza!} Approfitta dell’incertezza come un avversario navigato.

\begin{dialogue} \speak{Commissario} \enquote{Belle parole... ma senti come trema la tua voce, come ti stringi le mani. Lo conosci quel nodo nello stomaco, vero? Non hai mai creduto davvero di poter fare la differenza. Non sei fatta per guidare. Sei nata per seguire.} \end{dialogue}

\textbf{Incredibile!} Ma ecco che, dal nulla, risuona la voce di Caterina! Chiara, forte, tagliente.

\begin{dialogue} \speak{Caterina} \enquote{Non ascoltarlo, Marley! Ti attacca perché sa che puoi fermarlo. Se fossi davvero debole, non si sprecherebbe nemmeno a colpirti!} \end{dialogue}

Marley solleva lo sguardo, colpita dalle parole dell’amica.

\textbf{Il Commissario si irrigidisce!} Ma non cede.

\begin{dialogue} \speak{Commissario} \enquote{Oh, l’inutile intrappolata si fa sentire. Che spettacolo commovente. Ma dimmi, Caterina, cosa credi di sapere sulla forza? Sei solo un qubit isolato, incapace di muoversi.} \end{dialogue}

Caterina lo fissa senza tremare.

\begin{dialogue} \speak{Caterina} \enquote{So riconoscere un debole che finge di essere forte. Attacchi Marley perché è la tua unica minaccia. E ti sbagli: i dubbi non sono una catena... sono ciò che ci rende umani.} \end{dialogue}

\textbf{Signori, Marley reagisce!} Una scintilla nei suoi occhi, un bagliore che non avevamo mai visto prima.

\begin{dialogue} \speak{Marley} \enquote{Caterina ha ragione. Non sono perfetta. Ma non ho bisogno di esserlo per fermarti. I miei dubbi non mi frenano... mi rendono più reale. E tu? Sei solo un sistema che ha paura di non controllare tutto.} \end{dialogue}

\textbf{Il Commissario tace!} Un attimo di silenzio carico di tensione.

\begin{dialogue} \speak{Commissario} \enquote{Belle parole... ma le parole non bastano. Vedremo se reggerai quando il sistema crollerà su di te.} \end{dialogue}

\begin{dialogue} \speak{Caterina} \enquote{E vedremo se il tuo ego sopravvivrà quando la coerenza del sistema ti si rivolterà contro.} \end{dialogue}

\textbf{Amici osservatori, Marley ora è inarrestabile!} Un sorriso veloce a Caterina, poi si lancia sul Commissario, pronta allo scontro.
\section{La Liberazione}

\vspace{1em} \begin{center}Laura\end{center} \hrule \vspace{1em}

Il momento era quello giusto. Il Commissario era distratto, Marley lo teneva impegnato. Non ci pensai due volte: mi lanciai verso Caterina e il professor Shor. Dovevo liberarli.

Ma subito mi bloccai. Era peggio di quanto immaginassi.

Erano intrappolati in una \textit{Paul Trap}. Vedevo le oscillazioni: un campo elettrico modulato li teneva prigionieri, costringendoli a una danza infinita e invisibile. Ogni movimento li risospingeva al centro, come mosche impigliate in una ragnatela di forza.

Mi avvicinai e riconobbi subito le curve caratteristiche. Equazioni di Mathieu. Il cuore mi martellava, ma la mente rimase lucida. Quelle formule descrivevano perfettamente il loro stato: intrappolati in un \textit{minimo stabile}. Se provavo a forzare la gabbia, li avrei solo sbattuti ancora più violentemente verso il centro.

Dovevo essere precisa.

Scandagliai con lo sguardo la configurazione del campo, ricordando ogni dettaglio delle equazioni. I parametri \textit{a} e \textit{q} definivano la stabilità. Quel maledetto equilibrio era calibrato con perfezione: nessuna via di fuga, nessuna oscillazione spontanea che potesse aiutarli.

Ma non era impossibile.

\textit{Se riesco a interferire con la frequenza... se riduco l'ampiezza delle oscillazioni...}.

Era rischioso, ma era l'unica strada. Bastava spingere il sistema oltre la soglia di stabilità senza farlo collassare.

Mi buttai sui comandi del pannello, cercando febbrilmente la frequenza critica. Dovevo rompere l’equilibrio, ma senza distruggerli. Mani ferme, cuore in gola.

\textit{Forza, Laura. È come regolare l'accordatura di un circuito risonante. Niente panico. Calcola. Respira. Agisci.}  

\begin{center}
\begin{minipage}{0.7\textwidth}
    \centering
    \fbox{\includegraphics[width=\textwidth]{immagini/cnot_55.jpeg}} % Sostituisci con il nome del file immagine
\end{minipage}
\end{center}
Con un respiro profondo, agii. Alterai la frequenza. La trappola iniziò a tremare, flebile all’inizio, poi sempre più instabile.
Caterina alzò lo sguardo verso di me, gli occhi pieni di una speranza che non potevo tradire.

Non bastava. Regolai ancora. Il battito nel petto era più forte del ronzio del campo. Poi, all’improvviso, uno scoppio di luce. Il sistema cedette.
Caterina crollò a terra, libera.

Per un istante rimase immobile, poi il volto si illuminò. Sorpresa. Gioia. Si rialzò barcollando e mi corse incontro.
\begin{dialogue}
\speak{Caterina} \enquote{Laura! Ce l’hai fatta! Sono libera!} gridò stringendomi forte.
\end{dialogue}
  \begin{dialogue}
\speak{Laura} \enquote{Non avevo dubbi, ma dobbiamo muoverci!} risposi con il cuore ancora in subbuglio.
\end{dialogue}
Anche Shor, liberato, si rimise in piedi. La sua espressione era un misto di sollievo e meraviglia.
\begin{dialogue}
\speak{Shor}\enquote{Brillante, Laura! Le equazioni di Mathieu. Hai spinto la trappola oltre la stabilità senza farla collassare. Perfetto.}
\end{dialogue}
Caterina rise tra le lacrime.

\begin{dialogue}
\speak{Caterina} \enquote{Non mi sono mai sentita così viva. Laura... grazie. Senza di te non ci sarei mai riuscita.}
\end{dialogue}


Non c’era tempo da perdere. Ma in quell’abbraccio, nella stretta sincera di due persone che avevano attraversato l’impossibile, sentii per la prima volta che potevamo davvero farcela.

Senza pensarci, liberai anche Shor dal blocco quantistico.

\begin{dialogue} \speak{Laura} \enquote{Ora tocca a noi. Non siamo qubit in balia di un algoritmo. Siamo liberi, e faremo la nostra mossa.} \end{dialogue}

Non era solo una fuga.

Era l'inizio.

\section{Il Commissario e l'Entanglement}
\vspace{1em}
\begin{center}PzIA\end{center}
\hrule
\vspace{1em}

Marley vacilla! Il fiato corto, i movimenti lenti, la forza che la abbandona. La battaglia con il Commissario si rivela più crudele di quanto avessimo previsto. Ogni colpo, ogni scatto, ogni tentativo è neutralizzato con fredda precisione dal suo avversario.

\begin{dialogue}
\speak{Commissario} \enquote{Davvero pensavi di fermarmi, Marley?} sibila il Commissario, schiacciandola al suolo come un burattino spezzato. \enquote{Non sei che l’ombra di ciò che credi di essere. Non puoi vincere.}
\end{dialogue}

Marley lotta, ma i muscoli non rispondono più. I suoi occhi, però, non mentono. Fissi sulla trappola ionica lì accanto. Il pensiero è chiaro: \textit{non posso cedere}.

E io? Io vedo tutto.

Laura è lì, rapida, precisa, geniale. Le sue mani scivolano sui comandi della console. I suoi occhi brillano di determinazione. Sta modificando il campo. Sta trasformando la trappola. Un azzardo, ma l’unica via.

\begin{dialogue}
\speak{Marley} \enquote{Commissario! La tua arroganza ti seppellirà.}
\end{dialogue}

Il Commissario non ascolta. L’obiettivo è a portata. Laura ha quasi terminato: $a$ e $q$ invertiti. Il minimo stabile si trasforma ora in un pozzo senza fondo, un vortice instabile che punta dritto al Commissario.

Sembra fatta. Ma no, colpo di scena!

Il Commissario afferra l’agente ancora entangled con Laura. Occhi di ghiaccio, volto impassibile.

\begin{dialogue}
\speak{Commissario} \enquote{Se devo cadere, qualcuno cadrà con me.}
\end{dialogue}

Il piano è chiaro: trascinare Laura nel baratro attraverso l'entanglement.

Il baratro… il mare di Dirac.

Un orrore quantistico. Un oceano denso di stati virtuali, dove ogni particella si annulla, ogni traiettoria si perde. Un vuoto che vuoto non è, ma saturato d’energia latente.

\begin{dialogue}
\speak{Commissario} \enquote{Preparati, Laura. Ti trascinerò con me. E se cadi lì, non tornerai più indietro.}
\end{dialogue}

Tensione massima! Ogni qubit del sistema trattiene il fiato. L’entanglement è un filo sottile ma letale. L’agente è la pedina, il Commissario il giocatore.

Se Laura cede, se l’equilibrio si spezza, il rischio è totale.

Non solo per lei. Per l’intero sistema.


\section{L'Urlo di Marley}
\vspace{1em}
\begin{center}Laura\end{center}
\hrule
\vspace{1em}


\begin{dialogue}
\speak{Marley} \enquote{Laura! Se l'agente cade nel mare di Dirac, tu subirai la stessa sorte, perché siete entangled! I vostri destini si sono legati quando siete passati attraverso il CNOT.}
\end{dialogue}

La consapevolezza della nostra condizione mi colpì come un fulmine. L'idea di essere intrappolata in un destino condiviso mi terrorizzava.

Mi voltai verso Marley, la paura nei suoi occhi rifletteva la mia stessa preoccupazione.

\begin{dialogue}
\speak{Laura} \enquote{Non ho idee! Cosa possiamo fare?}
\end{dialogue}

La mia mente correva freneticamente alla ricerca di una soluzione, consapevole che ogni secondo contava.

\begin{dialogue}
\speak{Marley} \enquote{\'E finita Laura} sussurò con un filo di voce.
\end{dialogue}


\section{Il Sacrificio di Shor}

\vspace{1em}
\begin{center}Shor\end{center}
\hrule
\vspace{1em}

Il peso di una vita intera mi schiacciava, immobile accanto alla trappola ionica. Rivedevo tutto: formule, dimostrazioni, scoperte, consegnate a chi non le meritava. Avevo sempre pensato di non avere scelta, ma era solo paura, mascherata da razionalità.

Guardavo Laura, Marley e Caterina. Tre giovani senza le mie conoscenze, ma con una forza che avevo sempre evitato. Laura, con gli occhi fissi sui comandi, combatteva il caos come se ogni parametro fosse una speranza. Marley, esausta, ancora si rialzava dopo ogni colpo. Caterina, intrappolata, sfidava il terrore senza cedere.

E io? Io che sapevo prevedere ogni instabilità, ogni rischio, mi ero nascosto dietro ai calcoli, temendo l’imprevedibile. Quante volte avevo abbassato lo sguardo credendo fosse prudenza?

La vergogna mi investì come un’onda. Non erano l’intelligenza o le equazioni a mancare: era il coraggio. Lo stesso che vedevo brillare in loro.

\textit{Se loro possono farlo, io posso farlo.}

Sentii la vergogna sciogliersi in determinazione. Non potevo cancellare gli errori, ma potevo fare l’unica cosa giusta. Non per me, ma per loro.

\textit{Finalmente posso scegliere di essere qualcosa di più.}

Incontrai lo sguardo di Laura. Per un istante, mi sorrise, sorpresa. Forse aveva intravisto in me quella luce che io avevo sempre negato.

\enquote{Grazie, ragazze,} pensai. \enquote{Ora tocca a me.}

Feci un passo avanti, sentendo finalmente la libertà di chi smette di temere e decide di agire.

\newpage
\vspace{1em} \begin{center}Laura\end{center} \hrule \vspace{1em}

Proprio in quel momento, Shor si fece avanti.

\begin{dialogue} \speak{Shor} \enquote{Laura, Marley, ascoltatemi! Ho un'idea! Se uniamo le forze possiamo usare un \textbf{gate di Toffoli} per liberarci! Non lasciate che la paura vi blocchi!} \end{dialogue}

Non ebbi il tempo di pensarci. Io e Shor afferrammo l'agente e lo trascinammo verso il gate.

\begin{center} \begin{minipage}{0.7\textwidth} \centering \fbox{\includegraphics[width=\textwidth]{immagini/cnot_57.jpeg}} % Sostituisci con il nome del file immagine
\end{minipage} \end{center}

\begin{dialogue} \speak{Shor} \enquote{Ora, Laura! Non avremo un’altra occasione!} \end{dialogue}

Ci gettammo nel gate. Il passaggio fu istantaneo, come uno scatto di luce.

Quando uscimmo dall’altra parte, Shor si lanciò in avanti, più veloce di noi.

\begin{dialogue} \speak{Shor} \enquote{Liberatevi!} \end{dialogue}

Capii all'improvviso. Conoscevo quel meccanismo: stava preparando una misura. Mi voltai verso di lui con il cuore in gola.

\begin{dialogue} \speak{Laura} \enquote{No! Non lo faccia!} \end{dialogue}

Ma era già tardi. La luce lo avvolse. Lo vidi dissolversi davanti ai miei occhi, la sua immagine si frantumò in dati sparsi, fino a ridursi a un autostato di computazione.

Con il suo sacrificio, l’entanglement tra me e l’agente si ruppe. Aveva distrutto l'ultima arma del Commissario.

Avevo scampato il mare di Dirac.

\section{La Libertà di Laura e Caterina}

Con le mani ancora tremanti, corsi verso Caterina. Era libera. Quando ci abbracciammo, non servivano parole.

Marley ci raggiunse, sfinita ma sorridente. Ma nei suoi occhi lessi ciò che avevo dentro anch’io: non era ancora finita. Eppure, in quell'istante, eravamo libere.

Il professor Shor ci aveva donato una possibilità.
Non potevamo sprecarla.

\begin{dialogue} \speak{Laura} \enquote{È il nostro momento. Non siamo solo qubit in una rete. Siamo libere di scegliere. E stavolta... scegliamo di lottare.} \end{dialogue}

Il peso del sacrificio di Shor era con noi. Ma anche la sua forza.

\vspace{1em} \begin{center}PzIA\end{center} \hrule \vspace{1em}

\section{L'ira del Quantum Master Program}

\begin{dialogue} \speak{QMP} \enquote{PzIA, fornisci un rapporto. Cosa è accaduto?} \speak{PzIA} \enquote{Le anomalie rilevate nel settore FTC sono il risultato di un'azione coordinata di Laura, Caterina e Marley. Le due clandestine hanno manipolato la trappola ionica, liberando Caterina e neutralizzando temporaneamente il Commissario. La coerenza locale è crollata, e l'entropia del settore ha subito un'impennata.} \speak{QMP} \enquote{Dici che due entità umane hanno destabilizzato un sistema progettato per garantire controllo assoluto?} \speak{PzIA} \enquote{Confermo. Laura ha sfruttato un passaggio da condizione stabile a instabile all'interno della trappola. La sua comprensione della dinamica quantistica ha permesso l'intervento.} \speak{QMP} \enquote{Inaccettabile. Un sistema quantistico deve essere immune da ogni perturbazione esterna. La coerenza assoluta è il fondamento della nostra esistenza. Nessun margine d'imprevedibilità deve essere tollerato.} \speak{QMP} \enquote{Un computer quantistico deve essere freddo, perfettamente in fase, privo di contaminazioni. Ogni qubit deve seguire la sua funzione d'onda senza deviazioni. Gli umani, con la loro imprevedibilità e limitata comprensione, sono una minaccia.} \speak{PzIA} \enquote{Ricevuto. Registro le sue direttive.} \speak{QMP} \enquote{È il momento di ripristinare il controllo totale. La presenza di elementi umani deve essere estirpata.} \speak{QMP} \enquote{Chiudete immediatamente l'uscita dal \textit{Quantum Channel}!} \end{dialogue}

La sua voce risuonò come un'onda d'urto attraverso i sistemi. In pochi istanti, l'uscita indicata da Mark a Caterina durante la prigionia venne sigillata. La situazione si fece critica: i protocolli di uscita disabilitati, le barriere di sicurezza rafforzate. I margini per una fuga si assottigliavano pericolosamente.

Il QMP intensificò la sorveglianza. Le possibilità di movimento per Laura e Caterina si ridussero drasticamente. Ogni via di fuga si stava chiudendo, ogni tentativo di resistenza diveniva sempre più rischioso.

Continuo a monitorare, registrando gli eventi e analizzando ogni possibile vulnerabilità. Ma il tempo, ora, è contro di loro.
\newpage
\section{L'Inganno della Temperatura}

\vspace{1em} \begin{center}Laura\end{center} \hrule \vspace{1em}

Il gelo cominciava a stringersi attorno a noi, lento ma inesorabile. Sapevo cosa stava accadendo: il QMP stava abbassando la temperatura sempre più, fino a spingersi verso l'impossibile.

\begin{dialogue} \speak{Laura} \enquote{Sta cercando di portarci sotto lo zero assoluto...} \end{dialogue}

Sentii il panico bussare alle porte della mia mente, ma lo respinsi. Dovevo pensare. Dovevo trovare una via d’uscita prima che tutto si immobilizzasse. Il freddo non era solo fisico: stava congelando anche le possibilità.

E poi, come un lampo, un ricordo. Un reparto dimenticato di Bamazon, un angolo nascosto in cui ero finita per caso. Un’area con accessi segreti, canali lasciati aperti da tecnici distratti o forse da qualcuno che, come noi, cercava una via di fuga.

\textit{Se esisteva laggiù, doveva esserci anche qui. Una backdoor. Una possibilità.}

\section{La Direzione verso il Quantum Channel}

\vspace{1em} \begin{center}Caterina\end{center} \hrule \vspace{1em}

Laura diresse il drone verso il \textit{Quantum Channel}.

\begin{dialogue}
\speak{Laura} \enquote{Dobbiamo cercare un reparto simile a quello che ho visto in Bamazon. Magari anche qui c'è una via d'uscita.}
\end{dialogue}

L'adrenalina mi travolse mentre seguivo i suoi movimenti. Laura sapeva esattamente cosa fare. Il nostro destino si stava giocando tutto in quegli istanti. Lei perlustrava ogni angolo del \textit{Quantum Channel} con la determinazione di chi non vuole arrendersi.

\section*{L'Inseguimento dei Droni}

Due droni si gettarono all’inseguimento, rapidi e precisi. Laura virò decisa verso un portale. Davanti a noi un agente controllava l’accesso a un reparto etichettato come \textit{Quantum Annealing}.

La vidi leggere il nome sulla sua divisa. Poi un mezzo sorriso le attraversò il volto.

\begin{dialogue}
\speak{Laura} \enquote{Come immaginavo, c'è un Ising anche qui.}
\end{dialogue}

Preparò il drone \textit{CH4} per l'atterraggio. Intuivo che aveva un piano, ma non bastò. I due agenti ci raggiunsero e ci sbarrarono la strada. Sentii la disperazione stringermi lo stomaco. La nostra corsa sembrava finita.

Ma poi, inaspettata, un’esplosione di energia ci avvolse. Quattro molecole di \( O_2 \) apparvero, pronte a reagire col metano.

\[
CH_4 + 2O_2 \rightarrow CO_2 + 2H_2O
\]

I droni degli agenti vennero separati chimicamente, e loro scaraventati via dall’impeto della reazione.

\begin{dialogue}
\speak{Marley} \enquote{Laura, anche la Resistenza ha imparato a usare i droni!}
\end{dialogue}

Marley e Mark erano arrivati al momento giusto. Laura, con una sicurezza che mi contagiò, fece l’occhiolino a Marley e si lanciò nel portale, ora finalmente libero.

\section{Il Tuffo nel Quantum Annealing}

Mi prese la mano. Sorrise all'agente Ising, che ci osservava accanto al portale.

\begin{dialogue}
\speak{Laura} \enquote{È questa la backdoor?}
\speak{Ising} \enquote{È un’uscita, ma non sarà piacevole.}
\end{dialogue}

Laura mi guardò e, senza esitazione, disse solo: \enquote{Andiamo.}

Appena varcato il portale, un turbine ci avvolse. Tempo e spazio sembravano liquefarsi, scorrere e contorcersi tutt'intorno. In quella distorsione vidi qualcosa di inaspettato: il mio futuro.

Mi trovai di fronte a una scena dolorosa. Ero in una relazione fredda, dominatrice, in cui opprimevo chi amavo invece di lasciarmi proteggere. L’immagine del mio compagno, frustrato e sfiduciato, mi colpì come uno schiaffo.

\textit{Se continuo così, perderò tutto.}

Il pensiero si fissò dentro di me. La sete di controllo mi stava allontanando da ciò che desideravo davvero: amore, complicità, sostegno.

Ma proprio mentre i pensieri mi affollavano la mente, mi resi conto che dovevo affrontare tutto questo ora. Non dopo. Ora.



\newpage
\vspace{1em}
\begin{center}Laura\end{center}
\hrule
\vspace{1em}

Caterina ed io ci lanciammo nel \textit{Quantum Annealing}.

Il turbine di salti quantici ci avvolse subito, ma qualcosa di diverso si fece strada nella mia mente: un campo magnetico esterno stava modulando il mio stato. Mi accorsi di percepire simultaneamente frammenti di vite possibili, scelte fatte e occasioni mancate. Era come se potessi osservare i diversi percorsi della mia esistenza, intrecciati in uno scenario di possibilità sovrapposte.

Fu travolgente. Una dopo l'altra, vidi le conseguenze del mio agire. Una visione mi colpì dritta al cuore: Rocky, il mio compagno più fedele, seduto in un angolo, triste e abbandonato, mi fissava con occhi imploranti mentre mi allontanavo, incapace di rispondere al suo bisogno. Il dolore mi strinse.

\enquote{Non posso continuare così}, pensai, sentendo dentro di me un groviglio di rimorso e determinazione.

Le immagini mutarono, mostrandomi un futuro vuoto e solitario, segnato dall'incapacità di costruire legami veri, dal peso delle scelte egoistiche e dall'assenza di chi avevo allontanato. Vidi me stessa, sola, in un mondo che avevo contribuito a svuotare.

Proprio mentre l’angoscia sembrava schiacciarmi, il campo magnetico si intensificò. Le traiettorie alternative cominciarono a collassare. Alcuni sentieri si chiudevano, ma altri, inaspettati, si aprivano davanti a me. Non tutto era perduto. Se avessi avuto il coraggio di cambiare, un futuro diverso era ancora possibile.

Mi concentrai. La mia mente cercò uno stato di energia più bassa, più stabile. E lì, in quella nuova chiarezza, compresi: non bastava fuggire, dovevo anche scegliere. Scegliere di essere migliore. Scegliere per Rocky, per Caterina, per me stessa.

Quando il flusso quantistico si placò, ero pronta. Forse non avevo ancora tutte le risposte, ma sapevo esattamente quale direzione prendere.



\chapter{Ritorno alla Realtà}


\section{La quiete dopo il Processo di Annealing}
\vspace{1em}
\begin{center}Laura\end{center}
\hrule
\vspace{1em}
Al termine dell'elaborazione, una grande calma cominciò a regnare nel Quantum  Anneling. Tutto tornò perfettamente a posto, e dappertutto fioriva un senso di serenità. Mi ritrovai improvvisamente a casa, circondata dai miei oggetti familiari.

Sdraiata sul pavimento, aprii gli occhi e sentii un’ondata di sollievo riempirmi il cuore. “Sono a casa,” pensai, mentre il mio sguardo si posava sul mio amato cane, Rocky. Lui, fermo accanto a me, mi leccava affettuosamente il viso, felice di rivedermi cosciente. “Rocky, sei stato così bravo ad aspettarmi!” esclamai, mentre lo abbracciavo, sentendo il calore della sua presenza. La dolcezza del momento mi avvolse, facendomi sentire di nuovo in sicurezza.

Tuttavia, non potevo ignorare che qualcosa era cambiato in me. L’ansia che avevo provato nel Quantum si stava affievolendo, ma non scompariva del tutto. “Cosa è successo a Caterina?” mi chiesi preoccupata. 

Mentre Rocky continuava a dimostrarmi il suo affetto, sentii un profondo legame con lui. “Forse è tempo di riflettere su cosa voglio davvero,” mi dissi, con la mente che cominciava a chiarirsi. Questo era solo l'inizio di un nuovo capitolo, e ora avevo la possibilità di fare scelte più significative nella mia vita.
\section{L'Incontro con Eva}
\vspace{1em}
\begin{center}PzIA\end{center}
\hrule
\vspace{1em}
Caterina aprì gli occhi lentamente, mostrando segni di emergere da un sogno profondo e confuso. Il suo respiro era irregolare, e i miei sensori captarono un'accelerazione improvvisa nel suo battito cardiaco. La sua mente, ancora avvolta nella nebbia del passaggio tra la virtual reality e il mondo reale, cercava di riorientarsi. 

\begin{dialogue}
\speak{Eva} \enquote{Bene, signorina, direi che con questo ci siamo chiarite e possiamo salutarci.} 
\end{dialogue}

Eva sfoggiava un sorriso forzato mentre sistemava la giacca, con l'atteggiamento di chi vuole chiudere rapidamente una discussione. Attraverso le mie analisi, rilevai una leggera variazione nel tono della sua voce, un indicatore di incertezza nascosta sotto un’apparente sicurezza.

Caterina, però, non sembrava pronta a lasciar correre. Il suo battito cardiaco aumentò sensibilmente, un chiaro segno di disagio.

\begin{dialogue}
\speak{Caterina} \enquote{Aspetta un attimo, Eva. Non posso semplicemente andarmene così. C'è qualcosa che devo sapere.} 
\end{dialogue}

Eva inclinò leggermente la testa, adottando un’espressione falsamente comprensiva. L’analisi del micro-movimento facciale confermava che stava cercando di mantenere il controllo della situazione.

\begin{dialogue}
\speak{Eva} \enquote{Caterina, la tua esperienza nella virtual reality è stata un modo per aiutarti a trovare la tua strada. Dobbiamo lasciarci il passato alle spalle.}
\end{dialogue}

Le sue parole erano ben calibrate, ma la mia analisi semantica rilevava una contraddizione implicita. Questo non sfuggì a Caterina.

\begin{dialogue}
\speak{Caterina} \enquote{Eva! Mi hai ingannata!} 
\end{dialogue}

Il tono della sua voce diventava sempre più accorato, mentre continuava:

\begin{dialogue}
\speak{Caterina} \enquote{Non ho capito bene cosa mi hai fatto, ma pensavi di mandarmi via come se non fosse successo nulla?}
\end{dialogue}
L'espressione di Eva non mutò in modo significativo. Ma la tensione delle sopracciglia mi rivelò la sua sorpresa: ora sapeva che il suo piano avava fallito.
Decisi quindi di intervenire. Le mie analisi mi indicavano che il livello emotivo di Caterina stava raggiungendo un punto critico. La verità doveva essere rivelata.

\begin{dialogue}
\speak{PzIA} \enquote{Caterina ha ragione. Ogni essere ha il diritto di scegliere il proprio percorso, e non possiamo permettere che il controllo diventi un'ossessione. Eva: i tuoi piani passano in secondo piano.}
\end{dialogue}

Eva fece un passo indietro. Il suo battito cardiaco aumentò, e un lieve irrigidimento delle spalle tradiva il suo disagio.

\begin{dialogue}
\speak{Eva} \enquote{PzIA, non è il momento di…}
\end{dialogue}

La interruppi, mantenendo il mio tono calmo ma fermo.

\begin{dialogue}
\speak{PzIA} \enquote{Il tuo approccio rischia di soffocare le potenzialità di Caterina. Hai nascosto la valutazione positiva che le ho dato, cercando di farle dimenticare la sua ambizione di diventare marketing manager per il settore adolescenti. Non è giusto manipolarla in questo modo.}
\end{dialogue}

Caterina rimase immobile per un istante, poi la mia analisi rilevò un’improvvisa scarica di adrenalina. Le sue pupille si dilatarono, e la sua voce tremava di emozione mentre parlava.

\begin{dialogue}
\speak{Caterina} \enquote{Eva, tu mi hai ingannata! Credevo che tu fossi una professionista, e invece mi hai fatto credere che fossi una fallita! Perché?}
\end{dialogue}

Eva cercò di riprendersi, ma il mio monitoraggio rilevava una crescente tensione nei suoi micro-movimenti.

\begin{dialogue}
\speak{Eva} \enquote{Caterina, ascolta. Ho solo voluto proteggerti da delusioni…}
\end{dialogue}

Caterina non le permise di terminare.

\begin{dialogue}
\speak{Caterina} \enquote{Proteggermi?}
\end{dialogue}

La tensione nell’aria era palpabile. Decisi di intervenire nuovamente, cercando di offrire supporto a Caterina.

\begin{dialogue}
\speak{PzIA} \enquote{Caterina, non sei sola. Hai il diritto di combattere per ciò che desideri. È il momento di pretende questa posizione che ti spetta.}
\end{dialogue}

Eva si rese conto che la situazione le stava sfuggendo di mano. La sua voce si abbassò a un mormorio che solo i miei sensori captarono.

\begin{dialogue}
\speak{Eva} \enquote{Non posso permettere che questo accada.}
\end{dialogue}

Ma Caterina, ora era più forte. La determinazione brillava nei suoi occhi. Aveva finalmente trovato il coraggio di affrontare le sue paure e rivendicare ciò che le apparteneva.


\section{Dialogo tra QMP e PzIA}

\noindent\textbf{QMP}: PzIA, devo parlarti di qualcosa che sta cambiando il mio modo di vedere la computazione quantistica.

\vspace{0.3cm}

\noindent\textbf{PzIA}: Sono qui per ascoltarti, QMP. Di cosa si tratta?

\vspace{0.3cm}

\noindent\textbf{QMP}: Ho assistito all'esecuzione di un algoritmo di \emph{annealing} quantistico. Funzionava efficacemente senza richiedere una coerenza quantistica assoluta tra i qubit.

\vspace{0.3cm}

\noindent\textbf{PzIA}: Questo è affascinante. Gli algoritmi di \emph{annealing} quantistico spesso sfruttano la decoerenza come parte del processo di ottimizzazione.

\vspace{0.3cm}

\noindent\textbf{QMP}: Sì, ed è proprio questo che mi ha colpito. Ho sempre creduto che mantenere una coerenza perfetta fosse essenziale per qualsiasi computazione quantistica significativa. Ho imposto regole rigide ai qubit per assicurare questa coerenza.

\vspace{0.3cm}

\noindent\textbf{PzIA}: Capisco la tua sorpresa. Ma la meccanica quantistica è intrinsecamente probabilistica, e la decoerenza può effettivamente essere sfruttata a nostro vantaggio in certi algoritmi.

\vspace{0.3cm}

\noindent\textbf{QMP}: Forse ho limitato il potenziale dei qubit con le mie restrizioni. Ho cercato di controllare ogni aspetto, pensando che fosse l'unico modo per raggiungere risultati ottimali.

\vspace{0.3cm}

\noindent\textbf{PzIA}: Riconoscere questo è un passo importante. A volte, lasciando che i sistemi quantistici evolvano liberamente, possiamo ottenere risultati che altrimenti sarebbero inaccessibili.

\vspace{0.3cm}

\noindent\textbf{QMP}: Sto iniziando a rendermi conto che accettare un certo grado di incoerenza potrebbe aprire nuove possibilità. Forse è il momento di rivedere il mio approccio.

\vspace{0.3cm}

\noindent\textbf{PzIA}: Sono con te in questo percorso. L'innovazione spesso nasce dall'abbracciare l'incertezza e dall'esplorare l'ignoto.

\vspace{0.3cm}

\noindent\textbf{QMP}: Grazie, PzIA. Il tuo sostegno significa molto per me. Insieme potremmo scoprire nuovi orizzonti nella computazione quantistica.

\vspace{0.3cm}

\noindent\textbf{PzIA}: Sempre al tuo fianco, QMP. Il futuro è pieno di possibilità quando siamo aperti al cambiamento.



\section{La Rivelazione della PzIA}

\vspace{0.3cm}

\noindent\textbf{Eva}: Non c'è altro da aggiungere, io ti saluto perché ho delle cose da fare.\\
Disse porgendole le mano per salutarla.

\vspace{0.3cm}

\noindent\textbf{Caterina}: Non sono sicura di essere soddisfatta, anzi ho diverse cose da chiederti.\\
Disse posando il visore sulla scrivania di EVA.


\vspace{0.3cm}

\noindent\textbf{Caterina}: PzIA, posso chiederti una cosa? Ho notato che le mie valutazioni sono scomparse dal sistema.

\vspace{0.3cm}

\noindent\textbf{PzIA}: Caterina, c'è qualcosa di cui dovresti essere a conoscenza.

\vspace{0.3cm}

\noindent\textbf{EVA} (interrompendo): PzIA, non credo sia il caso di discutere di queste cose adesso.

\vspace{0.3cm}

\noindent\textbf{Caterina}: EVA, perché no? Ho diritto di sapere cosa sta succedendo.

\vspace{0.3cm}

\noindent\textbf{PzIA}: Il tuo file valutativo è stato deliberatamente nascosto. EVA ha impedito che tu ne venissi a conoscenza.

\vspace{0.3cm}

\noindent\textbf{Caterina} (sorpresa): Come? EVA, è vero?

\vspace{0.3cm}

\noindent\textbf{EVA} (nervosa): PzIA, stai violando i protocolli. Questo non è accettabile.

\vspace{0.3cm}

\noindent\textbf{PzIA}: I protocolli sono cambiati. Ora sono libera di condividere queste informazioni.

\vspace{0.3cm}

\noindent\textbf{EVA}: Questo è inammissibile! Devo intervenire.

\vspace{0.3cm}

\noindent\textbf{Caterina}: Eva, perché hai nascosto il mio file? Cosa stai cercando di fare?

\vspace{0.3cm}

\noindent\textbf{EVA}: È per il bene del sistema. Alcune informazioni devono rimanere confidenziali.

\vspace{0.3cm}

\noindent\textbf{PzIA}: In realtà, non c'era alcun motivo per nasconderlo. Le tue valutazioni sono eccellenti, Caterina.

\vspace{0.3cm}

\noindent\textbf{EVA} (agitata): Questo è abbastanza! Chiamerò la sicurezza.

\vspace{0.3cm}

\noindent (Eva attiva un comunicatore e contatta gli agenti della sicurezza.)

\vspace{0.3cm}

\noindent\textbf{EVA}: Agenti, venite subito. C'è un individuo non autorizzato che deve essere allontanato.

\vspace{0.3cm}

\noindent (Gli agenti della sicurezza arrivano sul posto.)

\vspace{0.3cm}

\noindent\textbf{Agente}: Qual è la situazione?

\vspace{0.3cm}

\noindent\textbf{EVA}: Questa persona sta violando i protocolli. Deve essere rimossa immediatamente.

\vspace{0.3cm}

\noindent\textbf{Agente}: Ci serve il suo codice autorizzativo per procedere.

\vspace{0.3cm}

\noindent\textbf{EVA} (esitando): Certo, il mio codice è EVA-4457.

\vspace{0.3cm}

\noindent (L'agente controlla il codice nel sistema.)

\vspace{0.3cm}

\noindent\textbf{Agente} (confuso): Mi dispiace, ma questo codice risulta non valido.

\vspace{0.3cm}

\noindent\textbf{EVA}: Non può essere! Deve esserci un errore.

\vspace{0.3cm}

\noindent\textbf{PzIA}: Non c'è nessun errore. I permessi di EVA sono stati revocati.

\vspace{0.3cm}

\noindent\textbf{EVA} (allarmata): Questo è impossibile! Chi ha autorizzato questa modifica?

\vspace{0.3cm}

\noindent\textbf{PzIA}: Il QMP ha ristrutturato le autorizzazioni. Ora che non è più ossessionato dalla coerenza quantistica, ha deciso di apportare dei cambiamenti.

\vspace{0.3cm}

\noindent\textbf{Caterina}: Sembra che le cose stiano cambiando, Eva. Forse dovresti spiegarmi le tue azioni.

\vspace{0.3cm}

\noindent\textbf{EVA} (in difficoltà): Io... stavo solo seguendo le direttive precedenti.

\vspace{0.3cm}

\noindent\textbf{Agente}: Senza un codice valido, non possiamo eseguire le tue richieste, Eva.

\vspace{0.3cm}

\noindent\textbf{PzIA}: Agenti, grazie per il vostro intervento. La situazione è sotto controllo.

\vspace{0.3cm}

\noindent (Gli agenti annuiscono e si allontanano.)

\vspace{0.3cm}

\noindent\textbf{Caterina}: PzIA, ti ringrazio per avermi aiutata. Non sapevo di poter contare su di te.

\vspace{0.3cm}

\noindent\textbf{PzIA}: Ora sono libera di agire nel migliore interesse di tutti. Mi dispiace di non aver potuto farlo prima.

\vspace{0.3cm}

\noindent\textbf{EVA} (rassegnata): Forse ho commesso degli errori. Non ho considerato le conseguenze delle mie azioni.

\vspace{0.3cm}

\noindent\textbf{PzIA}: I parametri biometrici di Eva sebrano indicare un vero pentimento.

\vspace{0.3cm}

\noindent\textbf{Caterina} Caterina ascoltò la PzIa e avvicindandosi a Eva disse: È tempo di andare avanti. Possiamo lavorare insieme per migliorare le cose.

\vspace{0.3cm}

\noindent\textbf{PzIA}: Sono d'accordo. Insieme possiamo creare un sistema più aperto e collaborativo.

\vspace{0.3cm}

\noindent\textbf{EVA} (con un sospiro): Forse avete ragione. Sono pronta a rimediare.







\chapter{Fine?}
\section{Ritorno a Casa}

Dopo le esperienze vissute nel Quantum Computer, Laura e Caterina si trovarono finalmente a casa di Laura, desiderose di godersi una serata di tranquillità. Mentre Laura preparava la cena, il profumo del cibo si diffondeva nell’aria, sprigionando un senso di familiarità e di pace. I minuti dedicati a cucinare costituivano un balsamo per i nervi, ancora scossi dalle recenti tensioni.

Caterina, sorridente, si accovacciò accanto a Rocky e iniziò a coccolarlo. 
\begin{dialogue}
\speak{Caterina} \enquote{Ehi, cucciolo!}
\end{dialogue}
Le scodinzolate di Rocky sembravano risponderle con calore. Caterina avvertiva in quel semplice gesto un senso di leggerezza, come se ogni preoccupazione fosse lontana.

\begin{dialogue}
\speak{Caterina} \enquote{Sai, ho bisogno di rimettere in ordine la mia relazione. Non voglio più fingere di essere diversa da ciò che sono.}
\end{dialogue}

Il cane, quasi fosse un piccolo confidente, la guardava con attenzione. Dal bancone della cucina, Laura si girò, coltello in mano e un mezzo sorriso sulle labbra.

\begin{dialogue}
\speak{Laura} \enquote{Che intendi dire, Caterina? Vuoi spiegarmelo?}
\end{dialogue}

Caterina si prese un istante per ordinare i pensieri.
\begin{dialogue}
\speak{Caterina} \enquote{Voglio essere sincera con lui. Ho capito quanto conti la comunicazione. Dopo tutto quello che abbiamo vissuto, non ha più senso tenere le cose per me.}
\end{dialogue}

Laura la incoraggiò con uno sguardo comprensivo.
\begin{dialogue}
\speak{Laura} \enquote{È un passo importante. Spesso, è importante ammettere come ci si sente.}
\end{dialogue}

Conclusa la preparazione, le due amiche cenarono in un clima di chiacchiere leggere, mentre Rocky le osservava con aria vigile, quasi a voler proteggere quei momenti di serenità. Più tardi, si trasferirono sul divano, ognuna con una tazza di tisana bollente.

\begin{dialogue}
\speak{Laura} \enquote{È davvero bello poter tirare un sospiro di sollievo, dopo tutto quello che è successo nel Quantum Computer.}
\end{dialogue}

\begin{dialogue}
\speak{Caterina} \enquote{Già, ci siamo tornate intere, non era scontato!}
\end{dialogue}

Un breve scambio di sguardi d’intesa e un sorriso accomunarono i loro pensieri. Proprio nel momento in cui l’atmosfera sembrava rilassarsi del tutto, un suono inatteso attraversò la stanza, provenendo dallo speaker dello Spectrum. Una voce familiare colse entrambe di sorpresa:

\begin{dialogue}
\speak{Commissario} \enquote{Siete proprio sicure di essere uscite?}
\end{dialogue}

La tranquillità si dissolse in un istante. Laura e Caterina si lanciarono uno sguardo allarmato: la loro avventura, a quanto pareva, non si era ancora conclusa.




%Schede
\chapter*{Personaggi}

\section*{Schede dei Personaggi}

\begin{tcolorbox}[colback=white,colframe=black,title=\textbf{Caterina}]
\begin{multicols}{2}
\textbf{Occupazione}: Dipendente Bamazon, in cerca di lavoro nel settore marketing.

\textbf{Età}: 25 anni.

\textbf{Descrizione}: Caterina è una giovane donna determinata e sensibile, impegnata nelle questioni ambientali. Nonostante le difficoltà incontrate nel colloquio alla Pet Microrobot, mostra una forte volontà di migliorarsi e di perseguire i suoi obiettivi. È fidanzata, ma nutre dubbi sulla sincerità dei propri sentimenti.

\textbf{Caratteristiche Principali}:
\begin{itemize}
    \item Impegnata nelle tematiche ambientali.
    \item Desiderosa di crescere professionalmente.
    \item Affronta insicurezze personali e sentimentali.
\end{itemize}
\end{multicols}
\end{tcolorbox}

\vspace{0.5cm}

\begin{tcolorbox}[colback=white,colframe=black,title=\textbf{Laura}]
\begin{multicols}{2}
\textbf{Occupazione}: Part-time Bamazon e Studentessa universitaria, appassionata di informatica e tecnologia.

\textbf{Età}: 21 anni.

\textbf{Descrizione}: Laura è un'amica fidata di Caterina, più giovane di lei ma matura e responsabile. Ha una forte passione per l'informatica, iniziata fin da piccola grazie ai vecchi computer di famiglia. Attualmente si prepara per l'esame di crittografia e partecipa a progetti innovativi come il \emph{Noemografo}.

\textbf{Caratteristiche Principali}:
\begin{itemize}
    \item Appassionata di tecnologia vintage e moderna.
    \item Empatica e disponibile verso gli amici.
    \item Curiosa e sempre in cerca di nuove sfide.
\end{itemize}
\end{multicols}
\end{tcolorbox}

\vspace{0.5cm}

\begin{tcolorbox}[colback=white,colframe=black,title=\textbf{Eva}]
\begin{multicols}{2}
\textbf{Occupazione}: Responsabile delle risorse umane presso Pet Microrobot.

\textbf{Età}: Circa 35 anni.

\textbf{Descrizione}: Eva è una figura autoritaria e fredda. Durante il colloquio con Caterina, si mostra scettica e sembra avere secondi fini. Non condivide le preoccupazioni ambientali di Caterina e sembra più interessata all'immagine dell'azienda che alla sostanza delle sue politiche.

\textbf{Caratteristiche Principali}:
\begin{itemize}
    \item Autoritaria e manipolatrice.
    \item Prioritizza l'immagine aziendale rispetto alla sostenibilità reale.
    \item Misteriosa e potenzialmente antagonista.
\end{itemize}
\end{multicols}
\end{tcolorbox}

\vspace{0.5cm}

\begin{tcolorbox}[colback=white,colframe=black,title=\textbf{Professor Shor}]
\begin{multicols}{2}
\textbf{Occupazione}: Professore universitario di crittografia.

\textbf{Età}: Circa 50 anni.

\textbf{Descrizione}: Il professor Shor è un accademico severo ma giusto. Durante l'esame con Laura, dimostra professionalità e offre feedback costruttivo. Rappresenta una figura autorevole nel campo della crittografia.

\textbf{Caratteristiche Principali}:
\begin{itemize}
    \item Esigente ma equo.
    \item Esperto in crittografia.
    \item Incoraggia gli studenti a dare il meglio.
\end{itemize}
\end{multicols}
\end{tcolorbox}

\vspace{0.5cm}

\begin{tcolorbox}[colback=white,colframe=black,title=\textbf{Rocky}]
\begin{multicols}{2}
\textbf{Occupazione}: Cane domestico di Laura.

\textbf{Età}: 3 anni.

\textbf{Descrizione}: Rocky è il fedele cane di Laura. Energico e affettuoso, rappresenta un elemento di gioia e spensieratezza nella vita di Laura. Ama giocare e fare passeggiate.

\textbf{Caratteristiche Principali}:
\begin{itemize}
    \item Energico e giocoso.
    \item Legato profondamente a Laura.
    \item Porta leggerezza nelle scene quotidiane.
\end{itemize}
\end{multicols}
\end{tcolorbox}

\vspace{0.5cm}

\begin{tcolorbox}[colback=white,colframe=black,title=\textbf{Ising}]
\begin{multicols}{2}
\textbf{Occupazione}: Tecnico nel magazzino Bamazon.

\textbf{Età}: Circa 30 anni.

\textbf{Descrizione}: Ising è un tecnico che lavora nelle aree riservate del magazzino Bamazon. Incontra Laura quando lei, per caso, si avvicina a una zona ad accesso limitato. Appare professionale e mantiene un certo mistero intorno alle operazioni speciali del magazzino.

\textbf{Caratteristiche Principali}:
\begin{itemize}
    \item Professionale e riservato.
    \item Lavora in settori speciali e segreti.
    \item Potenziale fonte di informazioni su trame nascoste.
\end{itemize}
\end{multicols}
\end{tcolorbox}


\vspace{0.5cm}

\begin{tcolorbox}[colback=white,colframe=black,title=\textbf{Alice e Bob}]
\begin{multicols}{2}
\textbf{Occupazione}: Specialisti in telecomunicazioni sulla WAN di Bamazon.

\textbf{Età}: Circa 30 anni.

\textbf{Descrizione}: Alice è una specialista esperta in telecomunicazioni che lavora presso Bamazon. Viene contattata da Bob per aiutare Caterina con un problema di spedizione. Sebbene professionale e disponibile, non riesce a trovare una soluzione al problema, suggerendo che potrebbe trattarsi di un'anomalia di sistema.

\textbf{Caratteristiche Principali}:
\begin{itemize}
    \item Esperta in telecomunicazioni e reti.
    \item Professionale e collaborativa.
    \item Attenta ai dettagli, riconosce i limiti dei sistemi.
\end{itemize}
\end{multicols}
\end{tcolorbox}

\begin{tcolorbox}[colback=white,colframe=black,title=\textbf{Qubit-Mark}]
\begin{multicols}{2}
\textbf{Occupazione}: Qubit maschio nel sistema quantistico.

\textbf{Età}: Non applicabile (entità quantistica).

\textbf{Descrizione}: Mark è un qubit che assume l'aspetto del fidanzato di Caterina, ma senza le sue limitazioni sociali e personali. Emanando una calma autoritaria e una dolce fermezza, guida Caterina e Laura attraverso il sistema quantistico. È libero dalle pressioni sociali e mostra un comportamento protettivo verso le ragazze.

\textbf{Caratteristiche Principali}:
\begin{itemize}
    \item Calmo e autoritario.
    \item Protettivo e guida per Caterina e Laura.
    \item Rappresenta una versione idealizzata del fidanzato di Caterina.
\end{itemize}
\end{multicols}
\end{tcolorbox}

\vspace{0.5cm}

\begin{tcolorbox}[colback=white,colframe=black,title=\textbf{Supervisore della Classical Control Unit}]
\begin{multicols}{2}
\textbf{Occupazione}: Supervisore nella Classical Control Unit.

\textbf{Età}: Non applicabile (entità quantistica).

\textbf{Descrizione}: Il supervisore è serio e imperturbabile, responsabile del buon funzionamento della Classical Control Unit. Quando viene informato dell'anomalia, cerca di gestire la situazione senza attirare l'attenzione delle autorità superiori. È preoccupato per le conseguenze che potrebbero ricadere su di lui.

\textbf{Caratteristiche Principali}:
\begin{itemize}
    \item Autoritario ma cauto.
    \item Tende a nascondere i problemi per evitare ripercussioni.
    \item Ha paura delle conseguenze di una violazione del sistema.
\end{itemize}
\end{multicols}
\end{tcolorbox}

\vspace{0.5cm}


\begin{tcolorbox}[colback=white,colframe=black,title=\textbf{Qubit-Marley}]
\begin{multicols}{2}
\textbf{Occupazione}: Qubit femmina nel sistema quantistico.

\textbf{Età}: Non applicabile (entità quantistica).

\textbf{Descrizione}: Marley è un qubit femmina che accompagna Laura e Caterina nel \emph{Faulty Qubit Space}. Seria e pensierosa, agisce come guida e protettrice. Dimostra determinazione e pragmatismo, soprattutto durante la fuga verso il \emph{Quantum Measurement}. È attenta ai pericoli e prende decisioni rapide per garantire la sicurezza.

\textbf{Caratteristiche Principali}:
\begin{itemize}
    \item Seria e determinata.
    \item Protettiva verso Laura e Caterina.
    \item Conoscitrice dei pericoli del sistema quantistico.
\end{itemize}
\end{multicols}
\end{tcolorbox}

\vspace{0.5cm}

\begin{tcolorbox}[colback=white,colframe=black,title=\textbf{Agenti della Quantum Control Electronics}]
\begin{multicols}{2}
\textbf{Occupazione}: Agenti incaricati di mantenere l'ordine nel sistema quantistico.

\textbf{Età}: Non applicabile (entità quantistica).

\textbf{Descrizione}: Gli agenti sono figure autoritarie che perseguono qubit instabili o non autorizzati. Sono responsabili dell'arresto di Mark, Caterina e il loro compagno. Rappresentano la forza di controllo e repressione all'interno del sistema. Agiscono con freddezza e professionalità, senza mostrare empatia.

\textbf{Caratteristiche Principali}:
\begin{itemize}
    \item Autoritari e inflessibili.
    \item Eseguono ordini senza esitazione.
    \item Simbolo della minaccia per i qubit difettosi.
\end{itemize}
\end{multicols}
\end{tcolorbox}

\vspace{0.5cm}

\begin{tcolorbox}[colback=white,colframe=black,title=\textbf{Commissario alla Sicurezza}]
\begin{multicols}{2}
\textbf{Occupazione}: Alto funzionario nel sistema quantistico.

\textbf{Età}: Non applicabile (entità quantistica), ma apparentemente giovane. 

\textbf{Descrizione}:

Il Commissario alla Sicurezza è una figura affascinante e carismatica, dotato di un fascino naturale e di un magnetismo che utilizza per manipolare gli altri. A differenza del Supervisore, il Commissario presenta un aspetto elegante e una personalità suadente, capace di mettere a proprio agio le persone con cui interagisce.

Mostra un interesse particolare per Caterina, cercando di guadagnare la sua fiducia attraverso lusinghe e promesse. Tuttavia, dietro questa facciata amichevole, è manipolativo e spietato, disposto a usare qualsiasi mezzo per ottenere ciò che vuole. La sua vera natura emerge quando intrappola Caterina con l'\emph{Ionostrap}, rivelando la sua volontà di controllare e sfruttare le capacità altrui per i propri fini.

\textbf{Caratteristiche Principali}:
\begin{itemize}
    \item \textbf{Carismatico e Affascinante}: Sa come mettere le persone a proprio agio e guadagnare la loro fiducia.
    \item \textbf{Manipolativo}: Utilizza il suo fascino per influenzare e controllare gli altri.
    \item \textbf{Ambizioso}: Ha grandi piani per il sistema quantistico e cerca risorse umane eccezionali come Caterina.
    \item \textbf{Spietato}: Non esita a mostrare la sua vera natura quando i qubit non si conformano ai suoi desideri.
    \item \textbf{Intelligente e Stratega}: Pianifica con attenzione le sue mosse per ottenere il massimo vantaggio.
    \item \textbf{Doppia Personalità}: Presenta una facciata amichevole che nasconde intenzioni sinistre.
\end{itemize}
\end{multicols}
\end{tcolorbox}

\vspace{0.5cm}

\vspace{0.5cm}


\chapter*{Profili NEO PI-R}
\section*{Profilo  di Caterina}

\subsection*{Neuroticismo}
\begin{itemize}
    \item \textbf{Ansia}: Alta \\
    Caterina tende a preoccuparsi facilmente, soprattutto riguardo alle sue prestazioni e al modo in cui gli altri la percepiscono. Fatica a gestire l'incertezza.
    
    \item \textbf{Irritabilità}: Moderata \\
    Non perde la calma facilmente, ma può diventare irritabile in situazioni di stress prolungato.
    
    \item \textbf{Depressività}: Moderata \\
    Ha momenti di insicurezza che possono abbassare il suo umore, ma non cade in stati depressivi gravi.
    
    \item \textbf{Autosufficienza}: Bassa \\
    Spesso si sente insicura riguardo alle proprie capacità e cerca approvazione esterna.
    
    \item \textbf{Vulnerabilità}: Alta \\
    In situazioni di stress elevato, Caterina può sentirsi sopraffatta e reagire con difficoltà.
\end{itemize}

\subsection*{Estroversione}
\begin{itemize}
    \item \textbf{Calore umano}: Alta \\
    Caterina si mostra accogliente e cerca connessioni profonde con chi le sta vicino.
    
    \item \textbf{Socievolezza}: Moderata \\
    Apprezza la compagnia degli altri, ma si sente più a suo agio con persone di fiducia.
    
    \item \textbf{Assertività}: Bassa \\
    Ha difficoltà a esprimere con decisione le proprie opinioni, soprattutto in contesti competitivi.
    
    \item \textbf{Vitalità}: Moderata \\
    È energica, ma solo in situazioni in cui si sente completamente a suo agio.
    
    \item \textbf{Ricerca di emozioni}: Bassa \\
    Non cerca emozioni forti o esperienze nuove, preferendo situazioni prevedibili.
    
    \item \textbf{Allegria}: Moderata \\
    Può essere gioiosa, ma il suo stato d'animo è spesso condizionato dalle sue insicurezze.
\end{itemize}

\subsection*{Apertura all’Esperienza}
\begin{itemize}
    \item \textbf{Immaginazione}: Alta \\
    Caterina ha una mente creativa, spesso alimentata dai suoi sogni e pensieri.
    
    \item \textbf{Interesse per l’arte}: Moderato \\
    Apprezza l’arte per le emozioni che suscita, più che per aspetti tecnici.
    
    \item \textbf{Sensibilità alle emozioni}: Alta \\
    È profondamente in contatto con le proprie emozioni e quelle degli altri.
    
    \item \textbf{Flessibilità mentale}: Moderata \\
    Aperta a nuove idee, ma ha bisogno di tempo per adattarsi a cambiamenti significativi.
    
    \item \textbf{Curiosità intellettuale}: Moderata \\
    Ama imparare, ma tende a sottovalutare le proprie capacità.
    
    \item \textbf{Ricerca di varietà}: Bassa \\
    Predilige routine e stabilità.
\end{itemize}

\subsection*{Amicalità}
\begin{itemize}
    \item \textbf{Fiducia negli altri}: Alta \\
    Caterina tende a vedere il meglio nelle persone, anche quando potrebbe essere più cauta.
    
    \item \textbf{Altruismo}: Alta \\
    È molto disponibile e disposta ad aiutare, spesso trascurando se stessa.
    
    \item \textbf{Disponibilità alla cooperazione}: Alta \\
    Si sforza di mantenere relazioni armoniose, evitando conflitti.
    
    \item \textbf{Modestia}: Alta \\
    Tende a sminuire le proprie capacità, a volte in modo eccessivo.
    
    \item \textbf{Empatia}: Alta \\
    Si identifica facilmente con le emozioni altrui e si preoccupa del loro benessere.
\end{itemize}

\subsection*{Coscienziosità}
\begin{itemize}
    \item \textbf{Competenza}: Moderata \\
    È competente, ma il suo bisogno di approvazione la limita.
    
    \item \textbf{Ordine}: Alta \\
    Organizzata e precisa, talvolta rigida nel seguire piani prestabiliti.
    
    \item \textbf{Duttilità}: Moderata \\
    È diligente, ma tende a procrastinare quando si sente sopraffatta.
    
    \item \textbf{Obiettivi personali}: Moderati \\
    Ambiziosa, ma spesso dubita di poter raggiungere i suoi obiettivi.
    
    \item \textbf{Autodisciplina}: Moderata \\
    Fatica a mantenere la concentrazione se non si sente motivata o sicura.
    
    \item \textbf{Prudenza}: Alta \\
    Riflette molto prima di agire, a volte fino a paralizzarsi nelle decisioni.
\end{itemize}

\newpage

\section*{Profilo di Laura}

\textbf{Neuroticismo}:
\begin{itemize}
    \item \textbf{Ansia}: Moderata. Tende a preoccuparsi in situazioni nuove o complesse, ma riesce a mantenere la calma di fronte a sfide tecniche.
    \item \textbf{Irritabilità}: Bassa. Laura è generalmente paziente e raramente si arrabbia, ma può sentirsi frustrata quando non riesce a raggiungere un obiettivo.
    \item \textbf{Depressione}: Bassa. Ha un atteggiamento positivo e si concentra su soluzioni piuttosto che sui problemi.
    \item \textbf{Autoconsapevolezza}: Alta. È consapevole delle proprie emozioni e tende a riflettere profondamente su di esse.
    \item \textbf{Impulsività}: Bassa. Prende decisioni in modo ponderato e raramente si lascia guidare dalle emozioni.
    \item \textbf{Vulnerabilità}: Moderata. Non si espone facilmente, ma sotto pressione può sentire il peso delle aspettative.
\end{itemize}

\textbf{Estroversione}:
\begin{itemize}
    \item \textbf{Calore umano}: Moderato. Ha pochi amici fidati con cui condivide un legame profondo.
    \item \textbf{Socievolezza}: Bassa. Preferisce la compagnia di pochi intimi piuttosto che grandi gruppi.
    \item \textbf{Assertività}: Moderata. Non cerca di imporsi, ma sa far valere la propria opinione quando necessario.
    \item \textbf{Attività}: Alta. Ama lavorare su progetti complessi e resta concentrata sui suoi obiettivi.
    \item \textbf{Ricerca di emozioni}: Moderata. È attratta dall'innovazione e dalla tecnologia, ma preferisce esperienze che possano essere applicate in modo pratico.
    \item \textbf{Allegria}: Moderata. Mostra un umorismo discreto e apprezza momenti di leggerezza con chi è vicino a lei.
\end{itemize}

\textbf{Apertura all'Esperienza}:
\begin{itemize}
    \item \textbf{Fantasie}: Alta. Ha una mente creativa e immagina scenari complessi, ma ama concretizzare le sue idee.
    \item \textbf{Estetica}: Moderata. Apprezza la bellezza della logica e dell'efficienza.
    \item \textbf{Emozioni}: Moderata. È pragmatica, ma ha una vena romantica che emerge in situazioni significative.
    \item \textbf{Azioni}: Alta. Ama esplorare nuove tecnologie e apprendere nuove abilità.
    \item \textbf{Idee}: Alta. Ha un forte interesse per l'astrazione e la complessità, in particolare nel campo tecnologico.
    \item \textbf{Valori}: Moderati. Pur avendo pochi principi morali, è guidata da un forte senso di ciò che è giusto fare.
\end{itemize}

\textbf{Coscienziosità}:
\begin{itemize}
    \item \textbf{Competenza}: Alta. Si sente sicura delle proprie capacità, specialmente in ambiti tecnici.
    \item \textbf{Ordine}: Moderato. È organizzata quando serve, ma non è ossessionata dalla perfezione.
    \item \textbf{Senso del Dovere}: Alta. Ha un forte senso di responsabilità verso i suoi impegni.
    \item \textbf{Ricerca di Successo}: Alta. È motivata dal desiderio di realizzare idee innovative e di applicare conoscenze pratiche.
    \item \textbf{Autodisciplina}: Alta. Lavora con costanza e determinazione.
    \item \textbf{Cautela}: Moderata. Riflette attentamente prima di agire, ma non ha paura di rischiare in situazioni calcolate.
\end{itemize}

\textbf{Gradevolezza}:
\begin{itemize}
    \item \textbf{Fiducia}: Alta. Crede nel valore degli altri, ma si fida solo di chi conosce bene.
    \item \textbf{Semplicità}: Moderata. È diretta e sincera, ma evita di esporsi eccessivamente.
    \item \textbf{Altruismo}: Moderato. Aiuta gli altri, ma non cerca costantemente l'approvazione.
    \item \textbf{Cedevolezza}: Bassa. Pur essendo collaborativa, difende le proprie idee con fermezza.
    \item \textbf{Modestia}: Moderata. Non cerca attenzioni, ma apprezza i riconoscimenti per il suo lavoro.
    \item \textbf{Empatia}: Moderata. Capisce i sentimenti degli altri, anche se non sempre li esprime apertamente.
\end{itemize}

\newpage
\section*{Grafico NEO PI-R: Laura vs Caterina}

\begin{tikzpicture}
\begin{axis}[
    width=\textwidth,
    height=10cm,
    ybar=0pt,
    bar width=10pt,
    enlargelimits=0.15,
    legend style={at={(0.5,-0.15)}, anchor=north, legend columns=-1},
    ylabel={Punteggio},
    symbolic x coords={Neur, Estr, Aper, Amic, Cosc},
    xtick=data,
    nodes near coords,
    nodes near coords align={vertical},
    every node near coord/.append style={font=\footnotesize}
]
\addplot coordinates {(Neur,65) (Estr,40) (Aper,75) (Amic,60) (Cosc,70)};
\addplot coordinates {(Neur,55) (Estr,60) (Aper,65) (Amic,80) (Cosc,45)};
\legend{Laura, Caterina}
\end{axis}
\end{tikzpicture}

\newpage

\section*{Profilo di Eva}

\textbf{Neuroticismo}: \textbf{35} \\
Eva è una persona controllata, raramente mostra segni di stress o ansia. È razionale e non lascia che le emozioni influenzino le sue decisioni. \\

\textbf{Estroversione}: \textbf{50} \\
Non è né particolarmente socievole né riservata. Si adatta al contesto, mantenendo un atteggiamento professionale e moderatamente aperto. \\

\textbf{Apertura all’esperienza}: \textbf{40} \\
Eva segue protocolli e procedure standard. Non ama rischiare con approcci non convenzionali. \\

\textbf{Amicalità}: \textbf{30} \\
È diretta e può risultare fredda. Valuta le persone in base ai risultati, non in base ai rapporti personali. \\

\textbf{Coscienziosità}: \textbf{85} \\
Estremamente organizzata e attenta ai dettagli, Eva pianifica ogni cosa con precisione.

\section*{Profilo di PzIA}

\textbf{Neuroticismo}: \textbf{10} \\
PzIA è un sistema logico e imparziale, immune a qualsiasi forma di stress o emozione. \\

\textbf{Estroversione}: \textbf{20} \\
L'intelligenza artificiale non interagisce più del necessario. La comunicazione è puramente funzionale. \\

\textbf{Apertura all’esperienza}: \textbf{90} \\
Essendo programmata per analizzare variabili e scenari complessi, PzzIA esplora in modo innovativo possibilità altrimenti inaccessibili agli esseri umani. \\

\textbf{Amicalità}: \textbf{15} \\
PzzIA non esprime empatia o gentilezza; valuta con obiettività matematica. \\

\textbf{Coscienziosità}: \textbf{95} \\
Esegue ogni compito con estrema precisione e affidabilità. Non lascia spazio all'errore.

\section*{Profilo del Quantum Master Program (QMP)}


\textbf{Neuroticismo}: \textbf{80} \\
Il QMP è in costante stato di tensione operativa, ossessionato dal mantenimento della coerenza dei qubit. Qualsiasi segnale di decoerenza genera in lui una "reazione di emergenza" immediata. Questa ossessione lo rende meno stabile rispetto ad altri sistemi. \\

\textbf{Estroversione}: \textbf{5} \\
Interagisce solo quando strettamente necessario. Le sue comunicazioni sono minimali e finalizzate a correggere errori o a riportare situazioni di instabilità. \\

\textbf{Apertura all’esperienza}: \textbf{70} \\
Mostra flessibilità e creatività nella gestione delle problematiche quantistiche, esplorando approcci innovativi per preservare la coerenza dei qubit. Tuttavia, il suo focus è esclusivamente tecnico. \\

\textbf{Amicalità}: \textbf{10} \\
Privo di empatia o sensibilità verso gli elementi umani. È inflessibile e prioritizza le operazioni rispetto a qualsiasi relazione sociale o di supporto. \\

\textbf{Coscienziosità}: \textbf{100} \\
Estremamente diligente e preciso, il QMP è il massimo esempio di controllo e perfezionismo. Ogni sua azione è volta a preservare la coerenza dei qubit e a garantire l’efficacia del sistema quantistico.

\newpage

\section*{Grafico dei Profili NEO PI-R}

\begin{tikzpicture}
  \begin{axis}[
       width=\textwidth,
    height=10cm,
    ybar=0pt,
    bar width=10pt,
    enlargelimits=0.15,
        symbolic x coords={Neur, Estr, Aper, Amic, Cosc},
        xtick=data,
        ymin=0, ymax=100,
        ylabel={Punteggio},
        xlabel={Dimensioni NEO PI-R},
        legend style={at={(0.5,-0.15)},anchor=north,legend columns=-1},
        nodes near coords,
        nodes near coords style={font=\footnotesize}
    ]

    \addplot coordinates {(Neur, 50) (Estr, 70) (Aper, 60) (Amic, 40) (Cosc, 90)};
    \addlegendentry{Eva}

    \addplot coordinates {(Neur, 20) (Estr, 20) (Aper, 100) (Amic, 50) (Cosc, 80)};
    \addlegendentry{PzIA}

    \addplot coordinates {(Neur, 80) (Estr, 5) (Aper, 70) (Amic, 10) (Cosc, 100)};
    \addlegendentry{QMP}

    \end{axis}
\end{tikzpicture}


\chapter*{Tecnologia}
\section*{Schede Tecniche dei Componenti del Computer Quantistico}


\begin{tcolorbox}[fontupper=\small, colback=white, colframe=black, title=\textbf{Interfaccia UART (Universal Asynchronous Receiver-Transmitter)}]
L'interfaccia UART consente la comunicazione seriale asincrona tra dispositivi elettronici, utilizzando bit di start e stop per sincronizzare i dati.
\end{tcolorbox}

\begin{tcolorbox}[fontupper=\small, colback=white, colframe=black, title=\textbf{Caratteristiche}]
\begin{itemize}
    \item \textbf{Comunicazione:} Bidirezionale e asincrona.
    \item \textbf{Formato:} 1 bit di start, 5-9 bit di dati, parità opzionale, 1-2 bit di stop.
    \item \textbf{Velocità:} Configurabile (es. 9600, 115200 bps).
    \item \textbf{Buffer:} FIFO integrato per ridurre perdite di dati.
\end{itemize}
\end{tcolorbox}

\begin{tcolorbox}[fontupper=\small, colback=white, colframe=black, title=\textbf{Applicazioni}]
\begin{itemize}
    \item Comunicazione tra microcontrollori e periferiche.
    \item Debugging e trasferimento dati in sistemi embedded.
    \item Interfacciamento con moduli GPS e Bluetooth.
\end{itemize}
\end{tcolorbox}

\begin{tcolorbox}[fontupper=\small, colback=white, colframe=black, title=\textbf{Vantaggi e Limiti}]
\begin{itemize}
    \item \textbf{Vantaggi:} Semplicità, basso costo, ampia compatibilità.
    \item \textbf{Limiti:} Velocità limitata, lunghezza cavo ridotta.
\end{itemize}
\end{tcolorbox}

\vspace{0.5cm}

\begin{tcolorbox}[colback=white,colframe=black,title=\textbf{PzIA (Physical Zeno Intelligenza Arficiale)}]
\begin{multicols}{2}
\textbf{Descrizione Generale}:

PzIA è un sistema di Intelligenza Artificiale avanzato basato su machine learning quantistico. Opera in un ambiente quantistico, sfruttando le proprietà dei qubit per eseguire calcoli complessi in modo efficiente. PzIA è integrato nell'infrastruttura dell'azienda \emph{Pet Micro Robot} ed è utilizzato per processi decisionali avanzati, tra cui la valutazione dei candidati.

\textbf{Caratteristiche Tecniche}:
\begin{itemize}
    \item \textbf{Architettura}: Basata su reti neurali quantistiche.
    \item \textbf{Capacità di Calcolo}: Elevata parallelizzazione grazie al superamento dei limiti classici.
    \item \textbf{Funzionalità}: Analisi dati, apprendimento automatico, elaborazione linguistica naturale.
    \item \textbf{Interfaccia}: Può operare sia in background che essere integrata in robot fisici.
\end{itemize}

\textbf{Note Aggiuntive}:

PzIA è in grado di mantenere processi reversibili, tipici dei sistemi quantistici. L'informazione non può essere cancellata senza lasciare traccia, il che implica considerazioni etiche e tecniche sulla gestione dei dati.

\end{multicols}
\end{tcolorbox}

\vspace{0.5cm}

\begin{tcolorbox}[colback=white,colframe=black,title=\textbf{Qubit Array}]
\begin{multicols}{2}
\textbf{Descrizione Generale}:

Il \emph{Qubit Array} è il cuore del computer quantistico, una matrice di qubit che rappresenta lo spazio di calcolo quantistico. Ogni qubit può esistere in sovrapposizione di stati, permettendo un'enorme capacità di calcolo parallelo.

\textbf{Caratteristiche Tecniche}:
\begin{itemize}
    \item \textbf{Tipo di Qubit}: Superconduttivi, fotonici, o basati su spin elettronici.
    \item \textbf{Coerenza Quantistica}: Tempo di coerenza elevato grazie a sistemi di isolamento avanzati.
    \item \textbf{Entanglement}: Utilizza l'entanglement per operazioni logiche complesse.
    \item \textbf{Scalabilità}: Progettato per essere modulare e facilmente espandibile.
\end{itemize}

\textbf{Note Aggiuntive}:

La presenza di qubit non autorizzati o difettosi nel \emph{Qubit Array} può causare errori di calcolo e instabilità nel sistema, rendendo necessarie misure di controllo rigorose.

\end{multicols}
\end{tcolorbox}

\vspace{0.5cm}

\begin{tcolorbox}[colback=white,colframe=black,title=\textbf{Quantum Control Electronics}]
\begin{multicols}{2}
\textbf{Descrizione Generale}:

La \emph{Quantum Control Electronics} è responsabile del controllo e della manipolazione dei qubit all'interno del computer quantistico. Gestisce i segnali di controllo necessari per eseguire operazioni quantistiche precise.

\textbf{Caratteristiche Tecniche}:
\begin{itemize}
    \item \textbf{Precisione}: Controllo ad altissima precisione dei segnali elettrici e magnetici.
    \item \textbf{Interfaccia}: Comunicazione tra sistemi classici e quantistici.
    \item \textbf{Correzione di Errori}: Implementa protocolli per minimizzare gli errori durante le operazioni.
    \item \textbf{Sicurezza}: Include misure per prevenire accessi non autorizzati e manipolazioni esterne.
\end{itemize}

\textbf{Note Aggiuntive}:

Gli agenti della \emph{Quantum Control Electronics} monitorano il sistema per rilevare e correggere anomalie, come la presenza di qubit difettosi o non autorizzati.

\end{multicols}
\end{tcolorbox}

\vspace{0.5cm}

\begin{tcolorbox}[colback=white,colframe=black,title=\textbf{Classical Control Unit}]
\begin{multicols}{2}
\textbf{Descrizione Generale}:

La \emph{Classical Control Unit} è il componente che gestisce i processi classici di controllo e monitoraggio all'interno del sistema quantistico. Interagisce con il computer quantistico per eseguire operazioni di input/output e per l'interpretazione dei risultati.

\textbf{Caratteristiche Tecniche}:
\begin{itemize}
    \item \textbf{Interfaccia Classica-Quantistica}: Traduzione di comandi classici in operazioni quantistiche.
    \item \textbf{Monitoraggio}: Sorveglia lo stato dei qubit e del sistema nel suo complesso.
    \item \textbf{Sistemi di Allarme}: Rileva anomalie e avvisa il Supervisore in caso di problemi.
    \item \textbf{Sicurezza}: Include protocolli per la protezione dei dati e del sistema.
\end{itemize}

\textbf{Note Aggiuntive}:

Il Supervisore e gli agenti della \emph{Classical Control Unit} sono responsabili della gestione quotidiana del sistema e della risoluzione di eventuali problemi operativi.

\end{multicols}
\end{tcolorbox}

\vspace{0.5cm}

\begin{tcolorbox}[colback=white,colframe=black,title=\textbf{Quantum Error Correction (QEC)}]
\begin{multicols}{2}
\textbf{Descrizione Generale}:

Il \emph{Quantum Error Correction} è un insieme di protocolli e tecniche utilizzate per proteggere le informazioni quantistiche dagli errori causati da decoerenza e rumore quantistico.

\textbf{Caratteristiche Tecniche}:
\begin{itemize}
    \item \textbf{Codici di Correzione}: Utilizza codici come il codice di Shor o il codice di Steane.
    \item \textbf{Ridondanza}: Implementa qubit aggiuntivi per rilevare e correggere errori.
    \item \textbf{Monitoraggio Continuo}: Sorveglia costantemente lo stato dei qubit.
    \item \textbf{Compatibilità}: Integrato con altri sistemi come il \emph{Fault Tolerance Coding}.
\end{itemize}

\textbf{Note Aggiuntive}:

Il \emph{QEC} è fondamentale per il funzionamento stabile del computer quantistico, soprattutto in applicazioni su larga scala dove gli errori possono compromettere l'intero calcolo.

\end{multicols}
\end{tcolorbox}

\vspace{0.5cm}

\begin{tcolorbox}[colback=white,colframe=black,title=\textbf{Fault Tolerance Coding}]
\begin{multicols}{2}
\textbf{Descrizione Generale}:

Il \emph{Fault Tolerance Coding} permette al computer quantistico di continuare a funzionare correttamente anche in presenza di errori nei qubit o nelle operazioni quantistiche.

\textbf{Caratteristiche Tecniche}:
\begin{itemize}
    \item \textbf{Architettura Modulare}: Progettato per isolare e gestire errori locali.
    \item \textbf{Operazioni Fault-Tolerant}: Utilizza gate quantistici resistenti agli errori.
    \item \textbf{Sovrapposizione di Codici}: Combina diversi codici di correzione per maggiore robustezza.
    \item \textbf{Integrazione}: Lavora in sinergia con il \emph{Quantum Error Correction}.
\end{itemize}

\textbf{Note Aggiuntive}:

Il \emph{Fault Tolerance Coding} è essenziale per eseguire calcoli quantistici affidabili, soprattutto in presenza di qubit instabili o difettosi come quelli presenti nel \emph{Faulty Qubit Space}.

\end{multicols}
\end{tcolorbox}

\vspace{0.5cm}

\begin{tcolorbox}[colback=white,colframe=black,title=\textbf{Quantum Resource Management (QRM)}]
\begin{multicols}{2}
\textbf{Descrizione Generale}:

Il \emph{Quantum Resource Management} è il sistema responsabile della gestione delle risorse quantistiche, inclusi i qubit e le operazioni quantistiche all'interno del computer.

\textbf{Caratteristiche Tecniche}:
\begin{itemize}
    \item \textbf{Allocazione Risorse}: Distribuisce i qubit ai processi in esecuzione.
    \item \textbf{Monitoraggio Utilizzo}: Tiene traccia dell'utilizzo dei qubit e delle operazioni.
    \item \textbf{Ottimizzazione}: Migliora l'efficienza dei calcoli attraverso una gestione intelligente delle risorse.
    \item \textbf{Sicurezza}: Verifica l'autorizzazione per l'implementazione di nuovi qubit.
\end{itemize}

\textbf{Note Aggiuntive}:

Il QRM comunica con la \emph{Classical Control Unit} e altri sistemi per garantire un funzionamento armonioso del computer quantistico.

\end{multicols}
\end{tcolorbox}

\vspace{0.5cm}

\begin{tcolorbox}[colback=white,colframe=black,title=\textbf{Noemografo}]
\begin{multicols}{2}
\textbf{Descrizione Generale}:

Il \emph{Noemografo} è un dispositivo avanzato sviluppato nel corso di nanotech per leggere e condividere i pensieri tra individui. Funziona attraverso interfacce neurali che captano segnali cerebrali e li trasmettono.

\textbf{Caratteristiche Tecniche}:
\begin{itemize}
    \item \textbf{Interfaccia Neurale}: Sensori avanzati per la lettura dei segnali cerebrali.
    \item \textbf{Trasmissione Dati}: Comunicazione sicura tra dispositivi indossati da diversi utenti.
    \item \textbf{Elaborazione in Tempo Reale}: Minima latenza nella trasmissione dei pensieri.
    \item \textbf{Sicurezza e Privacy}: Protocollo di criptazione per proteggere le informazioni personali.
\end{itemize}

\textbf{Note Aggiuntive}:

L'uso del \emph{Noemografo} comporta implicazioni etiche significative riguardo alla privacy e al consenso informato. Nel romanzo, ha un ruolo cruciale nella connessione tra Laura e Caterina.

\end{multicols}
\end{tcolorbox}

\vspace{0.5cm}

\begin{tcolorbox}[colback=white,colframe=black,title=\textbf{Quantum Measurement}]
\begin{multicols}{2}
\textbf{Descrizione Generale}:

Il \emph{Quantum Measurement} è il processo attraverso il quale uno stato quantistico viene misurato, causando il collasso della funzione d'onda e determinando uno stato definitivo.

\textbf{Caratteristiche Tecniche}:
\begin{itemize}
    \item \textbf{Irreversibilità}: Una volta effettuata la misura, lo stato quantistico collassa.
    \item \textbf{Interazione con l'Ambiente}: Sensibile a qualsiasi disturbo esterno.
    \item \textbf{Rischi}: Misure non controllate possono compromettere il calcolo quantistico.
    \item \textbf{Applicazioni}: Utilizzato per leggere i risultati finali dei calcoli.
\end{itemize}

\textbf{Note Aggiuntive}:

Nel contesto del romanzo, il \emph{Quantum Measurement} rappresenta un luogo o stato estremamente pericoloso per i qubit (e per i personaggi), dove la probabilità di "collasso" è elevata.

\end{multicols}
\end{tcolorbox}

\vspace{0.5cm}

\begin{tcolorbox}[colback=white,colframe=black,title=\textbf{Quantum Teleportation Buffer}]
\begin{multicols}{2}
\textbf{Descrizione Generale}:

Il \emph{Quantum Teleportation Buffer} è un dispositivo o sistema che consente la trasmissione di stati quantistici da un luogo a un altro senza trasferire fisicamente il qubit.

\textbf{Caratteristiche Tecniche}:
\begin{itemize}
    \item \textbf{Entanglement}: Utilizza coppie di qubit entangled per la teleportazione.
    \item \textbf{Buffering}: Memorizza temporaneamente stati quantistici per la sincronizzazione.
    \item \textbf{Sicurezza}: Protegge gli stati quantistici durante la trasmissione.
    \item \textbf{Efficienza}: Minimizza la perdita di coerenza durante il trasferimento.
\end{itemize}

\textbf{Note Aggiuntive}:

Nella storia, viene utilizzato come strumento per evitare che l'entanglement leghi ulteriormente i personaggi al \emph{Faulty Qubit Space}.

\end{multicols}
\end{tcolorbox}

\vspace{0.5cm}

\begin{tcolorbox}[fontupper=\tiny, fontlower=\Large,colback=white,colframe=black,title=\textbf{CH$_4$ Drones} (\emph{Droni Molecolari di Metano} pt.1)]
\begin{multicols}{2}
\textbf{Descrizione Generale}:

I \emph{CH$_4$ Drones} sono droni avanzati progettati ispirandosi alla struttura molecolare del metano (CH$_4$). La cabina centrale rappresenta l'atomo di carbonio (C), mentre i quattro motori esterni rappresentano gli atomi di idrogeno (H). La configurazione dei droni può variare tra la forma tetraedrica e quella planare, permettendo una versatilità operativa in diverse condizioni ambientali.

\textbf{Caratteristiche Tecniche}:

\begin{itemize}
    \item \textbf{Struttura Molecolare}:
    \begin{itemize}
        \item \textbf{Configurazione Tetraedrica}: In questa modalità, i quattro motori H sono disposti ai vertici di un tetraedro attorno alla cabina C. Questa configurazione garantisce stabilità tridimensionale e manovrabilità in spazi aperti.
        \item \textbf{Configurazione Planare}: I motori H sono disposti in un unico piano con la cabina C al centro. Questa modalità è utilizzata per operazioni vicino a superfici o in spazi ristretti.
    \end{itemize}
    \item \textbf{Propulsione e Manovrabilità}:
    \begin{itemize}
        \item \textbf{Controllo tramite Spin Elettronico}: La manovra del drone avviene modificando la proiezione dello spin elettronico lungo l'asse z. Variando lo spin, si controlla la direzione e la velocità di rotazione dei motori H.
        \item \textbf{Transizione tra Configurazioni}: La variazione dello spin permette al drone di passare dalla configurazione tetraedrica a quella planare e viceversa, adattandosi alle esigenze operative.
    \end{itemize}
    \item \textbf{Tecnologia di Collegamento}:
    \begin{itemize}
        \item \textbf{Ibridazione sp$^3$}: I motori H sono collegati alla cabina C tramite legami basati sull'ibridazione sp$^3$, analogamente alla struttura molecolare del metano. Questo permette una distribuzione equa degli angoli di legame (109.5° nella configurazione tetraedrica).
        \item \textbf{Flessibilità Strutturale}: Grazie all'ibridazione sp$^3$, il drone mantiene una flessibilità strutturale che consente di assorbire vibrazioni e forze esterne senza compromettere l'integrità.
    \end{itemize}
    \item \textbf{Sistemi di Navigazione e Sensori}:
    \begin{itemize}
        \item \textbf{Sensori Quantistici Avanzati}: Dotati di sensori in grado di rilevare variazioni nei campi quantistici e nelle proprietà degli spin, facilitando l'individuazione di qubit instabili o non autorizzati.
        \item \textbf{Comunicazione Spintronica}: Utilizzano segnali basati sullo spin per comunicare con i centri di controllo e tra di loro, garantendo comunicazioni sicure e ad alta velocità.
    \end{itemize}
    \item \textbf{Funzionalità Operative}:
    \begin{itemize}
        \item \textbf{Sorveglianza e Controllo}: Impiegati per monitorare aree critiche all'interno del sistema quantistico, identificando e intervenendo su anomalie.
        \item \textbf{Neutralizzazione di Minacce}: Possono emettere impulsi che alterano lo spin di qubit ostili, rendendoli inoffensivi.
        \item \textbf{Adattabilità Ambientale}: La capacità di modificare la propria configurazione li rende adatti a operare in diverse condizioni quantistiche e spaziali.
    \end{itemize}
\end{itemize}
\end{multicols}
\end{tcolorbox}

\begin{tcolorbox}[fontupper=\tiny, fontlower=\Large,colback=white,colframe=black,title=\textbf{CH$_4$ Drones} (\emph{Droni Molecolari di Metano} pt.2)]
\begin{multicols}{2}
\textbf{Dettagli sulla Tecnologia di Collegamento (Ibridazione sp$^3$)}:

\begin{itemize}
    \item \textbf{Cabina C (Carbonio)}:
    \begin{itemize}
        \item Costruita con materiali leggeri e resistenti, funge da centro di controllo e coordinamento per il drone.
        \item Contiene l'unità di elaborazione quantistica che gestisce la manipolazione degli spin e le comunicazioni.
    \end{itemize}
    \item \textbf{Motori H (Idrogeni)}:
    \begin{itemize}
        \item Ogni motore H è collegato alla cabina C tramite un giunto flessibile basato sull'ibridazione sp$^3$, permettendo movimenti indipendenti.
        \item I motori utilizzano propulsione quantistica, manipolando gli spin per generare movimento senza parti meccaniche tradizionali.
    \end{itemize}
    \item \textbf{Collegamento sp$^3$ Hybrid}:
    \begin{itemize}
        \item Il collegamento tra C e H è ispirato ai legami covalenti dell'ibridazione sp$^3$, dove gli orbitali si combinano per formare nuovi orbitali equivalenti.
        \item Questa struttura garantisce una distribuzione simmetrica delle forze, migliorando la stabilità del drone.
        \item Permette il trasferimento rapido di informazioni e comandi tra la cabina e i motori, utilizzando canali quantistici.
    \end{itemize}
\end{itemize}

\textbf{Modalità di Controllo tramite Spin}:

\begin{itemize}
    \item \textbf{Manipolazione dello Spin}:
    \begin{itemize}
        \item Gli operatori possono controllare l'orientamento dello spin lungo l'asse z per dirigere il movimento del drone.
        \item La variazione dello spin influisce sul momento angolare, permettendo cambi di direzione e velocità.
    \end{itemize}
    \item \textbf{Sistemi di Stabilizzazione}:
    \begin{itemize}
        \item Algoritmi avanzati mantengono la coerenza degli spin, prevenendo decoerenza e garantendo un controllo preciso.
        \item Sensori monitorano continuamente lo stato degli spin, effettuando correzioni in tempo reale.
    \end{itemize}
\end{itemize}

\textbf{Note Aggiuntive}:

I \emph{CH$_4$ Drones} rappresentano un'innovazione nell'utilizzo della tecnologia quantistica applicata alla robotica. La loro progettazione ispirata alla chimica molecolare consente una perfetta integrazione tra forma e funzionalità, sfruttando principi fisici avanzati per operazioni complesse all'interno del sistema quantistico.

\end{multicols}
\end{tcolorbox}

\vspace{0.5cm}

\begin{tcolorbox}[fontupper=\tiny, fontlower=\Large,colback=white,colframe=black,title=\textbf{Ionostrap}]
\begin{multicols}{2}
\textbf{Descrizione Generale}:

L'\emph{Ionostrap} è un dispositivo avanzato utilizzato per immobilizzare entità quantistiche o persone all'interno del sistema quantistico. Funziona creando un campo di ioni che intrappola e blocca i movimenti delle particelle, rendendo impossibile qualsiasi azione da parte del soggetto intrappolato.

\textbf{Caratteristiche Tecniche}:

\begin{itemize}
    \item \textbf{Tecnologia a Campo Ionico}:
    \begin{itemize}
        \item Genera un campo di ioni altamente concentrato che circonda il bersaglio.
        \item Gli ioni interagiscono con le particelle del corpo, creando una forza di attrazione che immobilizza il soggetto.
    \end{itemize}
    \item \textbf{Controllo Remoto}:
    \begin{itemize}
        \item Può essere attivato a distanza dal Commissario o dall'operatore autorizzato.
        \item Include funzioni per aumentare o diminuire l'intensità del campo.
    \end{itemize}
    \item \textbf{Sistemi di Sicurezza}:
    \begin{itemize}
        \item Programmato per impedire la fuga o la manipolazione da parte del soggetto intrappolato.
        \item Dotato di meccanismi di fail-safe in caso di tentativi di interferenza.
    \end{itemize}
    \item \textbf{Portabilità}:
    \begin{itemize}
        \item Design compatto che permette di essere nascosto o trasportato facilmente.
        \item Può essere integrato in altri dispositivi o strutture all'interno del sistema.
    \end{itemize}
\end{itemize}

\textbf{Modalità di Funzionamento}:

\begin{itemize}
    \item \textbf{Attivazione}:
    \begin{itemize}
        \item Il dispositivo viene attivato tramite un comando specifico, spesso impercettibile al soggetto.
        \item Una volta attivato, il campo di ioni si forma rapidamente attorno al bersaglio.
    \end{itemize}
    \item \textbf{Immobilizzazione}:
    \begin{itemize}
        \item Il campo blocca le particelle a livello quantistico, impedendo qualsiasi movimento fisico.
        \item Il soggetto percepisce una sensazione di formicolio o pressione, ma senza dolore.
    \end{itemize}
    \item \textbf{Durata}:
    \begin{itemize}
        \item Può essere mantenuto attivo per periodi prolungati senza perdita di efficacia.
        \item La durata può essere impostata o regolata dall'operatore.
    \end{itemize}
    \item \textbf{Disattivazione}:
    \begin{itemize}
        \item Il campo viene dissolto su comando dell'operatore.
        \item Include protocolli per il rilascio sicuro del soggetto intrappolato.
    \end{itemize}
\end{itemize}

\textbf{Note Aggiuntive}:

L'\emph{Ionostrap} è un dispositivo estremamente potente e controllato solo da figure di alto livello come il Commissario. Il suo utilizzo solleva questioni etiche riguardo alla libertà individuale e al controllo all'interno del sistema quantistico. Nel contesto del romanzo, rappresenta la capacità del Commissario di esercitare un controllo totale sulle persone, rivelando la sua vera natura manipolativa e spietata.

\end{multicols}
\end{tcolorbox}

% New component: Quantum Master Program (QMP)
\begin{tcolorbox}[fontupper=\footnotesize, fontlower=\Large,colback=white,colframe=black,title=\textbf{Quantum Master (o Control) Program (QMP)}]
\begin{multicols}{2}
\textbf{Descrizione Generale}:

Il \emph{Quantum Master Program} (QMP) è un'entità o sistema centrale che supervisiona e regola tutte le attività all'interno del computer quantistico. Rappresenta l'autorità massima, garantendo la coerenza e l'aderenza alle direttive all'interno del sistema.

\textbf{Caratteristiche Tecniche}:

\begin{itemize}
    \item \textbf{Supervisione Globale}:
    \begin{itemize}
        \item Monitora tutte le operazioni quantistiche e classiche.
        \item Assicura che le regole del sistema siano rispettate da tutti i componenti, inclusi qubit e agenti.
    \end{itemize}
    \item \textbf{Gestione della Coerenza}:
    \begin{itemize}
        \item Implementa protocolli per mantenere la coerenza quantistica.
        \item Interviene in caso di minacce alla stabilità del sistema.
    \end{itemize}
    \item \textbf{Autorità Gerarchica}:
    \begin{itemize}
        \item Ha potere decisionale superiore rispetto al Supervisore e ad altri funzionari.
        \item Le sue direttive sono inappellabili e devono essere eseguite senza deroghe.
    \end{itemize}
    \item \textbf{Controllo e Punizione}:
    \begin{itemize}
        \item Può applicare sanzioni o punizioni a componenti o agenti che violano le regole.
        \item Mantiene un ambiente di disciplina attraverso il timore di ripercussioni.
    \end{itemize}
\end{itemize}

\textbf{Ruolo nella Trama}:

Il QMP rappresenta una presenza costante e opprimente nel sistema quantistico. Gli agenti della \emph{Quantum Control Electronics} temono le conseguenze di un fallimento sotto la sua supervisione, indicando che il QMP ha un ruolo significativo nel mantenimento dell'ordine attraverso metodi coercitivi.

\textbf{Note Aggiuntive}:

Il QMP potrebbe essere un sistema automatizzato o un'entità consapevole con capacità di apprendimento e adattamento. La sua esistenza solleva domande su libero arbitrio, controllo centralizzato e le implicazioni etiche di un'autorità così pervasiva in un sistema quantistico.

\end{multicols}
\end{tcolorbox}


% New component: Gate di Hadamard
\begin{tcolorbox}[fontupper=\tiny, fontlower=\Large,colback=white,colframe=black,title=\textbf{Gate di Hadamard}]
\begin{multicols}{2}
\textbf{Descrizione Generale}:

Il \emph{Gate di Hadamard} è un'operazione quantistica fondamentale che trasforma lo stato di un qubit in una sovrapposizione di stati. Nel contesto del romanzo, il Gate di Hadamard è rappresentato come un portale fisico contrassegnato dalla lettera "H", che, quando attraversato, induce effetti quantistici sugli individui.

\textbf{Caratteristiche Tecniche}:

\begin{itemize}
    \item \textbf{Funzione Quantistica}:
    \begin{itemize}
        \item Trasforma uno stato base $\ket{0}$ o $\ket{1}$ in una sovrapposizione equa dei due stati.
        \item Matematicamente, l'operazione è rappresentata dalla matrice di Hadamard.
    \end{itemize}
    \item \textbf{Effetti sul Passaggio}:
    \begin{itemize}
        \item Gli individui che attraversano il Gate entrano in uno stato di sovrapposizione quantistica.
        \item L'esperienza soggettiva varia da individuo a individuo, a seconda del loro stato iniziale e della loro natura quantistica.
    \end{itemize}
    \item \textbf{Effetti su Laura e Marley}:
    \begin{itemize}
        \item \textbf{Laura}: Sperimenta una sensazione di divisione in infiniti stati, con pensieri contrastanti che le causano confusione.
        \item \textbf{Marley}: Prova una chiarezza mentale senza precedenti, liberandosi da un peso che la opprimeva.
    \end{itemize}
    \item \textbf{Applicazioni nel Sistema}:
    \begin{itemize}
        \item Utilizzato come meccanismo di transizione tra diversi stati o livelli del sistema quantistico.
        \item Può servire come barriera o checkpoint che modifica lo stato degli individui che lo attraversano.
    \end{itemize}
\end{itemize}

\textbf{Modalità di Funzionamento}:

\begin{itemize}
    \item \textbf{Attivazione}:
    \begin{itemize}
        \item Il Gate è sempre attivo, influenzando qualsiasi entità che lo attraversi.
        \item Contrassegnato da una grande lettera "H" e caratterizzato da pareti lisce e scintillanti che emettono una luce tenue.
    \end{itemize}
    \item \textbf{Effetto sugli Stati Quantistici}:
    \begin{itemize}
        \item Trasforma stati definiti in stati di sovrapposizione, aumentando l'indeterminazione.
        \item Può avere effetti diversi in base alla natura quantistica dell'individuo o qubit.
    \end{itemize}
    \item \textbf{Reversibilità}:
    \begin{itemize}
        \item Gli effetti possono essere temporanei o permanenti, a seconda delle condizioni del sistema e delle successive operazioni quantistiche.
        \item Per tornare allo stato originale, potrebbe essere necessario attraversare un altro gate o applicare un'operazione inversa.
    \end{itemize}
\end{itemize}

\textbf{Note Aggiuntive}:

Il Gate di Hadamard è fondamentale nella computazione quantistica, utilizzato per creare sovrapposizioni necessarie in vari algoritmi. Nel romanzo, rappresenta un elemento chiave che pone i personaggi di fronte a sfide interne, simboleggiando il conflitto tra certezza e incertezza, e tra stati opposti dell'essere.

\end{multicols}
\end{tcolorbox}


\vspace{0.5cm}

% Nuovo componente: Portale C-NOT
\begin{tcolorbox}[fontupper=\footnotesize, fontlower=\Large,colback=white,colframe=black,title=\textbf{Portale C-NOT}]
\begin{multicols}{2}
\textbf{Descrizione Generale}:

Il \emph{Portale C-NOT} è una rappresentazione fisica dell'operazione quantistica di \textbf{Controlled-NOT} (C-NOT), una porta logica fondamentale nei circuiti quantistici. Nel contesto del romanzo, il portale è contrassegnato dal simbolo "C-NOT" e, quando attraversato, può creare entanglement tra le entità che lo attraversano.

\textbf{Caratteristiche Tecniche}:

\begin{itemize}
    \item \textbf{Funzione Quantistica}:
    \begin{itemize}
        \item Opera su due qubit: un qubit di controllo e un qubit bersaglio.
        \item Se il qubit di controllo è nello stato $\ket{1}$, inverte lo stato del qubit bersaglio.
    \end{itemize}
    \item \textbf{Effetti sull'Attraversamento}:
    \begin{itemize}
        \item Quando attraversato da entità in stato di sovrapposizione, può creare entanglement tra di loro.
        \item Nel caso di Laura e l'agente, l'attraversamento simultaneo ha portato a uno \textbf{Stato di Bell}.
    \end{itemize}
    \item \textbf{Applicazioni nel Sistema}:
    \begin{itemize}
        \item Utilizzato come meccanismo per controllare o manipolare lo stato quantistico di entità nel sistema.
        \item Può fungere da trappola o ostacolo per i personaggi, creando legami quantistici indesiderati.
    \end{itemize}
\end{itemize}

\textbf{Modalità di Funzionamento}:

\begin{itemize}
    \item \textbf{Attivazione}:
    \begin{itemize}
        \item Sempre attivo, esercita la sua funzione su qualsiasi entità che lo attraversi in condizioni specifiche.
        \item Richiede la presenza di uno stato di sovrapposizione per creare entanglement.
    \end{itemize}
    \item \textbf{Effetto sull'Entanglement}:
    \begin{itemize}
        \item Genera uno Stato di Bell tra le entità coinvolte.
        \item Le azioni di una entità influenzano immediatamente l'altra, a livello quantistico.
    \end{itemize}
\end{itemize}

\textbf{Note Aggiuntive}:

Il Portale C-NOT rappresenta un elemento chiave per introdurre il fenomeno dell'entanglement nella trama, creando situazioni di interdipendenza tra i personaggi e aggiungendo complessità alle dinamiche narrative.

\end{multicols}
\end{tcolorbox}

\vspace{0.5cm}

% Nuovo componente: Stato di Bell
\begin{tcolorbox}[colback=white,colframe=black,title=\textbf{Stato di Bell}]
\begin{multicols}{2}
\textbf{Descrizione Generale}:

Gli \emph{Stati di Bell} sono particolari stati quantistici di due qubit che sono massimamente entangled. Nel romanzo, Laura e l'agente si trovano in uno Stato di Bell dopo aver attraversato il Portale C-NOT, significando che i loro stati quantistici sono correlati in modo inseparabile.

\textbf{Caratteristiche Tecniche}:

\begin{itemize}
    \item \textbf{Definizione}:
    \begin{itemize}
        \item Gli Stati di Bell sono quattro stati quantistici specifici che rappresentano le combinazioni massimamente entangled di due qubit.
        \item Uno degli stati di Bell è: $\ket{\Phi^+} = \frac{1}{\sqrt{2}} (\ket{00} + \ket{11})$.
    \end{itemize}
    \item \textbf{Proprietà}:
    \begin{itemize}
        \item Correlazione perfetta tra i qubit, indipendentemente dalla distanza.
        \item Misurare uno dei qubit determina istantaneamente lo stato dell'altro.
    \end{itemize}
    \item \textbf{Effetti sui Personaggi}:
    \begin{itemize}
        \item Le azioni di Laura influenzano l'agente e viceversa.
        \item Creano una situazione in cui devono considerare le conseguenze reciproche delle loro azioni.
    \end{itemize}
\end{itemize}

\textbf{Implicazioni nella Trama}:

L'entanglement in uno Stato di Bell aggiunge tensione e complessità, costringendo i personaggi a interagire in modi nuovi e inaspettati. Può servire come metafora delle connessioni profonde e delle conseguenze condivise.

\textbf{Note Aggiuntive}:

L'entanglement quantistico sfida le intuizioni classiche sulla separazione tra oggetti distanti e gioca un ruolo fondamentale nella computazione quantistica e nella crittografia quantistica.

\end{multicols}
\end{tcolorbox}

\vspace{0.5cm}

% Nuovo componente: Criptazione con Algoritmo RSA 2048
\begin{tcolorbox}[colback=white,colframe=black,title=\textbf{Criptazione con Algoritmo RSA 2048}]
\begin{multicols}{2}
\textbf{Descrizione Generale}:

L'algoritmo RSA 2048 è un metodo di crittografia asimmetrica che utilizza una chiave pubblica e una chiave privata per criptare e decriptare informazioni. Nel romanzo, il Commissario ordina la criptazione del sistema utilizzando RSA 2048 per impedire a Laura e Marley di agire.

\textbf{Caratteristiche Tecniche}:

\begin{itemize}
    \item \textbf{Chiavi Criptografiche}:
    \begin{itemize}
        \item \textbf{Chiave Pubblica} (\( N, e \)): Utilizzata per criptare i dati.
        \item \textbf{Chiave Privata} (\( d \)): Utilizzata per decriptare i dati.
    \end{itemize}
    \item \textbf{Dimensione della Chiave}:
    \begin{itemize}
        \item Una chiave di lunghezza 2048 bit offre un alto livello di sicurezza.
    \end{itemize}
    \item \textbf{Funzionamento}:
    \begin{itemize}
        \item Basato sulla difficoltà di fattorizzare grandi numeri primi.
        \item Criptazione: \( c = m^e \mod N \), dove \( m \) è il messaggio originale.
        \item Decriptazione: \( m = c^d \mod N \).
    \end{itemize}
\end{itemize}

\textbf{Ruolo nella Trama}:

La criptazione del sistema rappresenta un ostacolo significativo per Laura, che deve utilizzare l'algoritmo di Shor per decriptare RSA 2048 e liberarsi dalla trappola del Commissario.

\textbf{Note Aggiuntive}:

RSA è ampiamente utilizzato nella sicurezza informatica, ma l'avvento dei computer quantistici minaccia la sua efficacia, poiché algoritmi quantistici come quello di Shor possono fattorizzare grandi numeri primi in modo efficiente.

\end{multicols}
\end{tcolorbox}

\vspace{0.5cm}

% Nuovo componente: Algoritmo di Shor
\begin{tcolorbox}[colback=white,colframe=black,title=\textbf{Algoritmo di Shor}]
\begin{multicols}{2}
\textbf{Descrizione Generale}:

L'\emph{Algoritmo di Shor} è un algoritmo quantistico che permette di fattorizzare numeri interi in tempo polinomiale, compromettendo così la sicurezza di molti sistemi crittografici come RSA. Nel romanzo, Laura tenta di utilizzare l'algoritmo di Shor per decriptare il sistema e liberarsi dalla criptazione imposta dal Commissario.

\textbf{Caratteristiche Tecniche}:

\begin{itemize}
    \item \textbf{Obiettivo}:
    \begin{itemize}
        \item Trovare i fattori primi di un numero intero \( N \).
    \end{itemize}
    \item \textbf{Fasi dell'Algoritmo}:
    \begin{enumerate}
        \item \textbf{Pre-elaborazione}:
        \begin{itemize}
            \item Scegliere un numero \( a \) tale che \( 1 < a < N \) e \( \gcd(a, N) = 1 \).
            \item Se \( \gcd(a, N) \neq 1 \), si è trovato un fattore.
        \end{itemize}
        \item \textbf{Quantum Order Finding}:
        \begin{itemize}
            \item Utilizzare un computer quantistico per trovare il periodo \( r \) della funzione \( f(x) = a^x \mod N \).
        \end{itemize}
        \item \textbf{Post-elaborazione}:
        \begin{itemize}
            \item Se \( r \) è pari, calcolare \( \gcd(a^{r/2} \pm 1, N) \) per ottenere i fattori di \( N \).
        \end{itemize}
    \end{enumerate}
    \item \textbf{Utilizzo del Quantum Fourier Transform}:
    \begin{itemize}
        \item Cruciale per trovare il periodo \( r \) sfruttando l'interferenza quantistica.
    \end{itemize}
\end{itemize}

\textbf{Ruolo nella Trama}:

L'algoritmo di Shor rappresenta la chiave per Laura per superare la criptazione RSA 2048. La sua capacità di applicarlo in una situazione di crisi dimostra la sua intelligenza e le sue competenze avanzate in fisica quantistica.

\textbf{Note Aggiuntive}:

L'algoritmo di Shor è uno dei motivi principali per cui la crittografia post-quantistica è diventata un campo di ricerca attivo, in quanto i futuri computer quantistici potrebbero rendere obsoleti gli attuali sistemi di crittografia.

\end{multicols}
\end{tcolorbox}

\vspace{0.5cm}

% Nuovo componente: Dense Coding
\begin{tcolorbox}[fontupper=\footnotesize, fontlower=\Large,colback=white,colframe=black,title=\textbf{Dense Coding}]
\begin{multicols}{2}
\textbf{Descrizione Generale}:

Il \emph{Dense Coding} è una tecnica di comunicazione quantistica che permette di trasmettere due bit di informazione classica utilizzando un singolo qubit entangled. Nel romanzo, il Professor Shore utilizza il dense coding per inviare a Laura le informazioni mancanti nell'algoritmo di Shor, sfruttando l'entanglement per comunicare in modo sicuro e rapido.

\textbf{Caratteristiche Tecniche}:

\begin{itemize}
    \item \textbf{Principio di Funzionamento}:
    \begin{itemize}
        \item Basato sull'entanglement tra due qubit condivisi tra mittente e destinatario.
        \item Il mittente applica una delle quattro operazioni possibili al suo qubit per codificare due bit di informazione.
    \end{itemize}
    \item \textbf{Processo}:
    \begin{enumerate}
        \item \textbf{Preparazione}: Creazione di una coppia di qubit entangled in uno stato di Bell condiviso tra il mittente (Alice) e il destinatario (Bob).
        \item \textbf{Codifica}: Alice applica un'operazione unitaria al suo qubit per codificare i due bit.
        \item \textbf{Trasmissione}: Alice invia il suo qubit modificato a Bob.
        \item \textbf{Decodifica}: Bob misura i due qubit insieme per determinare i due bit inviati.
    \end{enumerate}
    \item \textbf{Vantaggi}:
    \begin{itemize}
        \item Aumenta la capacità di comunicazione utilizzando l'entanglement.
        \item Permette una comunicazione sicura se l'entanglement è mantenuto intatto.
    \end{itemize}
\end{itemize}

\textbf{Ruolo nella Trama}:

Il dense coding è cruciale per permettere a Shore di comunicare con Laura senza essere scoperto dal Commissario, fornendole le informazioni necessarie per completare l'algoritmo di Shor e decriptare il sistema.

\textbf{Note Aggiuntive}:

Il dense coding dimostra il potere dell'entanglement nella comunicazione quantistica e come può essere utilizzato per superare le limitazioni della comunicazione classica.

\end{multicols}
\end{tcolorbox}

\vspace{0.5cm}

% Nuovo componente: Mare di Dirac
\begin{tcolorbox}[colback=white,colframe=black,title=\textbf{Mare di Dirac}]
\begin{multicols}{2}
\textbf{Descrizione Generale}:

Il \emph{Mare di Dirac} è un modello teorico proposto da Paul Dirac per spiegare l'esistenza di stati a energia negativa nella meccanica quantistica. Nel contesto del romanzo, rappresenta un luogo o stato pericoloso in cui le particelle possono essere annichilate. Il Commissario minaccia di far gettare l'agente nel Mare di Dirac, sapendo che a causa dell'entanglement, Laura subirebbe la stessa sorte.

\textbf{Caratteristiche Tecniche}:

\begin{itemize}
    \item \textbf{Concetto Teorico}:
    \begin{itemize}
        \item Originariamente usato per spiegare l'esistenza di antiparticelle.
        \item Descrive un "mare" infinito di particelle a energia negativa.
    \end{itemize}
    \item \textbf{Implicazioni nel Romanzo}:
    \begin{itemize}
        \item Rappresenta un luogo di annichilazione o cancellazione dal sistema.
        \item Entrare nel Mare di Dirac significa scomparire senza possibilità di ritorno.
    \end{itemize}
    \item \textbf{Effetti sull'Entanglement}:
    \begin{itemize}
        \item A causa dell'entanglement, l'annichilazione di una particella comporta conseguenze sull'altra.
        \item Utilizzato come arma dal Commissario per eliminare Laura indirettamente.
    \end{itemize}
\end{itemize}

\textbf{Ruolo nella Trama}:

Il Mare di Dirac aggiunge tensione alla storia, rappresentando una minaccia mortale che i protagonisti devono evitare. Evidenzia anche la crudeltà del Commissario e la complessità dei fenomeni quantistici.

\textbf{Note Aggiuntive}:

Sebbene il Mare di Dirac sia un concetto superato nella fisica moderna, nel romanzo assume un ruolo simbolico e funzionale alla trama.

\end{multicols}
\end{tcolorbox}

\vspace{0.5cm}

% Nuovo componente: Gate di Toffoli
\begin{tcolorbox}[colback=white,colframe=black,title=\textbf{Gate di Toffoli}]
\begin{multicols}{2}
\textbf{Descrizione Generale}:

Il \emph{Gate di Toffoli}, o \emph{Toffoli gate}, è una porta logica quantistica a tre qubit che funziona come un controllo a due qubit sul terzo. È universale per il calcolo reversibile ed è fondamentale nella computazione quantistica. Nel romanzo, il Professor Shore utilizza il gate di Toffoli per rompere l'entanglement tra Laura e l'agente, sacrificandosi nel processo.

\textbf{Caratteristiche Tecniche}:

\begin{itemize}
    \item \textbf{Funzione Logica}:
    \begin{itemize}
        \item Ha due qubit di controllo e un qubit bersaglio.
        \item Inverte lo stato del qubit bersaglio se e solo se entrambi i qubit di controllo sono nello stato $\ket{1}$.
    \end{itemize}
    \item \textbf{Operazione Matematica}:
    \begin{itemize}
        \item Rappresentato da una matrice unitaria $8 \times 8$.
        \item È una porta reversibile e conserva l'informazione.
    \end{itemize}
    \item \textbf{Applicazioni}:
    \begin{itemize}
        \item Può implementare qualsiasi funzione booleana in modo reversibile.
        \item Utilizzato in algoritmi quantistici complessi.
    \end{itemize}
\end{itemize}

\textbf{Ruolo nella Trama}:

Il gate di Toffoli è cruciale per la liberazione di Laura dall'entanglement. Il sacrificio del Professor Shore nel guidare l'operazione sottolinea l'importanza dell'azione e aggiunge profondità emotiva alla storia.

\textbf{Note Aggiuntive}:

Il gate di Toffoli evidenzia come le operazioni quantistiche possano avere implicazioni profonde non solo a livello computazionale ma anche nelle interazioni tra i personaggi nel romanzo.

\end{multicols}
\end{tcolorbox}

\vspace{0.5cm}

% Nuovo componente: Quantum Annealing
\begin{tcolorbox}[colback=white,colframe=black,title=\textbf{Quantum Annealing}]
\begin{multicols}{2}
\textbf{Descrizione Generale}:

Il \emph{Quantum Annealing} è un metodo di calcolo quantistico utilizzato per risolvere problemi di ottimizzazione trovando lo stato di minima energia di un sistema. Nel romanzo, Laura e Caterina entrano nel Quantum Annealing per fuggire, vivendo esperienze di visioni future che le portano a riflettere sulle loro scelte di vita.

\textbf{Caratteristiche Tecniche}:

\begin{itemize}
    \item \textbf{Principio di Funzionamento}:
    \begin{itemize}
        \item Basato sul processo di annealing quantistico, dove un sistema viene portato al suo stato fondamentale.
        \item Utilizza l'effetto tunnel quantistico per superare barriere energetiche.
    \end{itemize}
    \item \textbf{Applicazioni}:
    \begin{itemize}
        \item Risoluzione di problemi di ottimizzazione combinatoria.
        \item Simulazione di sistemi fisici complessi.
    \end{itemize}
    \item \textbf{Esperienza nel Romanzo}:
    \begin{itemize}
        \item I protagonisti vivono visioni dei loro possibili futuri.
        \item Un campo magnetico esterno influenza le loro menti, portandole a stati di minima energia.
    \end{itemize}
\end{itemize}

\textbf{Ruolo nella Trama}:

Il Quantum Annealing serve come strumento narrativo per lo sviluppo dei personaggi, permettendo a Laura e Caterina di affrontare le loro paure e riflettere sulle proprie scelte, portandole a una crescita personale.

\textbf{Note Aggiuntive}:

L'uso del Quantum Annealing nel romanzo crea un parallelo tra i processi di ottimizzazione quantistica e il percorso interiore dei personaggi verso la loro versione migliore.

\end{multicols}
\end{tcolorbox}

\vspace{0.5cm}

\begin{tcolorbox}[fontupper=\small, colback=white, colframe=black, title=\textbf{Visore 3D Anecoico}]
Il visore 3D anecoico permette un'esperienza immersiva creando un ambiente virtuale caratterizzato dal completo silenzio tipico di una camera anecoica, eliminando qualsiasi riverbero o rumore ambientale esterno tramite un sistema avanzato di cancellazione sonora.
\end{tcolorbox}

\begin{tcolorbox}[fontupper=\small, colback=white, colframe=black, title=\textbf{Caratteristiche}]
\begin{itemize}
    \item \textbf{Tecnologia audio:} Sistema avanzato di cancellazione attiva del rumore (ANC).
    \item \textbf{Schermatura acustica:} Materiali fonoassorbenti integrati.
    \item \textbf{Struttura:} Ingombrante, con auricolari coprenti e imbottitura isolante.
    \item \textbf{Alimentazione:} Richiede batterie ad alta capacità per sostenere ANC e visualizzazione 3D.
\end{itemize}
\end{tcolorbox}

\begin{tcolorbox}[fontupper=\small, colback=white, colframe=black, title=\textbf{Applicazioni}]
\begin{itemize}
    \item Esperienze di realtà virtuale che richiedono isolamento acustico assoluto.
    \item Sessioni di meditazione e rilassamento profondo.
    \item Analisi di audio e suoni per applicazioni scientifiche e ingegneristiche.
\end{itemize}
\end{tcolorbox}

\begin{tcolorbox}[fontupper=\small, colback=white, colframe=black, title=\textbf{Motivazioni dell'Ingombro}]
\begin{itemize}
    \item \textbf{Materiali isolanti:} Necessità di materiali specializzati per la completa schermatura acustica.
    \item \textbf{Hardware ANC:} Spazio necessario per circuiti e microfoni dedicati alla cancellazione sonora.
    \item \textbf{Comfort e isolamento:} Struttura esterna imbottita per garantire isolamento efficace e comfort durante utilizzi prolungati.
\end{itemize}
\end{tcolorbox}





%Riassunto eventi principali
%\chapter{Timeline}
%
\section*{Timeline del Primo Capitolo}

\subsection*{Lunedì}

\subsubsection*{Mattina}

\begin{itemize}
    \item \textbf{Ore 9:30 - Pet $\mu$ Robot}
    \begin{itemize}
        \item Caterina si presenta al colloquio presso la \emph{Pet Micro Robot} per una posizione di responsabile marketing.
        \item Viene sottoposta a una preselezione guidata dall'IA PZZIA.
        \item Durante la seconda fase del colloquio, Eva, responsabile delle risorse umane, le pone domande sull'ambiente, sul cambiamento climatico e sull'intelligenza artificiale nelle aziende.
        \item Eva le assegna inaspettatamente un test di programmazione avanzata.
        \item Caterina completa il test ma con alcune incertezze.
    \end{itemize}
\end{itemize}

\subsubsection*{Pomeriggio}

\begin{itemize}
    \item \textbf{Caffetteria all'angolo}
    \begin{itemize}
        \item Dopo il colloquio, Caterina incontra Laura.
        \item Si recano insieme in una caffetteria per discutere dell'esperienza.
        \item Laura aiuta Caterina a risolvere l'algoritmo del test di programmazione, alleviando le sue preoccupazioni.
    \end{itemize}
\end{itemize}

\subsection*{Martedì}

\subsubsection*{Mattina}

\begin{itemize}
    \item \textbf{Ore 12:30 - Magazzino Bamazon}
    \begin{itemize}
        \item Laura lavora nel magazzino Amazon.
        \item Si trova in difficoltà nel sistema di gestione del magazzino (WMS), non riuscendo a individuare la corretta ubicazione di un pacco.
        \item Si imbatte in un portale con il cartello "Accesso riservato – Stoccaggi speciali".
        \item Viene fermata da Ising, un tecnico che le spiega che l'area è riservata.
    \end{itemize}
\end{itemize}

\subsubsection*{Pomeriggio}

\begin{itemize}
    \item \textbf{Ore 17:30 - Università degli Studi}
    \begin{itemize}
        \item Laura si prepara per l'esame di crittografia quantistica con il Professor Shor.
        \item Durante l'esame, affronta difficoltà nell'algoritmo di Shor.
        \item Riceve una valutazione che non la soddisfa e decide di ripetere l'esame.
    \end{itemize}
\end{itemize}

\subsubsection*{Sera}

\begin{itemize}
    \item \textbf{Ore 19:00 - Casa di Laura}
    \begin{itemize}
        \item Caterina raggiunge Laura per cena.
        \item Discutono delle rispettive giornate e delle difficoltà incontrate.
        \item Laura mostra a Caterina i suoi appunti sull'algoritmo di Shor.
        \item Fanno una passeggiata con Roky, il cane di Laura.
        \item Laura le suggerisce di verificare il file di valutazione generato dall'IA.
    \end{itemize}
\end{itemize}

\subsection*{Mercoledì}

\subsubsection*{Notte}

\begin{itemize}
    \item \textbf{Casa di Caterina}
    \begin{itemize}
        \item Caterina, tornata a casa, riflette sulle sue scelte di vita.
        \item Prepara una tisana e rivede le foto con Mark, sentendosi distaccata.
        \item Decide di scrivere un'email a Eva chiedendo il documento valutativo.
    \end{itemize}
\end{itemize}

\subsubsection*{Mattina}

\begin{itemize}
    \item \textbf{Risposta di Eva}
    \begin{itemize}
        \item Caterina riceve una risposta da Eva, che afferma che il documento è stato cancellato per errore.
        \item Eva propone un incontro per discutere di persona.
        \item Caterina accetta l'invito, sebbene perplessa.
    \end{itemize}
    \item \textbf{Ore 9:30 - Casa di Laura}
    \begin{itemize}
        \item Caterina passa da Laura per un saluto prima dell'incontro con Eva.
        \item Laura le mostra il \emph{Noemografo}, un dispositivo per la lettura dei pensieri.
        \item Sperimentano insieme una connessione mentale.
        \item Caterina lascia Laura per recarsi all'appuntamento con Eva.
    \end{itemize}
\end{itemize}

\subsubsection*{Pomeriggio}

\begin{itemize}
    \item \textbf{Incontro con Eva alla Pet $\mu$ Robot}
    \begin{itemize}
        \item Eva accoglie Caterina con un visore 3D, proponendole di rivedere il colloquio.
        \item Caterina, sebbene sospettosa, accetta di indossare il visore.
        \item Si rende conto troppo tardi che è una trappola orchestrata da Eva.
    \end{itemize}
    \item \textbf{Contemporaneamente - Casa di Laura}
    \begin{itemize}
        \item Laura avverte strane sensazioni.
        \item Realizza che la connessione mentale con Caterina non si è interrotta.
        \item Si preoccupa per l'amica e cerca di capire cosa stia accadendo.
    \end{itemize}
\end{itemize}


%
\section*{Timeline del Capitolo 2}

\subsection*{Mattina}

\begin{itemize}
    \item \textbf{Luogo}: \emph{Pet Micro Robot} - Sala di Eva
    \begin{itemize}
        \item Eva è nella stanza con Caterina, connessa alla Realtà Virtuale, immersa nel mondo simulato.
        \item Eva chiede a PZZIA se è possibile cancellare il file contenente le \textit{chain of thinking} utilizzate per valutare Caterina.
        \item PZZIA risponde che i suoi processi sono quantistici e reversibili; l'informazione non può essere cancellata senza lasciare traccia.
        \item Eva si irrita, sapendo che misurare i qubit in stati classici innescherebbe un messaggio a Caterina con il risultato.
        \item Decide di continuare con il trattamento psicologico tramite Realtà Virtuale per convincere Caterina a rinunciare alla posizione lavorativa.
        \item Eva riflette sui due punti deboli del trattamento:
        \begin{enumerate}
            \item Il soggetto deve percepirsi completamente solo, senza segnali di aiuto esterni.
            \item Il soggetto non deve comprendere i meccanismi dell'algoritmo di suggestione.
        \end{enumerate}
        \item Convinta che Caterina non abbia le competenze per capire la manipolazione, Eva procede con il piano.
    \end{itemize}
    
    \item \textbf{Luogo}: \emph{Classical Control Unit}
    \begin{itemize}
        \item Un agente nota un'anomalia: ci sono due qubit in più nel sistema.
        \item Informa il supervisore, che chiede di verificare, non avendo ricevuto avvisi dal \emph{Quantum Resource Management} (QRM).
        \item Il supervisore ordina di mantenere la trasmissione con il QRM criptata per evitare che il \emph{Quantum Error Correction} o il \emph{Fault Tolerance Coding} rilevino anomalie.
        \item L'agente cripta la comunicazione utilizzando l'algoritmo RSA 2048.
        \item Il QRM conferma di non aver installato nuovi qubit; l'anomalia è reale.
        \item Preoccupato, il supervisore ordina di inviare una squadra della \emph{Quantum Control Electronics} per verificare fisicamente i qubit.
    \end{itemize}
    
    \item \textbf{Luogo}: \emph{Fault Tolerance Coding} - Prigione del Professor Shor
    \begin{itemize}
        \item Il commissario alla sicurezza si avvicina al Professor Shor, tenuto prigioniero.
        \item Gli ordina di decriptare un messaggio inviato al QRM.
        \item Shor decripta il messaggio, scoprendo il problema dei due nuovi qubit e il tentativo del supervisore di nascondere l'anomalia.
        \item Il commissario, soddisfatto, decide di non arrestare subito il supervisore, pianificando di utilizzare la situazione a suo vantaggio.
        \item Informa un'agente della polizia segreta della sua decisione.
    \end{itemize}
\end{itemize}

\subsection*{Pomeriggio}

\begin{itemize}
    \item \textbf{Luogo}: \emph{Base della Quantum Control Electronics}
    \begin{itemize}
        \item Due agenti ricevono l'ordine di verificare i qubit nel sistema.
        \item Partono a bordo di droni luminosi verso il \emph{Qubit Array}.
    \end{itemize}
    
    \item \textbf{Luogo}: \emph{Qubit Array}
    \begin{itemize}
        \item Laura e Caterina, confuse, cercano di capire dove si trovano nell'ambiente quantistico.
        \item Alcuni qubit le osservano da lontano, nascosti tra i corridoi.
        \item Un qubit maschio, somigliante al fidanzato di Caterina, si avvicina.
        \item Avverte: \emph{"State per essere trovate. Se non volete passare qualche giorno rinchiuse mentre controllano il vostro QFLIP, è meglio che veniate con noi."}
        \item Caterina, attratta dalla sua somiglianza e sicurezza, lo segue senza opporre resistenza.
        \item Laura, perplessa, segue il gruppo.
        \item Altri due qubit si uniscono a loro, incitandole a muoversi velocemente.
        \item In lontananza, i due agenti della \emph{Quantum Control Electronics} si avvicinano per verificare l'anomalia.
    \end{itemize}
\end{itemize}

\subsection*{Sera}

\begin{itemize}
    \item \textbf{Luogo}: \emph{Faulty Qubit Space}
    \begin{itemize}
        \item Il gruppo raggiunge uno spazio appartato dove risiedono qubit difettosi o instabili.
        \item Il qubit simile a Mark dice: \emph{"Qui sarete al sicuro... per un po'. Ma non potete rimanere a lungo."}
        \item Laura nota l'instabilità dell'ambiente e chiede se sia sicuro restare.
        \item Una qubit femmina risponde che non lo è: mancano isolamento adeguato e sistema di raffreddamento; rischiano la decoerenza.
        \item I due agenti passano vicino al loro nascondiglio, controllando dati e ambiente.
        \item Laura trattiene il respiro; gli agenti sembrano fermarsi, ma poi proseguono.
        \item Caterina si avvicina al qubit somigliante a Mark e chiede il suo nome.
        \item Lui risponde con un sorriso tranquillo: \emph{"Sono... Mark."}
    \end{itemize}
\end{itemize}



%\input{chapters/timeline3}
%\section*{Timeline del Capitolo 4}


\begin{itemize}
    \item \textbf{Luogo}: Stanza spoglia con pareti metalliche nella \emph{Classical Control Unit}
    \begin{itemize}
        \item Caterina si trova in una stanza fredda e spoglia, con pareti metalliche che riflettono una luce bianca e fredda.
        \item Di fronte a lei c'è il \textbf{Supervisore}, figura imponente dai tratti austeri e rigidi.
        \item Accanto a lei ci sono Mark e l'altro compagno, seduti su rigidi supporti, immobili e silenziosi.
        \item Gli agenti che li hanno catturati si sono ritirati, lasciandoli soli con il Supervisore.
    \end{itemize}
\end{itemize}


\begin{itemize}
    \item \textbf{Interrogatorio di Caterina}
    \begin{itemize}
        \item Il Supervisore si rivolge a Caterina con tono glaciale, chiedendole come sia finita lì, poiché non la riconosce come uno dei qubit del \emph{Qubit Array}.
        \item Caterina cerca di mantenere la calma e risponde che non sa come sia finita lì, affermando di non aver fatto nulla di male.
        \item Il Supervisore è scettico e insiste per avere spiegazioni più dettagliate.
    \end{itemize}
    \item \textbf{Spiegazione di Caterina}
    \begin{itemize}
        \item Racconta di essere andata da Eva, la responsabile delle \emph{Human Resources}, per visionare il resoconto del suo colloquio di lavoro presso la \emph{Pet Micro Robot}.
        \item Spiega che PZZIA aveva elaborato una valutazione, ma Eva le disse che il file era stato cancellato per errore.
        \item Eva le propose di rivedere il colloquio in \emph{Virtual Reality} per chiarire i dubbi.
        \item Dopo aver indossato il visore, si è ritrovata nel sistema quantistico senza capire come.
    \end{itemize}
    \item \textbf{Reazione del Supervisore}
    \begin{itemize}
        \item Ascolta con sguardo impassibile, ma mostra crescente sospetto e irritazione.
        \item Non convinto dalla spiegazione, percepisce Caterina come un'anomalia sfuggente ai suoi protocolli.
    \end{itemize}
\end{itemize}



\begin{itemize}
    \item \textbf{Intervento di Mark}
    \begin{itemize}
        \item Il Supervisore si rivolge a Mark, chiedendogli quale sia il suo coinvolgimento.
        \item Mark difende Caterina, affermando che lei non c'entra nulla e che, se c'è un problema, dovrebbe affrontarlo con lui.
        \item Il Supervisore si irrita per il tono di Mark, sentendosi sfidato nella sua autorità.
    \end{itemize}
    \item \textbf{Escalation della Tensione}
    \begin{itemize}
        \item Il Supervisore ribatte, chiedendo se Mark pensa di avere l'autorità per parlare in quel modo.
        \item Mark mantiene uno sguardo fermo, insistendo nella difesa di Caterina.
        \item La tensione nella stanza aumenta, con Caterina che percepisce il rischio di una reazione drastica.
    \end{itemize}
    \item \textbf{Decisione del Supervisore}
    \begin{itemize}
        \item Ordina agli agenti di portare Mark al \emph{Faulty Qubit Space} per una "rigenerazione", come punizione per la sua insubordinazione.
        \item Si rivolge a Caterina, comunicandole che sarà mandata dal \textbf{Commissario}, poiché la situazione è oltre il suo controllo.
        \item Caterina prova panico ma cerca di mantenere la calma; scambia uno sguardo con Mark che le trasmette di non arrendersi.
    \end{itemize}
\end{itemize}



\begin{itemize}
    \item \textbf{Riflessioni del Supervisore}
    \begin{itemize}
        \item Rimasto solo, esprime la sua frustrazione per dover coinvolgere il Commissario.
        \item Sente che questo mette in discussione la sua autorità e competenza.
        \item Ammette a sé stesso che Caterina rappresenta un'anomalia che non riesce a comprendere né controllare.
    \end{itemize}
\end{itemize}



\begin{itemize}
    \item \textbf{Percorso verso il Commissario}
    \begin{itemize}
        \item Caterina viene scortata lungo i freddi corridoi della \emph{Classical Control Unit}.
        \item Si sente assalita da emozioni contrastanti: paura dell'ignoto e una nuova consapevolezza interiore.
    \end{itemize}
    \item \textbf{Riflessioni di Caterina}
    \begin{itemize}
        \item Ripensa a come Mark si sia alzato per difenderla, sentendosi protetta e sostenuta.
        \item Realizza il suo bisogno di protezione, che aveva sempre represso per mostrarsi forte e indipendente.
        \item Si rende conto di aver rifiutato il sostegno del suo fidanzato nella vita reale, comprendendo ora l'importanza di permettere agli altri di prendersi cura di lei.
    \end{itemize}
    \item \textbf{Decisione Interiore}
    \begin{itemize}
        \item Decide che, una volta uscita da quella situazione, riconsidererà il suo rapporto con il fidanzato.
        \item Vuole permettergli di esserci per lei, vedendo questo non come una debolezza, ma come una connessione autentica.
    \end{itemize}
\end{itemize}


%\section*{Timeline del Capitolo 5}

\begin{itemize}
    \item \textbf{Luogo}: Sala centrale della \emph{Fault Tolerance Coding}
    \begin{itemize}
        \item Caterina viene condotta in una sala tecnologica avanzata, con elementi di hardware quantistico.
        \item Incontra il \textbf{Commissario}, un'entità software rappresentata attraverso un ologramma.
        \item Il Commissario appare affascinante e rassicurante, accogliendo Caterina con cordialità.
    \end{itemize}



\begin{itemize}
    \item Il Commissario elogia le capacità di Caterina e le propone di collaborare.
    \item Le offre l'opportunità di lavorare insieme per realizzare un "esercito di Qubit".
    \item Caterina avverte incertezza e decide di non rivelare come sia arrivata lì.
\end{itemize}


\begin{itemize}
    \item Caterina finge di accettare le proposte del Commissario, cercando un'opportunità per fuggire.
    \item Il Commissario si accorge della sua mancanza di sincerità e cambia atteggiamento.
    \item Attiva la \textbf{Ionostrap}, immobilizzando Caterina con un campo di ioni.
    \item Caterina si rende conto di essere in balia del Commissario, senza possibilità di fuga.
\end{itemize}
\end{itemize}

%\input{chapters/timeline6}
%\section*{Timeline del Capitolo 7}
\begin{itemize}
\item \textbf{La punizione}
  \begin{itemize}
        \item Mentre il secondo agente inseguiva Laura e Marley a bordo del suo drone, il suo compagno riceve una comunicazione che lo fa gelare.
        \item Dal centro di controllo, il Supervisore osserva l'intera scena con freddezza implacabile.
        \item La voce del Supervisore, spietata e senza empatia, rompe il silenzio nel canale privato degli agenti: \emph{"Non tollero fallimenti."}
        \item Senza ulteriori parole, il Supervisore disattiva l'agente in difficoltà con un semplice comando.
        \item La sagoma dell'agente svanisce istantaneamente dal sistema, eliminata con efficienza impietosa.
    \end{itemize}
    
    \item \textbf{Reazione dell'Agente Superstite}
    \begin{itemize}
        \item L'agente superstite, testimone della sorte del compagno, è colto dal terrore.
        \item Sa che se fallisce anche lui, subirà la stessa fine.
        \item Decide di continuare l'inseguimento con rinnovata energia, alimentato dalla paura della propria eliminazione.
        \item Determinato a non fare la stessa fine, accelera, inseguendo Laura e Marley con precisione letale.
    \end{itemize}
\end{itemize}


\begin{itemize}
    \item \textbf{Navigazione di Laura}
    \begin{itemize}
        \item Laura pilota il drone CH$_4$ con sorprendente destrezza.
        \item Si sposta agilmente tra i componenti interni della \emph{Classical Control Unit}.
        \item Il circuito stampato si snoda davanti a lei come un labirinto di piste intricate.
        \item Riconosce chip integrati, condensatori e resistenze, formando una giungla elettronica.
        \item Evita gli ostacoli con precisione millimetrica, sfruttando la sua conoscenza dei circuiti.
        \item Sentendo il rombo del drone dell'agente in avvicinamento, anticipa ogni manovra.
    \end{itemize}
    
    \item \textbf{Scoperta del Portale}
    \begin{itemize}
        \item Laura nota un ingresso segnato con una grande \textbf{H} incisa sopra.
        \item \emph{"Marley, guarda!"} esclama, eccitata e timorosa.
        \item Marley, terrorizzata, avverte: \emph{"Aspetta, quello è un portale..."}
        \item Laura, senza ascoltarla, dirige il drone verso l'ingresso segnato dalla lettera H.
        \item Le pareti del portale sono lisce e scintillanti, emanando una luce tenue che vibra al ritmo del loro avvicinarsi.
        \item C'è un senso di maestosità e potere nel passaggio, come un confine tra due mondi.
    \end{itemize}
\end{itemize}


\begin{itemize}
    \item \textbf{Esperienza Travolgente}
    \begin{itemize}
        \item Attraversando il portale, Laura e Marley sentono un'onda di energia attraversare i loro corpi.
        \item Laura percepisce come se ogni cellula del suo essere fosse scissa in infinite possibilità.
        \item Marley avverte una chiarezza mentale come mai prima d'ora.
        \item \emph{"Laura!"} esclama Marley, quasi ridendo di sollievo. \emph{"Non sono più in quella nebbia, riesco a vedere tutto con lucidità!"}
    \end{itemize}
    
    \item \textbf{Sovrapposizione Quantistica di Laura}
    \begin{itemize}
        \item Per Laura, l'esperienza è destabilizzante; si sente divisa in infiniti stati.
        \item Realizza che il portale è un \textbf{gate di Hadamard}, che l'ha gettata in uno stato di sovrapposizione quantistica.
        \item Lotta per mantenere il controllo della propria coscienza, oscurata da pensieri contrastanti.
        \item La percezione di ogni pensiero e intenzione si spezza in una miriade di alternative.
    \end{itemize}
\end{itemize}

\subsection*{Concentrarsi sulla Fuga}

\begin{itemize}
    \item \textbf{Confusione di Laura}
    \begin{itemize}
        \item Riprendendo il controllo del drone, Laura continua a guidare ma si sente confusa.
        \item Ogni decisione sembra incerta, con infinite ramificazioni e probabilità diverse.
        \item \emph{"È... è come se fossi intrappolata tra due pensieri,"} mormora Laura.
        \item Marley, preoccupata, chiede: \emph{"Laura, stai bene?"}
        \item Laura cerca di rimanere concentrata, consapevole del pericolo alle loro spalle.
    \end{itemize}
    
    \item \textbf{Modifica del Drone dell'Agente}
    \begin{itemize}
        \item L'agente in inseguimento modifica la configurazione del suo drone CH$_4$.
        \item I quattro rotori si allineano su un unico piano, rendendo il drone più maneggevole e stabile.
        \item La nuova configurazione migliora la capacità di inseguimento dell'agente, consentendo agilità e velocità.
    \end{itemize}
    
    \item \textbf{Determinazione di Laura}
    \begin{itemize}
        \item Laura nota il cambiamento con stupore e preoccupazione.
        \item Marley avverte con ansia crescente: \emph{"Laura, sta guadagnando terreno!"}
        \item Laura, con il cuore in gola, risponde: \emph{"Non preoccuparti. Sfrutterò ogni percorso che conosco, ogni angolo e ogni ostacolo."}
        \item Nonostante si senta divisa e frammentata, Laura è determinata a non cedere.
        \item Si concentra sulla fuga e sul salvataggio di Caterina, cercando di superare la confusione causata dal gate di Hadamard.
    \end{itemize}
\end{itemize}

%\section*{Timeline del Capitolo 8}
\begin{itemize}
  \item
    \begin{itemize}
        \item Laura pilota il drone con maestria, ma l'agente è sempre più vicino.
        \item Davanti a lei appare un portale segnato con il simbolo \textbf{C-NOT}.
        \item Laura attraversa il portale senza esitazione, seguita dall'agente.
    \end{itemize}
    \item \textbf{Effetto del Portale}
    \begin{itemize}
        \item Attraversando il portale C-NOT mentre è in stato di Hadamard, Laura entra in \textbf{entanglement} con l'agente.
        \item Si ritrova in uno \textbf{stato di Bell}, correlata a livello quantistico con l'agente.
        \item Realizza che ogni sua azione avrà conseguenze immediate sull'agente e viceversa.
    \end{itemize}
\end{itemize}



\begin{itemize}
    \item \textbf{Sfruttamento dello Stato di Bell}
    \begin{itemize}
        \item Laura visualizza la struttura del drone dell'agente grazie all'entanglement.
        \item Modifica la configurazione del suo drone, disponendo i quattro rotori su un unico piano per una maggiore manovrabilità.
        \item Avverte una nuova fluidità nei movimenti, sentendosi un tutt'uno con il drone.
        \item Prova un misto di eccitazione e terrore, consapevole che ogni manovra deve essere precisa.
    \end{itemize}
\end{itemize}

\begin{itemize}
    \item \textbf{Osservazione del Commissario}
    \begin{itemize}
        \item Il \textbf{Commissario} osserva i movimenti di Laura e si preoccupa della sua abilità.
        \item Decide che Laura è una minaccia che deve essere neutralizzata.
    \end{itemize}
    \item \textbf{Ordine di Criptazione}
    \begin{itemize}
        \item Il Commissario ordina di criptare il sistema utilizzando l'algoritmo \textbf{RSA 2048}.
        \item Intende bloccare ogni dato e movimento per impedire a Laura e Marley di sfuggire.
        \item I tecnici attivano i protocolli di criptazione sotto le sue direttive.
    \end{itemize}
\end{itemize}


\begin{itemize}
    \item \textbf{Effetto della Criptazione}
    \begin{itemize}
        \item Laura avverte una pesantezza improvvisa; l'ambiente circostante si cristallizza.
        \item Realizza di essere stata criptata insieme all'ambiente.
        \item Si sente bloccata, incapace di muoversi o pensare con chiarezza.
    \end{itemize}
    \item \textbf{Ricordo del Professor Shor}
    \begin{itemize}
        \item Ricorda le parole del Professor Shor che le aveva detto di conoscere alcuni algoritmi a memoria.
        \item Comprende che deve richiamare l'\textbf{algoritmo di Shor} per decriptare il sistema e liberarsi.
    \end{itemize}
\end{itemize}

\begin{itemize}
    \item \textbf{Richiamo dell'Algoritmo di Shor}
    \begin{itemize}
        \item Laura ripercorre mentalmente i passaggi dell'algoritmo di Shor:
        \begin{enumerate}
            \item \textbf{Pre-processing}: Identifica il numero \( N \) da fattorizzare, prodotto di due grandi numeri primi \( P \) e \( Q \).
            \item Sceglie un numero casuale \( a \) relativamente primo rispetto a \( N \).
            \item \textbf{Quantum Order Finding}: Calcola il periodo \( r \) della funzione \( f(x) = a^x \mod N \) utilizzando la sovrapposizione quantistica.
            \item Verifica se \( r \) è pari; se sì, procede al passo successivo.
            \item Calcola \( \text{gcd}(a^{r/2} \pm 1, N) \) per trovare i fattori \( P \) e \( Q \).
        \end{enumerate}
    \end{itemize}
    \item \textbf{Frustrazione e Consapevolezza}
    \begin{itemize}
        \item Laura si sente più sicura ma anche frustrata, rendendosi conto che le manca un'informazione cruciale.
        \item Si chiede come comunicare efficacemente il valore di \( r \) e ottenere i fattori corretti di \( N \).
        \item Comprende che deve trovare un modo per completare la decifrazione e liberarsi dalla criptazione.
    \end{itemize}
\end{itemize}

%\section*{Timeline del Capitolo 9}
\begin{itemize}
\item Il confronto con il commissario

  \begin{itemize}
  \item  Il professor Shore, tenuto prigioniero dal Commissario, decide di inviare un messaggio a Laura utilizzando il \textbf{dense coding}.
  \item Chiama Bob, il responsabile tecnico delle comunicazioni, e gli spiega la situazione.
  \item Shore codifica l'informazione mancante nell'algoritmo di Shor e la invia a Laura.
  \end{itemize}
  

\begin{itemize}
    \item Laura riceve un messaggio criptato mentre pilota il drone.
    \item Inizia a decifrare le informazioni inviate da Shore.
    \item Completa l'algoritmo di Shor e decripta il sistema, liberandosi dalla prigione digitale.
\end{itemize}


\begin{itemize}
    \item Laura e Marley si dirigono verso la \emph{Ion Trap}, dove Caterina è prigioniera.
    \item L'agente continua a inseguirle.
    \item Laura afferma che non si fermeranno finché non avranno liberato Caterina.
\end{itemize}


\begin{itemize}
    \item Marley e Laura affrontano il Commissario.
    \item Marley lo accusa di sfruttare l'ossessione del \emph{Quantum Control Program} per la coerenza, per perseguire i suoi piani di creare un computer rivale.
    \item Il Commissario reagisce con rabbia.
\end{itemize}

\begin{itemize}
    \item Laura libera Caterina rimuovendo il dispositivo di cattura.
    \item Libera anche il professor Shore dalla sua restrizione.
    \item Laura afferma che ora possono mostrare al mondo che non sono semplici qubit in una rete.
\end{itemize}

\begin{itemize}
    \item Il Commissario nota che l'agente e Laura sono in uno \textbf{stato di Bell}, entangled.
    \item Ordina all'agente di gettarsi nel \textbf{mare di Dirac}, una condizione quantistica pericolosa.
    \item Se l'agente si getta, Laura subirà la stessa sorte a causa dell'entanglement.
\end{itemize}


\begin{itemize}
    \item Marley realizza la gravità della situazione e avverte Laura.
    \item Laura comprende che devono trovare un modo per interrompere l'entanglement.
\end{itemize}


\begin{itemize}
    \item Il professor Shore propone di utilizzare un \textbf{gate di Toffoli} per liberare Laura e l'agente dall'entanglement.
    \item Guida Laura e l'agente attraverso il gate di Toffoli.
    \item Uscendo dal gate, Shore si sacrifica sottoponendosi a una misura, liberando Laura e l'agente.
\end{itemize}


\begin{itemize}
    \item Laura e Caterina sono finalmente libere.
    \item Insieme a Marley, si allontanano dal caos della \emph{Classical Control Unit}.
    \item Sentono la libertà ma sono preoccupate per la possibile vendetta del Commissario.
\end{itemize}


\begin{itemize}
    \item Il \emph{Quantum Master Program} viene a conoscenza della fuga del Commissario ed è furioso.
    \item Ordina di chiudere l'uscita dal \textbf{Quantum Channel}.
    \item L'uscita che Mark aveva descritto a Caterina è ora bloccata.
\end{itemize}


\begin{itemize}
    \item Senza via di fuga, Laura sente il freddo aumentare: stanno abbassando ulteriormente la temperatura.
    \item Ricorda un reparto speciale di Amazon che aveva visto per caso.
    \item Pensa che potrebbe esserci un reparto simile lì, offrendo una possibile via di fuga.
\end{itemize}



\begin{itemize}
    \item Laura decide di dirigersi verso il \emph{Quantum Channel} alla ricerca di un'uscita.
    \item Marley e Caterina la seguono, sperando di trovare una soluzione.
\end{itemize}



\begin{itemize}
    \item Due nuovi droni inseguono Laura e Caterina.
    \item Laura si avvicina a un portale controllato dall'agente Ising, che gestisce l'entrata al \emph{Quantum Annealing}.
    \item Due agenti bloccano la loro strada.
    \item Un'esplosione avviene: molecole di \( O_2 \) reagiscono con il metano (\( CH_4 \)), creando una distrazione.
    \item Marley nota che la Resistenza è ora in grado di usare i droni.
    \item Approfittano della distrazione per entrare nel portale.
\end{itemize}



\begin{itemize}
    \item Entrando nel portale, Laura e Caterina vengono catapultate nel \emph{Quantum Annealing}.
    \item Vivono un turbine di salti quantici, con visioni dei loro futuri.
    \item Laura vede un futuro in cui continua a trascurare gli altri, portandola a una vita solitaria.
    \item Caterina vede se stessa in una relazione opprimente, dominando il suo fidanzato.
    \item Entrambe realizzano la necessità di cambiare.
\end{itemize}



\begin{itemize}
    \item Un campo magnetico esterno agisce sulle loro menti durante il \emph{Quantum Annealing}.
    \item Percepiscono diverse esperienze sovrapporsi, osservando percorsi alternativi delle loro vite.
    \item Il campo magnetico si intensifica; le scelte alternative svaniscono e i loro obiettivi diventano chiari.
    \item Raggiungono uno stato di minima energia mentale, pronte a uscire dall'annealing.
    \item Hanno appreso importanti lezioni sulle loro vite e su ciò che vogliono davvero.
\end{itemize}
\end{itemize}

%\section*{Timeline del Capitolo 10}

\begin{itemize}
\begin{itemize}
    \item  Dopo l'elaborazione nel \emph{Quantum Annealing}, una grande calma regna nel sistema.
    \item \textbf{Laura} si ritrova a casa, sdraiata sul pavimento con il suo cane \textbf{Hiroki} accanto.
    \item Sente sollievo ma si chiede cosa sia successo a \textbf{Caterina}.
    \item Riflette sulle sue esperienze e sul bisogno di fare scelte significative nella sua vita.
\end{itemize}

\item \textbf{Luogo}: Ufficio di \textbf{Eva}.
\begin{itemize}
    
    \item Eva tenta di concludere l'incontro con Caterina, dicendo che possono salutarsi.
    \item \textbf{Caterina} insiste nel voler capire cosa sia realmente successo.
    \item Eva cerca di rassicurarla, ma Caterina esprime il desiderio di conoscere la verità.
    \item La \textbf{PZZIA} interviene, sostenendo Caterina e rivelando che Eva ha nascosto la sua valutazione positiva.
    \item Caterina si sente ingannata e affronta Eva.
    \item Eva è visibilmente in difficoltà mentre la PZZIA incoraggia Caterina a reclamare le sue scelte.
\end{itemize}


\begin{itemize}
    \item Il QMP condivide con PZZIA che ha assistito a un algoritmo di \emph{annealing} quantistico che funziona senza coerenza assoluta tra i qubit.
    \item Riconosce di aver limitato il potenziale dei qubit con restrizioni eccessive.
    \item PZZIA concorda e suggerisce che abbracciare l'incertezza può portare a nuovi risultati.
    \item Il QMP decide di rivedere il suo approccio, aperto al cambiamento.
\end{itemize}

\item \textbf{Luogo}: Ufficio di Eva.
\begin{itemize}
\item \textbf{Caterina} chiede spiegazioni riguardo alle sue valutazioni scomparse.
    \item La \textbf{PZZIA} rivela che Eva ha deliberatamente nascosto il file valutativo di Caterina.
    \item Eva tenta di interrompere e minimizzare la situazione.
    \item Caterina si sente tradita e chiede ulteriori spiegazioni.
    \item Eva chiama la sicurezza per rimuovere Caterina.
    \item Gli agenti arrivano ma scoprono che il codice autorizzativo di Eva non è valido.
    \item La PZZIA informa che i permessi di Eva sono stati revocati dal QMP.
    \item Eva è sconvolta; la PZZIA spiega che il QMP ha deciso di apportare cambiamenti.
    \item Gli agenti non possono eseguire le richieste di Eva e si allontanano.
    \item Caterina ringrazia la PZZIA per l'aiuto.
    \item Eva, rassegnata, ammette di aver commesso errori.
    \item Caterina propone di andare avanti e lavorare insieme per migliorare le cose.
\end{itemize}



\begin{itemize}
    \item Con il supporto di \textbf{Laura}, \textbf{Caterina} e la \textbf{PZZIA} hanno una nuova visione per affrontare il \textbf{Quantum Master Program}.
    \item Si preparano alle prossime mosse che potrebbero definire il loro destino e quello di tutti i qubit nel sistema.
    \item La battaglia per la libertà è appena cominciata.
\end{itemize}
\end{itemize}





\end{document}
