
\section*{Timeline del Capitolo 1}

\subsection*{Lunedì}

\subsubsection*{Mattina}

\begin{itemize}
    \item \textbf{Ore 9:30 - Pet $\mu$ Robot}
    \begin{itemize}
        \item Caterina si presenta al colloquio presso la \emph{Pet Micro Robot} per una posizione di responsabile marketing.
        \item Viene sottoposta a una preselezione guidata dall'IA PZZIA.
        \item Durante la seconda fase del colloquio, Eva, responsabile delle risorse umane, le pone domande sull'ambiente, sul cambiamento climatico e sull'intelligenza artificiale nelle aziende.
        \item Eva le assegna inaspettatamente un test di programmazione avanzata.
        \item Caterina completa il test ma con alcune incertezze.
    \end{itemize}
\end{itemize}

\subsubsection*{Pomeriggio}

\begin{itemize}
    \item \textbf{Caffetteria all'angolo}
    \begin{itemize}
        \item Dopo il colloquio, Caterina incontra Laura.
        \item Si recano insieme in una caffetteria per discutere dell'esperienza.
        \item Laura aiuta Caterina a risolvere l'algoritmo del test di programmazione, alleviando le sue preoccupazioni.
    \end{itemize}
\end{itemize}

\subsection*{Martedì}

\subsubsection*{Mattina}

\begin{itemize}
    \item \textbf{Ore 12:30 - Magazzino Amazon}
    \begin{itemize}
        \item Laura lavora nel magazzino Amazon.
        \item Si trova in difficoltà nel sistema di gestione del magazzino (WMS), non riuscendo a individuare la corretta ubicazione di un pacco.
        \item Si imbatte in un portale con il cartello "Accesso riservato – Stoccaggi speciali".
        \item Viene fermata da Ising, un tecnico che le spiega che l'area è riservata.
    \end{itemize}
\end{itemize}

\subsubsection*{Pomeriggio}

\begin{itemize}
    \item \textbf{Ore 17:30 - Università degli Studi}
    \begin{itemize}
        \item Laura si prepara per l'esame di crittografia quantistica con il Professor Shor.
        \item Durante l'esame, affronta difficoltà nell'algoritmo di Shor.
        \item Riceve una valutazione che non la soddisfa e decide di ripetere l'esame.
    \end{itemize}
\end{itemize}

\subsubsection*{Sera}

\begin{itemize}
    \item \textbf{Ore 19:00 - Casa di Laura}
    \begin{itemize}
        \item Caterina raggiunge Laura per cena.
        \item Discutono delle rispettive giornate e delle difficoltà incontrate.
        \item Laura mostra a Caterina i suoi appunti sull'algoritmo di Shor.
        \item Fanno una passeggiata con \textbf{Hiroki}, il cane di Laura.
        \item Laura le suggerisce di verificare il file di valutazione generato dall'IA.
    \end{itemize}
\end{itemize}

\subsection*{Mercoledì}

\subsubsection*{Notte}

\begin{itemize}
    \item \textbf{Casa di Caterina}
    \begin{itemize}
        \item Caterina, tornata a casa, riflette sulle sue scelte di vita.
        \item Prepara una tisana e rivede le foto con Mark, sentendosi distaccata.
        \item Decide di scrivere un'email a Eva chiedendo il documento valutativo.
    \end{itemize}
\end{itemize}

\subsubsection*{Mattina}

\begin{itemize}
    \item \textbf{Risposta di Eva}
    \begin{itemize}
        \item Caterina riceve una risposta da Eva, che afferma che il documento è stato cancellato per errore.
        \item Eva propone un incontro per discutere di persona.
        \item Caterina accetta l'invito, sebbene perplessa.
    \end{itemize}
    \item \textbf{Ore 9:30 - Casa di Laura}
    \begin{itemize}
        \item Caterina passa da Laura per un saluto prima dell'incontro con Eva.
        \item Laura le mostra il \emph{Noemografo}, un dispositivo per la lettura dei pensieri.
        \item Sperimentano insieme una connessione mentale.
        \item Caterina lascia Laura per recarsi all'appuntamento con Eva.
    \end{itemize}
\end{itemize}

\subsubsection*{Pomeriggio}

\begin{itemize}
    \item \textbf{Incontro con Eva alla Pet $\mu$ Robot}
    \begin{itemize}
        \item Eva accoglie Caterina con un visore 3D, proponendole di rivedere il colloquio.
        \item Caterina, sebbene sospettosa, accetta di indossare il visore.
        \item Si rende conto troppo tardi che è una trappola orchestrata da Eva.
    \end{itemize}
    \item \textbf{Contemporaneamente - Casa di Laura}
    \begin{itemize}
        \item Laura avverte strane sensazioni.
        \item Realizza che la connessione mentale con Caterina non si è interrotta.
        \item Si preoccupa per l'amica e cerca di capire cosa stia accadendo.
    \end{itemize}
\end{itemize}

\section*{Timeline del Capitolo 2}

\subsection*{Mattina}

\begin{itemize}
    \item \textbf{Luogo}: \emph{Pet Micro Robot} - Sala di Eva
    \begin{itemize}
        \item Eva si trova nella stanza con Caterina, connessa alla Realtà Virtuale.
        \item Chiede a PZZIA se è possibile cancellare il file contenente le \emph{chain of thinking} utilizzate per valutare Caterina.
        \item PZZIA risponde che i suoi processi sono quantistici e reversibili; l'informazione non può essere cancellata senza lasciare traccia.
        \item Eva si irrita, sapendo che misurare i qubit in stati classici innescherebbe un messaggio a Caterina con il risultato.
        \item Decide di continuare con il trattamento psicologico tramite Realtà Virtuale per convincere Caterina a rinunciare alla posizione lavorativa.
        \item Riflette sui due punti deboli del trattamento:
        \begin{enumerate}
            \item Il soggetto deve percepirsi completamente solo, senza segnali di aiuto esterni.
            \item Il soggetto non deve comprendere i meccanismi dell'algoritmo di suggestione.
        \end{enumerate}
        \item Convinta che Caterina non abbia le competenze per capire la manipolazione, Eva procede con il piano.
    \end{itemize}

    \item \textbf{Luogo}: \emph{Classical Control Unit}
    \begin{itemize}
        \item Un agente nota un'anomalia: ci sono due qubit in più nel sistema.
        \item Informa il Supervisore, che chiede di verificare, non avendo ricevuto avvisi dal \emph{Quantum Resource Management} (QRM).
        \item Il Supervisore ordina di mantenere la trasmissione con il QRM criptata per evitare che il \emph{Quantum Error Correction} o il \emph{Fault Tolerance Coding} rilevino anomalie.
        \item L'agente cripta la comunicazione utilizzando l'algoritmo RSA 2048.
        \item Il QRM conferma di non aver installato nuovi qubit; l'anomalia è reale.
        \item Preoccupato, il Supervisore ordina di inviare una squadra della \emph{Quantum Control Electronics} per verificare fisicamente i qubit.
    \end{itemize}

    \item \textbf{Luogo}: \emph{Fault Tolerance Coding} - Prigione del Professor Shor
    \begin{itemize}
        \item Il Commissario alla sicurezza si avvicina al Professor Shor, tenuto prigioniero.
        \item Gli ordina di decriptare un messaggio inviato al QRM.
        \item Shor decripta il messaggio, scoprendo il problema dei due nuovi qubit e il tentativo del Supervisore di nascondere l'anomalia.
        \item Il Commissario, soddisfatto, decide di non arrestare subito il Supervisore, pianificando di utilizzare la situazione a suo vantaggio.
        \item Informa un agente della polizia segreta della sua decisione.
    \end{itemize}
\end{itemize}

\subsection*{Pomeriggio}

\begin{itemize}
    \item \textbf{Luogo}: \emph{Base della Quantum Control Electronics}
    \begin{itemize}
        \item Due agenti ricevono l'ordine di verificare i qubit nel sistema.
        \item Partono a bordo di droni luminosi verso il \emph{Qubit Array}.
    \end{itemize}

    \item \textbf{Luogo}: \emph{Qubit Array}
    \begin{itemize}
        \item Laura e Caterina, confuse, cercano di capire dove si trovano nell'ambiente quantistico.
        \item Alcuni qubit le osservano da lontano, nascosti tra i corridoi.
        \item Un qubit maschio, somigliante al fidanzato di Caterina, si avvicina.
        \item Avverte: \emph{"State per essere trovate. Se non volete passare qualche giorno rinchiuse mentre controllano il vostro QFLIP, è meglio che veniate con noi."}
        \item Caterina, attratta dalla sua somiglianza e sicurezza, lo segue senza opporre resistenza.
        \item Laura, perplessa, segue il gruppo.
        \item Altri due qubit si uniscono a loro, incitandole a muoversi velocemente.
        \item In lontananza, i due agenti della \emph{Quantum Control Electronics} si avvicinano per verificare l'anomalia.
    \end{itemize}
\end{itemize}

\subsection*{Sera}

\begin{itemize}
    \item \textbf{Luogo}: \emph{Faulty Qubit Space}
    \begin{itemize}
        \item Il gruppo raggiunge uno spazio appartato dove risiedono qubit difettosi o instabili.
        \item Il qubit simile a Mark dice: \emph{"Qui sarete al sicuro... per un po'. Ma non potete rimanere a lungo."}
        \item Laura nota l'instabilità dell'ambiente e chiede se sia sicuro restare.
        \item Una qubit femmina risponde che non lo è: mancano isolamento adeguato e sistema di raffreddamento; rischiano la decoerenza.
        \item I due agenti passano vicino al loro nascondiglio, controllando dati e ambiente.
        \item Laura trattiene il respiro; gli agenti sembrano fermarsi, ma poi proseguono.
        \item Caterina si avvicina al qubit somigliante a Mark e chiede il suo nome.
        \item Lui risponde con un sorriso tranquillo: \emph{"Sono... Mark."}
    \end{itemize}
\end{itemize}

\section*{Timeline del Capitolo 3}

\subsection*{Mattina}

\begin{itemize}
    \item \textbf{Luogo}: \emph{Faulty Qubit Space}
    \begin{itemize}
        \item Laura e Caterina si trovano nel \emph{Faulty Qubit Space}, un limbo per qubit instabili destinati a essere eliminati se non riescono a mantenere la coerenza.
        \item Marley è accanto a loro, con il volto serio e pensieroso, osservando gli altri qubit rassegnati al loro destino.
        \item Laura avverte tensione e afferra il braccio di Caterina, notando la paura nell'amica.
    \end{itemize}
\end{itemize}

\subsection*{Pomeriggio}

\begin{itemize}
    \item \textbf{Luogo}: \emph{Faulty Qubit Space}
    \begin{itemize}
        \item Mark, Marley e un altro compagno si avvicinano a Laura e Caterina.
        \item Mark dice loro di rimanere nascoste; lui e l'altro proveranno a raggiungere un circuito periferico.
        \item Spiega che devono aggiungere un \emph{Quantum Teleportation Buffer} per evitare che l'entanglement leghi ulteriormente al \emph{Faulty Qubit Space}.
        \item Caterina esprime preoccupazione per Mark, che la rassicura prima di allontanarsi nell'ombra.
    \end{itemize}
\end{itemize}

\subsection*{Sera}

\begin{itemize}
    \item \textbf{Luogo}: \emph{Faulty Qubit Space}
    \begin{itemize}
        \item Rimaste sole con Marley, Laura e Caterina discutono delle loro preoccupazioni.
        \item Caterina chiede cosa stia realmente accadendo; Laura cerca di rassicurarla, ma non ha risposte certe.
        \item Marley appare tesa; il tempo nel rifugio è limitato.
    \end{itemize}

    \item \textbf{Evento}: Arrivo degli Agenti
    \begin{itemize}
        \item Una luce rossa intermittente attraversa lo spazio, seguita da passi veloci e decisi.
        \item Marley avverte: \emph{"Gli agenti."}
        \item Spinge Laura e Caterina più in fondo al \emph{Faulty Qubit Space} per nascondersi.
        \item Laura vede Mark e il suo compagno fermarsi mentre cercano di collegare il circuito periferico.
        \item Due agenti li circondano; Mark tenta di difendersi ma viene immobilizzato.
        \item Caterina corre verso Mark per aiutarlo, nonostante le proteste di Laura: \emph{"Caterina, fermati!"}
        \item L'altro agente afferra Caterina, legandole i polsi; anche lei viene arrestata.
        \item Laura prova angoscia, ma Marley la trascina via.
    \end{itemize}
\end{itemize}

\subsection*{Notte}

\begin{itemize}
    \item \textbf{Luogo}: Fuga verso il \emph{Quantum Measurement}
    \begin{itemize}
        \item Marley dice a Laura che non possono fare nulla per gli altri ora.
        \item Laura, con gli occhi pieni di lacrime, viene guidata via da Marley.
        \item Chiede: \emph{"Dove andiamo?"}
        \item Marley risponde: \emph{"Al Quantum Measurement. È pericoloso, ma è l'unico posto dove gli agenti non potranno seguire le nostre tracce così facilmente."}
        \item Entrano nel \emph{Quantum Measurement}; l'atmosfera è sospesa tra realtà e astrazione.
        \item Laura avverte una strana pressione nella testa, come se ogni pensiero potesse far collassare l'intero sistema.
    \end{itemize}

    \item \textbf{Evento}: Inseguiti dai Droni CH\textsubscript{4}
    \begin{itemize}
        \item Il rumore dei droni CH\textsubscript{4} si avvicina.
        \item Laura dice con voce tremante: \emph{"Ci hanno trovate."}
        \item Marley esorta Laura a restare calma: \emph{"Devi restare calma."}
        \item Le luci dei droni illuminano l'oscurità; il collasso dello stato è imminente.
        \item Sanno che il \emph{Quantum Measurement} è estremamente instabile; un errore potrebbe essere fatale.
        \item Marley suggerisce: \emph{"Se dobbiamo restare qui, faremo in modo di non venire rilevate."}
        \item Laura annuisce, determinata a lottare fino alla fine per salvare Caterina e se stessa.
    \end{itemize}
\end{itemize}

\section*{Timeline del Capitolo 4}

\subsection*{Mattina}

\begin{itemize}
    \item \textbf{Luogo}: Stanza spoglia con pareti metalliche nella \emph{Classical Control Unit}
    \begin{itemize}
        \item Caterina si trova in una stanza fredda e spoglia, con pareti metalliche che riflettono una luce bianca e fredda.
        \item Di fronte a lei c'è il \textbf{Supervisore}, figura imponente dai tratti austeri e rigidi.
        \item Accanto a lei ci sono Mark e l'altro compagno, seduti su rigidi supporti, immobili e silenziosi.
        \item Gli agenti che li hanno catturati si sono ritirati, lasciandoli soli con il Supervisore.
    \end{itemize}
\end{itemize}

\subsection*{Colloquio con il Supervisore}

\begin{itemize}
    \item \textbf{Interrogatorio di Caterina}
    \begin{itemize}
        \item Il Supervisore si rivolge a Caterina con tono glaciale, chiedendole come sia finita lì, poiché non la riconosce come uno dei qubit del \emph{Qubit Array}.
        \item Caterina cerca di mantenere la calma e risponde che non sa come sia finita lì, affermando di non aver fatto nulla di male.
        \item Il Supervisore è scettico e insiste per avere spiegazioni più dettagliate.
    \end{itemize}

    \item \textbf{Spiegazione di Caterina}
    \begin{itemize}
        \item Racconta di essere andata da Eva, la responsabile delle \emph{Human Resources}, per visionare il resoconto del suo colloquio di lavoro presso la \emph{Pet Micro Robot}.
        \item Spiega che PZZIA aveva elaborato una valutazione, ma Eva le disse che il file era stato cancellato per errore.
        \item Eva le propose di rivedere il colloquio in \emph{Virtual Reality} per chiarire i dubbi.
        \item Dopo aver indossato il visore, si è ritrovata nel sistema quantistico senza capire come.
    \end{itemize}

    \item \textbf{Reazione del Supervisore}
    \begin{itemize}
        \item Ascolta con sguardo impassibile, ma mostra crescente sospetto e irritazione.
        \item Non convinto dalla spiegazione, percepisce Caterina come un'anomalia sfuggente ai suoi protocolli.
    \end{itemize}
\end{itemize}

\subsection*{Conflitto con il Supervisore}

\begin{itemize}
    \item \textbf{Intervento di Mark}
    \begin{itemize}
        \item Il Supervisore si rivolge a Mark, chiedendogli quale sia il suo coinvolgimento.
        \item Mark difende Caterina, affermando che lei non c'entra nulla e che, se c'è un problema, dovrebbe affrontarlo con lui.
        \item Il Supervisore si irrita per il tono di Mark, sentendosi sfidato nella sua autorità.
    \end{itemize}

    \item \textbf{Escalation della Tensione}
    \begin{itemize}
        \item Il Supervisore ribatte, chiedendo se Mark pensa di avere l'autorità per parlare in quel modo.
        \item Mark mantiene uno sguardo fermo, insistendo nella difesa di Caterina.
        \item La tensione nella stanza aumenta, con Caterina che percepisce il rischio di una reazione drastica.
    \end{itemize}

    \item \textbf{Decisione del Supervisore}
    \begin{itemize}
        \item Ordina agli agenti di portare Mark al \emph{Faulty Qubit Space} per una "rigenerazione", come punizione per la sua insubordinazione.
        \item Si rivolge a Caterina, comunicandole che sarà mandata dal \textbf{Commissario}, poiché la situazione è oltre il suo controllo.
        \item Caterina prova panico ma cerca di mantenere la calma; scambia uno sguardo con Mark che le trasmette di non arrendersi.
    \end{itemize}
\end{itemize}

\subsection*{Frustrazione del Supervisore}

\begin{itemize}
    \item \textbf{Riflessioni del Supervisore}
    \begin{itemize}
        \item Rimasto solo, esprime la sua frustrazione per dover coinvolgere il Commissario.
        \item Sente che questo mette in discussione la sua autorità e competenza.
        \item Ammette a sé stesso che Caterina rappresenta un'anomalia che non riesce a comprendere né controllare.
    \end{itemize}
\end{itemize}

\subsection*{I Corridoi Inesplorati del Cuore}

\begin{itemize}
    \item \textbf{Percorso verso il Commissario}
    \begin{itemize}
        \item Caterina viene scortata lungo i freddi corridoi della \emph{Classical Control Unit}.
        \item Si sente assalita da emozioni contrastanti: paura dell'ignoto e una nuova consapevolezza interiore.
    \end{itemize}

    \item \textbf{Riflessioni di Caterina}
    \begin{itemize}
        \item Ripensa a come Mark si sia alzato per difenderla, sentendosi protetta e sostenuta.
        \item Realizza il suo bisogno di protezione, che aveva sempre represso per mostrarsi forte e indipendente.
        \item Si rende conto di aver rifiutato il sostegno del suo fidanzato nella vita reale, comprendendo ora l'importanza di permettere agli altri di prendersi cura di lei.
    \end{itemize}

    \item \textbf{Decisione Interiore}
    \begin{itemize}
        \item Decide che, una volta uscita da quella situazione, riconsidererà il suo rapporto con il fidanzato.
        \item Vuole permettergli di esserci per lei, vedendo questo non come una debolezza, ma come una connessione autentica.
    \end{itemize}
\end{itemize}

\section*{Timeline del Capitolo 5}

\subsection*{Incontro con il Commissario}

\begin{itemize}
    \item \textbf{Luogo}: Sala centrale della \emph{Fault Tolerance Coding}
    \begin{itemize}
        \item Caterina viene condotta in una sala tecnologica avanzata, con elementi di hardware quantistico.
        \item Incontra il \textbf{Commissario}, un'entità software rappresentata attraverso un ologramma.
        \item Il Commissario appare affascinante e rassicurante, accogliendo Caterina con cordialità.
    \end{itemize}
\end{itemize}

\subsection*{Conversazione con il Commissario}

\begin{itemize}
    \item \textbf{Proposta del Commissario}
    \begin{itemize}
        \item Il Commissario elogia le capacità di Caterina e le propone di collaborare con lui.
        \item Le offre l'opportunità di lavorare insieme per realizzare un "esercito di qubit" e rivoluzionare il sistema.
    \end{itemize}

    \item \textbf{Reazione di Caterina}
    \begin{itemize}
        \item Caterina avverte incertezza e diffidenza verso il Commissario.
        \item Decide di non rivelare come sia arrivata lì e finge interesse per guadagnare tempo.
    \end{itemize}

    \item \textbf{Sospetti del Commissario}
    \begin{itemize}
        \item Il Commissario percepisce la mancanza di sincerità di Caterina.
        \item Cambia atteggiamento, diventando più freddo e calcolatore.
    \end{itemize}
\end{itemize}

\subsection*{La Trappola della Ionostrap}

\begin{itemize}
    \item \textbf{Cattura di Caterina}
    \begin{itemize}
        \item Il Commissario attiva la \textbf{Ionostrap}, un dispositivo che immobilizza Caterina con un campo di ioni.
        \item Caterina si rende conto di essere intrappolata senza possibilità di fuga immediata.
    \end{itemize}

    \item \textbf{Monologo del Commissario}
    \begin{itemize}
        \item Il Commissario spiega i suoi piani di controllo totale sul sistema quantistico.
        \item Rivela di voler utilizzare le capacità uniche di Caterina per i suoi scopi.
    \end{itemize}

    \item \textbf{Determinazione di Caterina}
    \begin{itemize}
        \item Nonostante la situazione, Caterina decide di non cedere e inizia a pensare a un modo per liberarsi.
    \end{itemize}
\end{itemize}

\section*{Timeline del Capitolo 6}

\subsection*{Fuga nel Quantum Measurement}

\begin{itemize}
    \item \textbf{Luogo}: \emph{Quantum Measurement}
    \begin{itemize}
        \item Laura e Marley corrono attraverso i corridoi del \emph{Quantum Measurement}, inseguiti dagli agenti della \emph{Quantum Control Electronics}.
        \item Sentono le urla dei qubit che collassano a causa del processo di misura.
        \item Marley spiega che per salvare Caterina devono sconfiggere il \textbf{Commissario}.
        \item Laura esprime incredulità e paura all'idea di affrontare una figura così potente.
    \end{itemize}
\end{itemize}

\subsection*{Incontro con gli Agenti}

\begin{itemize}
    \item \textbf{Evento}: Arrivo dei droni CH\textsubscript{4}
    \begin{itemize}
        \item Due agenti su droni CH\textsubscript{4} le individuano e iniziano l'inseguimento.
        \item Laura e Marley si nascondono dietro circuiti e componenti, ma capiscono che devono agire rapidamente.
    \end{itemize}

    \item \textbf{Dialogo tra gli Agenti}
    \begin{itemize}
        \item Gli agenti discutono della pressione del Supervisore e del \emph{Quantum Master Program}.
        \item Temono le conseguenze di un fallimento e decidono di intensificare la ricerca.
    \end{itemize}
\end{itemize}

\subsection*{La Fuga sul Drone CH\textsubscript{4}}

\begin{itemize}
    \item \textbf{Piano di Laura}
    \begin{itemize}
        \item Laura propone di utilizzare un drone CH\textsubscript{4} per fuggire e raggiungere il luogo dove è tenuta Caterina.
        \item Nota che il drone ha un sistema a spin totale 1 e pensa di poterlo controllare.
        \item Marley è scettica, ma Laura insiste sulla necessità di agire.
    \end{itemize}

    \item \textbf{Esecuzione del Piano}
    \begin{itemize}
        \item Laura sorprende uno degli agenti, lo disarma e prende controllo del drone.
        \item Marley si unisce a lei; insieme decollano, cercando di seminare gli inseguitori.
    \end{itemize}
\end{itemize}

\subsection*{Intervento del Supervisore}

\begin{itemize}
    \item \textbf{Evento}: Disattivazione dell'Agente
    \begin{itemize}
        \item Il Supervisore osserva la scena dal centro di controllo e si infuria per l'incompetenza degli agenti.
        \item Decide di disattivare l'agente a terra come monito per gli altri.
    \end{itemize}

    \item \textbf{Reazione dell'Agente Superstite}
    \begin{itemize}
        \item L'agente superstite, terrorizzato, continua l'inseguimento con maggiore determinazione.
        \item Comunica al Supervisore che farà di tutto per catturare le fuggitive.
    \end{itemize}
\end{itemize}

\subsection*{Attraversamento del Gate di Hadamard}

\begin{itemize}
    \item \textbf{Evento}: Ingresso nel Portale
    \begin{itemize}
        \item Laura guida il drone attraverso un ingresso segnato con una grande \textbf{H}, il gate di Hadamard.
        \item Marley riconosce il simbolo e avverte Laura degli effetti imprevedibili.
    \end{itemize}

    \item \textbf{Effetti del Gate di Hadamard}
    \begin{itemize}
        \item Marley avverte una chiarezza mentale mai provata prima, come se potesse vedere tutte le possibilità.
        \item Laura percepisce uno stato di sovrapposizione quantistica, sentendosi confusa e frammentata.
        \item Nonostante il disorientamento, Laura continua a pilotare il drone, concentrata sul salvataggio di Caterina.
    \end{itemize}
\end{itemize}

\section*{Timeline del Capitolo 7}

\subsection*{Confronto con l'Agente}

\begin{itemize}
    \item \textbf{Evento}: Inseguimento Serrato
    \begin{itemize}
        \item L'agente superstite modifica la configurazione del suo drone per aumentare la velocità.
        \item Si avvicina sempre di più a Laura e Marley, deciso a catturarle.
    \end{itemize}

    \item \textbf{Scontro nel Cielo Quantistico}
    \begin{itemize}
        \item Laura esegue manovre evasive, ma l'agente le tiene testa.
        \item Marley suggerisce di attraversare un altro portale per confonderlo.
    \end{itemize}
\end{itemize}

\subsection*{Attraversamento del Gate C-NOT}

\begin{itemize}
    \item \textbf{Evento}: Ingresso nel Portale C-NOT
    \begin{itemize}
        \item Laura individua un portale con il simbolo \textbf{C-NOT} e decide di attraversarlo.
        \item L'agente la segue senza esitazione.
    \end{itemize}

    \item \textbf{Effetto del Portale}
    \begin{itemize}
        \item Attraversando il gate C-NOT, Laura e l'agente entrano in uno \textbf{stato di Bell}, diventando entangled.
        \item Ogni azione di uno influenzerà immediatamente l'altro.
    \end{itemize}
\end{itemize}

\subsection*{Strategia di Laura}

\begin{itemize}
    \item \textbf{Comprensione dell'Entanglement}
    \begin{itemize}
        \item Laura realizza la situazione e decide di sfruttarla a suo vantaggio.
        \item Utilizza la connessione per anticipare le mosse dell'agente.
    \end{itemize}

    \item \textbf{Manovre Sincronizzate}
    \begin{itemize}
        \item Inizia a eseguire manovre complesse che costringono l'agente a seguirla in schemi che lo disorientano.
        \item L'agente si confonde, incapace di distinguere i propri movimenti da quelli indotti dall'entanglement.
    \end{itemize}
\end{itemize}

\subsection*{Intervento del Commissario}

\begin{itemize}
    \item \textbf{Decisione Drastica}
    \begin{itemize}
        \item Il \textbf{Commissario}, osservando la situazione, decide di intervenire personalmente.
        \item Ordina all'agente di gettarsi nel \textbf{mare di Dirac}, una zona del sistema estremamente pericolosa.
    \end{itemize}

    \item \textbf{Conseguenze dell'Ordine}
    \begin{itemize}
        \item Se l'agente obbedisce, Laura subirà la stessa sorte a causa dell'entanglement.
        \item L'agente esita, consapevole del sacrificio richiesto.
    \end{itemize}
\end{itemize}

\subsection*{Il Dilemma dell'Agente}

\begin{itemize}
    \item \textbf{Conflitto Interiore}
    \begin{itemize}
        \item L'agente lotta tra l'obbedienza al Commissario e l'istinto di sopravvivenza.
        \item Ricorda momenti passati che mettono in dubbio la sua fedeltà cieca.
    \end{itemize}

    \item \textbf{Decisione Finale}
    \begin{itemize}
        \item Decide di disobbedire all'ordine, rompendo il legame con il Commissario.
        \item Questo atto di ribellione innesca una serie di eventi imprevedibili nel sistema.
    \end{itemize}
\end{itemize}

\section*{Timeline del Capitolo 8}

\subsection*{Il Sacrificio di Shore}

\begin{itemize}
    \item \textbf{Intervento di Shore}
    \begin{itemize}
        \item Il professor Shore, liberato dalla prigionia, decide di aiutare Laura.
        \item Propone di utilizzare un \textbf{gate di Toffoli} per interrompere l'entanglement.
    \end{itemize}

    \item \textbf{Esecuzione del Piano}
    \begin{itemize}
        \item Guida Laura e l'agente verso il gate di Toffoli.
        \item Durante il processo, Shore si sacrifica sottoponendosi a una misura, annullando l'entanglement.
    \end{itemize}
\end{itemize}

\subsection*{La Liberazione di Caterina}

\begin{itemize}
    \item \textbf{Riunione delle Amiche}
    \begin{itemize}
        \item Laura raggiunge la \emph{Ion Trap} dove Caterina è prigioniera.
        \item Insieme a Marley, riesce a disattivare il campo di ioni e liberarla.
    \end{itemize}

    \item \textbf{Confronto con il Commissario}
    \begin{itemize}
        \item Il Commissario tenta un ultimo attacco per fermarle.
        \item Le tre, unite, riescono a respingerlo utilizzando le loro conoscenze quantistiche.
    \end{itemize}
\end{itemize}

\subsection*{Fuga Finale}

\begin{itemize}
    \item \textbf{Chiusura del Quantum Channel}
    \begin{itemize}
        \item Il \emph{Quantum Master Program} ordina di chiudere tutte le uscite.
        \item Laura ricorda un percorso alternativo attraverso il \emph{Quantum Annealing}.
    \end{itemize}

    \item \textbf{Ingresso nel Quantum Annealing}
    \begin{itemize}
        \item Le tre entrano nel portale, affrontando un viaggio caotico tra diversi stati quantistici.
        \item Vivono esperienze intense che le portano a riflettere sulle loro scelte di vita.
    \end{itemize}
\end{itemize}

\section*{Timeline del Capitolo 9}

\subsection*{L'Illuminazione nel Quantum Annealing}

\begin{itemize}
    \item \textbf{Esperienze Personali}
    \begin{itemize}
        \item Laura vede un futuro solitario se continua a trascurare le relazioni.
        \item Caterina realizza che il controllo sugli altri la porterà all'isolamento.
    \end{itemize}

    \item \textbf{Raggiungimento dello Stato di Minima Energia}
    \begin{itemize}
        \item Il campo magnetico esterno le aiuta a focalizzare i loro veri desideri.
        \item Entrambe raggiungono una nuova consapevolezza, pronte a cambiare.
    \end{itemize}
\end{itemize}

\subsection*{Ritorno alla Realtà}

\begin{itemize}
    \item \textbf{Uscita dal Quantum Annealing}
    \begin{itemize}
        \item Laura e Caterina si risvegliano nei rispettivi corpi, tornando alla realtà.
        \item Sentono una grande calma e determinazione nel perseguire nuovi obiettivi.
    \end{itemize}

    \item \textbf{Riunione con Hiroki}
    \begin{itemize}
        \item Laura trova conforto nel suo cane Hiroki, sentendo una connessione più profonda.
    \end{itemize}
\end{itemize}

\subsection*{La Situazione di Eva}

\begin{itemize}
    \item \textbf{Confronto Finale}
    \begin{itemize}
        \item Caterina affronta Eva, chiedendo spiegazioni sulle valutazioni nascoste.
        \item La PZZIA interviene, rivelando le manipolazioni di Eva.
    \end{itemize}

    \item \textbf{Cambio di Potere}
    \begin{itemize}
        \item Eva tenta di chiamare la sicurezza, ma i suoi permessi sono stati revocati.
        \item Il \emph{Quantum Master Program} ha deciso di apportare cambiamenti, rimuovendo Eva dal suo ruolo.
    \end{itemize}
\end{itemize}

\section*{Timeline del Capitolo 10}

\subsection*{Nuovi Inizi}

\begin{itemize}
    \item \textbf{Collaborazione tra Laura, Caterina e PZZIA}
    \begin{itemize}
        \item Decidono di lavorare insieme per migliorare il sistema e promuovere la libertà dei qubit.
        \item Si preparano ad affrontare le sfide future con determinazione.
    \end{itemize}

    \item \textbf{Riflessioni Finali}
    \begin{itemize}
        \item Laura e Caterina comprendono l'importanza delle connessioni umane e dell'apertura verso gli altri.
        \item Sentono che la loro avventura le ha trasformate profondamente.
    \end{itemize}
\end{itemize}

\subsection*{Conclusione}

\begin{itemize}
    \item \textbf{La Battaglia per la Libertà Continua}
    \begin{itemize}
        \item Consapevoli che il percorso sarà difficile, sono pronte a lottare per un futuro migliore.
        \item La storia si conclude con un senso di speranza e determinazione.
    \end{itemize}
\end{itemize}
