\section*{Timeline del Capitolo 8}
\begin{itemize}
  \item
    \begin{itemize}
        \item Laura pilota il drone con maestria, ma l'agente è sempre più vicino.
        \item Davanti a lei appare un portale segnato con il simbolo \textbf{C-NOT}.
        \item Laura attraversa il portale senza esitazione, seguita dall'agente.
    \end{itemize}
    \item \textbf{Effetto del Portale}
    \begin{itemize}
        \item Attraversando il portale C-NOT mentre è in stato di Hadamard, Laura entra in \textbf{entanglement} con l'agente.
        \item Si ritrova in uno \textbf{stato di Bell}, correlata a livello quantistico con l'agente.
        \item Realizza che ogni sua azione avrà conseguenze immediate sull'agente e viceversa.
    \end{itemize}
\end{itemize}



\begin{itemize}
    \item \textbf{Sfruttamento dello Stato di Bell}
    \begin{itemize}
        \item Laura visualizza la struttura del drone dell'agente grazie all'entanglement.
        \item Modifica la configurazione del suo drone, disponendo i quattro rotori su un unico piano per una maggiore manovrabilità.
        \item Avverte una nuova fluidità nei movimenti, sentendosi un tutt'uno con il drone.
        \item Prova un misto di eccitazione e terrore, consapevole che ogni manovra deve essere precisa.
    \end{itemize}
\end{itemize}

\begin{itemize}
    \item \textbf{Osservazione del Commissario}
    \begin{itemize}
        \item Il \textbf{Commissario} osserva i movimenti di Laura e si preoccupa della sua abilità.
        \item Decide che Laura è una minaccia che deve essere neutralizzata.
    \end{itemize}
    \item \textbf{Ordine di Criptazione}
    \begin{itemize}
        \item Il Commissario ordina di criptare il sistema utilizzando l'algoritmo \textbf{RSA 2048}.
        \item Intende bloccare ogni dato e movimento per impedire a Laura e Marley di sfuggire.
        \item I tecnici attivano i protocolli di criptazione sotto le sue direttive.
    \end{itemize}
\end{itemize}


\begin{itemize}
    \item \textbf{Effetto della Criptazione}
    \begin{itemize}
        \item Laura avverte una pesantezza improvvisa; l'ambiente circostante si cristallizza.
        \item Realizza di essere stata criptata insieme all'ambiente.
        \item Si sente bloccata, incapace di muoversi o pensare con chiarezza.
    \end{itemize}
    \item \textbf{Ricordo del Professor Shor}
    \begin{itemize}
        \item Ricorda le parole del Professor Shor che le aveva detto di conoscere alcuni algoritmi a memoria.
        \item Comprende che deve richiamare l'\textbf{algoritmo di Shor} per decriptare il sistema e liberarsi.
    \end{itemize}
\end{itemize}

\begin{itemize}
    \item \textbf{Richiamo dell'Algoritmo di Shor}
    \begin{itemize}
        \item Laura ripercorre mentalmente i passaggi dell'algoritmo di Shor:
        \begin{enumerate}
            \item \textbf{Pre-processing}: Identifica il numero \( N \) da fattorizzare, prodotto di due grandi numeri primi \( P \) e \( Q \).
            \item Sceglie un numero casuale \( a \) relativamente primo rispetto a \( N \).
            \item \textbf{Quantum Order Finding}: Calcola il periodo \( r \) della funzione \( f(x) = a^x \mod N \) utilizzando la sovrapposizione quantistica.
            \item Verifica se \( r \) è pari; se sì, procede al passo successivo.
            \item Calcola \( \text{gcd}(a^{r/2} \pm 1, N) \) per trovare i fattori \( P \) e \( Q \).
        \end{enumerate}
    \end{itemize}
    \item \textbf{Frustrazione e Consapevolezza}
    \begin{itemize}
        \item Laura si sente più sicura ma anche frustrata, rendendosi conto che le manca un'informazione cruciale.
        \item Si chiede come comunicare efficacemente il valore di \( r \) e ottenere i fattori corretti di \( N \).
        \item Comprende che deve trovare un modo per completare la decifrazione e liberarsi dalla criptazione.
    \end{itemize}
\end{itemize}
