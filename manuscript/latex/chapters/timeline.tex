
\section*{Timeline del Primo Capitolo}

\subsection*{Lunedì}

\subsubsection*{Mattina}

\begin{itemize}
    \item \textbf{Ore 9:30 - Pet $\mu$ Robot}
    \begin{itemize}
        \item Caterina si presenta al colloquio presso la \emph{Pet Micro Robot} per una posizione di responsabile marketing.
        \item Viene sottoposta a una preselezione guidata dall'IA PZZIA.
        \item Durante la seconda fase del colloquio, Eva, responsabile delle risorse umane, le pone domande sull'ambiente, sul cambiamento climatico e sull'intelligenza artificiale nelle aziende.
        \item Eva le assegna inaspettatamente un test di programmazione avanzata.
        \item Caterina completa il test ma con alcune incertezze.
    \end{itemize}
\end{itemize}

\subsubsection*{Pomeriggio}

\begin{itemize}
    \item \textbf{Caffetteria all'angolo}
    \begin{itemize}
        \item Dopo il colloquio, Caterina incontra Laura.
        \item Si recano insieme in una caffetteria per discutere dell'esperienza.
        \item Laura aiuta Caterina a risolvere l'algoritmo del test di programmazione, alleviando le sue preoccupazioni.
    \end{itemize}
\end{itemize}

\subsection*{Martedì}

\subsubsection*{Mattina}

\begin{itemize}
    \item \textbf{Ore 12:30 - Magazzino Bamazon}
    \begin{itemize}
        \item Laura lavora nel magazzino Amazon.
        \item Si trova in difficoltà nel sistema di gestione del magazzino (WMS), non riuscendo a individuare la corretta ubicazione di un pacco.
        \item Si imbatte in un portale con il cartello "Accesso riservato – Stoccaggi speciali".
        \item Viene fermata da Ising, un tecnico che le spiega che l'area è riservata.
    \end{itemize}
\end{itemize}

\subsubsection*{Pomeriggio}

\begin{itemize}
    \item \textbf{Ore 17:30 - Università degli Studi}
    \begin{itemize}
        \item Laura si prepara per l'esame di crittografia quantistica con il Professor Shor.
        \item Durante l'esame, affronta difficoltà nell'algoritmo di Shor.
        \item Riceve una valutazione che non la soddisfa e decide di ripetere l'esame.
    \end{itemize}
\end{itemize}

\subsubsection*{Sera}

\begin{itemize}
    \item \textbf{Ore 19:00 - Casa di Laura}
    \begin{itemize}
        \item Caterina raggiunge Laura per cena.
        \item Discutono delle rispettive giornate e delle difficoltà incontrate.
        \item Laura mostra a Caterina i suoi appunti sull'algoritmo di Shor.
        \item Fanno una passeggiata con Roky, il cane di Laura.
        \item Laura le suggerisce di verificare il file di valutazione generato dall'IA.
    \end{itemize}
\end{itemize}

\subsection*{Mercoledì}

\subsubsection*{Notte}

\begin{itemize}
    \item \textbf{Casa di Caterina}
    \begin{itemize}
        \item Caterina, tornata a casa, riflette sulle sue scelte di vita.
        \item Prepara una tisana e rivede le foto con Mark, sentendosi distaccata.
        \item Decide di scrivere un'email a Eva chiedendo il documento valutativo.
    \end{itemize}
\end{itemize}

\subsubsection*{Mattina}

\begin{itemize}
    \item \textbf{Risposta di Eva}
    \begin{itemize}
        \item Caterina riceve una risposta da Eva, che afferma che il documento è stato cancellato per errore.
        \item Eva propone un incontro per discutere di persona.
        \item Caterina accetta l'invito, sebbene perplessa.
    \end{itemize}
    \item \textbf{Ore 9:30 - Casa di Laura}
    \begin{itemize}
        \item Caterina passa da Laura per un saluto prima dell'incontro con Eva.
        \item Laura le mostra il \emph{Noemografo}, un dispositivo per la lettura dei pensieri.
        \item Sperimentano insieme una connessione mentale.
        \item Caterina lascia Laura per recarsi all'appuntamento con Eva.
    \end{itemize}
\end{itemize}

\subsubsection*{Pomeriggio}

\begin{itemize}
    \item \textbf{Incontro con Eva alla Pet $\mu$ Robot}
    \begin{itemize}
        \item Eva accoglie Caterina con un visore 3D, proponendole di rivedere il colloquio.
        \item Caterina, sebbene sospettosa, accetta di indossare il visore.
        \item Si rende conto troppo tardi che è una trappola orchestrata da Eva.
    \end{itemize}
    \item \textbf{Contemporaneamente - Casa di Laura}
    \begin{itemize}
        \item Laura avverte strane sensazioni.
        \item Realizza che la connessione mentale con Caterina non si è interrotta.
        \item Si preoccupa per l'amica e cerca di capire cosa stia accadendo.
    \end{itemize}
\end{itemize}

%%2

\section*{Timeline del Capitolo 2}

\subsection*{Mattina}

\begin{itemize}
    \item \textbf{Luogo}: \emph{Pet Micro Robot} - Sala di Eva
    \begin{itemize}
        \item Eva è nella stanza con Caterina, connessa alla Realtà Virtuale, immersa nel mondo simulato.
        \item Eva chiede a PZZIA se è possibile cancellare il file contenente le \textit{chain of thinking} utilizzate per valutare Caterina.
        \item PZZIA risponde che i suoi processi sono quantistici e reversibili; l'informazione non può essere cancellata senza lasciare traccia.
        \item Eva si irrita, sapendo che misurare i qubit in stati classici innescherebbe un messaggio a Caterina con il risultato.
        \item Decide di continuare con il trattamento psicologico tramite Realtà Virtuale per convincere Caterina a rinunciare alla posizione lavorativa.
        \item Eva riflette sui due punti deboli del trattamento:
        \begin{enumerate}
            \item Il soggetto deve percepirsi completamente solo, senza segnali di aiuto esterni.
            \item Il soggetto non deve comprendere i meccanismi dell'algoritmo di suggestione.
        \end{enumerate}
        \item Convinta che Caterina non abbia le competenze per capire la manipolazione, Eva procede con il piano.
    \end{itemize}
    
    \item \textbf{Luogo}: \emph{Classical Control Unit}
    \begin{itemize}
        \item Un agente nota un'anomalia: ci sono due qubit in più nel sistema.
        \item Informa il supervisore, che chiede di verificare, non avendo ricevuto avvisi dal \emph{Quantum Resource Management} (QRM).
        \item Il supervisore ordina di mantenere la trasmissione con il QRM criptata per evitare che il \emph{Quantum Error Correction} o il \emph{Fault Tolerance Coding} rilevino anomalie.
        \item L'agente cripta la comunicazione utilizzando l'algoritmo RSA 2048.
        \item Il QRM conferma di non aver installato nuovi qubit; l'anomalia è reale.
        \item Preoccupato, il supervisore ordina di inviare una squadra della \emph{Quantum Control Electronics} per verificare fisicamente i qubit.
    \end{itemize}
    
    \item \textbf{Luogo}: \emph{Fault Tolerance Coding} - Prigione del Professor Shor
    \begin{itemize}
        \item Il commissario alla sicurezza si avvicina al Professor Shor, tenuto prigioniero.
        \item Gli ordina di decriptare un messaggio inviato al QRM.
        \item Shor decripta il messaggio, scoprendo il problema dei due nuovi qubit e il tentativo del supervisore di nascondere l'anomalia.
        \item Il commissario, soddisfatto, decide di non arrestare subito il supervisore, pianificando di utilizzare la situazione a suo vantaggio.
        \item Informa un'agente della polizia segreta della sua decisione.
    \end{itemize}
\end{itemize}

\subsection*{Pomeriggio}

\begin{itemize}
    \item \textbf{Luogo}: \emph{Base della Quantum Control Electronics}
    \begin{itemize}
        \item Due agenti ricevono l'ordine di verificare i qubit nel sistema.
        \item Partono a bordo di droni luminosi verso il \emph{Qubit Array}.
    \end{itemize}
    
    \item \textbf{Luogo}: \emph{Qubit Array}
    \begin{itemize}
        \item Laura e Caterina, confuse, cercano di capire dove si trovano nell'ambiente quantistico.
        \item Alcuni qubit le osservano da lontano, nascosti tra i corridoi.
        \item Un qubit maschio, somigliante al fidanzato di Caterina, si avvicina.
        \item Avverte: \emph{"State per essere trovate. Se non volete passare qualche giorno rinchiuse mentre controllano il vostro QFLIP, è meglio che veniate con noi."}
        \item Caterina, attratta dalla sua somiglianza e sicurezza, lo segue senza opporre resistenza.
        \item Laura, perplessa, segue il gruppo.
        \item Altri due qubit si uniscono a loro, incitandole a muoversi velocemente.
        \item In lontananza, i due agenti della \emph{Quantum Control Electronics} si avvicinano per verificare l'anomalia.
    \end{itemize}
\end{itemize}

\subsection*{Sera}

\begin{itemize}
    \item \textbf{Luogo}: \emph{Faulty Qubit Space}
    \begin{itemize}
        \item Il gruppo raggiunge uno spazio appartato dove risiedono qubit difettosi o instabili.
        \item Il qubit simile a Mark dice: \emph{"Qui sarete al sicuro... per un po'. Ma non potete rimanere a lungo."}
        \item Laura nota l'instabilità dell'ambiente e chiede se sia sicuro restare.
        \item Una qubit femmina risponde che non lo è: mancano isolamento adeguato e sistema di raffreddamento; rischiano la decoerenza.
        \item I due agenti passano vicino al loro nascondiglio, controllando dati e ambiente.
        \item Laura trattiene il respiro; gli agenti sembrano fermarsi, ma poi proseguono.
        \item Caterina si avvicina al qubit somigliante a Mark e chiede il suo nome.
        \item Lui risponde con un sorriso tranquillo: \emph{"Sono... Mark."}
    \end{itemize}
\end{itemize}


%%3
\section*{Timeline del Capitolo 3}

\subsection*{Mattina}

\begin{itemize}
    \item \textbf{Luogo}: \emph{Faulty Qubit Space}
    \begin{itemize}
        \item Laura e Caterina si trovano nel \emph{Faulty Qubit Space}, un limbo per qubit instabili destinati a essere eliminati se non riescono a mantenere la coerenza.
        \item Marley è accanto a loro, con il volto serio e pensieroso, osservando gli altri qubit rassegnati al loro destino.
        \item Laura avverte tensione e afferra il braccio di Caterina, notando la paura nell'amica.
    \end{itemize}
\end{itemize}

\subsection*{Pomeriggio}

\begin{itemize}
    \item \textbf{Luogo}: \emph{Faulty Qubit Space}
    \begin{itemize}
        \item Mark, Marley e un altro compagno si avvicinano a Laura e Caterina.
        \item Mark dice loro di rimanere nascoste; lui e l'altro proveranno a raggiungere un circuito periferico.
        \item Spiega che devono aggiungere un \emph{Quantum Teleportation Buffer} per evitare che l'entanglement leghi ulteriormente al \emph{Faulty Qubit Space}.
        \item Caterina esprime preoccupazione per Mark, che la rassicura prima di allontanarsi nell'ombra.
    \end{itemize}
\end{itemize}

\subsection*{Sera}

\begin{itemize}
    \item \textbf{Luogo}: \emph{Faulty Qubit Space}
    \begin{itemize}
        \item Rimaste sole con Marley, Laura e Caterina discutono delle loro preoccupazioni.
        \item Caterina chiede cosa stia realmente accadendo; Laura cerca di rassicurarla, ma non ha risposte certe.
        \item Marley appare tesa; il tempo nel rifugio è limitato.
    \end{itemize}
    \item \textbf{Evento}: Arrivo degli Agenti
    \begin{itemize}
        \item Una luce rossa intermittente attraversa lo spazio, seguita da passi veloci e decisi.
        \item Marley avverte: \emph{"Gli agenti."}
        \item Spinge Laura e Caterina più in fondo al \emph{Faulty Qubit Space} per nascondersi.
        \item Laura vede Mark e il suo compagno fermarsi mentre cercano di collegare il circuito periferico.
        \item Due agenti li circondano; Mark tenta di difendersi ma viene immobilizzato.
        \item Caterina corre verso Mark per aiutarlo, nonostante le proteste di Laura: \emph{"Caterina, fermati!"}
        \item L'altro agente afferra Caterina, legandole i polsi; anche lei viene arrestata.
        \item Laura prova angoscia, ma Marley la trascina via.
    \end{itemize}
\end{itemize}

\subsection*{Notte}

\begin{itemize}
    \item \textbf{Luogo}: Fuga verso il \emph{Quantum Measurement}
    \begin{itemize}
        \item Marley dice a Laura che non possono fare nulla per gli altri ora.
        \item Laura, con gli occhi pieni di lacrime, viene guidata via da Marley.
        \item Chiede: \emph{"Dove andiamo?"}
        \item Marley risponde: \emph{"Al Quantum Measurement. È pericoloso, ma è l'unico posto dove gli agenti non potranno seguire le nostre tracce così facilmente."}
        \item Entrano nel \emph{Quantum Measurement}; l'atmosfera è sospesa tra realtà e astrazione.
        \item Laura avverte una strana pressione nella testa, come se ogni pensiero potesse far collassare l'intero sistema.
    \end{itemize}
    \item \textbf{Evento}: Inseguiti dai Droni CH$_4$
    \begin{itemize}
        \item Il rumore dei droni CH$_4$ si avvicina.
        \item Laura dice con voce tremante: \emph{"Ci hanno trovate."}
        \item Marley esorta Laura a restare calma: \emph{"Devi restare calma."}
        \item Le luci dei droni illuminano l'oscurità; il collasso dello stato è imminente.
        \item Sanno che il \emph{Quantum Measurement} è estremamente instabile; un errore potrebbe essere fatale.
        \item Marley suggerisce: \emph{"Se dobbiamo restare qui, faremo in modo di non venire rilevate."}
        \item Laura annuisce, determinata a lottare fino alla fine per salvare Caterina e se stessa.
    \end{itemize}
\end{itemize}
%%4
\section*{Timeline del Capitolo 4}

\subsection*{Mattina}

\begin{itemize}
    \item \textbf{Luogo}: Stanza spoglia con pareti metalliche nella \emph{Classical Control Unit}
    \begin{itemize}
        \item Caterina si trova in una stanza fredda e spoglia, con pareti metalliche che riflettono una luce bianca e fredda.
        \item Di fronte a lei c'è il \textbf{Supervisore}, figura imponente dai tratti austeri e rigidi.
        \item Accanto a lei ci sono Mark e l'altro compagno, seduti su rigidi supporti, immobili e silenziosi.
        \item Gli agenti che li hanno catturati si sono ritirati, lasciandoli soli con il Supervisore.
    \end{itemize}
\end{itemize}

\subsection*{Colloquio con il Supervisore}

\begin{itemize}
    \item \textbf{Interrogatorio di Caterina}
    \begin{itemize}
        \item Il Supervisore si rivolge a Caterina con tono glaciale, chiedendole come sia finita lì, poiché non la riconosce come uno dei qubit del \emph{Qubit Array}.
        \item Caterina cerca di mantenere la calma e risponde che non sa come sia finita lì, affermando di non aver fatto nulla di male.
        \item Il Supervisore è scettico e insiste per avere spiegazioni più dettagliate.
    \end{itemize}
    \item \textbf{Spiegazione di Caterina}
    \begin{itemize}
        \item Racconta di essere andata da Eva, la responsabile delle \emph{Human Resources}, per visionare il resoconto del suo colloquio di lavoro presso la \emph{Pet Micro Robot}.
        \item Spiega che PZZIA aveva elaborato una valutazione, ma Eva le disse che il file era stato cancellato per errore.
        \item Eva le propose di rivedere il colloquio in \emph{Virtual Reality} per chiarire i dubbi.
        \item Dopo aver indossato il visore, si è ritrovata nel sistema quantistico senza capire come.
    \end{itemize}
    \item \textbf{Reazione del Supervisore}
    \begin{itemize}
        \item Ascolta con sguardo impassibile, ma mostra crescente sospetto e irritazione.
        \item Non convinto dalla spiegazione, percepisce Caterina come un'anomalia sfuggente ai suoi protocolli.
    \end{itemize}
\end{itemize}

\subsection*{Conflitto con il Supervisore}

\begin{itemize}
    \item \textbf{Intervento di Mark}
    \begin{itemize}
        \item Il Supervisore si rivolge a Mark, chiedendogli quale sia il suo coinvolgimento.
        \item Mark difende Caterina, affermando che lei non c'entra nulla e che, se c'è un problema, dovrebbe affrontarlo con lui.
        \item Il Supervisore si irrita per il tono di Mark, sentendosi sfidato nella sua autorità.
    \end{itemize}
    \item \textbf{Escalation della Tensione}
    \begin{itemize}
        \item Il Supervisore ribatte, chiedendo se Mark pensa di avere l'autorità per parlare in quel modo.
        \item Mark mantiene uno sguardo fermo, insistendo nella difesa di Caterina.
        \item La tensione nella stanza aumenta, con Caterina che percepisce il rischio di una reazione drastica.
    \end{itemize}
    \item \textbf{Decisione del Supervisore}
    \begin{itemize}
        \item Ordina agli agenti di portare Mark al \emph{Faulty Qubit Space} per una "rigenerazione", come punizione per la sua insubordinazione.
        \item Si rivolge a Caterina, comunicandole che sarà mandata dal \textbf{Commissario}, poiché la situazione è oltre il suo controllo.
        \item Caterina prova panico ma cerca di mantenere la calma; scambia uno sguardo con Mark che le trasmette di non arrendersi.
    \end{itemize}
\end{itemize}

\subsection*{Frustrazione del Supervisore}

\begin{itemize}
    \item \textbf{Riflessioni del Supervisore}
    \begin{itemize}
        \item Rimasto solo, esprime la sua frustrazione per dover coinvolgere il Commissario.
        \item Sente che questo mette in discussione la sua autorità e competenza.
        \item Ammette a sé stesso che Caterina rappresenta un'anomalia che non riesce a comprendere né controllare.
    \end{itemize}
\end{itemize}

\subsection*{I Corridoi Inesplorati del Cuore}

\begin{itemize}
    \item \textbf{Percorso verso il Commissario}
    \begin{itemize}
        \item Caterina viene scortata lungo i freddi corridoi della \emph{Classical Control Unit}.
        \item Si sente assalita da emozioni contrastanti: paura dell'ignoto e una nuova consapevolezza interiore.
    \end{itemize}
    \item \textbf{Riflessioni di Caterina}
    \begin{itemize}
        \item Ripensa a come Mark si sia alzato per difenderla, sentendosi protetta e sostenuta.
        \item Realizza il suo bisogno di protezione, che aveva sempre represso per mostrarsi forte e indipendente.
        \item Si rende conto di aver rifiutato il sostegno del suo fidanzato nella vita reale, comprendendo ora l'importanza di permettere agli altri di prendersi cura di lei.
    \end{itemize}
    \item \textbf{Decisione Interiore}
    \begin{itemize}
        \item Decide che, una volta uscita da quella situazione, riconsidererà il suo rapporto con il fidanzato.
        \item Vuole permettergli di esserci per lei, vedendo questo non come una debolezza, ma come una connessione autentica.
    \end{itemize}
\end{itemize}

%%5
\section*{Timeline del Capitolo 5}

\subsection*{Incontro con il Commissario}

\begin{itemize}
    \item \textbf{Luogo}: Sala centrale della \emph{Fault Tolerance Coding}
    \begin{itemize}
        \item Caterina viene condotta in una sala tecnologica avanzata, con elementi di hardware quantistico.
        \item Incontra il \textbf{Commissario}, un'entità software rappresentata attraverso un ologramma.
        \item Il Commissario appare affascinante e rassicurante, accogliendo Caterina con cordialità.
    \end{itemize}
\end{itemize}

\subsection*{Conversazione con il Commissario}

\begin{itemize}
    \item Il Commissario elogia le capacità di Caterina e le propone di collaborare.
    \item Le offre l'opportunità di lavorare insieme per realizzare un "esercito di Qubit".
    \item Caterina avverte incertezza e decide di non rivelare come sia arrivata lì.
\end{itemize}

\subsection*{La Trappola della Ionostrap}

\begin{itemize}
    \item Caterina finge di accettare le proposte del Commissario, cercando un'opportunità per fuggire.
    \item Il Commissario si accorge della sua mancanza di sincerità e cambia atteggiamento.
    \item Attiva la \textbf{Ionostrap}, immobilizzando Caterina con un campo di ioni.
    \item Caterina si rende conto di essere in balia del Commissario, senza possibilità di fuga.
\end{itemize}
%%6

\section*{Timeline del Capitolo 6}

\subsection*{Fuga nel Quantum Measurement}

\begin{itemize}
    \item \textbf{Luogo}: \emph{Quantum Measurement}
    \begin{itemize}
        \item Laura e Marley corrono attraverso i corridoi del \emph{Quantum Measurement}, il rumore dei passi amplificato dall'eco metallico delle pareti.
        \item Sentono una serie di urla strazianti; Marley spiega che è il suono dei qubit che collassano a causa del processo di misura.
        \item Laura chiede preoccupata di Caterina; Marley risponde che potrebbe essere in pericolo e che devono muoversi in fretta.
        \item Marley afferma che per trovare Caterina devono sconfiggere il \textbf{Commissario}, che può negare loro l'accesso al sistema.
        \item Laura esprime incredulità riguardo al dover affrontare il Commissario; Marley spiega che è un fanatico dell'ordine e che devono agire per non permettere al \emph{Quantum Master Program} di vincere.
    \end{itemize}
\end{itemize}

\subsection*{Incontro con gli Agenti della Quantum Control Electronics}

\begin{itemize}
    \item \textbf{Evento}: Arrivo dei droni CH$_4$
    \begin{itemize}
        \item Un suono metallico e ronzante le fa sobbalzare; due droni CH$_4$ compaiono, pilotati da agenti della \emph{Quantum Control Electronics} inviati per trovarle.
        \item Laura e Marley si nascondono dietro circuiti e componenti, trattenendo il fiato.
        \item Gli agenti atterrano e iniziano a perlustrare l'area con movimenti calcolati e sguardi attenti.
        \item La tensione è alta; Laura realizza che devono agire rapidamente o saranno scoperte.
    \end{itemize}
\end{itemize}

\subsection*{I Due Agenti}

\begin{itemize}
    \item \textbf{Dialogo tra gli Agenti}
    \begin{itemize}
        \item Gli agenti discutono sulla possibile posizione di Laura e Marley, ipotizzando che si siano nascoste nel settore di stabilizzazione dei qubit.
        \item Esprimono preoccupazione per le conseguenze di un fallimento sotto gli occhi del Supervisore e del \emph{Quantum Master Program}.
        \item Decidono di concentrarsi e controllare l'area con attenzione per evitare errori.
    \end{itemize}
\end{itemize}

\subsection*{La Fuga sul Drone CH$_4$}

\begin{itemize}
    \item \textbf{Piano di Laura}
    \begin{itemize}
        \item Laura propone di fuggire utilizzando uno dei droni CH$_4$ per raggiungere il \emph{Faulty Qubit System} e salvare Caterina.
        \item Marley esprime dubbi, sottolineando che i droni hanno sistemi di sicurezza e che le probabilità di essere scoperte sono alte.
        \item Laura nota che il drone ha un sistema a spin totale 1 e pensa di poterlo controllare modificando la proiezione dello spin lungo l'asse Z.
    \end{itemize}
    \item \textbf{Esecuzione del Piano}
    \begin{itemize}
        \item Marley chiede a Laura come sappia queste cose; Laura evita di rispondere direttamente.
        \item Marley sospetta che Laura sia una \emph{Quantum Crafter}; Laura glissa, affermando che non è il momento di discuterne.
        \item Decidono di agire; Laura si avvicina al drone con determinazione.
        \item Con un balzo, Laura mette fuori gioco uno degli agenti e sale sul drone.
        \item Marley, inizialmente sorpresa, decide di unirsi a lei sul drone.
        \item Attivano il drone; i rotori iniziano a girare e si sollevano da terra, pronte a dirigersi verso il \emph{Faulty Qubit System} per salvare Caterina.
    \end{itemize}
\end{itemize}
%%7
\section*{Timeline del Capitolo 7}

\subsection*{Fuga nel Quantum Measurement}

\begin{itemize}
    \item \textbf{Luogo}: \emph{Quantum Measurement}
    \begin{itemize}
        \item Laura e Marley corrono attraverso i corridoi, inseguiti dagli agenti della \emph{Quantum Control Electronics}.
        \item Sentono le urla dei qubit che collassano a causa del processo di misura.
        \item Marley spiega che per salvare Caterina devono sconfiggere il \textbf{Commissario}.
    \end{itemize}
\end{itemize}

\subsection*{Incontro con gli Agenti}

\begin{itemize}
    \item \textbf{Evento}: Arrivo dei droni CH$_4$
    \begin{itemize}
        \item Due agenti su droni CH$_4$ le individuano e iniziano l'inseguimento.
        \item Laura e Marley si nascondono, ma capiscono che devono agire rapidamente.
    \end{itemize}
    \item \textbf{Dialogo tra gli Agenti}
    \begin{itemize}
        \item Gli agenti discutono della pressione del Supervisore e del \emph{Quantum Master Program}.
        \item Temono le conseguenze di un fallimento.
    \end{itemize}
\end{itemize}

\subsection*{La Fuga sul Drone CH$_4$}

\begin{itemize}
    \item \textbf{Piano di Laura}
    \begin{itemize}
        \item Laura propone di utilizzare un drone CH$_4$ per fuggire.
        \item Nota che il drone ha un sistema a spin totale 1 e pensa di poterlo controllare.
        \item Marley è scettica, ma Laura insiste.
    \end{itemize}
    \item \textbf{Esecuzione del Piano}
    \begin{itemize}
        \item Laura mette fuori gioco uno degli agenti e sale sul drone.
        \item Marley si unisce a lei; attivano il drone e decollano.
    \end{itemize}
\end{itemize}

\subsection*{Intervento del Supervisore}

\begin{itemize}
    \item \textbf{Evento}: Disattivazione dell'Agente
    \begin{itemize}
        \item Il Supervisore osserva la scena dal centro di controllo.
        \item Comunica agli agenti che non tollera fallimenti.
        \item Disattiva l'agente a terra con un semplice comando.
    \end{itemize}
    \item \textbf{Reazione dell'Agente Superstite}
    \begin{itemize}
        \item L'agente superstite, terrorizzato, continua l'inseguimento con maggiore determinazione.
    \end{itemize}
\end{itemize}

\subsection*{Attraversamento del Gate di Hadamard}

\begin{itemize}
    \item \textbf{Evento}: Ingresso nel Portale
    \begin{itemize}
        \item Laura guida il drone attraverso un ingresso segnato con una grande \textbf{H}.
        \item Marley riconosce che si tratta di un portale speciale.
    \end{itemize}
    \item \textbf{Effetti del Gate di Hadamard}
    \begin{itemize}
        \item Marley avverte una chiarezza mentale mai provata prima.
        \item Laura percepisce uno stato di sovrapposizione quantistica, sentendosi confusa e frammentata.
        \item Nonostante il disorientamento, Laura continua a pilotare il drone.
    \end{itemize}
\end{itemize}

\subsection*{Concentrarsi sulla Fuga}

\begin{itemize}
    \item \textbf{Inseguimento dell'Agente}
    \begin{itemize}
        \item L'agente modifica la configurazione del suo drone per aumentare la velocità.
        \item Marley avverte che l'agente sta guadagnando terreno.
    \end{itemize}
    \item \textbf{Determinazione di Laura}
    \begin{itemize}
        \item Laura decide di sfruttare la sua conoscenza dei percorsi per seminare l'agente.
        \item Lotta per mantenere la lucidità nonostante gli effetti del gate di Hadamard.
        \item Si concentra sul salvataggio di Caterina, determinata a non cedere.
    \end{itemize}
\end{itemize}
%%8
\section*{Timeline del Capitolo 8}

\subsection*{Inseguimento attraverso il Portale C-NOT}

\begin{itemize}
    \item \textbf{Evento}: Inseguimento dell'agente
    \begin{itemize}
        \item Laura pilota il drone con maestria, ma l'agente è sempre più vicino.
        \item Davanti a lei appare un portale segnato con il simbolo \textbf{C-NOT}.
        \item Laura attraversa il portale senza esitazione, seguita dall'agente.
    \end{itemize}
    \item \textbf{Effetto del Portale}
    \begin{itemize}
        \item Attraversando il portale C-NOT mentre è in stato di Hadamard, Laura entra in \textbf{entanglement} con l'agente.
        \item Si ritrova in uno \textbf{stato di Bell}, correlata a livello quantistico con l'agente.
        \item Realizza che ogni sua azione avrà conseguenze immediate sull'agente e viceversa.
    \end{itemize}
\end{itemize}

\subsection*{Laura Passa all'Azione}

\begin{itemize}
    \item \textbf{Sfruttamento dello Stato di Bell}
    \begin{itemize}
        \item Laura visualizza la struttura del drone dell'agente grazie all'entanglement.
        \item Modifica la configurazione del suo drone, disponendo i quattro rotori su un unico piano per una maggiore manovrabilità.
        \item Avverte una nuova fluidità nei movimenti, sentendosi un tutt'uno con il drone.
        \item Prova un misto di eccitazione e terrore, consapevole che ogni manovra deve essere precisa.
    \end{itemize}
\end{itemize}

\subsection*{Il Commissario Prende Misure Drastiche}

\begin{itemize}
    \item \textbf{Osservazione del Commissario}
    \begin{itemize}
        \item Il \textbf{Commissario} osserva i movimenti di Laura e si preoccupa della sua abilità.
        \item Decide che Laura è una minaccia che deve essere neutralizzata.
    \end{itemize}
    \item \textbf{Ordine di Criptazione}
    \begin{itemize}
        \item Il Commissario ordina di criptare il sistema utilizzando l'algoritmo \textbf{RSA 2048}.
        \item Intende bloccare ogni dato e movimento per impedire a Laura e Marley di sfuggire.
        \item I tecnici attivano i protocolli di criptazione sotto le sue direttive.
    \end{itemize}
\end{itemize}

\subsection*{Laura Intrappolata nella Criptazione}

\begin{itemize}
    \item \textbf{Effetto della Criptazione}
    \begin{itemize}
        \item Laura avverte una pesantezza improvvisa; l'ambiente circostante si cristallizza.
        \item Realizza di essere stata criptata insieme all'ambiente.
        \item Si sente bloccata, incapace di muoversi o pensare con chiarezza.
    \end{itemize}
    \item \textbf{Ricordo del Professor Shor}
    \begin{itemize}
        \item Ricorda le parole del Professor Shor che le aveva detto di conoscere alcuni algoritmi a memoria.
        \item Comprende che deve richiamare l'\textbf{algoritmo di Shor} per decriptare il sistema e liberarsi.
    \end{itemize}
\end{itemize}

\subsection*{Riflessione di Laura}

\begin{itemize}
    \item \textbf{Richiamo dell'Algoritmo di Shor}
    \begin{itemize}
        \item Laura ripercorre mentalmente i passaggi dell'algoritmo di Shor:
        \begin{enumerate}
            \item \textbf{Pre-processing}: Identifica il numero \( N \) da fattorizzare, prodotto di due grandi numeri primi \( P \) e \( Q \).
            \item Sceglie un numero casuale \( a \) relativamente primo rispetto a \( N \).
            \item \textbf{Quantum Order Finding}: Calcola il periodo \( r \) della funzione \( f(x) = a^x \mod N \) utilizzando la sovrapposizione quantistica.
            \item Verifica se \( r \) è pari; se sì, procede al passo successivo.
            \item Calcola \( \text{gcd}(a^{r/2} \pm 1, N) \) per trovare i fattori \( P \) e \( Q \).
        \end{enumerate}
    \end{itemize}
    \item \textbf{Frustrazione e Consapevolezza}
    \begin{itemize}
        \item Laura si sente più sicura ma anche frustrata, rendendosi conto che le manca un'informazione cruciale.
        \item Si chiede come comunicare efficacemente il valore di \( r \) e ottenere i fattori corretti di \( N \).
        \item Comprende che deve trovare un modo per completare la decifrazione e liberarsi dalla criptazione.
    \end{itemize}
\end{itemize}
%%9
\section*{Timeline del Capitolo 9}

\subsection*{Il Messaggio di Shore}

\begin{itemize}
    \item \textbf{Evento}: Il professor Shore, tenuto prigioniero dal Commissario, decide di inviare un messaggio a Laura utilizzando il \textbf{dense coding}.
    \item Chiama Bob, il responsabile tecnico delle comunicazioni, e gli spiega la situazione.
    \item Shore codifica l'informazione mancante nell'algoritmo di Shor e la invia a Laura.
\end{itemize}

\subsection*{La Decifrazione}

\begin{itemize}
    \item Laura riceve un messaggio criptato mentre pilota il drone.
    \item Inizia a decifrare le informazioni inviate da Shore.
    \item Completa l'algoritmo di Shor e decripta il sistema, liberandosi dalla prigione digitale.
\end{itemize}

\subsection*{La Fuga e l'Arrivo alla Ion Trap}

\begin{itemize}
    \item Laura e Marley si dirigono verso la \emph{Ion Trap}, dove Caterina è prigioniera.
    \item L'agente continua a inseguirle.
    \item Laura afferma che non si fermeranno finché non avranno liberato Caterina.
\end{itemize}

\subsection*{L'Accusa al Commissario}

\begin{itemize}
    \item Marley e Laura affrontano il Commissario.
    \item Marley lo accusa di sfruttare l'ossessione del \emph{Quantum Control Program} per la coerenza, per perseguire i suoi piani di creare un computer rivale.
    \item Il Commissario reagisce con rabbia.
\end{itemize}

\subsection*{La Liberazione}

\begin{itemize}
    \item Laura libera Caterina rimuovendo il dispositivo di cattura.
    \item Libera anche il professor Shore dalla sua restrizione.
    \item Laura afferma che ora possono mostrare al mondo che non sono semplici qubit in una rete.
\end{itemize}

\subsection*{Il Commissario e l'Entanglement}

\begin{itemize}
    \item Il Commissario nota che l'agente e Laura sono in uno \textbf{stato di Bell}, entangled.
    \item Ordina all'agente di gettarsi nel \textbf{mare di Dirac}, una condizione quantistica pericolosa.
    \item Se l'agente si getta, Laura subirà la stessa sorte a causa dell'entanglement.
\end{itemize}

\subsection*{L'Urlo di Marley}

\begin{itemize}
    \item Marley realizza la gravità della situazione e avverte Laura.
    \item Laura comprende che devono trovare un modo per interrompere l'entanglement.
\end{itemize}

\subsection*{Il Sacrificio di Shore}

\begin{itemize}
    \item Il professor Shore propone di utilizzare un \textbf{gate di Toffoli} per liberare Laura e l'agente dall'entanglement.
    \item Guida Laura e l'agente attraverso il gate di Toffoli.
    \item Uscendo dal gate, Shore si sacrifica sottoponendosi a una misura, liberando Laura e l'agente.
\end{itemize}

\subsection*{La Libertà di Laura e Caterina}

\begin{itemize}
    \item Laura e Caterina sono finalmente libere.
    \item Insieme a Marley, si allontanano dal caos della \emph{Classical Control Unit}.
    \item Sentono la libertà ma sono preoccupate per la possibile vendetta del Commissario.
\end{itemize}

\subsection*{L'Ira del Quantum Master Program}

\begin{itemize}
    \item Il \emph{Quantum Master Program} viene a conoscenza della fuga del Commissario ed è furioso.
    \item Ordina di chiudere l'uscita dal \textbf{Quantum Channel}.
    \item L'uscita che Mark aveva descritto a Caterina è ora bloccata.
\end{itemize}

\subsection*{L'Inganno della Temperatura}

\begin{itemize}
    \item Senza via di fuga, Laura sente il freddo aumentare: stanno abbassando ulteriormente la temperatura.
    \item Ricorda un reparto speciale di Amazon che aveva visto per caso.
    \item Pensa che potrebbe esserci un reparto simile lì, offrendo una possibile via di fuga.
\end{itemize}

\subsection*{La Direzione verso il Quantum Channel}

\begin{itemize}
    \item Laura decide di dirigersi verso il \emph{Quantum Channel} alla ricerca di un'uscita.
    \item Marley e Caterina la seguono, sperando di trovare una soluzione.
\end{itemize}

\subsection*{L'Inseguimento dei Droni}

\begin{itemize}
    \item Due nuovi droni inseguono Laura e Caterina.
    \item Laura si avvicina a un portale controllato dall'agente Ising, che gestisce l'entrata al \emph{Quantum Annealing}.
    \item Due agenti bloccano la loro strada.
    \item Un'esplosione avviene: molecole di \( O_2 \) reagiscono con il metano (\( CH_4 \)), creando una distrazione.
    \item Marley nota che la Resistenza è ora in grado di usare i droni.
    \item Approfittano della distrazione per entrare nel portale.
\end{itemize}

\subsection*{Il Tuffo nel Quantum Annealing}

\begin{itemize}
    \item Entrando nel portale, Laura e Caterina vengono catapultate nel \emph{Quantum Annealing}.
    \item Vivono un turbine di salti quantici, con visioni dei loro futuri.
    \item Laura vede un futuro in cui continua a trascurare gli altri, portandola a una vita solitaria.
    \item Caterina vede se stessa in una relazione opprimente, dominando il suo fidanzato.
    \item Entrambe realizzano la necessità di cambiare.
\end{itemize}

\subsection*{L'Illuminazione nel Turbine}

\begin{itemize}
    \item Un campo magnetico esterno agisce sulle loro menti durante il \emph{Quantum Annealing}.
    \item Percepiscono diverse esperienze sovrapporsi, osservando percorsi alternativi delle loro vite.
    \item Il campo magnetico si intensifica; le scelte alternative svaniscono e i loro obiettivi diventano chiari.
    \item Raggiungono uno stato di minima energia mentale, pronte a uscire dall'annealing.
    \item Hanno appreso importanti lezioni sulle loro vite e su ciò che vogliono davvero.
\end{itemize}
%%10\section*{Timeline del Capitolo 10}

\subsection*{La Calma dopo il Processo di Annealing}

\begin{itemize}
    \item \textbf{Evento}: Dopo l'elaborazione nel \emph{Quantum Annealing}, una grande calma regna nel sistema.
    \item \textbf{Laura} si ritrova a casa, sdraiata sul pavimento con il suo cane \textbf{Hiroki} accanto.
    \item Sente sollievo ma si chiede cosa sia successo a \textbf{Caterina}.
    \item Riflette sulle sue esperienze e sul bisogno di fare scelte significative nella sua vita.
\end{itemize}

\subsection*{L'Incontro con Eva}

\begin{itemize}
    \item \textbf{Luogo}: Ufficio di \textbf{Eva}.
    \item Eva tenta di concludere l'incontro con Caterina, dicendo che possono salutarsi.
    \item \textbf{Caterina} insiste nel voler capire cosa sia realmente successo.
    \item Eva cerca di rassicurarla, ma Caterina esprime il desiderio di conoscere la verità.
    \item La \textbf{PZZIA} interviene, sostenendo Caterina e rivelando che Eva ha nascosto la sua valutazione positiva.
    \item Caterina si sente ingannata e affronta Eva.
    \item Eva è visibilmente in difficoltà mentre la PZZIA incoraggia Caterina a reclamare le sue scelte.
\end{itemize}

\subsection*{Dialogo tra QMP e PZZIA}

\begin{itemize}
    \item \textbf{Personaggi}: \textbf{Quantum Master Program (QMP)} e \textbf{PZZIA}.
    \item Il QMP condivide con PZZIA che ha assistito a un algoritmo di \emph{annealing} quantistico che funziona senza coerenza assoluta tra i qubit.
    \item Riconosce di aver limitato il potenziale dei qubit con restrizioni eccessive.
    \item PZZIA concorda e suggerisce che abbracciare l'incertezza può portare a nuovi risultati.
    \item Il QMP decide di rivedere il suo approccio, aperto al cambiamento.
\end{itemize}

\subsection*{La Rivelazione della PZZIA}

\begin{itemize}
    \item \textbf{Luogo}: Ufficio di Eva.
    \item \textbf{Caterina} chiede spiegazioni riguardo alle sue valutazioni scomparse.
    \item La \textbf{PZZIA} rivela che Eva ha deliberatamente nascosto il file valutativo di Caterina.
    \item Eva tenta di interrompere e minimizzare la situazione.
    \item Caterina si sente tradita e chiede ulteriori spiegazioni.
    \item Eva chiama la sicurezza per rimuovere Caterina.
    \item Gli agenti arrivano ma scoprono che il codice autorizzativo di Eva non è valido.
    \item La PZZIA informa che i permessi di Eva sono stati revocati dal QMP.
    \item Eva è sconvolta; la PZZIA spiega che il QMP ha deciso di apportare cambiamenti.
    \item Gli agenti non possono eseguire le richieste di Eva e si allontanano.
    \item Caterina ringrazia la PZZIA per l'aiuto.
    \item Eva, rassegnata, ammette di aver commesso errori.
    \item Caterina propone di andare avanti e lavorare insieme per migliorare le cose.
\end{itemize}

\subsection*{Conclusione}

\begin{itemize}
    \item Con il supporto di \textbf{Laura}, \textbf{Caterina} e la \textbf{PZZIA} hanno una nuova visione per affrontare il \textbf{Quantum Master Program}.
    \item Si preparano alle prossime mosse che potrebbero definire il loro destino e quello di tutti i qubit nel sistema.
    \item La battaglia per la libertà è appena cominciata.
\end{itemize}


