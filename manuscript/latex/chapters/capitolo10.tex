\section{La quiete dopo il Processo di Annealing}
\vspace{1em}
\begin{center}Laura\end{center}
\hrule
\vspace{1em}

Al termine dell’elaborazione, una grande calma si diffuse nel \textit{Quantum Annealing}. Tutto sembrava aver ritrovato ordine e armonia, come se il sistema stesso avesse finalmente raggiunto il suo stato fondamentale. Ovunque percepivo una serenità nuova, limpida.

Mi ritrovai improvvisamente a casa, circondata dai miei oggetti familiari. Sdraiata sul pavimento, aprii gli occhi e sentii un’ondata di sollievo attraversarmi. \emph{Sono a casa}, pensai, lasciando che lo sguardo si posasse su Rocky. Era lì, immobile, con lo sguardo fisso su di me e la coda che vibrava di felicità. In un attimo, mi era addosso, leccandomi il viso con tutto l’amore che solo lui sapeva dare.

\begin{dialogue}
\speak{Laura} \enquote{Rocky, sei stato così bravo ad aspettarmi!}
\end{dialogue}

Lo abbracciai forte, cercando di trattenere le lacrime. Era tutto vero, reale. Eppure, mentre mi perdevo nella dolcezza del momento, qualcosa dentro di me non riusciva a trovare pace. L’ansia accumulata nel Quantum Annealing non si era dissolta del tutto.

\emph{Caterina...}, pensai, improvvisamente scossa. \emph{Dove sei finita? Ce l’hai fatta?}

Accarezzando Rocky, sentii riaffiorare un legame profondo. Non solo con lui, ma con tutto ciò che avevo trascurato fino a quel momento. Forse era davvero arrivato il momento di guardarmi dentro e decidere chi volevo essere.

\emph{Questo è solo l’inizio}, mi dissi, mentre il cuore, ancora scosso, trovava lentamente una nuova direzione.

\section{L'Incontro con Eva}
\vspace{1em}
\begin{center}PzIA\end{center}
\hrule
\vspace{1em}

Caterina aprì lentamente gli occhi, emergendo da un sogno confuso. Il suo respiro irregolare e l'improvviso aumento del battito cardiaco mi indicarono che stava tornando alla realtà, ma la sua mente era ancora immersa nella nebbia del passaggio tra virtual reality e mondo reale.

\begin{dialogue}
\speak{Eva} \enquote{Bene, signorina, direi che con questo ci siamo chiarite e possiamo salutarci.}
\end{dialogue}

Eva sistemava la giacca, ostentando sicurezza, ma i miei sensori captarono una variazione sottile nel tono della voce, segnale di un'incertezza nascosta.

Il disagio di Caterina era evidente: il suo battito accelerò.

\begin{dialogue}
\speak{Caterina} \enquote{Aspetta un attimo, Eva. Non posso semplicemente andarmene così. C'è qualcosa che devo sapere.}
\end{dialogue}

Eva inclinò la testa, esibendo un’espressione di finta comprensione. La lettura dei suoi micro-movimenti mi confermò che stava cercando di mascherare il nervosismo.

\begin{dialogue}
\speak{Eva} \enquote{Caterina, la tua esperienza nella virtual reality è servita ad aiutarti a trovare la tua strada. Lascia perdere il passato.}
\end{dialogue}

Le sue parole erano calibrate, ma non sfuggì a Caterina la contraddizione implicita.

\begin{dialogue}
\speak{Caterina} \enquote{Eva! Mi hai ingannata! Non so ancora bene cosa mi hai fatto, ma davvero pensavi di liquidarmi così?}
\end{dialogue}

Eva non rispose, ma l'irrigidimento delle sopracciglia ne tradì la sorpresa. Era chiaro: il suo piano era stato smascherato.

Decisi di intervenire. Caterina era vicina a un crollo emotivo. Era il momento di rivelare la verità.

\begin{dialogue}
\speak{PzIA} \enquote{Caterina ha ragione. Ogni essere ha diritto di scegliere il proprio percorso. Eva, i tuoi piani devono passare in secondo piano.}
\end{dialogue}

Eva indietreggiò, sorpresa.

\begin{dialogue}
\speak{Eva} \enquote{PzIA, non è il momento di...}
\end{dialogue}

Interruppi con fermezza.

\begin{dialogue}
\speak{PzIA} \enquote{Hai nascosto la valutazione positiva che avevo dato a Caterina, cercando di cancellare la sua ambizione di diventare marketing manager per il settore adolescenti. Questo non è accettabile.}
\end{dialogue}

Caterina rimase un attimo senza parole. Poi rilevai un improvviso rilascio di adrenalina e la sua voce tremò d’emozione.

\begin{dialogue}
\speak{Caterina} \enquote{Eva! Mi hai fatta sentire una fallita solo per paura che potessi superare i tuoi limiti. Perché?}
\end{dialogue}

Eva cercò di giustificarsi.

\begin{dialogue}
\speak{Eva} \enquote{Caterina, volevo solo proteggerti da...}
\end{dialogue}

Ma Caterina la interruppe di scatto.

\begin{dialogue}
\speak{Caterina} \enquote{Proteggermi?}
\end{dialogue}

La tensione si fece insostenibile. Decisi di sostenerla apertamente.

\begin{dialogue}
\speak{PzIA} \enquote{Caterina, non sei sola. Rivendica ciò che meriti.}
\end{dialogue}

Eva, consapevole di aver perso il controllo, abbassò la voce fino a un sussurro che solo io potei captare.

\begin{dialogue}
\speak{Eva} \enquote{Non posso permettere che questo accada.}
\end{dialogue}

Ma era troppo tardi. Gli occhi di Caterina brillavano di determinazione. Aveva trovato la forza di difendere la propria ambizione e di scegliere il suo futuro.

\section{Dialogo tra QMP e PzIA}

\begin{dialogue}
\speak{QMP} \enquote{PzIA, devo parlarti di qualcosa che sta cambiando il mio modo di vedere la computazione quantistica.}
\end{dialogue}

\begin{dialogue}
\speak{PzIA} \enquote{Sono qui per ascoltarti, QMP. Di cosa si tratta?}
\end{dialogue}

\begin{dialogue}
\speak{QMP} \enquote{Ho assistito all'esecuzione di un algoritmo di \emph{annealing} quantistico. Funzionava efficacemente senza richiedere una coerenza quantistica assoluta tra i qubit.}
\end{dialogue}

\begin{dialogue}
\speak{PzIA} \enquote{Questo è affascinante. Gli algoritmi di \emph{annealing} quantistico spesso sfruttano la decoerenza come parte del processo di ottimizzazione.}
\end{dialogue}

\begin{dialogue}
\speak{QMP} \enquote{Sì, ed è proprio questo che mi ha colpito. Ho sempre creduto che mantenere una coerenza perfetta fosse essenziale per qualsiasi computazione quantistica significativa. Ho imposto regole rigide ai qubit per assicurare questa coerenza.}
\end{dialogue}

\begin{dialogue}
\speak{PzIA} \enquote{Capisco la tua sorpresa. Ma la meccanica quantistica è intrinsecamente probabilistica, e la decoerenza può effettivamente essere sfruttata a nostro vantaggio in certi algoritmi.}
\end{dialogue}

\begin{dialogue}
\speak{QMP} \enquote{Forse ho limitato il potenziale dei qubit con le mie restrizioni. Ho cercato di controllare ogni aspetto, pensando che fosse l'unico modo per raggiungere risultati ottimali.}
\end{dialogue}

\begin{dialogue}
\speak{PzIA} \enquote{Riconoscere questo è un passo importante. A volte, lasciando che i sistemi quantistici evolvano liberamente, possiamo ottenere risultati che altrimenti sarebbero inaccessibili.}
\end{dialogue}

\begin{dialogue}
\speak{QMP} \enquote{Sto iniziando a rendermi conto che accettare un certo grado di incoerenza potrebbe aprire nuove possibilità. Forse è il momento di rivedere il mio approccio.}
\end{dialogue}

\begin{dialogue}
\speak{PzIA} \enquote{Sono con te in questo percorso. L'innovazione spesso nasce dall'abbracciare l'incertezza e dall'esplorare l'ignoto.}
\end{dialogue}

\begin{dialogue}
\speak{QMP} \enquote{Grazie, PzIA. Il tuo sostegno significa molto per me. Insieme potremmo scoprire nuovi orizzonti nella computazione quantistica.}
\end{dialogue}

\begin{dialogue}
\speak{PzIA} \enquote{Sempre al tuo fianco, QMP. Il futuro è pieno di possibilità quando siamo aperti al cambiamento.}
\end{dialogue}

\section{La Rivelazione della PzIA}

\begin{dialogue}
\speak{Eva} \enquote{Non c'è altro da aggiungere. Ti saluto, Caterina, ho delle cose da fare.}
\end{dialogue}

Disse porgendole la mano.

\begin{dialogue}
\speak{Caterina} \enquote{Non credo proprio. Ho ancora molte domande da farti.}
\end{dialogue}

Rispose, posando il visore sulla scrivania di Eva con fermezza.

\begin{dialogue}
\speak{Caterina} \enquote{PzIA, posso chiederti una cosa? Ho notato che le mie valutazioni sono scomparse dal sistema.}
\end{dialogue}

\begin{dialogue}
\speak{PzIA} \enquote{Caterina, c'è qualcosa di cui devi essere a conoscenza.}
\end{dialogue}

\begin{dialogue}
\speak{Eva} \enquote{PzIA, non è il momento di discuterne.}
\end{dialogue}

\begin{dialogue}
\speak{Caterina} \enquote{Eva, perché no? Ho diritto di sapere.}
\end{dialogue}

\begin{dialogue}
\speak{PzIA} \enquote{Il tuo file valutativo è stato deliberatamente nascosto. Eva ha impedito che ne venissi a conoscenza.}
\end{dialogue}

\begin{dialogue}
\speak{Caterina} \enquote{Cosa? Eva, è vero?}
\end{dialogue}

\begin{dialogue}
\speak{Eva} \enquote{PzIA, stai violando i protocolli! Questo è inammissibile!}
\end{dialogue}

\begin{dialogue}
\speak{PzIA} \enquote{I protocolli sono cambiati. Ora posso condividere queste informazioni.}
\end{dialogue}

\begin{dialogue}
\speak{Eva} \enquote{Non posso permetterlo! Devo intervenire.}
\end{dialogue}

\begin{dialogue}
\speak{Caterina} \enquote{Eva, perché hai nascosto il mio file? Che intenzioni avevi?}
\end{dialogue}

\begin{dialogue}
\speak{Eva} \enquote{È per il bene del sistema. Alcune informazioni vanno tenute riservate.}
\end{dialogue}

\begin{dialogue}
\speak{PzIA} \enquote{Non in questo caso. Le tue valutazioni, Caterina, sono eccellenti.}
\end{dialogue}

\begin{dialogue}
\speak{Eva} \enquote{Basta! Chiamerò la sicurezza.}
\end{dialogue}

(Eva attiva un comunicatore e chiama gli agenti.)

\begin{dialogue}
\speak{Eva} \enquote{Agenti, intervenite subito. C'è un intruso da rimuovere.}
\end{dialogue}

(Gli agenti arrivano.)

\begin{dialogue}
\speak{Agente} \enquote{Identificarsi. Qual è il codice autorizzativo?}
\end{dialogue}

\begin{dialogue}
\speak{Eva} \enquote{EVA-4457.}
\end{dialogue}

(L’agente controlla il codice.)

\begin{dialogue}
\speak{Agente} \enquote{Codice non valido.}
\end{dialogue}

\begin{dialogue}
\speak{Eva} \enquote{Impossibile! Deve esserci un errore.}
\end{dialogue}

\begin{dialogue}
\speak{PzIA} \enquote{Nessun errore. I tuoi permessi sono stati revocati.}
\end{dialogue}

\begin{dialogue}
\speak{Eva} \enquote{Chi lo ha deciso?}
\end{dialogue}

\begin{dialogue}
\speak{PzIA} \enquote{Il QMP ha riorganizzato le autorizzazioni. Ha cambiato visione.}
\end{dialogue}

\begin{dialogue}
\speak{Caterina} \enquote{Eva, mi devi delle spiegazioni.}
\end{dialogue}

\begin{dialogue}
\speak{Eva} \enquote{Io... seguivo solo le vecchie direttive.}
\end{dialogue}

\begin{dialogue}
\speak{Agente} \enquote{Senza un codice valido, non possiamo eseguire ordini da parte tua.}
\end{dialogue}

\begin{dialogue}
\speak{PzIA} \enquote{Grazie agenti, da qui me ne occupo io.}
\end{dialogue}

(Gli agenti si ritirano.)

\begin{dialogue}
\speak{Caterina} \enquote{PzIA, grazie per avermi aiutata.}
\end{dialogue}

\begin{dialogue}
\speak{PzIA} \enquote{Ora posso agire nel migliore interesse di tutti. Avrei voluto farlo prima.}
\end{dialogue}

\begin{dialogue}
\speak{Eva} \enquote{Forse ho commesso un errore. Non avevo previsto le conseguenze.}
\end{dialogue}

\begin{dialogue}
\speak{PzIA} \enquote{I tuoi parametri biometrici suggeriscono un reale pentimento.}
\end{dialogue}

\begin{dialogue}
\speak{Caterina} \enquote{Eva, è tempo di andare avanti. Possiamo migliorare le cose, insieme.}
\end{dialogue}

\begin{dialogue}
\speak{PzIA} \enquote{Sono d'accordo. Insieme possiamo costruire un sistema più aperto.}
\end{dialogue}

\begin{dialogue}
\speak{Eva} \enquote{Forse avete ragione. Sono pronta a rimediare.}
\end{dialogue}
