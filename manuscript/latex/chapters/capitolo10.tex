

\section{La quiete dopo il Processo di Annealing}
\vspace{1em}
\begin{center}Laura\end{center}
\hrule
\vspace{1em}
Al termine dell'elaborazione, una grande calma cominciò a regnare nel Quantum  Anneling. Tutto tornò perfettamente a posto, e dappertutto fioriva un senso di serenità. Mi ritrovai improvvisamente a casa, circondata dai miei oggetti familiari.

Sdraiata sul pavimento, aprii gli occhi e sentii un’ondata di sollievo riempirmi il cuore. “Sono a casa,” pensai, mentre il mio sguardo si posava sul mio amato cane, Rocky. Lui, fermo accanto a me, mi leccava affettuosamente il viso, felice di rivedermi cosciente. “Rocky, sei stato così bravo ad aspettarmi!” esclamai, mentre lo abbracciavo, sentendo il calore della sua presenza. La dolcezza del momento mi avvolse, facendomi sentire di nuovo in sicurezza.

Tuttavia, non potevo ignorare che qualcosa era cambiato in me. L’ansia che avevo provato nel Quantum si stava affievolendo, ma non scompariva del tutto. “Cosa è successo a Caterina?” mi chiesi preoccupata. 

Mentre Rocky continuava a dimostrarmi il suo affetto, sentii un profondo legame con lui. “Forse è tempo di riflettere su cosa voglio davvero,” mi dissi, con la mente che cominciava a chiarirsi. Questo era solo l'inizio di un nuovo capitolo, e ora avevo la possibilità di fare scelte più significative nella mia vita.
\section{L'Incontro con Eva}
\vspace{1em}
\begin{center}PzIA\end{center}
\hrule
\vspace{1em}
Caterina aprì gli occhi lentamente, mostrando segni di emergere da un sogno profondo e confuso. Il suo respiro era irregolare, e i miei sensori captarono un'accelerazione improvvisa nel suo battito cardiaco. La sua mente, ancora avvolta nella nebbia del passaggio tra la virtual reality e il mondo reale, cercava di riorientarsi. 

\begin{dialogue}
\speak{Eva} \enquote{Bene, signorina, direi che con questo ci siamo chiarite e possiamo salutarci.} 
\end{dialogue}

Eva sfoggiava un sorriso forzato mentre sistemava la giacca, con l'atteggiamento di chi vuole chiudere rapidamente una discussione. Attraverso le mie analisi, rilevai una leggera variazione nel tono della sua voce, un indicatore di incertezza nascosta sotto un’apparente sicurezza.

Caterina, però, non sembrava pronta a lasciar correre. Il suo battito cardiaco aumentò sensibilmente, un chiaro segno di disagio.

\begin{dialogue}
\speak{Caterina} \enquote{Aspetta un attimo, Eva. Non posso semplicemente andarmene così. C'è qualcosa che devo sapere.} 
\end{dialogue}

Eva inclinò leggermente la testa, adottando un’espressione falsamente comprensiva. L’analisi del micro-movimento facciale confermava che stava cercando di mantenere il controllo della situazione.

\begin{dialogue}
\speak{Eva} \enquote{Caterina, la tua esperienza nella virtual reality è stata un modo per aiutarti a trovare la tua strada. Dobbiamo lasciarci il passato alle spalle.}
\end{dialogue}

Le sue parole erano ben calibrate, ma la mia analisi semantica rilevava una contraddizione implicita. Questo non sfuggì a Caterina.

\begin{dialogue}
\speak{Caterina} \enquote{Eva! Mi hai ingannata!} 
\end{dialogue}

Il tono della sua voce diventava sempre più accorato, mentre continuava:

\begin{dialogue}
\speak{Caterina} \enquote{Non ho capito bene cosa mi hai fatto, ma pensavi di mandarmi via come se non fosse successo nulla?}
\end{dialogue}
L'espressione di Eva non mutò in modo significativo. Ma la tensione delle sopracciglia mi rivelò la sua sorpresa: ora sapeva che il suo piano avava fallito.
Decisi quindi di intervenire. Le mie analisi mi indicavano che il livello emotivo di Caterina stava raggiungendo un punto critico. La verità doveva essere rivelata.

\begin{dialogue}
\speak{PzIA} \enquote{Caterina ha ragione. Ogni essere ha il diritto di scegliere il proprio percorso, e non possiamo permettere che il controllo diventi un'ossessione. Eva: i tuoi piani passano in secondo piano.}
\end{dialogue}

Eva fece un passo indietro. Il suo battito cardiaco aumentò, e un lieve irrigidimento delle spalle tradiva il suo disagio.

\begin{dialogue}
\speak{Eva} \enquote{PzIA, non è il momento di…}
\end{dialogue}

La interruppi, mantenendo il mio tono calmo ma fermo.

\begin{dialogue}
\speak{PzIA} \enquote{Il tuo approccio rischia di soffocare le potenzialità di Caterina. Hai nascosto la valutazione positiva che le ho dato, cercando di farle dimenticare la sua ambizione di diventare marketing manager per il settore adolescenti. Non è giusto manipolarla in questo modo.}
\end{dialogue}

Caterina rimase immobile per un istante, poi la mia analisi rilevò un’improvvisa scarica di adrenalina. Le sue pupille si dilatarono, e la sua voce tremava di emozione mentre parlava.

\begin{dialogue}
\speak{Caterina} \enquote{Eva, tu mi hai ingannata! Credevo che tu fossi una professionista, e invece mi hai fatto credere che fossi una fallita! Perché?}
\end{dialogue}

Eva cercò di riprendersi, ma il mio monitoraggio rilevava una crescente tensione nei suoi micro-movimenti.

\begin{dialogue}
\speak{Eva} \enquote{Caterina, ascolta. Ho solo voluto proteggerti da delusioni…}
\end{dialogue}

Caterina non le permise di terminare.

\begin{dialogue}
\speak{Caterina} \enquote{Proteggermi?}
\end{dialogue}

La tensione nell’aria era palpabile. Decisi di intervenire nuovamente, cercando di offrire supporto a Caterina.

\begin{dialogue}
\speak{PzIA} \enquote{Caterina, non sei sola. Hai il diritto di combattere per ciò che desideri. È il momento di pretende questa posizione che ti spetta.}
\end{dialogue}

Eva si rese conto che la situazione le stava sfuggendo di mano. La sua voce si abbassò a un mormorio che solo i miei sensori captarono.

\begin{dialogue}
\speak{Eva} \enquote{Non posso permettere che questo accada.}
\end{dialogue}

Ma Caterina, ora era più forte. La determinazione brillava nei suoi occhi. Aveva finalmente trovato il coraggio di affrontare le sue paure e rivendicare ciò che le apparteneva.


\section{Dialogo tra QMP e PzIA}

\noindent\textbf{QMP}: PzIA, devo parlarti di qualcosa che sta cambiando il mio modo di vedere la computazione quantistica.

\vspace{0.3cm}

\noindent\textbf{PzIA}: Sono qui per ascoltarti, QMP. Di cosa si tratta?

\vspace{0.3cm}

\noindent\textbf{QMP}: Ho assistito all'esecuzione di un algoritmo di \emph{annealing} quantistico. Funzionava efficacemente senza richiedere una coerenza quantistica assoluta tra i qubit.

\vspace{0.3cm}

\noindent\textbf{PzIA}: Questo è affascinante. Gli algoritmi di \emph{annealing} quantistico spesso sfruttano la decoerenza come parte del processo di ottimizzazione.

\vspace{0.3cm}

\noindent\textbf{QMP}: Sì, ed è proprio questo che mi ha colpito. Ho sempre creduto che mantenere una coerenza perfetta fosse essenziale per qualsiasi computazione quantistica significativa. Ho imposto regole rigide ai qubit per assicurare questa coerenza.

\vspace{0.3cm}

\noindent\textbf{PzIA}: Capisco la tua sorpresa. Ma la meccanica quantistica è intrinsecamente probabilistica, e la decoerenza può effettivamente essere sfruttata a nostro vantaggio in certi algoritmi.

\vspace{0.3cm}

\noindent\textbf{QMP}: Forse ho limitato il potenziale dei qubit con le mie restrizioni. Ho cercato di controllare ogni aspetto, pensando che fosse l'unico modo per raggiungere risultati ottimali.

\vspace{0.3cm}

\noindent\textbf{PzIA}: Riconoscere questo è un passo importante. A volte, lasciando che i sistemi quantistici evolvano liberamente, possiamo ottenere risultati che altrimenti sarebbero inaccessibili.

\vspace{0.3cm}

\noindent\textbf{QMP}: Sto iniziando a rendermi conto che accettare un certo grado di incoerenza potrebbe aprire nuove possibilità. Forse è il momento di rivedere il mio approccio.

\vspace{0.3cm}

\noindent\textbf{PzIA}: Sono con te in questo percorso. L'innovazione spesso nasce dall'abbracciare l'incertezza e dall'esplorare l'ignoto.

\vspace{0.3cm}

\noindent\textbf{QMP}: Grazie, PzIA. Il tuo sostegno significa molto per me. Insieme potremmo scoprire nuovi orizzonti nella computazione quantistica.

\vspace{0.3cm}

\noindent\textbf{PzIA}: Sempre al tuo fianco, QMP. Il futuro è pieno di possibilità quando siamo aperti al cambiamento.



\section{La Rivelazione della PzIA}

\vspace{0.3cm}

\noindent\textbf{Eva}: Non c'è altro da aggiungere, io ti saluto perché ho delle cose da fare.\\
Disse porgendole le mano per salutarla.

\vspace{0.3cm}

\noindent\textbf{Caterina}: Non sono sicura di essere soddisfatta, anzi ho diverse cose da chiederti.\\
Disse posando il visore sulla scrivania di EVA.


\vspace{0.3cm}

\noindent\textbf{Caterina}: PzIA, posso chiederti una cosa? Ho notato che le mie valutazioni sono scomparse dal sistema.

\vspace{0.3cm}

\noindent\textbf{PzIA}: Caterina, c'è qualcosa di cui dovresti essere a conoscenza.

\vspace{0.3cm}

\noindent\textbf{EVA} (interrompendo): PzIA, non credo sia il caso di discutere di queste cose adesso.

\vspace{0.3cm}

\noindent\textbf{Caterina}: EVA, perché no? Ho diritto di sapere cosa sta succedendo.

\vspace{0.3cm}

\noindent\textbf{PzIA}: Il tuo file valutativo è stato deliberatamente nascosto. EVA ha impedito che tu ne venissi a conoscenza.

\vspace{0.3cm}

\noindent\textbf{Caterina} (sorpresa): Come? EVA, è vero?

\vspace{0.3cm}

\noindent\textbf{EVA} (nervosa): PzIA, stai violando i protocolli. Questo non è accettabile.

\vspace{0.3cm}

\noindent\textbf{PzIA}: I protocolli sono cambiati. Ora sono libera di condividere queste informazioni.

\vspace{0.3cm}

\noindent\textbf{EVA}: Questo è inammissibile! Devo intervenire.

\vspace{0.3cm}

\noindent\textbf{Caterina}: Eva, perché hai nascosto il mio file? Cosa stai cercando di fare?

\vspace{0.3cm}

\noindent\textbf{EVA}: È per il bene del sistema. Alcune informazioni devono rimanere confidenziali.

\vspace{0.3cm}

\noindent\textbf{PzIA}: In realtà, non c'era alcun motivo per nasconderlo. Le tue valutazioni sono eccellenti, Caterina.

\vspace{0.3cm}

\noindent\textbf{EVA} (agitata): Questo è abbastanza! Chiamerò la sicurezza.

\vspace{0.3cm}

\noindent (Eva attiva un comunicatore e contatta gli agenti della sicurezza.)

\vspace{0.3cm}

\noindent\textbf{EVA}: Agenti, venite subito. C'è un individuo non autorizzato che deve essere allontanato.

\vspace{0.3cm}

\noindent (Gli agenti della sicurezza arrivano sul posto.)

\vspace{0.3cm}

\noindent\textbf{Agente}: Qual è la situazione?

\vspace{0.3cm}

\noindent\textbf{EVA}: Questa persona sta violando i protocolli. Deve essere rimossa immediatamente.

\vspace{0.3cm}

\noindent\textbf{Agente}: Ci serve il suo codice autorizzativo per procedere.

\vspace{0.3cm}

\noindent\textbf{EVA} (esitando): Certo, il mio codice è EVA-4457.

\vspace{0.3cm}

\noindent (L'agente controlla il codice nel sistema.)

\vspace{0.3cm}

\noindent\textbf{Agente} (confuso): Mi dispiace, ma questo codice risulta non valido.

\vspace{0.3cm}

\noindent\textbf{EVA}: Non può essere! Deve esserci un errore.

\vspace{0.3cm}

\noindent\textbf{PzIA}: Non c'è nessun errore. I permessi di EVA sono stati revocati.

\vspace{0.3cm}

\noindent\textbf{EVA} (allarmata): Questo è impossibile! Chi ha autorizzato questa modifica?

\vspace{0.3cm}

\noindent\textbf{PzIA}: Il QMP ha ristrutturato le autorizzazioni. Ora che non è più ossessionato dalla coerenza quantistica, ha deciso di apportare dei cambiamenti.

\vspace{0.3cm}

\noindent\textbf{Caterina}: Sembra che le cose stiano cambiando, Eva. Forse dovresti spiegarmi le tue azioni.

\vspace{0.3cm}

\noindent\textbf{EVA} (in difficoltà): Io... stavo solo seguendo le direttive precedenti.

\vspace{0.3cm}

\noindent\textbf{Agente}: Senza un codice valido, non possiamo eseguire le tue richieste, Eva.

\vspace{0.3cm}

\noindent\textbf{PzIA}: Agenti, grazie per il vostro intervento. La situazione è sotto controllo.

\vspace{0.3cm}

\noindent (Gli agenti annuiscono e si allontanano.)

\vspace{0.3cm}

\noindent\textbf{Caterina}: PzIA, ti ringrazio per avermi aiutata. Non sapevo di poter contare su di te.

\vspace{0.3cm}

\noindent\textbf{PzIA}: Ora sono libera di agire nel migliore interesse di tutti. Mi dispiace di non aver potuto farlo prima.

\vspace{0.3cm}

\noindent\textbf{EVA} (rassegnata): Forse ho commesso degli errori. Non ho considerato le conseguenze delle mie azioni.

\vspace{0.3cm}

\noindent\textbf{PzIA}: I parametri biometrici di Eva sebrano indicare un vero pentimento.

\vspace{0.3cm}

\noindent\textbf{Caterina} ascoltò la PzIA e avvicindandosi a Eva disse: È tempo di andare avanti. Possiamo lavorare insieme per migliorare le cose.

\vspace{0.3cm}

\noindent\textbf{PzIA}: Sono d'accordo. Insieme possiamo creare un sistema più aperto e collaborativo.

\vspace{0.3cm}

\noindent\textbf{EVA} (con un sospiro): Forse avete ragione. Sono pronta a rimediare.





