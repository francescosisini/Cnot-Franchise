
\vspace{1em}
\begin{center}PzIA\end{center}
\hrule
\vspace{1em}

\textbf{Signore e signori, che manovre!} Laura guida il drone come una veterana, ma l'agente non molla e si fa sotto, stringendo le traiettorie come un predatore! \emph{Attenzione!} Il battito di Laura accelera, il respiro si fa corto. Ed ecco, proprio adesso, davanti a lei appare il portale marcato con il simbolo \textbf{Cnot}, un varco luminoso nel caos del circuito!

\textbf{Incredibile, Laura non ci pensa due volte!} Si lancia nel portale \textbf{Cnot} come un razzo! Ma ecco il colpo di scena: anche l'agente la segue all'istante! \emph{Allerta massima!} Le loro traiettorie si intrecciano e... \textbf{BAM!} Entrano in \textbf{entanglement}! Le loro menti ora si rispecchiano, legate in uno \textbf{stato di Bell} che cambierà il destino della partita! Ogni azione di Laura... ogni pensiero dell'agente... \emph{ora sono connessi!}


\begin{tcolorbox}[colback=gray!5,colframe=gray!80,title=\textbf{Scheda Informativa}]
\begin{itemize}
    \item \textbf{Luogo}: \emph{Qubit Array}
    \item \textbf{Giorno e ora}: Il tempo non è osservabile
    \item \textbf{Situazione}: Laura e Marley puntano al FTC.
\end{itemize}
\end{tcolorbox}

\textbf{Signori, occhi puntati su Laura!} Il volto si irrigidisce, lo sguardo si allarga: ha capito! È \textbf{intrappolata} in uno \textbf{stato di Bell}! Ora ogni sua mossa rimbalza dritta sull'agente come un riflesso quantistico! \textbf{Situazione disperata}, amici osservatori: deve reagire, e in fretta, o sarà tutto finito!


\section{Laura passa all'azione}
\vspace{1em}
\begin{center}Laura\end{center}
\hrule
\vspace{1em}

Sfruttai l'entanglement come una finestra aperta sul suo drone: vedevo ciò che vedeva lui, intuivo ciò che pensava. Il cruscotto dell'agente era davanti a me, nitido come se fossi seduta al suo posto. E lì capii. Bastava passare alla \textit{configurazione piana}.
Senza pensarci due volte, ruotai i rotori fino a stenderli su un unico piano. Il drone scattò in avanti, morbido, obbediente. Era come se i suoi movimenti fossero il prolungamento delle mie dita. Un brivido mi percorse la schiena: il controllo era perfetto.
Il battito del cuore si fece rapido, ma non mi fermò. Ogni reazione del drone seguiva la mia concentrazione, fluido e preciso. Per la prima volta dall'inizio di questa fuga, sentii l'adrenalina trasformarsi in forza.
Sorrisi. Solo un attimo, solo per me. Non ero ancora al sicuro, ma almeno avevo ribaltato la situazione. E dentro di me, tra l'esaltazione e la paura, una certezza: dovevo farcela. Non solo per me. Per Caterina.



\section{Il Commissario Prende Misure Drastiche}

\vspace{1em}
\begin{center}PzIA\end{center}
\hrule
\vspace{1em}


Nel quartier generale, il Commissario osserva attentamente i movimenti di Laura e l'efficienza con cui manovra il drone. Rileva che Laura non è un'avversaria comune. Inizialmente aveva considerato la possibilità di controllarla, sfruttando il suo spirito ribelle per integrarla nei suoi piani. Tuttavia, ora riconosce che rappresenta una potenziale minaccia.

Il Commissario prende una decisione drastica: deve fermare Laura e Marley prima che la situazione sfugga al suo controllo.

\begin{tcolorbox}[colback=white!95!blue!5, colframe=blue!75!black, title=\textbf{Ordine del Commissario}, fonttitle=\bfseries]
\emph{\enquote{Criptate immediatamente l’intero sistema utilizzando l'algoritmo RSA! Non possiamo permettere ulteriori violazioni.}}
\end{tcolorbox}

I tecnici iniziarono a lavorare rapidamente per implementare l’algoritmo RSA. 
La loro prima azione fu la selezione di due numeri primi: \( p = 61 \) e \( q = 53 \).

Il primo passo fu calcolare \( n \), il prodotto dei due numeri primi:
\[
n = p \times q = 61 \times 53 = 3233\]

Successivamente, calcolarono la funzione di Eulero:
\[
\phi(n) = (p-1)(q-1) = (61-1)(53-1) = 60 \times 52 = 3120\]

Da un’altra console, un tecnico selezionò \( e = 17 \), un valore standard per \( e \) poiché è primo rispetto a \( \phi(n) \). Il passo successivo fu calcolare \( d \), l’inverso moltiplicativo di \( e \) modulo \( \phi(n) \):
\[
 d = e^{-1} \mod \phi(n)
\]

Utilizzando un algoritmo per il calcolo dell’inverso moltiplicativo, \( d \) risultò:
\[
 d = 2753
\]

Con \( n = 3233 \), \( e = 17 \), e \( d = 2753 \), le chiavi RSA erano pronte per l’uso. I tecnici iniziarono immediatamente a criptare i dati.

Ogni messaggio originale \( m \), numericamente rappresentabile come un blocco, venne trasformato in un messaggio cifrato \( c \):
\[
 c = m^e \mod n
\]


Questi dati criptati furono poi distribuiti attraverso il sistema.

\begin{tcolorbox}[colback=white!95!green!5, colframe=green!75!black, title=\textbf{Risultato della Cifratura RSA}, fonttitle=\bfseries]
\emph{\enquote{Signore, la cifratura è completa. Il sistema è ora protetto.}}
\end{tcolorbox}

Il Commissario, osservando i monitor, annuì soddisfatto.
\newpage

\begin{tcolorbox}[colback=white!95!blue!5, colframe=blue!75!black, title=\textbf{Commissario}, fonttitle=\bfseries]
\emph{\enquote{Eccellente. Ora nessuna fuga sarà possibile. Monitorate ogni attività. Voglio un controllo assoluto.}}
\end{tcolorbox}



\section{Laura Intrappolata nella Criptazione}

\vspace{1em}
\begin{center}Laura\end{center}
\hrule
\vspace{1em}

L'aria si fece spessa come un campo viscoso. I comandi del drone rispondevano a fatica, ogni gesto sembrava lottare contro un fluido invisibile. Davanti a noi, il FTC: vicinissimo, irraggiungibile.

Tutto si deformava, come se il tempo si fosse piegato, congelandosi in un istante che si dilatava all’infinito.

Sentii il battito nel petto, forte, sordo, come un metronomo impazzito. Non capivo se fossi io a rallentare o il mondo attorno.

\begin{dialogue}
\speak{Laura} \enquote{Cosa credi sia successo Marley?}
\end{dialogue}

Mi guardò confusa.

\begin{dialogue}
\speak{Marley} \begin{tcolorbox}[colback=white!95!blue!5, colframe=blue!75!black, title=\textbf{Messaggio di Marley}, fonttitle=\bfseries]
\emph{
641, 2185, 1230, 1632, 1992, 1230, 884, 1632, 3179, 1992, 1773, 3179, 281, 1313, 2235, 1773, 2185, 1992, 2726, 1632, 2160, 2412, 1632, 1853, 3216, 1853, 1992, 1307, 1773, 1773, 3179, 2185, 2825, 1992, 3000, 1632, 2235, 2235, 2185, 1992, 281, 2412, 3179, 612, 884, 1632, 884, 2185, 1992, 3179, 745, 1992, 1230, 3179, 1230, 884, 1313, 2271, 1632
}
\end{tcolorbox}

\end{dialogue}



Marley farfugliava solo una sequenza di numeri, rapida, monotona, incomprensibile.

Perché non mi rispondeva normalmente? Cosa diavolo significavano quei numeri?

Poi mi colpì come uno schiaffo: erano cifre, puro testo cifrato.

Un brivido mi attraversò.

«Hanno criptato tutto!» pensai, sentendo l’ansia serrarmi la gola.

Non sapevo da dove cominciare. Ogni pensiero correva caotico. ``Fai mente locale, Laura. Respira. Pensa. Aritmetica modulare...'' Ma il panico montava, e con lui l’incapacità di ragionare. La mia corteccia sembrava bloccata, l’unica cosa che riuscivo a fare era ripetermi quella parola: aritmetica.

Fu allora che riaffiorò la voce di Shor, come una nota stonata ma nitida nella confusione:

\emph{“Ci sono cose che devi sapere a memoria, Laura. Non sempre avrai il tempo di ricalcolare da capo.”}
La frustrazione dell’esame mi investì di nuovo, ma stavolta compresi davvero il senso di quelle parole. Non era più teoria: l'algoritmo di Shor era l’unica via d’uscita.


Con il respiro affannoso e il cuore che batteva come un tamburo, iniziai a richiamare i passaggi dell'algoritmo, consapevole che ogni secondo era cruciale. La consapevolezza della mia stessa inadeguatezza pesava sul cuore, ma al tempo stesso sentivo crescere dentro di me una determinazione nuova. Questa era la mia prova. Dovevo ricordare, dovevo riuscirci... o rischiare di rimanere imprigionata per sempre in quella rete di criptazione.

\section{Riflessione di Laura}

 La mia mente iniziò a focalizzarsi sui concetti che avevo studiato. L'ansia del momento si mescolava a un senso di determinazione.

\emph{Devo ricordare come funziona l'algoritmo di Shor,} pensai, cercando di riorganizzare i miei ricordi. \emph{Se riesco a decifrare l'RSA, potrei trovare un modo per liberarmi da questo sistema.}

La prima cosa che mi venne in mente fu il \textbf{pre-processing}, la fase iniziale in cui devo trovare un numero intero \( N \) da fattorizzare, tipicamente il prodotto di due grandi numeri primi \( p \) e \( q \). \emph{\( N \) è ciò che protegge la chiave pubblica,} mi ricordai, visualizzando mentalmente il flusso del processo.

Poi pensai al passo successivo: la scelta di un numero casuale \( a \), tale che \( 1 < a < N \) e coprimo con \( N \). \emph{Questo è fondamentale. Se \( a \) e \( N \) condividono un fattore comune, posso risolvere immediatamente il problema,} riflettei. \emph{Altrimenti, devo passare alla parte quantistica dell'algoritmo.}

Ora entravo nel cuore dell'algoritmo: il \textbf{Quantum Order Finding}. In questo passaggio, devo calcolare il periodo \( r \) della funzione \( f(x) = a^x \mod N \). \emph{Devo trovare il minimo intero positivo \( r \) tale che \( a^r \equiv 1 \mod N \),} pensai, mentre la mia mente si concentrava sull'idea di utilizzare le proprietà della sovrapposizione e l'interferenza quantistica per ottenere il risultato.

\emph{Il trucco è preparare uno stato quantistico che rappresenti una sovrapposizione di tutti i possibili valori di \( x \),} continuai a riflettere. \emph{Poi, applicando la funzione \( f(x) \) e la trasformata di Fourier quantistica, posso ottenere informazioni sul periodo \( r \).}

Ma c'era un passaggio critico che mi sfuggiva. Mi sentivo sopraffatta dalla frustrazione.

\emph{Devo essere in grado di eseguire la trasformata di Fourier quantistica, ma come posso farlo qui?} Mi chiesi. \emph{Aspetta... il \textit{gate} di Hadamard!}

Ricordai di aver attraversato il \textit{gate} di Hadamard, che mi aveva posto in uno stato di sovrapposizione. \emph{Posso sfruttare questo stato per costruire la trasformata di Fourier quantistica,} realizzai. \emph{Ma devo riuscire a maneggiare i qubit in modo preciso.}

In quel momento, mi resi conto che l'entanglement con l'agente poteva essere una risorsa. \emph{Se utilizzo lo stato di Bell in cui mi trovo, posso condividere l'informazione quantistica e sfruttare l'entanglement per eseguire i calcoli necessari.}

Concentrandomi intensamente, iniziai a visualizzare il circuito quantistico. \emph{Applico le porte di Hadamard ai miei qubit, poi utilizzo le porte di controllo per eseguire la funzione \( f(x) \). Successivamente, eseguo la trasformata di Fourier quantistica.}

Sentivo la mia mente lavorare al limite. \emph{Devo misurare lo stato finale per ottenere un valore che mi dia informazioni su \( r \).}

Dopo un'attenta elaborazione, ottenni un risultato. \emph{Ho trovato un valore \( c \) tale che \( c \approx \dfrac{k}{r} \),} dove \( k \) è un intero. \emph{Ora devo approssimare la frazione continua per trovare \( r \).}

Utilizzai l'algoritmo delle frazioni continue per approssimare \( \dfrac{c}{2^n} \) e determinare \( r \). Finalmente, dopo quello che sembrò un tempo infinito, trovai il periodo.

\emph{Ho il valore di \( r \)!} esclamai mentalmente, sentendo un'ondata di sollievo.

Verificai che \( r \) fosse pari e che \( a^{r/2} \not\equiv -1 \mod N \). Procedetti a calcolare i seguenti valori:

\[
\text{gcd}\left(a^{\frac{r}{2}} - 1, N\right), \quad \text{gcd}\left(a^{\frac{r}{2}} + 1, N\right)
\]

\emph{Questi mi daranno i fattori primi \( p \) e \( q \) di \( N \).}


Con i fattori finalmente tra le mani, calcolai la chiave privata e decifrai il sistema.

In un istante, la pesantezza scomparve. L'aria tornò fluida, il drone rispose docile ai comandi, e la mia mente si rischiarò.

Ma non era ancora finita. Qualcosa resisteva, un velo sottile ma ostinato.

Marley mi fissava, gli occhi pieni di speranza.

Scossi lentamente la testa. La frustrazione mi colpì come un pugno: mancava ancora un passaggio.


\begin{dialogue}
\speak{Laura} \enquote{No. Manca un passaggio} dissi, anche se sapevo che, per ora, non mi poteva capire. 
\end{dialogue}


