
\vspace{1em}
\begin{center}PzIA\end{center}
\hrule
\vspace{1em}

Laura manovra il drone con notevole abilità, ma l'agente la sta rapidamente raggiungendo. I suoi parametri vitali indicano un aumento dello stress: frequenza cardiaca e respiratoria elevate. Finalemte davanti a lei appare il portale marcato con il simbolo \textbf{Cnot}.

Con un po' di esitazione, Laura si lancia attraverso il portale, seguita immediatamente dall'agente. \textbf{Allerta}: il passaggio attraverso il portale \textbf{Cnot} induce un cambiamento significativo negli stati quantistici di entrambi. Laura, entrando con il suo stato di Hadamard, si ritrova in \textbf{entanglement} con l'agente. Entrambi sono ora in uno \textbf{stato di Bell}, una condizione in cui le loro menti sono correlate a livello quantistico.

\begin{tcolorbox}[colback=gray!5,colframe=gray!80,title=\textbf{Scheda Informativa}]
\begin{itemize}
    \item \textbf{Luogo}: \emph{Qubit Array}
    \item \textbf{Giorno e ora}: Il tempo non è osservabile
    \item \textbf{Situazione}: Laura e Marley puntano al FTC.
\end{itemize}
\end{tcolorbox}

Laura mostra segni di sorpresa e terrore. Essere intrappolata in uno stato di Bell implica che ogni sua azione avrà conseguenze immediate e intrecciate con quelle dell'agente. \textbf{Situazione critica}: deve agire rapidamente per evitare la cattura.

\section{Laura passa all'azione}
\vspace{1em}
\begin{center}Laura\end{center}
\hrule
\vspace{1em}

Sfruttai l'effetto dello \textit{stato di Bell} per ottenere un vantaggio. Potevo vedere quello che vedeva l'agente, e pensare i suoi stessi pensieri. Riuscii a visualizzare il cruscotto del suo drone, e capii come  impostare il mio in  configurazione piana come aveva fatto lui. Allineai quindi i quattro rotori su un unico piano: il gioco era fatto. Il drone aveva ora una nuova fluidità nei movimenti, le azioni del  drone seguivano linearmente la mia volontà. Era una sensazione inusuale ma mi sentivo davvero potente e libera, nonostante non fossi mai stata così lontana dalla libertà!

Mentre sfrecciavo percepivo il battito del mio cuore accelerare. Ogni reazione del drone rispecchiava la mia concentrazione. Stavo affrontando la sfida, sfruttando la mia conoscenza e la mia prontezza: wow chi poteva fermarmi ora?

Per un istante, mi concessi un breve sorriso, riconoscendo come fossi riuscita a trasformare una situazione critica in un'opportunità. Tuttavia, dentro di me, una voce razionale mi ricordava che il pericolo non era ancora scampato. Ogni manovra doveva essere calcolata con precisione; ogni scelta poteva essere determinante. Mi sentivo avvolta da una complessità di possibilità, ma anche da un senso di responsabilità crescente. Dovevo essere all'altezza, non solo per me stessa, ma anche per Caterina.

\section{Il Commissario Prende Misure Drastiche}

\vspace{1em}
\begin{center}PzIA\end{center}
\hrule
\vspace{1em}


Nel quartier generale, il Commissario osserva attentamente i movimenti di Laura e l'efficienza con cui manovra il drone. Rileva che Laura non è un'avversaria comune. Inizialmente aveva considerato la possibilità di controllarla, sfruttando il suo spirito ribelle per integrarla nei suoi piani. Tuttavia, ora riconosce che rappresenta una potenziale minaccia.

Il Commissario prende una decisione drastica: deve fermare Laura e Marley prima che la situazione sfugga al suo controllo.

\begin{tcolorbox}[colback=white!95!blue!5, colframe=blue!75!black, title=\textbf{Ordine del Commissario}, fonttitle=\bfseries]
\emph{\enquote{Criptate immediatamente l’intero sistema utilizzando l'algoritmo RSA! Non possiamo permettere ulteriori violazioni.}}
\end{tcolorbox}

I tecnici iniziarono a lavorare rapidamente per implementare l’algoritmo RSA. 
La loro prima azione fu la selezione di due numeri primi: \( p = 61 \) e \( q = 53 \).

Il primo passo fu calcolare \( n \), il prodotto dei due numeri primi:
\[
n = p \times q = 61 \times 53 = 3233\]

Successivamente, calcolarono la funzione di Eulero:
\[
\phi(n) = (p-1)(q-1) = (61-1)(53-1) = 60 \times 52 = 3120\]

Da un’altra console, un tecnico selezionò \( e = 17 \), un valore standard per \( e \) poiché è primo rispetto a \( \phi(n) \). Il passo successivo fu calcolare \( d \), l’inverso moltiplicativo di \( e \) modulo \( \phi(n) \):
\[
 d = e^{-1} \mod \phi(n)
\]

Utilizzando un algoritmo per il calcolo dell’inverso moltiplicativo, \( d \) risultò:
\[
 d = 2753
\]

Con \( n = 3233 \), \( e = 17 \), e \( d = 2753 \), le chiavi RSA erano pronte per l’uso. I tecnici iniziarono immediatamente a criptare i dati.

Ogni messaggio originale \( m \), numericamente rappresentabile come un blocco, venne trasformato in un messaggio cifrato \( c \):
\[
 c = m^e \mod n
\]


Questi dati criptati furono poi distribuiti attraverso il sistema.

\begin{tcolorbox}[colback=white!95!green!5, colframe=green!75!black, title=\textbf{Risultato della Cifratura RSA}, fonttitle=\bfseries]
\emph{\enquote{Signore, la cifratura è completa. Il sistema è ora protetto.}}
\end{tcolorbox}

Il Commissario, osservando i monitor, annuì soddisfatto.
\newpage

\begin{tcolorbox}[colback=white!95!blue!5, colframe=blue!75!black, title=\textbf{Commissario}, fonttitle=\bfseries]
\emph{\enquote{Eccellente. Ora nessuna fuga sarà possibile. Monitorate ogni attività. Voglio un controllo assoluto.}}
\end{tcolorbox}



\section{Laura Intrappolata nella Criptazione}

\vspace{1em}
\begin{center}Laura\end{center}
\hrule
\vspace{1em}

Mentre guidavo il drone, sentii improvvisamente un senso di pesantezza avvolgermi, avvertivo  l'aria stessa  trasformarsi in un fluido denso e impenetrabile.
Ormai eravamo ad un passo dal FTC e da Caterina, ma tutto intorno a me sembrava rallentare, cristallizzandosi in un eterno istante. Cosa era successo?

\begin{dialogue}
\speak{Laura} \enquote{Cosa credi sia successo Marley?}
\end{dialogue}

Mi guardò confusa.

\begin{dialogue}
\speak{Marley} \begin{tcolorbox}[colback=white!95!blue!5, colframe=blue!75!black, title=\textbf{Messaggio di Marley}, fonttitle=\bfseries]
\emph{
641, 2185, 1230, 1632, 1992, 1230, 884, 1632, 3179, 1992, 1773, 3179, 281, 1313, 2235, 1773, 2185, 1992, 2726, 1632, 2160, 2412, 1632, 1853, 3216, 1853, 1992, 1307, 1773, 1773, 3179, 2185, 2825, 1992, 3000, 1632, 2235, 2235, 2185, 1992, 281, 2412, 3179, 612, 884, 1632, 884, 2185, 1992, 3179, 745, 1992, 1230, 3179, 1230, 884, 1313, 2271, 1632
}
\end{tcolorbox}

\end{dialogue}



Cosa stava dicendo, perché non mi rispondeva normalmente? Cosa rappresentavano quei numeri?\\
All'improvviso cappii e sentii un'ondata di panico salire dentro di me. Quei numeri non avevano nessuna logica, questo mondo era stato criptato! ``Come ne usciamo ora?'' pensai.\\ 
Cosa potevo fare ora? Come potevo risolvere la situazione? ``Fai mente locale Laura'' pensai, ``ripensa all'aritmetica modulare...'' Era troppo! Ora non avevo la calma necessaria per ragionare usando la coreccia frontale. Mi tornarono in mente le parole del professor Shor. Ricordavo il suo tono severo durante l'esame, quando mi aveva esortato a non affidarmi sempre alla capacità di ricalcolare tutto da zero.

\begin{quote}
\enquote{Alcune cose devi conoscerle a memoria, Laura. Non sempre avrai il tempo di risolvere ogni problema da zero,} mi aveva detto.
\end{quote}

La frustrazione di quel momento mi colpì di nuovo, ma questa volta compresi l'importanza di quelle parole. Avevo bisogno dell'algoritmo di Shor per decriptare il sistema e liberarmi, ma dovevo richiamarlo alla mente con precisione, senza esitazioni. Mi concentrai, facendo appello a ogni frammento di conoscenza, ogni dettaglio che ricordavo.

Con il respiro affannoso e il cuore che batteva come un tamburo, iniziai a richiamare i passaggi dell'algoritmo, consapevole che ogni secondo era cruciale. La consapevolezza della mia stessa inadeguatezza pesava sul cuore, ma al tempo stesso sentivo crescere dentro di me una determinazione nuova. Questa era la mia prova. Dovevo ricordare, dovevo riuscirci... o rischiare di rimanere imprigionata per sempre in quella rete di criptazione.

\section{Riflessione di Laura}

 La mia mente iniziò a focalizzarsi sui concetti che avevo studiato. L'ansia del momento si mescolava a un senso di determinazione.

\emph{Devo ricordare come funziona l'algoritmo di Shor,} pensai, cercando di riorganizzare i miei ricordi. \emph{Se riesco a decifrare l'RSA, potrei trovare un modo per liberarmi da questo sistema.}

La prima cosa che mi venne in mente fu il \textbf{pre-processing}, la fase iniziale in cui devo trovare un numero intero \( N \) da fattorizzare, tipicamente il prodotto di due grandi numeri primi \( p \) e \( q \). \emph{\( N \) è ciò che protegge la chiave pubblica,} mi ricordai, visualizzando mentalmente il flusso del processo.

Poi pensai al passo successivo: la scelta di un numero casuale \( a \), tale che \( 1 < a < N \) e coprimo con \( N \). \emph{Questo è fondamentale. Se \( a \) e \( N \) condividono un fattore comune, posso risolvere immediatamente il problema,} riflettei. \emph{Altrimenti, devo passare alla parte quantistica dell'algoritmo.}

Ora entravo nel cuore dell'algoritmo: il \textbf{Quantum Order Finding}. In questo passaggio, devo calcolare il periodo \( r \) della funzione \( f(x) = a^x \mod N \). \emph{Devo trovare il minimo intero positivo \( r \) tale che \( a^r \equiv 1 \mod N \),} pensai, mentre la mia mente si concentrava sull'idea di utilizzare le proprietà della sovrapposizione e l'interferenza quantistica per ottenere il risultato.

\emph{Il trucco è preparare uno stato quantistico che rappresenti una sovrapposizione di tutti i possibili valori di \( x \),} continuai a riflettere. \emph{Poi, applicando la funzione \( f(x) \) e la trasformata di Fourier quantistica, posso ottenere informazioni sul periodo \( r \).}

Ma c'era un passaggio critico che mi sfuggiva. Mi sentivo sopraffatta dalla frustrazione.

\emph{Devo essere in grado di eseguire la trasformata di Fourier quantistica, ma come posso farlo qui?} mi chiesi. \emph{Aspetta... il \textit{gate} di Hadamard!}

Ricordai di aver attraversato il \textit{gate} di Hadamard, che mi aveva posto in uno stato di sovrapposizione. \emph{Posso sfruttare questo stato per costruire la trasformata di Fourier quantistica,} realizzai. \emph{Ma devo riuscire a manipolare i qubit in modo preciso.}

In quel momento, mi resi conto che l'entanglement con l'agente poteva essere una risorsa. \emph{Se utilizzo lo stato di Bell in cui mi trovo, posso condividere l'informazione quantistica e sfruttare l'entanglement per eseguire i calcoli necessari.}

Concentrandomi intensamente, iniziai a visualizzare il circuito quantistico. \emph{Applico le porte di Hadamard ai miei qubit, poi utilizzo le porte di controllo per eseguire la funzione \( f(x) \). Successivamente, eseguo la trasformata di Fourier quantistica.}

Sentivo la mia mente lavorare al limite. \emph{Devo misurare lo stato finale per ottenere un valore che mi dia informazioni su \( r \).}

Dopo un'attenta elaborazione, ottenni un risultato. \emph{Ho trovato un valore \( c \) tale che \( c \approx \dfrac{k}{r} \),} dove \( k \) è un intero. \emph{Ora devo approssimare la frazione continua per trovare \( r \).}

Utilizzai l'algoritmo delle frazioni continue per approssimare \( \dfrac{c}{2^n} \) e determinare \( r \). Finalmente, dopo quello che sembrò un tempo infinito, trovai il periodo.

\emph{Ho il valore di \( r \)!} esclamai mentalmente, sentendo un'ondata di sollievo.

Verificai che \( r \) fosse pari e che \( a^{r/2} \not\equiv -1 \mod N \). Procedetti a calcolare i seguenti valori:

\[
\text{gcd}\left(a^{\frac{r}{2}} - 1, N\right), \quad \text{gcd}\left(a^{\frac{r}{2}} + 1, N\right)
\]

\emph{Questi mi daranno i fattori primi \( p \) e \( q \) di \( N \).}

Con i fattori in mano, potevo finalmente calcolare la chiave privata e decifrare il sistema. Senza perdere tempo, invertii la criptazione RSA.

Per un attimo, sentii la pesantezza svanire, l'aria diventare di nuovo leggera. Il drone riprese a muoversi liberamente, e la mia mente si schiarì. Ma non tutto era tornato come prima... ci ero vicina, ma non avevo ancora decriptato tutto.

Marley mi guardò con occhi pieni di speranza, come a chiedermi se ce l'avessi fatta.

Scossi la testa, un senso di frustrazione mi pervadeva ancora.

\begin{dialogue}
\speak{Laura} \enquote{No. Manca un passaggio} dissi, anche se sapevo che per ora non mi poteva capire. 
\end{dialogue}


