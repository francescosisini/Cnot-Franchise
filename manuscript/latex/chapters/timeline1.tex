
\section*{Timeline del Primo Capitolo}

\subsection*{Lunedì}

\subsubsection*{Mattina}

\begin{itemize}
    \item \textbf{Ore 9:30 - Pet $\mu$ Robot}
    \begin{itemize}
        \item Caterina si presenta al colloquio presso la \emph{Pet Micro Robot} per una posizione di responsabile marketing.
        \item Viene sottoposta a una preselezione guidata dall'IA PZZIA.
        \item Durante la seconda fase del colloquio, Eva, responsabile delle risorse umane, le pone domande sull'ambiente, sul cambiamento climatico e sull'intelligenza artificiale nelle aziende.
        \item Eva le assegna inaspettatamente un test di programmazione avanzata.
        \item Caterina completa il test ma con alcune incertezze.
    \end{itemize}
\end{itemize}

\subsubsection*{Pomeriggio}

\begin{itemize}
    \item \textbf{Caffetteria all'angolo}
    \begin{itemize}
        \item Dopo il colloquio, Caterina incontra Laura.
        \item Si recano insieme in una caffetteria per discutere dell'esperienza.
        \item Laura aiuta Caterina a risolvere l'algoritmo del test di programmazione, alleviando le sue preoccupazioni.
    \end{itemize}
\end{itemize}

\subsection*{Martedì}

\subsubsection*{Mattina}

\begin{itemize}
    \item \textbf{Ore 12:30 - Magazzino Bamazon}
    \begin{itemize}
        \item Laura lavora nel magazzino Amazon.
        \item Si trova in difficoltà nel sistema di gestione del magazzino (WMS), non riuscendo a individuare la corretta ubicazione di un pacco.
        \item Si imbatte in un portale con il cartello "Accesso riservato – Stoccaggi speciali".
        \item Viene fermata da Ising, un tecnico che le spiega che l'area è riservata.
    \end{itemize}
\end{itemize}

\subsubsection*{Pomeriggio}

\begin{itemize}
    \item \textbf{Ore 17:30 - Università degli Studi}
    \begin{itemize}
        \item Laura si prepara per l'esame di crittografia quantistica con il Professor Shor.
        \item Durante l'esame, affronta difficoltà nell'algoritmo di Shor.
        \item Riceve una valutazione che non la soddisfa e decide di ripetere l'esame.
    \end{itemize}
\end{itemize}

\subsubsection*{Sera}

\begin{itemize}
    \item \textbf{Ore 19:00 - Casa di Laura}
    \begin{itemize}
        \item Caterina raggiunge Laura per cena.
        \item Discutono delle rispettive giornate e delle difficoltà incontrate.
        \item Laura mostra a Caterina i suoi appunti sull'algoritmo di Shor.
        \item Fanno una passeggiata con Roky, il cane di Laura.
        \item Laura le suggerisce di verificare il file di valutazione generato dall'IA.
    \end{itemize}
\end{itemize}

\subsection*{Mercoledì}

\subsubsection*{Notte}

\begin{itemize}
    \item \textbf{Casa di Caterina}
    \begin{itemize}
        \item Caterina, tornata a casa, riflette sulle sue scelte di vita.
        \item Prepara una tisana e rivede le foto con Mark, sentendosi distaccata.
        \item Decide di scrivere un'email a Eva chiedendo il documento valutativo.
    \end{itemize}
\end{itemize}

\subsubsection*{Mattina}

\begin{itemize}
    \item \textbf{Risposta di Eva}
    \begin{itemize}
        \item Caterina riceve una risposta da Eva, che afferma che il documento è stato cancellato per errore.
        \item Eva propone un incontro per discutere di persona.
        \item Caterina accetta l'invito, sebbene perplessa.
    \end{itemize}
    \item \textbf{Ore 9:30 - Casa di Laura}
    \begin{itemize}
        \item Caterina passa da Laura per un saluto prima dell'incontro con Eva.
        \item Laura le mostra il \emph{Noemografo}, un dispositivo per la lettura dei pensieri.
        \item Sperimentano insieme una connessione mentale.
        \item Caterina lascia Laura per recarsi all'appuntamento con Eva.
    \end{itemize}
\end{itemize}

\subsubsection*{Pomeriggio}

\begin{itemize}
    \item \textbf{Incontro con Eva alla Pet $\mu$ Robot}
    \begin{itemize}
        \item Eva accoglie Caterina con un visore 3D, proponendole di rivedere il colloquio.
        \item Caterina, sebbene sospettosa, accetta di indossare il visore.
        \item Si rende conto troppo tardi che è una trappola orchestrata da Eva.
    \end{itemize}
    \item \textbf{Contemporaneamente - Casa di Laura}
    \begin{itemize}
        \item Laura avverte strane sensazioni.
        \item Realizza che la connessione mentale con Caterina non si è interrotta.
        \item Si preoccupa per l'amica e cerca di capire cosa stia accadendo.
    \end{itemize}
\end{itemize}

