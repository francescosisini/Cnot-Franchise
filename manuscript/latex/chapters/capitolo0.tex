\section{Giorno 1}

Laura si fermò davanti alla grande vetrata della Pet Microrobot, osservando con attenzione il logo luccicante dell'azienda. Aveva accompagnato Caterina al colloquio per una posizione di responsabile marketing per il settore adolescenti, un'opportunità che sembrava perfetta per la sua amica.

"Ce la farai, stai tranquilla", disse Laura cercando di trasmettere fiducia.

Caterina annuì nervosamente, il suo sguardo perso tra la folla di impiegati e visitatori che entravano e uscivano dalla grande hall.

---

Caterina entrò nella stanza e si sedette di fronte a Eva, la responsabile delle risorse umane di Pet Microrobot. Lo sguardo di Eva era attento, freddo, e i suoi occhiali riflettevano lo schermo del tablet che teneva in mano. Sul display, c'erano le risposte di Caterina ai test di valutazione gestiti da PZZIA, il software di intelligenza artificiale basato su machine learning quantistico. Il sistema, presente ovunque all'interno dell'azienda, operava in gran parte in background, ma poteva essere integrato anche in robot quando necessario.

"Prima di procedere," iniziò Eva, incrociando le mani sul tavolo, "vorrei discutere delle tue risposte riguardo al cambiamento climatico e all'ambiente. Ma prima dimmi: cosa pensi riguardo alla presenza di IA nelle aziende?"

Caterina sentì il cuore accelerare leggermente, ma mantenne un tono fermo.

"Sono profondamente impegnata nelle iniziative ambientali. Ho partecipato a progetti di sensibilizzazione locale e ho sostenuto campagne per la riduzione dell'impronta di carbonio nelle aziende con cui ho collaborato. Credo che ogni settore, compreso quello tecnologico, debba fare la sua parte per ridurre le emissioni e rendere più sostenibile l'industria."

Fece una pausa, cercando di calibrare la seconda parte della risposta.

"Quanto all'azienda, penso che robot e intelligenza artificiale, come PZZIA, possano fare molto per ottimizzare i processi e ridurre gli sprechi. Tuttavia, credo che il vero potenziale emerga quando esseri umani e macchine collaborano. L'AI è potente, ma è la creatività umana a dare un valore aggiunto che la macchina non può replicare."

Eva annuì, senza dare segni evidenti di approvazione o disapprovazione. Il tablet rimase silenzioso nelle sue mani. PZZIA non si attivò per ulteriori domande, come se avesse già raccolto tutte le informazioni necessarie durante la valutazione precedente.

---

Eva incalzò: "E cosa ne pensi dell'adozione dell'elettrico al posto dei combustibili fossili nei nostri processi produttivi?"

Caterina si prese un momento per riflettere, poi rispose con sicurezza.

"Sono molto attenta al clima e all'impatto ambientale. Tuttavia, credo che le innovazioni adottate debbano davvero ridurre le emissioni, non soltanto dare l'impressione all'utente finale di essere lui a non produrre inquinamento. Va bene l'elettrico, ma solo se l'energia utilizzata proviene da generatori certificati come il fotovoltaico, l'idroelettrico, e altre fonti rinnovabili."

Eva ascoltò la risposta senza interromperla, ma dentro di sé rifletteva. Questa posizione non le andava bene. Lei aveva intenzione di spingere l'azienda verso la certificazione, ma senza preoccuparsi del reale impatto sulle emissioni di CO2. Quello che contava, per lei, era l'immagine che l'azienda avrebbe proiettato verso l'esterno, non la vera sostenibilità delle operazioni.

---

"Interessante," disse Eva, con voce piatta. Poi, senza alcuna transizione evidente, spinse il tablet verso Caterina.

"Prima di concludere, vorrei che risolvesse un problema di programmazione avanzata. Deve implementare un algoritmo di ricerca. Ha dieci minuti."

Caterina si irrigidì per un attimo, sorpresa dalla richiesta improvvisa, ma si rimise subito a concentrarsi. Lesse rapidamente la descrizione del problema sullo schermo, mentre nella sua mente cominciava a prendere forma un'idea per la soluzione. Le sue dita si mossero con sicurezza sulla tastiera. Il codice fluiva, le strutture di controllo e i cicli erano ben organizzati, ma man mano che si avvicinava alla fine, cominciò a notare piccole discrepanze. Qualcosa non quadrava del tutto.

Non c’era più tempo per rivedere tutto. Consegnò il tablet ad Eva con un sospiro appena percettibile.

Eva lo osservò per un istante, scorrendo il codice con sguardo veloce ma attento. Poi, senza dire nulla, sollevò lo sguardo su Caterina. Sorrise appena, con un’espressione sorniona, quasi impercettibile.

"Grazie, Caterina. Riceverà notizie a breve."

Caterina uscì dalla stanza, il cuore pesante. Sapeva che qualcosa non era andato per il verso giusto con l'algoritmo. Quella piccola incertezza continuava a tormentarla mentre attraversava il corridoio, dove Laura l'aspettava, ignara della sfida appena conclusa.

---

Caterina uscì dall'edificio e trovò Laura che l'aspettava dall'altro lato della strada. Si avvicinò, visibilmente provata dal colloquio.

Laura sapeva quando era il momento di non fare troppe domande. Si limitò a un sorriso e le fece cenno di seguirla verso la piccola caffetteria all'angolo.

Entrarono e ordinarono un cappuccino e una pastina ciascuna. Laura si sistemò al tavolo, osservando Caterina che sembrava persa nei suoi pensieri.

"Allora, com'è andata?" chiese Laura con la giusta dose di curiosità, senza insistere troppo.

Caterina sospirò, girando il cucchiaino nella tazza.

"Non lo so... mi hanno chiesto delle cose sull'ambiente, sui robot, l'intelligenza artificiale... e poi c'è stato il test di programmazione."

Laura sollevò un sopracciglio, cercando di mantenere un tono neutro.

"Un test di programmazione? Per una posizione di marketing?"

"Sì," rispose Caterina, senza alzare lo sguardo. "Mi ha chiesto di implementare un algoritmo di ricerca. Non sono sicura di averlo fatto bene."

Laura annuì, riflettendo su quanto fosse insolito. Decise di non commentare troppo, rispettando l'umore di Caterina.

"Vuoi spiegarmelo? Magari lo risolviamo insieme."

Caterina esitò un attimo, ma poi prese un tovagliolo e iniziò a spiegare il problema. Laura ascoltò attentamente, scarabocchiando qualche appunto sul tovagliolo.

Dopo qualche minuto, Laura sorrise leggermente.

"Credo che l'abbiamo risolto," disse, evitando di sembrare troppo sicura. Sapeva che, nonostante tutto, doveva mostrare rispetto per l'esperienza di Caterina. "Non era troppo difficile, ma capisco che sia strano chiedere una cosa del genere per il tuo ruolo."

Caterina guardò il tovagliolo, sollevata.

"Grazie. Anche se non sono sicura di aver fatto bene al colloquio... almeno ora so che potevo farcela."

Laura fece un cenno con la testa.

"Non ti preoccupare troppo. Sei più in gamba di quanto credi. E poi, chi si aspetterebbe un test di programmazione per un ruolo di marketing?"

Caterina sorrise, finalmente rilassata. Laura aveva saputo darle il supporto di cui aveva bisogno, senza forzare la mano. Il loro legame era saldo, nonostante la differenza d'età, e quel momento lo confermava.

"Dai, lasciamo stare per un po'," disse Laura con un sorriso. "Godiamoci il cappuccino e la pastina. Il resto si vedrà."

---

\section{Giorno 2}

\subsection{Laura nel magazzino Amazon}

Il magazzino di Amazon sembrava non finire mai. Anche dopo mesi di lavoro, continuavo a scoprire nuove sezioni che non avevo mai visto prima. Oggi, però, ero in difficoltà con un pacco che non riuscivo a classificare. Il codice logistico non corrispondeva a nessuna delle sezioni a cui ero abituata. Continuavo a controllare il mio scanner, ma niente. Quel maledetto codice mi stava portando in una parte del magazzino completamente sconosciuta.

Mi ritrovai davanti a un portale che non avevo mai visto prima. Sopra di esso c’era un cartello che diceva: "Accesso riservato – Stoccaggi speciali". Alzai il pacchetto per osservarlo meglio, e notai un piccolo codice aggiuntivo su un angolo. "Speciale", lessi sottovoce. Non avevo mai avuto a che fare con pacchi del genere.

Avvicinai il barcode al lettore accanto al portale. Sentii un bip e un ronzio. Il portale sembrava iniziare ad aprirsi lentamente, e stavo per fare un passo avanti quando una voce mi bloccò.

"Ferma!" disse qualcuno alle mie spalle.

Mi girai di scatto e vidi un uomo avvicinarsi rapidamente. Indossava una tuta da tecnico, e notai che aveva "Ising" cucito sopra il petto.

"Questo è un reparto riservato a stoccaggi speciali. Chi ti ha autorizzato ad accedere qui?" mi chiese con tono serio.

Per un momento rimasi in silenzio, cercando di trovare una risposta.

"Nessuno mi ha autorizzato," risposi alla fine. "Il codice del pacco mi ha portato qui."

Ising mi guardò con una certa curiosità, anche se sembrava mantenere il suo atteggiamento professionale.

"Quello non è un pacco ordinario," mi disse, questa volta in tono più gentile. "Se non hai l’autorizzazione, dovresti riportarlo all'ufficio di smistamento."

Annuì, anche se dentro di me ero confusa. Che cosa poteva avere di speciale quel pacco? Perché era così importante?

"Capisco," risposi, prendendo il pacco e girandomi per tornare indietro.

Mentre camminavo verso l'ufficio di smistamento, non potevo fare a meno di chiedermi che segreti nascondesse quella sezione del magazzino. Avevo la sensazione che ci fosse molto di più sotto la superficie, qualcosa che forse non avrei mai dovuto scoprire.

% \textbf{[Aggiunta]} Decisi di tornare indietro. "Scusi, Ising, posso farle una domanda?" chiesi, voltandomi verso di lui.

Ising si fermò e mi guardò attentamente. "Dimmi pure," rispose.

"Che cosa c'è realmente dietro quel portale? Sembra diverso da tutto il resto del magazzino."

Lui esitò per un istante, poi abbassò la voce. "Non dovrei dirlo, ma lì custodiamo merci molto particolari, prototipi e tecnologia avanzata. Non è un'area per personale non autorizzato."

La mia curiosità era ormai alle stelle. "Capisco. Mi scusi per l'intrusione."

Ising annuì. "Fai attenzione. E se dovessi avere altri pacchi con codici strani, avvisami direttamente."

"Certamente, lo farò."

---

\section{Caterina riceve il risultato del colloquio}

\subsection{L'incontro al magazzino}

Stavo per uscire dal magazzino quando vidi \textbf{Caterina} camminare verso di me. Lavorava temporaneamente al reparto spedizione, in cerca di un'occupazione stabile.

"Ciao, allora? Hai ricevuto notizie?" le chiesi mentre ci incrociavamo.

Lei annuì lentamente.

"Mi hanno scritto che non sono stata assunta," disse, cercando di nascondere la delusione. "Potevo fare di meglio, soprattutto con quel test di programmazione. Vorrei prepararmi meglio e ricandidarmi... Secondo te come potrei prepararmi per la programmazione?"

La guardai con empatia. "Beh, potremmo studiare insieme. So che può sembrare complicato all'inizio, ma con un po' di pratica migliorerai sicuramente."

Lei sorrise appena. "Ti ringrazio. Ne avrei davvero bisogno."

Stavo per continuare, ma guardai l'orologio e mi accorsi di essere in ritardo per l’esame di crittografia.

"Caterina, devo correre all’università. Ci vediamo stasera da me alle 19? Così ne parliamo con calma," dissi, affrettandomi verso l’uscita.

Caterina annuì.

"A dopo," rispose, con un sorriso leggermente malinconico.

---

\section{L'esame di crittografia}

\subsection{L'attesa dell'esame}

Arrivai di corsa all’università, con il fiato corto. Mi sedetti nella sala d'attesa insieme agli altri studenti. Alcuni discutevano a bassa voce delle domande d’esame. Li ascoltai distrattamente, ma quello che dicevano mi fece rendere conto di aver trascurato qualche dettaglio importante nel mio studio. Mi sentii sopraffatta da un’ondata d'ansia.

% \textbf{[Aggiunta]} Cercai di concentrarmi, ripassando mentalmente gli algoritmi principali. "Se solo avessi dedicato più tempo allo studio invece di smanettare con il \emph{Noemografo}..." pensai tra me e me.

Un collega si avvicinò. "Tutto bene, Laura?"

"Sì, solo un po' nervosa," risposi forzando un sorriso.

"Vedrai che andrà bene. Hai sempre ottenuto ottimi risultati."

Annuii, ma l'ansia non diminuiva.

\subsection{L'esame con il professor Shor}

Quando il professore chiamò il mio nome, esitai per un attimo. Il \textbf{professor Shor} era lì davanti a me. Non potevo più tornare indietro.

"Buonasera, signorina," mi salutò cortesemente. "È pronta?"

"Sì, professore," risposi, cercando di sembrare sicura di me, anche se non lo ero del tutto.

"Mi può dire qual è la complessità dell'algoritmo classico per la fattorizzazione di un numero intero?" mi chiese Shor, fissandomi con uno sguardo attento.

Il mio cervello si bloccò. Cercai di ragionare, parlando lentamente, come se stessi cercando di mettere insieme i pezzi.

"Ehm... se non erro, la migliore fattorizzazione classica conosciuta ha una complessità sub-esponenziale, più precisamente di tipo $O\left(e^{(c \cdot \ln n)^{1/3} (\ln \ln n)^{2/3}}\right)$, dove $c$ è una costante."

Shor annuì leggermente. "Bene, e sa dirmi perché l'algoritmo quantistico è più efficiente?"

"Sì, l'algoritmo di fattorizzazione di Shor ha una complessità polinomiale rispetto al numero di cifre di $n$, quindi $O((\log n)^3)$, il che lo rende esponenzialmente più veloce rispetto agli algoritmi classici."

Il professore sembrò soddisfatto. "Ottimo. Vedo che ha compreso i concetti fondamentali."

Mi rilassai leggermente, ma sapevo che avrei potuto rispondere meglio fin dall'inizio.

Shor mi guardò con comprensione.

"Le offro un 24. Ha dimostrato conoscenza, ma deve lavorare sulla sicurezza nelle risposte."

Esitai un attimo. Non era il voto che speravo, ma potevo accettarlo.

"Accetto, professore. Grazie."

---

\section{Il ritorno a casa e Rocky}

\subsection{Passeggiata con Rocky}

Tornai a casa di fretta. \textbf{Rocky} mi accolse scodinzolando come al solito, pieno di energia. Lo presi al guinzaglio per portarlo fuori a fare pipì, ma non avevo molto tempo. Dovevo ancora preparare la cena prima che Caterina arrivasse.

Rocky voleva giocare, ma io lo strattonai verso casa.

"Dai, Rocky, non oggi..." gli dissi, cercando di non farlo sembrare un rimprovero.

Mi guardò con occhi tristi mentre rientravamo.

"Domani giochiamo, te lo prometto," aggiunsi, anche se non ero sicura che avrebbe capito.

---

\section{Cena con Caterina}

Erano quasi le 19 quando Caterina arrivò a casa di Laura. Prima di raggiungerla, era passata dalla sua abitazione, dove il fidanzato l'aveva rimproverata.

"Non ti confidi mai con me," le aveva detto. "Sembra che tu dia più importanza alle tue amiche che al nostro rapporto."

Quelle parole l'avevano colpita più di quanto volesse ammettere. Per la prima volta, Caterina si era interrogata sulla sincerità dei propri sentimenti. Voleva davvero sposarsi? Non riusciva a togliersi quella domanda dalla testa mentre si dirigeva verso casa di Laura, e il peso di quel pensiero le impediva di godersi l'aria fresca della sera.

Laura la accolse con un sorriso stanco, impegnata negli ultimi preparativi per la cena. La cucina era inondata dal profumo di sugo e spezie.

"Ciao, Caterina! Vieni, stavo finendo di preparare."

Caterina si tolse la giacca e la sistemò su una sedia.

"Grazie, Laura. Dove sta tua sorella? Non la vedo in giro."

Laura girò un mestolo nella pentola.

"Valentina? Ah, è a Vienna. Starà lì per sei mesi per via del suo dottorato di ricerca. Si è trasferita giusto la settimana scorsa."

"Sei mesi lontana da casa... Dev'essere dura," rispose Caterina, riflettendo ad alta voce. "Ma almeno ha un obiettivo chiaro. A volte mi chiedo cosa sto facendo io."

Laura si voltò verso di lei, notando la nota di tristezza nella sua voce.

"Siediti, la cena è quasi pronta. Mi sembra che tu abbia bisogno di sfogarti."

Sedute a tavola, le due amiche si scambiarono uno sguardo complice. La differenza d'età tra loro portava una dinamica di reciproco rispetto, ma in quel momento Caterina si sentiva come se avesse bisogno di un consiglio da qualcuno che sapesse ascoltare.

"Non so, Laura... ho ricevuto una comunicazione ufficiale dalla \emph{Pet Micro Robot}, ho fallito il colloquio. Sono un po' giù di morale."

Laura sollevò un sopracciglio.

"Ma non è detto che sia un fallimento..."

Caterina scosse la testa.

"Credo che la \textbf{PZZIA} mi abbia valutata bene, ma Eva, la responsabile delle risorse umane, sembrava intenzionata a farmi crollare. Alla fine anche quel test di programmazione avanzata. Che senso aveva?"

Laura appoggiò la forchetta e la guardò con aria perplessa.

"Un test di programmazione... Ma non ti candidavi per una posizione di marketing?"

"Sì, esattamente," rispose Caterina, spingendo il piatto leggermente più avanti. "Non so perché mi abbia chiesto di fare un test così tecnico. Non mi è sembrato neanche pertinente."

Laura rifletté per un attimo.

"Strano davvero. Forse volevano testare la tua capacità di pensiero logico, ma anche così... è un po' fuori luogo per un ruolo del genere."

Dopo cena, decisero di fare una passeggiata per schiarirsi le idee. Rocky, il cane di Laura, si illuminò alla vista del guinzaglio, saltellando per la stanza.

Mentre camminavano nel parco, Caterina si sfogò ancora.

"Sai, non imparerò mai a programmare. Tutti questi algoritmi, strutture dati... è tutto così complicato per me."

Laura la guardò con un sorriso.

"Non dire così. Anche io ho imparato da zero, e non è stato semplice. Ho iniziato da piccola, programmando i vecchi computer di famiglia. Sai, lo \emph{ZX Spectrum} e il \emph{Commodore 64}."

Caterina si fermò un attimo, sorpresa.

"Davvero? Dove li avevi trovati?"

Laura rise.

"Erano cimeli di famiglia, probabilmente di mio zio. Li avevo trovati in soffitta e ho deciso di riportarli in vita. Ho costruito nuovi alimentatori, cavi per i monitor... ed è così che ho iniziato a programmare."

"Monitor?" Caterina la guardò con un'espressione di stupore. "Pensavo fossero ormai un pezzo di antiquariato."

"Sì, lo erano," rispose Laura ridendo. "Ma è stato così che ho imparato. Era una sfida, ma mi ha dato grandi soddisfazioni."

Dopo un momento di silenzio, Laura cambiò argomento, tornando a un pensiero che l'aveva colpita prima.

"A proposito, sei riuscita a controllare il file di valutazione generato dall'IA?"

Caterina scrollò le spalle.

"Non ho ricevuto nulla. Nessun file allegato alla comunicazione."

Laura si fermò, pensierosa.

"È strano. Con la nuova legge, tutti dovrebbero ricevere sempre una \emph{chain of thinking} allegata alle decisioni delle IA. Questo mi sembra davvero sospetto."

"Già, non so cosa pensare," disse Caterina, con una punta di frustrazione nella voce. "Forse c'è stato un errore..."

Camminarono in silenzio per un po', mentre Rocky scodinzolava felice, ignaro delle preoccupazioni che turbinavano nelle menti delle due amiche. Quella sera sembrava carica di domande senza risposta.

% \textbf{[Aggiunta]} Caterina si fermò all'improvviso.

"Laura, posso chiederti una cosa?"

"Certo, dimmi."

"Secondo te, è normale avere dubbi sul proprio futuro... anche in amore?"

Laura la guardò con attenzione. "Cosa intendi?"

"Il mio fidanzato dice che non mi confido abbastanza con lui. Forse ha ragione. Ma non so se sono pronta a fare il grande passo."

"È una decisione importante," rispose Laura con dolcezza. "Devi ascoltare te stessa e capire cosa desideri davvero."

Caterina annuì, grata per l'ascolto.

---

\chapter{Giorno 3: Connessioni inaspettate}

Dopo la passeggiata, Caterina tornò a casa, ma l'inquietudine per la mancanza del documento valutativo continuava a tormentarla. Decise di contattare Eva la mattina successiva, inviandole un'email. La risposta arrivò inaspettatamente veloce:

"\emph{Caterina, purtroppo il documento è stato cancellato per errore, quindi non posso fornirlo. Tuttavia, possiamo fissare un appuntamento domani per discutere di persona. Cordiali saluti, Eva.}"

Caterina sospirò. Non era la risposta che sperava, ma almeno avrebbe avuto l'opportunità di chiedere spiegazioni. Quella sera, decise di tornare a casa di Laura. Quando arrivò, trovò l’amica seduta alla scrivania, con uno dei suoi vecchi computer accesi.

Laura alzò lo sguardo e sorrise.

"Ciao, Cate. Come è andata la giornata?"

Caterina si sedette sul divano, osservando curiosa l’attività di Laura.

"Ho scritto a Eva. Dice che il documento è stato cancellato, ma mi ha dato appuntamento per domani. Vedremo cosa mi dirà."

Laura annuì, ma non sembrava troppo sorpresa.

"Immaginavo. A volte certi sistemi fanno più danni di quanto dovrebbero." Poi indicò il vecchio computer sul tavolo. "Guarda cosa ho rispolverato. Ho deciso di rimettermi su questi vecchi cimeli per prepararmi all'esame di crittografia."

Caterina si sporse in avanti, osservando con interesse.

"Che roba è questa? Non pensavo che li usassi ancora. Mi sembra di essere tornata negli anni '80."

Laura rise.

"Sì, fa un po' quell'effetto, vero? Sto cercando di collegare uno strumento che stiamo sviluppando nel corso di nanotech, il \emph{Noemografo}, a questi vecchi sistemi. Volevo vedere se riesco a farli dialogare."

Caterina aveva solo vagamente sentito parlare del \emph{Noemografo}, ma non l'aveva mai visto in azione.

"Il Noemografo? Non l'ho mai usato. Come funziona?"

Laura si alzò, andò verso una piccola scrivania laterale e tornò con due strani dispositivi, simili a cuffie ma più complessi.

"È un dispositivo che stiamo sviluppando per leggere i pensieri. Viene usato per applicazioni in nanotech, ma sto provando a integrarlo in questi sistemi per una sfida personale."

Senza dire altro, Laura porse uno dei dispositivi a Caterina.

"Prova. Io ne indosso uno, tu l'altro. Vediamo se funziona."

Caterina guardò il dispositivo con un misto di curiosità e nervosismo.

"Sei sicura?"

"Sì, fidati. Non è pericoloso," disse Laura, sorridendo. "In pratica ci colleghiamo per qualche attimo. Puoi sentire i miei pensieri e io i tuoi. Solo per un breve momento, però."

Caterina indossò il Noemografo e, quasi istantaneamente, un senso di connessione profonda attraversò la sua mente. Per qualche secondo, era come se le barriere tra loro due fossero scomparse. Poteva percepire frammenti di pensieri di Laura: insicurezze legate all'esame di crittografia, piccoli momenti della sua giornata... ma c'era qualcosa di più.

Allo stesso tempo, Laura sentiva i pensieri di Caterina. L'ansia per il colloquio con Eva, la frustrazione per il file scomparso, ma, più sorprendente, poteva scorgere i pensieri nascosti sui problemi d'amore di Caterina. Laura si rese conto delle preoccupazioni della sua amica riguardo al fidanzato, dei suoi dubbi sul matrimonio. Ma decise di non dire nulla. Quando il collegamento si interruppe, Laura si limitò a sorridere.

"Funziona, vero?" disse, con tono casuale, mentre si toglieva il Noemografo.

Caterina si tolse il dispositivo e annuì.

"Sì... è stato strano, ma affascinante."

Laura fece finta di non aver notato nulla di personale.

"Beh, è solo un piccolo esperimento. Ma è incredibile quanto la tecnologia possa avvicinarci, non trovi?"

Caterina, ancora un po' scossa dall'esperienza, decise di non parlare dei suoi pensieri personali. Aveva altre cose di cui preoccuparsi per ora.

"Sì, lo è. E domani... vedremo cosa dirà Eva."

---

\chapter{Giorno 4: La trappola di Eva}

Caterina era al lavoro nel reparto spedizioni, intenta a preparare gli ultimi pacchi della giornata. La mente era ancora affollata dai pensieri riguardanti il colloquio con Eva e l'assenza del file di valutazione. Tuttavia, cercava di concentrarsi sul lavoro, mantenendo il ritmo.

Mentre stava etichettando un pacco, all'improvviso ebbe una visione nitida: vide chiaramente le mani di Laura che digitavano sui tasti di gomma del vecchio \emph{ZX Spectrum}. Rimase immobile per un istante, sbigottita dall'esperienza.

Come era possibile?

Scosse la testa, cercando di non darle troppo peso e tornò al lavoro, ma la sensazione rimase. Stava lottando con una spedizione che non riusciva a completare. C'era qualcosa che non andava con il riferimento del destinatario, e nonostante i vari tentativi, non riusciva a trovare la soluzione. Decise quindi di rivolgersi ai suoi colleghi che si occupavano delle telecomunicazioni sulla WAN di Amazon.

Si avvicinò a Roberto, e gli spiegò la situazione.

"Non riesco a trovare il corretto riferimento per questa spedizione," disse, mostrando il codice sul suo schermo. "Hai modo di darmi una mano?"

Roberto la guardò con un'espressione concentrata.

"Sembra un problema complesso," disse, poi si girò verso il suo terminale. "Meglio chiamare Alice, lei potrebbe avere la soluzione." Aprì un canale di comunicazione criptato per evitare qualsiasi rischio di violazione dei dati sensibili.

Pochi istanti dopo, Alice rispose alla chiamata.

"Ciao Robi, cosa c'è?"

"Abbiamo un problema con una spedizione," spiegò. "Potresti dare un'occhiata al riferimento? È strano, non riusciamo a collegarlo correttamente."

Alice esaminò i dati che le aveva inviato, ma dopo diversi tentativi, nemmeno lei riuscì a risolvere la questione.

"Mi dispiace, ma non riesco a trovare una soluzione. Potrebbe essere un problema di sistema..."

% \textbf{[Aggiunta]} Caterina pensò al pacco misterioso che Laura aveva menzionato. "Potrebbe essere collegato a qualche stoccaggio speciale?" chiese, quasi tra sé e sé.

Roberto la guardò curioso. "Perché lo chiedi?"

"Oh, niente. Solo un pensiero. Forse dovremmo verificare con il reparto speciale."

Roberto scosse la testa. "Non abbiamo accesso a quei dati. Sono riservati."

Caterina sospirò. Il tempo stringeva e doveva lasciare il lavoro per l'appuntamento con Eva.

"Grazie lo stesso, Alice, ci proverò di nuovo più tardi. Devo andare ora, ma la spedizione resterà a metà per ora."

Roberto annuì.

"Va bene, Caterina. Se riesci a tornare prima, vediamo se nel frattempo il sistema si sblocca."

Caterina lasciò il reparto, frustrata per non essere riuscita a risolvere il problema, ma con la mente già concentrata sull'incontro imminente con Eva.

Il pensiero del file con le \emph{cot} che Eva non le aveva inviato continuava a tormentarla. Salì su un drone taxi, osservando la città scorrere sotto di lei.

Mentre il drone si muoveva, un'altra visione apparve nella sua mente: lo schermo del \emph{ZX Spectrum} era chiarissimo, e riusciva persino a distinguere delle righe di codice Assembly. Questa volta non poteva ignorarlo. Il turbamento cresceva dentro di lei, e cominciò a pensare che forse le loro menti, la sua e quella di Laura, fossero ancora connesse dal \emph{Noemografo}. Stava davvero vedendo ciò che stava facendo Laura in quel preciso istante?

Il desiderio di chiamare Laura per verificare era forte, ma prima che potesse farlo, il drone taxi si fermò. Era arrivata alla \emph{PET Micro Robot}. Caterina scese e si avviò all'ingresso.

Eva la ricevette con un sorriso fin troppo cordiale.

"Caterina, benvenuta! Mi dispiace ancora per il disguido con il file," disse, con tono apparentemente sincero. "Capisco perfettamente i tuoi dubbi, per questo ho pensato di mostrarti qualcosa che potrebbe rassicurarti."

Caterina annuì, ma qualcosa nel comportamento di Eva non la convinceva del tutto.

"Ho qui una registrazione in 3D del tuo colloquio, sia con me che con \emph{PZZIA}," continuò Eva. "Per fartela vedere, dovrai indossare questo visore 3D. È un modello un po' datato, ma farà il suo lavoro."

Caterina guardò il visore con esitazione. Era un modello di alcuni anni prima, e qualcosa non le quadrava. Tuttavia, i modi sicuri e controllati di Eva la convinsero a seguire le sue istruzioni. Indossò il visore e sentì una leggera vibrazione mentre lo schermo passava dalla modalità \emph{Augmented Reality} a quella \emph{Virtual Reality}. Mentre la transizione avveniva, Caterina colse un rapido sguardo di soddisfazione sul volto di Eva. Fu in quel momento che capì: era caduta in una trappola.

Contemporaneamente, a casa sua, Laura era intenta a salvare il programma su \emph{Micro Drive}. Le sue dita si muovevano con familiarità sulla tastiera del \emph{ZX Spectrum}, ma improvvisamente si sentì mancare. Una sensazione di vertigine e di perdita di peso la travolse, come se stesse per svanire da un momento all'altro.

La connessione tra le due menti non era affatto spezzata. E ora, sembrava che qualcosa di più grande stesse per accadere.

