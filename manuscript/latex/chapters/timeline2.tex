
\section*{Timeline del Capitolo 2}

\subsection*{Mattina}

\begin{itemize}
    \item \textbf{Luogo}: \emph{Pet Micro Robot} - Sala di Eva
    \begin{itemize}
        \item Eva è nella stanza con Caterina, connessa alla Realtà Virtuale, immersa nel mondo simulato.
        \item Eva chiede a PZZIA se è possibile cancellare il file contenente le \textit{chain of thinking} utilizzate per valutare Caterina.
        \item PZZIA risponde che i suoi processi sono quantistici e reversibili; l'informazione non può essere cancellata senza lasciare traccia.
        \item Eva si irrita, sapendo che misurare i qubit in stati classici innescherebbe un messaggio a Caterina con il risultato.
        \item Decide di continuare con il trattamento psicologico tramite Realtà Virtuale per convincere Caterina a rinunciare alla posizione lavorativa.
        \item Eva riflette sui due punti deboli del trattamento:
        \begin{enumerate}
            \item Il soggetto deve percepirsi completamente solo, senza segnali di aiuto esterni.
            \item Il soggetto non deve comprendere i meccanismi dell'algoritmo di suggestione.
        \end{enumerate}
        \item Convinta che Caterina non abbia le competenze per capire la manipolazione, Eva procede con il piano.
    \end{itemize}
    
    \item \textbf{Luogo}: \emph{Classical Control Unit}
    \begin{itemize}
        \item Un agente nota un'anomalia: ci sono due qubit in più nel sistema.
        \item Informa il supervisore, che chiede di verificare, non avendo ricevuto avvisi dal \emph{Quantum Resource Management} (QRM).
        \item Il supervisore ordina di mantenere la trasmissione con il QRM criptata per evitare che il \emph{Quantum Error Correction} o il \emph{Fault Tolerance Coding} rilevino anomalie.
        \item L'agente cripta la comunicazione utilizzando l'algoritmo RSA 2048.
        \item Il QRM conferma di non aver installato nuovi qubit; l'anomalia è reale.
        \item Preoccupato, il supervisore ordina di inviare una squadra della \emph{Quantum Control Electronics} per verificare fisicamente i qubit.
    \end{itemize}
    
    \item \textbf{Luogo}: \emph{Fault Tolerance Coding} - Prigione del Professor Shor
    \begin{itemize}
        \item Il commissario alla sicurezza si avvicina al Professor Shor, tenuto prigioniero.
        \item Gli ordina di decriptare un messaggio inviato al QRM.
        \item Shor decripta il messaggio, scoprendo il problema dei due nuovi qubit e il tentativo del supervisore di nascondere l'anomalia.
        \item Il commissario, soddisfatto, decide di non arrestare subito il supervisore, pianificando di utilizzare la situazione a suo vantaggio.
        \item Informa un'agente della polizia segreta della sua decisione.
    \end{itemize}
\end{itemize}

\subsection*{Pomeriggio}

\begin{itemize}
    \item \textbf{Luogo}: \emph{Base della Quantum Control Electronics}
    \begin{itemize}
        \item Due agenti ricevono l'ordine di verificare i qubit nel sistema.
        \item Partono a bordo di droni luminosi verso il \emph{Qubit Array}.
    \end{itemize}
    
    \item \textbf{Luogo}: \emph{Qubit Array}
    \begin{itemize}
        \item Laura e Caterina, confuse, cercano di capire dove si trovano nell'ambiente quantistico.
        \item Alcuni qubit le osservano da lontano, nascosti tra i corridoi.
        \item Un qubit maschio, somigliante al fidanzato di Caterina, si avvicina.
        \item Avverte: \emph{"State per essere trovate. Se non volete passare qualche giorno rinchiuse mentre controllano il vostro QFLIP, è meglio che veniate con noi."}
        \item Caterina, attratta dalla sua somiglianza e sicurezza, lo segue senza opporre resistenza.
        \item Laura, perplessa, segue il gruppo.
        \item Altri due qubit si uniscono a loro, incitandole a muoversi velocemente.
        \item In lontananza, i due agenti della \emph{Quantum Control Electronics} si avvicinano per verificare l'anomalia.
    \end{itemize}
\end{itemize}

\subsection*{Sera}

\begin{itemize}
    \item \textbf{Luogo}: \emph{Faulty Qubit Space}
    \begin{itemize}
        \item Il gruppo raggiunge uno spazio appartato dove risiedono qubit difettosi o instabili.
        \item Il qubit simile a Mark dice: \emph{"Qui sarete al sicuro... per un po'. Ma non potete rimanere a lungo."}
        \item Laura nota l'instabilità dell'ambiente e chiede se sia sicuro restare.
        \item Una qubit femmina risponde che non lo è: mancano isolamento adeguato e sistema di raffreddamento; rischiano la decoerenza.
        \item I due agenti passano vicino al loro nascondiglio, controllando dati e ambiente.
        \item Laura trattiene il respiro; gli agenti sembrano fermarsi, ma poi proseguono.
        \item Caterina si avvicina al qubit somigliante a Mark e chiede il suo nome.
        \item Lui risponde con un sorriso tranquillo: \emph{"Sono... Mark."}
    \end{itemize}
\end{itemize}


