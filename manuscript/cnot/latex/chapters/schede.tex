
\section*{Schede dei Personaggi}

\begin{tcolorbox}[colback=white,colframe=black,title=\textbf{Caterina}]
\begin{multicols}{2}
\textbf{Occupazione}: Dipendente Bamazon, in cerca di lavoro nel settore marketing.

\textbf{Età}: 25 anni.

\textbf{Descrizione}: Caterina è una giovane donna determinata e sensibile, impegnata nelle questioni ambientali. Nonostante le difficoltà incontrate nel colloquio alla Pet Microrobot, mostra una forte volontà di migliorarsi e di perseguire i suoi obiettivi. È fidanzata, ma nutre dubbi sulla sincerità dei propri sentimenti.

\textbf{Caratteristiche Principali}:
\begin{itemize}
    \item Impegnata nelle tematiche ambientali.
    \item Desiderosa di crescere professionalmente.
    \item Affronta insicurezze personali e sentimentali.
\end{itemize}
\end{multicols}
\end{tcolorbox}

\vspace{0.5cm}

\begin{tcolorbox}[colback=white,colframe=black,title=\textbf{Laura}]
\begin{multicols}{2}
\textbf{Occupazione}: Part-time Bamazon e Studentessa universitaria, appassionata di informatica e tecnologia.

\textbf{Età}: 21 anni.

\textbf{Descrizione}: Laura è un'amica fidata di Caterina, più giovane di lei ma matura e responsabile. Ha una forte passione per l'informatica, iniziata fin da piccola grazie ai vecchi computer di famiglia. Attualmente si prepara per l'esame di crittografia e partecipa a progetti innovativi come il \emph{Noemografo}.

\textbf{Caratteristiche Principali}:
\begin{itemize}
    \item Appassionata di tecnologia vintage e moderna.
    \item Empatica e disponibile verso gli amici.
    \item Curiosa e sempre in cerca di nuove sfide.
\end{itemize}
\end{multicols}
\end{tcolorbox}

\vspace{0.5cm}

\begin{tcolorbox}[colback=white,colframe=black,title=\textbf{Eva}]
\begin{multicols}{2}
\textbf{Occupazione}: Responsabile delle risorse umane presso Pet Microrobot.

\textbf{Età}: Circa 35 anni.

\textbf{Descrizione}: Eva è una figura autoritaria e fredda. Durante il colloquio con Caterina, si mostra scettica e sembra avere secondi fini. Non condivide le preoccupazioni ambientali di Caterina e sembra più interessata all'immagine dell'azienda che alla sostanza delle sue politiche.

\textbf{Caratteristiche Principali}:
\begin{itemize}
    \item Autoritaria e manipolatrice.
    \item Prioritizza l'immagine aziendale rispetto alla sostenibilità reale.
    \item Misteriosa e potenzialmente antagonista.
\end{itemize}
\end{multicols}
\end{tcolorbox}

\vspace{0.5cm}

\begin{tcolorbox}[colback=white,colframe=black,title=\textbf{Professor Shor}]
\begin{multicols}{2}
\textbf{Occupazione}: Professore universitario di crittografia.

\textbf{Età}: Circa 50 anni.

\textbf{Descrizione}: Il professor Shor è un accademico severo ma giusto. Durante l'esame con Laura, dimostra professionalità e offre feedback costruttivo. Rappresenta una figura autorevole nel campo della crittografia.

\textbf{Caratteristiche Principali}:
\begin{itemize}
    \item Esigente ma equo.
    \item Esperto in crittografia.
    \item Incoraggia gli studenti a dare il meglio.
\end{itemize}
\end{multicols}
\end{tcolorbox}

\vspace{0.5cm}

\begin{tcolorbox}[colback=white,colframe=black,title=\textbf{Rocky}]
\begin{multicols}{2}
\textbf{Occupazione}: Cane domestico di Laura.

\textbf{Età}: 3 anni.

\textbf{Descrizione}: Rocky è il fedele cane di Laura. Energico e affettuoso, rappresenta un elemento di gioia e spensieratezza nella vita di Laura. Ama giocare e fare passeggiate.

\textbf{Caratteristiche Principali}:
\begin{itemize}
    \item Energico e giocoso.
    \item Legato profondamente a Laura.
    \item Porta leggerezza nelle scene quotidiane.
\end{itemize}
\end{multicols}
\end{tcolorbox}

\vspace{0.5cm}

\begin{tcolorbox}[colback=white,colframe=black,title=\textbf{Ising}]
\begin{multicols}{2}
\textbf{Occupazione}: Tecnico nel magazzino Bamazon.

\textbf{Età}: Circa 30 anni.

\textbf{Descrizione}: Ising è un tecnico che lavora nelle aree riservate del magazzino Bamazon. Incontra Laura quando lei, per caso, si avvicina a una zona ad accesso limitato. Appare professionale e mantiene un certo mistero intorno alle operazioni speciali del magazzino.

\textbf{Caratteristiche Principali}:
\begin{itemize}
    \item Professionale e riservato.
    \item Lavora in settori speciali e segreti.
    \item Potenziale fonte di informazioni su trame nascoste.
\end{itemize}
\end{multicols}
\end{tcolorbox}


\vspace{0.5cm}

\begin{tcolorbox}[colback=white,colframe=black,title=\textbf{Alice e Bob}]
\begin{multicols}{2}
\textbf{Occupazione}: Specialisti in telecomunicazioni sulla WAN di Bamazon.

\textbf{Età}: Circa 30 anni.

\textbf{Descrizione}: Alice è una specialista esperta in telecomunicazioni che lavora presso Bamazon. Viene contattata da Bob per aiutare Caterina con un problema di spedizione. Sebbene professionale e disponibile, non riesce a trovare una soluzione al problema, suggerendo che potrebbe trattarsi di un'anomalia di sistema.

\textbf{Caratteristiche Principali}:
\begin{itemize}
    \item Esperta in telecomunicazioni e reti.
    \item Professionale e collaborativa.
    \item Attenta ai dettagli, riconosce i limiti dei sistemi.
\end{itemize}
\end{multicols}
\end{tcolorbox}

\begin{tcolorbox}[colback=white,colframe=black,title=\textbf{Qubit-Mark}]
\begin{multicols}{2}
\textbf{Occupazione}: Qubit maschio nel sistema quantistico.

\textbf{Età}: Non applicabile (entità quantistica).

\textbf{Descrizione}: Mark è un qubit che assume l'aspetto del fidanzato di Caterina, ma senza le sue limitazioni sociali e personali. Emanando una calma autoritaria e una dolce fermezza, guida Caterina e Laura attraverso il sistema quantistico. È libero dalle pressioni sociali e mostra un comportamento protettivo verso le ragazze.

\textbf{Caratteristiche Principali}:
\begin{itemize}
    \item Calmo e autoritario.
    \item Protettivo e guida per Caterina e Laura.
    \item Rappresenta una versione idealizzata del fidanzato di Caterina.
\end{itemize}
\end{multicols}
\end{tcolorbox}

\vspace{0.5cm}

\begin{tcolorbox}[colback=white,colframe=black,title=\textbf{Supervisore della Classical Control Unit}]
\begin{multicols}{2}
\textbf{Occupazione}: Supervisore nella Classical Control Unit.

\textbf{Età}: Non applicabile (entità quantistica).

\textbf{Descrizione}: Il supervisore è serio e imperturbabile, responsabile del buon funzionamento della Classical Control Unit. Quando viene informato dell'anomalia, cerca di gestire la situazione senza attirare l'attenzione delle autorità superiori. È preoccupato per le conseguenze che potrebbero ricadere su di lui.

\textbf{Caratteristiche Principali}:
\begin{itemize}
    \item Autoritario ma cauto.
    \item Tende a nascondere i problemi per evitare ripercussioni.
    \item Ha paura delle conseguenze di una violazione del sistema.
\end{itemize}
\end{multicols}
\end{tcolorbox}

\vspace{0.5cm}


\begin{tcolorbox}[colback=white,colframe=black,title=\textbf{Qubit-Marley}]
\begin{multicols}{2}
\textbf{Occupazione}: Qubit femmina nel sistema quantistico.

\textbf{Età}: Non applicabile (entità quantistica).

\textbf{Descrizione}: Marley è un qubit femmina che accompagna Laura e Caterina nel \emph{Faulty Qubit Space}. Seria e pensierosa, agisce come guida e protettrice. Dimostra determinazione e pragmatismo, soprattutto durante la fuga verso il \emph{Quantum Measurement}. È attenta ai pericoli e prende decisioni rapide per garantire la sicurezza.

\textbf{Caratteristiche Principali}:
\begin{itemize}
    \item Seria e determinata.
    \item Protettiva verso Laura e Caterina.
    \item Conoscitrice dei pericoli del sistema quantistico.
\end{itemize}
\end{multicols}
\end{tcolorbox}

\vspace{0.5cm}

\begin{tcolorbox}[colback=white,colframe=black,title=\textbf{Agenti della Quantum Control Electronics}]
\begin{multicols}{2}
\textbf{Occupazione}: Agenti incaricati di mantenere l'ordine nel sistema quantistico.

\textbf{Età}: Non applicabile (entità quantistica).

\textbf{Descrizione}: Gli agenti sono figure autoritarie che perseguono qubit instabili o non autorizzati. Sono responsabili dell'arresto di Mark, Caterina e il loro compagno. Rappresentano la forza di controllo e repressione all'interno del sistema. Agiscono con freddezza e professionalità, senza mostrare empatia.

\textbf{Caratteristiche Principali}:
\begin{itemize}
    \item Autoritari e inflessibili.
    \item Eseguono ordini senza esitazione.
    \item Simbolo della minaccia per i qubit difettosi.
\end{itemize}
\end{multicols}
\end{tcolorbox}

\vspace{0.5cm}

\begin{tcolorbox}[colback=white,colframe=black,title=\textbf{Commissario alla Sicurezza}]
\begin{multicols}{2}
\textbf{Occupazione}: Alto funzionario nel sistema quantistico.

\textbf{Età}: Non applicabile (entità quantistica), ma apparentemente giovane. 

\textbf{Descrizione}:

Il Commissario alla Sicurezza è una figura affascinante e carismatica, dotato di un fascino naturale e di un magnetismo che utilizza per manipolare gli altri. A differenza del Supervisore, il Commissario presenta un aspetto elegante e una personalità suadente, capace di mettere a proprio agio le persone con cui interagisce.

Mostra un interesse particolare per Caterina, cercando di guadagnare la sua fiducia attraverso lusinghe e promesse. Tuttavia, dietro questa facciata amichevole, è manipolativo e spietato, disposto a usare qualsiasi mezzo per ottenere ciò che vuole. La sua vera natura emerge quando intrappola Caterina con l'\emph{Ionostrap}, rivelando la sua volontà di controllare e sfruttare le capacità altrui per i propri fini.

\textbf{Caratteristiche Principali}:
\begin{itemize}
    \item \textbf{Carismatico e Affascinante}: Sa come mettere le persone a proprio agio e guadagnare la loro fiducia.
    \item \textbf{Manipolativo}: Utilizza il suo fascino per influenzare e controllare gli altri.
    \item \textbf{Ambizioso}: Ha grandi piani per il sistema quantistico e cerca risorse umane eccezionali come Caterina.
    \item \textbf{Spietato}: Non esita a mostrare la sua vera natura quando i qubit non si conformano ai suoi desideri.
    \item \textbf{Intelligente e Stratega}: Pianifica con attenzione le sue mosse per ottenere il massimo vantaggio.
    \item \textbf{Doppia Personalità}: Presenta una facciata amichevole che nasconde intenzioni sinistre.
\end{itemize}
\end{multicols}
\end{tcolorbox}

\vspace{0.5cm}

\vspace{0.5cm}

