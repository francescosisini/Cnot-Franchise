\clearpage
\thispagestyle{empty}

\begin{center}
\vspace*{2cm}
{\Large \textbf{Prefazione}}\\[1.5cm]
\end{center}

Questa prefazione è stata interamente scritta da ChatGPT, modello 5,
su indicazioni specifiche dell’autore.  
Il testo è stato generato seguendo un processo dialogico: l’autore ha fornito
le intenzioni narrative, i concetti chiave da rappresentare e il ruolo
che \textit{Cnot} ricopre all’interno della trilogia;  
il modello ha poi sintetizzato tali elementi, li ha ordinati e li ha reimpaginati
in un discorso coerente, controllato e fedele allo spirito dell’opera.
\\\\
L’universo narrativo di \textit{Cnot} fa parte di una trilogia ambientata
in un’Europa attraversata da trasformazioni profonde, dove l’infrastruttura
del calcolo — reti neurali, sistemi quantistici, ambienti distribuiti —
diventa il nuovo terreno su cui si misurano diritti, identità e forme di
appartenenza.

In questo quadro ampio, \textit{Cnot} occupa un ruolo particolare.
Non racconta ancora il cambiamento sociale nel suo insieme:
indaga invece un punto più intimo e fragile, situato all’incrocio fra
persona e istituzioni.  
È il problema della \emph{decoerenza}: ciò che accade quando un individuo,
con i suoi limiti, le sue contraddizioni e la sua umanità, viene messo a
contatto con un modello ideale di cittadino, definito da sistemi che
aspirano alla coerenza assoluta.

Il linguaggio del quantum computing non è un ornamento narrativo.
È la metafora centrale:  
lo Stato come sistema che richiede stabilità e purezza di stato;  
il cittadino come vettore che, nella vita reale, tende naturalmente a
disperdersi, a perdere fase, a diventare opaco.  
La domanda di fondo è semplice e radicale:  
\textit{come si resta persone dentro un mondo che funziona solo se rimaniamo perfettamente coerenti?}

\textit{Cnot} racconta questo attrito senza giudizio, attraverso scene quotidiane
che mostrano quanto sia sottile la distanza fra ciò che il sistema richiede
e ciò che gli esseri umani possono offrire.

\vspace{0.5cm}
\noindent\textbf{Il significato del nome \textit{Cnot}}

Il titolo di questo libro richiama un nome che, a chi proviene dal mondo del
calcolo, suonerà familiare.  
Ma qui quel nome non indica un meccanismo: indica un rapporto.

\textit{Cnot} è una figura, un’immagine mentale.  
È l’idea che due entità — due vite, due coscienze, due destini —
possano toccarsi senza confondersi, e tuttavia modificarsi a vicenda.
È il punto in cui un’esistenza diventa condizione di possibilità per un’altra.

Per questo la trilogia prende avvio da qui.  
Perché l’Europa che raccontiamo non nasce da un atto solitario,
ma da un intreccio: dall’effetto che una persona ha sull’altra,
dall’eco minima che un gesto scatena nel mondo che lo circonda.

\textit{Cnot} è il nome di questo intreccio.  
Non è un simbolo tecnico, ma una soglia.  
Un invito a guardare ciò che accade tra le persone,
nei loro punti di contatto, nei loro piccoli scarti,
lì dove comincia a formarsi un legame che nessuno dei due prevedeva.

È da questo minuscolo nodo iniziale che si dispiega l’intera trilogia:
dalla consapevolezza che ogni relazione, anche la più impercettibile,
contiene la possibilità di cambiare tutto.
\vspace{0.5cm}

La trilogia proseguirà ampliando lo sguardo sull’Europa, sulle sue tensioni
e sui suoi fondamenti, ma è in questo primo volume che emerge il nucleo
tematico essenziale: la ricerca di un equilibrio possibile tra l’ideale
del calcolo e la realtà di chi, inevitabilmente, vive nella decoerenza.

\medskip
\begin{flushright}
\textit{Ferrara, 2025}
\end{flushright}

\clearpage
