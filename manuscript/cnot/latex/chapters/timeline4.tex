\section*{Timeline del Capitolo 4}


\begin{itemize}
    \item \textbf{Luogo}: Stanza spoglia con pareti metalliche nella \emph{Classical Control Unit}
    \begin{itemize}
        \item Caterina si trova in una stanza fredda e spoglia, con pareti metalliche che riflettono una luce bianca e fredda.
        \item Di fronte a lei c'è il \textbf{Supervisore}, figura imponente dai tratti austeri e rigidi.
        \item Accanto a lei ci sono Mark e l'altro compagno, seduti su rigidi supporti, immobili e silenziosi.
        \item Gli agenti che li hanno catturati si sono ritirati, lasciandoli soli con il Supervisore.
    \end{itemize}
\end{itemize}


\begin{itemize}
    \item \textbf{Interrogatorio di Caterina}
    \begin{itemize}
        \item Il Supervisore si rivolge a Caterina con tono glaciale, chiedendole come sia finita lì, poiché non la riconosce come uno dei qubit del \emph{Qubit Array}.
        \item Caterina cerca di mantenere la calma e risponde che non sa come sia finita lì, affermando di non aver fatto nulla di male.
        \item Il Supervisore è scettico e insiste per avere spiegazioni più dettagliate.
    \end{itemize}
    \item \textbf{Spiegazione di Caterina}
    \begin{itemize}
        \item Racconta di essere andata da Eva, la responsabile delle \emph{Human Resources}, per visionare il resoconto del suo colloquio di lavoro presso la \emph{Pet Micro Robot}.
        \item Spiega che PZZIA aveva elaborato una valutazione, ma Eva le disse che il file era stato cancellato per errore.
        \item Eva le propose di rivedere il colloquio in \emph{Virtual Reality} per chiarire i dubbi.
        \item Dopo aver indossato il visore, si è ritrovata nel sistema quantistico senza capire come.
    \end{itemize}
    \item \textbf{Reazione del Supervisore}
    \begin{itemize}
        \item Ascolta con sguardo impassibile, ma mostra crescente sospetto e irritazione.
        \item Non convinto dalla spiegazione, percepisce Caterina come un'anomalia sfuggente ai suoi protocolli.
    \end{itemize}
\end{itemize}



\begin{itemize}
    \item \textbf{Intervento di Mark}
    \begin{itemize}
        \item Il Supervisore si rivolge a Mark, chiedendogli quale sia il suo coinvolgimento.
        \item Mark difende Caterina, affermando che lei non c'entra nulla e che, se c'è un problema, dovrebbe affrontarlo con lui.
        \item Il Supervisore si irrita per il tono di Mark, sentendosi sfidato nella sua autorità.
    \end{itemize}
    \item \textbf{Escalation della Tensione}
    \begin{itemize}
        \item Il Supervisore ribatte, chiedendo se Mark pensa di avere l'autorità per parlare in quel modo.
        \item Mark mantiene uno sguardo fermo, insistendo nella difesa di Caterina.
        \item La tensione nella stanza aumenta, con Caterina che percepisce il rischio di una reazione drastica.
    \end{itemize}
    \item \textbf{Decisione del Supervisore}
    \begin{itemize}
        \item Ordina agli agenti di portare Mark al \emph{Faulty Qubit Space} per una "rigenerazione", come punizione per la sua insubordinazione.
        \item Si rivolge a Caterina, comunicandole che sarà mandata dal \textbf{Commissario}, poiché la situazione è oltre il suo controllo.
        \item Caterina prova panico ma cerca di mantenere la calma; scambia uno sguardo con Mark che le trasmette di non arrendersi.
    \end{itemize}
\end{itemize}



\begin{itemize}
    \item \textbf{Riflessioni del Supervisore}
    \begin{itemize}
        \item Rimasto solo, esprime la sua frustrazione per dover coinvolgere il Commissario.
        \item Sente che questo mette in discussione la sua autorità e competenza.
        \item Ammette a sé stesso che Caterina rappresenta un'anomalia che non riesce a comprendere né controllare.
    \end{itemize}
\end{itemize}



\begin{itemize}
    \item \textbf{Percorso verso il Commissario}
    \begin{itemize}
        \item Caterina viene scortata lungo i freddi corridoi della \emph{Classical Control Unit}.
        \item Si sente assalita da emozioni contrastanti: paura dell'ignoto e una nuova consapevolezza interiore.
    \end{itemize}
    \item \textbf{Riflessioni di Caterina}
    \begin{itemize}
        \item Ripensa a come Mark si sia alzato per difenderla, sentendosi protetta e sostenuta.
        \item Realizza il suo bisogno di protezione, che aveva sempre represso per mostrarsi forte e indipendente.
        \item Si rende conto di aver rifiutato il sostegno del suo fidanzato nella vita reale, comprendendo ora l'importanza di permettere agli altri di prendersi cura di lei.
    \end{itemize}
    \item \textbf{Decisione Interiore}
    \begin{itemize}
        \item Decide che, una volta uscita da quella situazione, riconsidererà il suo rapporto con il fidanzato.
        \item Vuole permettergli di esserci per lei, vedendo questo non come una debolezza, ma come una connessione autentica.
    \end{itemize}
\end{itemize}

